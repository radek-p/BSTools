
%/* //////////////////////////////////////////////////// */
%/* This file is a part of the BSTools procedure package */
%/* written by Przemyslaw Kiciak.                        */
%/* //////////////////////////////////////////////////// */

\newpage
\section{Operacje na ci"agach w"ez"l"ow}

W tym punkcie s"a zamieszczone opisy procedur dokonuj"acych r"o"rnych
pomocniczych dzia"la"n na ci"agach w"ez"l"ow, takich jak wyszukiwanie,
tworzenie nowych ci"ag"ow i~obliczanie ich d"lugo"sci (przed utworzeniem, co
przydaje si"e w~celu zarezerwowania tablic o~odpowiedniej d"lugo"sci).

\subsection{Przeszukiwanie ci"agu w"ez"l"ow}

Ci"agi w"ez"l"ow $u_0,\ldots,u_N$ przekazywane w~tablicach jako parametry
wszystkich procedur musz"a by"c niemalej"ace i~musz"a spe"lnia"c warunek
$u_n<u_{N-n}$. Odpowiedzialno"s"c za spe"lnienie tych warunk"ow spoczywa na
procedurach wywo"luj"acych (bo ka"rdorazowe sprawdzanie zabiera czas).

\vspace{\bigskipamount}
\cprog{%
int mbs\_KnotMultiplicityf ( int lastknot, const float *knots, \\
\ind{28}float t );}
Procedura \texttt{mbs\_KnotMultiplicityf} otrzymuje
tablic"e \texttt{knots}, kt"ora zawiera niemalej"acy ci"ag liczb
o~d"lugo"sci $N+1$, gdzie $N$~jest warto"sci"a parametru \texttt{lastknot}.
Warto"sci"a procedury jest liczba wyst"apie"n liczby \texttt{t} w~tym ci"agu.


\vspace{\bigskipamount}
\cprog{%
int mbs\_FindKnotIntervalf ( int degree, \\
\ind{28}int lastknot, const float *knots, \\
\ind{28}float t, int *mult );}
\begin{sloppypar}
Procedura \texttt{mbs\_FindKnotIntervalf} otrzymuje tablic"e \texttt{knots},
kt"ora zawiera niemalej"acy ci"ag liczb o d"lugo"sci $N+1$ (liczba $N$ jest
warto"sci"a parametru \texttt{lastknot}). Je"sli parametr \texttt{degree} ma
warto"s"c $-1$, to warto"sci"a procedury jest indeks $k$ do tablicy, taki "re
$\mbox{\texttt{knots[$k$]}}\leq\mbox{\texttt{t}}<\mbox{\texttt{knots[$k+1$]}}$.
Mo"re te"r by"c $-1$ je"sli $t<\mbox{\texttt{knots[$0$]}}$ lub
$N$ je"sli $t\geq\mbox{\texttt{knots[$N$]}}$.
\end{sloppypar}

Je"sli warto"s"c $n$ parametru \texttt{degree} jest nieujemna, to najmniejsza
zwracana warto"s"c procedury mo"re by"c r"owna~$n$, a~najwi"eksza $N-n-1$.
Nast"epuje domniemanie, "re procedura jest wywo"lana w~celu wyznaczenia
przedzia"lu mi"edzy w"ez"lami, kt"oremu odpowiada pewien wielomian lub "luk
wielomianowy opisuj"acy funkcj"e lub krzyw"a sklejan"a stopnia~$n$. Po
znalezieniu numeru tego "luku mo"rna oblicza"c jego punkty (np.\ algorytmem
de~Boora). W~ten spos"ob je"sli $t\notin[u_n,u_{N-1})$, to b"ed"a obliczane
punkty pierwszego albo ostatniego "luku opisuj"acego krzyw"a.

Parametr \texttt{mult} s"lu"ry do przekazania krotno"sci w"ez"la~$t$. Je"sli
jego warto"sci"a jest wska"znik pusty (\texttt{NULL}), to parametr ten jest
ignorowany. W~przeciwnym razie je"sli $t=u_k$ (dla $k$ r"ownego warto"sci
procedury), to zmienna \texttt{*mult} otrzymuje warto"s"c r"own"a liczbie
wyst"apie"n liczby $t$ w~podanym ci"agu w"ez"l"ow.



\subsection{Tworzenie ci"ag"ow w"ez"l"ow}

\cprog{%
int mbs\_NumKnotIntervalsf ( int degree, int lastknot, \\
\ind{28}const float *knots );}
\hspace*{\parindent}Procedura \texttt{mbs\_NumKnotIntervalsf} oblicza,
z~ilu przedzia"l"ow mi"edzy
w"ez"lami sk"lada si"e dziedzina funkcji (albo krzywych) sklejanych stopnia
\texttt{degree}, okre"slonych dla ci"agu w"ez"l"ow o~d"lugo"sci
\texttt{*lastknot+1}, podanego w tablicy \texttt{knots}.


\vspace{\bigskipamount}
\cprog{%
int mbs\_LastknotMaxInsf ( int degree, int lastknot, \\
\ind{26}const float *knots, \\
\ind{26}int *numknotintervals );}
Procedura \texttt{mbs\_LastknotMaxInsf} oblicza indeks ostatniego w"ez"la
reprezentacji krzywych, kt"ora zostanie utworzona przez
procedur"e \texttt{mbs\_MaxKnotInsf}.

\vspace{\bigskipamount}
\cprog{%
int mbs\_NumMaxKnotsf ( int degree, int lastknot, \\
\ind{23}const float *knots );}
Procedura \texttt{mbs\_NumMaxKnotsf} oblicza d"lugo"s"c ci"agu w"ez"l"ow
potrzebn"a do reprezentowania w lokalnych bazach Bernsteina stopnia
\texttt{degree} funkcji lub krzywej sklejanej okre"slonej dla ci"agu
w"ez"l"ow o d"lugo"sci \texttt{lastknot+1}, podanego w tablicy
\texttt{knots}.

\vspace{\bigskipamount}
\cprog{%
void mbs\_SetKnotPatternf ( int lastinknot, const float *inknots, \\
\ind{27}int multipl, \\
\ind{27}int *lastoutknot, float *outknots );}
Procedura \texttt{mbs\_SetKnotPatternf} s"lu"ry do wygenerowania ci"agu
w"ez"l"ow, kt"ory sk"lada si"e z~liczb podanych w~tablicy \texttt{inknots}
(o~d"lugo"sci $\mathord{\mbox{\texttt{lastinknot}}}+1$),
ale w~kt"orym ka"rdy w"eze"l ma krotno"s"c \texttt{multipl}.

Ci"ag w"ez"l"ow
jest wpisywany do tablicy \texttt{outknots}, a~indeks ostatniego w"ez"la
jest zwracana poprzez parametr \texttt{*lastoutknot}.


\newpage
\subsection{Reparametryzacja krzywych i~p"lat"ow}

%\vspace{\bigskipamount}
\cprog{%
void mbs\_TransformAffKnotsf ( int degree, int lastknot, \\
\ind{30}const float *inknots, \\
\ind{30}float a, float b, float *outknots );}
\begin{sloppypar}
\hspace*{\parindent}Procedura \texttt{mbs\_TransformAffKnotsf} dokonuje
przekszta"lcenia
afinicznego dziedziny krzywej B-sklejanej, tj.\ oblicza ci"ag w"ez"l"ow
zwi"azany z~now"a dziedzin"a, co~jest r"ownowa"rne reparametryzacji krzywej.
,,Star"a'' dziedzin"a jest przedzia"l
$[u_n,u_{N-n}]$, za"s now"a przedzia"l $[a,b]$. Parametr \texttt{degree}
okre"sla stopie"n $n$ krzywej, parametr \texttt{lastknot} okre"sla indeks
ostatniego w"ez"la, tablica \texttt{inknots} zawiera ci"ag w"ez"l"ow
$u_0,\ldots,u_N$.
\end{sloppypar}

Parametry \texttt{a}, \texttt{b} okre"slaj"a brzegi przedzia"lu $[a,b]$,
przy czym powinien by"c spe"lniony warunek $a<b$ (procedura go nie sprawdza).
Nowy ci"ag w"ez"l"ow procedura wpisuje do tablicy \texttt{outknots}.

Parametry \texttt{inknots} i~\texttt{outknots} mog"a wskazywa"c dwie r"o"rne
(roz"l"aczne) tablice o~d"lugo"sci $N+1$, lub te"r t"e sam"a tablic"e.
W~drugim przypadku procedura dokonuje zmiany reprezentacji
(reparametryzacji) krzywej ,,w~miejscu''.

\vspace{\bigskipamount}
\cprog{%
void mbs\_multiReverseBSCurvef ( int degree, int lastknot, \\
\ind{32}float *knots, \\
\ind{32}int ncurves, int spdimen, \\
\ind{32}int pitch, float *ctlpoints );}
Procedura \texttt{mbs\_multiReverseBSCurvef} dokonuje reparametryzacji
krzywych \mbox{B-sklejanych} stopnia $n$, polegaj"acej na podstawieniu
parametru $-t$ w~miejsce~$t$.

Parametr \texttt{degree} okre"sla stopie"n $n$ krzywych. Parametry
\texttt{lastknot} i~\texttt{knots} opisuj"a ci"ag w"ez"l"ow, na kt"orym
oparta jest reprezentacja krzywych. Parametr \texttt{ncurves} okre"sla
liczb"e krzywych, a~\texttt{spdimen} wymiar przestrzeni, w~kt"orej le"r"a
krzywe.

Je"sli parametr \texttt{knots} jest r"owny \texttt{NULL}, to procedura tylko
odwraca kolejno"s"c punkt"ow kontrolnych. Dzi"eki temu mo"rna jej u"ry"c do
,,odwracania'' krzywych (lub p"lat"ow) B\'{e}ziera. W~tym przypadku
warto"s"c parametru \texttt{lastknot} jest ignorowana (krzywa stopnia~$n$ ma
$n+1$ punkt"ow kontrolnych).

Parametr \texttt{pitch} okre"sla podzia"lk"e tablicy \texttt{ctlpoints},
w~kt"orej s"a podane punkty kontrolne krzywych.

Obliczenie realizowane jest ,,w~miejscu'' i~polega na zmianie znaku na
przeciwny i~odwr"oceniu kolejno"sci w"ez"l"ow oraz na odwr"oceniu
kolejno"sci punkt"ow kontrolnych ka"rdej krzywej. Obliczenie to jest
wykonywane bez b"l"ed"ow zaokr"agle"n.


\newpage
\subsection{Modyfikowanie w"ez"l"ow}

\cprog{%
int mbs\_SetKnotf ( int lastknot, float *knots, \\
\ind{19}int knotnum, int mult, float t );}
Procedura \texttt{mbs\_SetKnotf} zmienia w"eze"l w~danym ci"agu,
z~zachowaniem uporz"adkowania. Parametry:
\texttt{lastknot} --- numer ostatniego w"ez"la w~ci"agu, \texttt{knots} ---
wska"znik tablicy z~ci"agiem w"ez"l"ow, \texttt{knotnum} --- indeks zmienianego
w"ez"la w~ci"agu, \texttt{mult} --- krotno"s"c (now"a warto"s"c otrzymaj"a
w"ez"ly o~indeksach \texttt{knotnum-i+1 \ldots\ knotnum}),
\texttt{t} --- nowa warto"s"c w"ez"la.

Po zmianie w"ez"l"ow ci"ag jest sortowany. Warto"sci"a procedury jest
nowy indeks w"ez"la, a~dok"ladniej liczba~$k$, taka "re $t=u_k<u_{k+1}$.

Warto"s"c procedury $-1$ oznacza b"l"edn"a warto"s"c parametru \texttt{knotnum};
musi ona by"c od~$0$ do \texttt{lastknot}.


\vspace{\bigskipamount}
\cprog{%
int mbs\_SetKnotClosedf ( int degree, int lastknot, float *knots, \\
\ind{25}float T, int knotnum, int mult, float t );}
Procedura \texttt{mbs\_SetKnotClosedf} zmienia w"eze"l w~danym ci"agu,
z~zachowaniem uporz"adkowania i~okresowo"sci wymaganej przez reprezentacj"e
zamkni"etej krzywej B-sklejanej. Parametry: \texttt{degree} --- stopie"n krzywej,
\texttt{lastknot} --- numer ostatniego w"ez"la w~ci"agu, \texttt{knots} ---
wska"znik tablicy z~ci"agiem w"ez"l"ow, \texttt{T} --- d"lugo"s"c dziedziny
krzywej (po zmianie ma by"c \texttt{knots[lastknot-degree]-knots[degree]})
 \texttt{knotnum} --- indeks zmienianego
w"ez"la w~ci"agu, \texttt{mult} --- krotno"s"c (now"a warto"s"c otrzymaj"a
w"ez"ly o~indeksach \texttt{knotnum-i+1 \ldots\ knotnum}),
\texttt{t} --- nowa warto"s"c w"ez"la.

Po zmianie w"ez"l"ow ci"ag jest sortowany. Warto"sci"a procedury jest
nowy indeks w"ez"la, a~dok"ladniej liczba~$k$, taka "re $t=u_k<u_{k+1}$.

Warto"s"c procedury $-1$ oznacza b"l"edn"a warto"s"c parametru \texttt{knotnum};
musi ona by"c od~$0$ do \texttt{lastknot}, lub \texttt{lastknot},
kt"ora musi by"c wi"eksza ni"r \texttt{3*degree}.


\subsection{Sprawdzanie poprawno"sci}

\cprog{%
boolean mbs\_ClosedKnotsCorrectf ( int degree, int lastknot, \\
\ind{34}float *knots, \\
\ind{34}float T, int K, float tol );}
Procedura \texttt{mbs\_ClosedKnotsCorrectf} sprawdza poprawno"s"c ci"agu
w"ez"l"ow przeznaczonego do reprezentowania zamkni"etej krzywej B-sklejanej.
Poprawny ci"ag musi by"c niemalej"acy i~spe"lnia"c warunek $u_{i+K}=u_i+T$
dla $i=1,\ldots,n$, gdzie $K=N-2n$, $N>3n$. Krotno"sci w"ez"l"ow nie mog"a
przekracza"c stopnia~$n$. Parametr \texttt{tol} okre"sla
tolerancj"e (tj.\ dopuszczaln"a r"o"rnic"e $u_{i+K}-T-u_i$);
musi to by"c ma"la liczba dodatnia, nie mo"re by"c~$0$ z~powodu b"l"ed"ow
zaokr"agle"n.


\newpage
\section{Obliczanie warto"sci funkcji B-sklejanych}

Wyznaczenie warto"sci funkcji B-sklejanych mo"re by"c potrzebne do
rozwi"azywania zada"n interpolacyjnych. Funkcje te s"a obliczane na
podstawie wzor"ow~(\ref{eq:BS:basis0}) i~(\ref{eq:BS:basisn}).

\vspace{\bigskipamount}
\cprog{%
void mbs\_deBoorBasisf ( int degree, int lastknot, \\
\ind{24}const float *knots, \\
\ind{24}float t, int *fnz, int *nnz, float *bfv );}

\begin{sloppypar}
Procedura \texttt{mbs\_deBoorBasisf} oblicza warto"sci funkcji B-sklejanych
stopnia $n$ (parametr \texttt{degree}) w~punkcie \texttt{t}. Funkcje
s"a okre"s\-lo\-ne przez podanie niemalej"acego ci"agu w"ez\-"l"ow
$u_0,\ldots,u_M$ w tablicy \texttt{knots}. Liczba w"ez\-"l"ow jest
r"owna \texttt{lastknot+1}.
Warto"s"c parametru \texttt{t} musi by"c liczb"a z przedzia"lu
$[u_n,u_{N-n}]$.
\end{sloppypar}

Obliczone warto"sci funkcji B-sklejanych s"a umieszczane w tablicy
\texttt{bfv}, przy czym warto"s"c parametru \texttt{*fnz} na wyj"sciu jest
r"owna numerowi pierwszej funkcji r"o"rnej od $0$ dla podanego \texttt{t};
tablica \texttt{bfv} musi mie"c d"lugo"s"c \texttt{degree+1}.

Parametr \texttt{*nnz} s"lu"ry do przekazania inormacji o~liczbie funkcji
bazowych r"o"rnych od $0$ w punkcie $t$. Zawarto"s"c miejsc w tablicy
\texttt{bfv} zaczynaj"ac od miejsca \texttt{*nnz} jest nieokre"slona
(ale procedura mo"re wpisywa"c po"srednie wyniki swoich oblicze"n do
pierwszych $\mbox{\texttt{degree}}+1$ miejsc w tej tablicy).


\newpage
\section{Wyznaczanie punkt"ow krzywych i p"lat"ow}

\subsection{Algorytm de Boora}

Algorytm de~Boora obliczania punktu $\bm{s}(t)$ krzywej $\bm{s}$ danej
wzorem~(\ref{eq:BScurve:def}), dla $t\in[u_k,u_{k+1})$,
$k\in\{n,\ldots,N-n-1\}$, polega na rekurencyjnym obliczeniu
punkt"ow $\bm{d}^{(j)}_i$ dla $j=1,\ldots,n-r$ oraz $i=k-n,\ldots,k-r$,
na podstawie wzoru
\begin{align}
  \bm{d}^{(j)}_i ={}&
  (1-\alpha^{(j)}_i)\bm{d}^{(j-1)}_{i-1} + \alpha^{(j)}_i\bm{d}^{(j-1)}_i, \\
  \mbox{gdzie}\quad \alpha^{(j)}_i ={}& \frac{t-u_i}{u_{i+n+1-j}-u_i}. \nonumber
\end{align}
Punkty $\bm{d}^{(0)}_i=\bm{d}_i$ s"a punktami kontrolnymi krzywej,
za"s liczba~$r$ jest liczba"a wyst"apie"n (krotno"sci"a) liczby $t$ w~ci"agu
w"ez"l"ow $u_0,\ldots,u_N$.

\vspace{\bigskipamount}
\cprog{%
int mbs\_multideBoorf ( int degree, int lastknot, \\
\ind{23}const float *knots, \\
\ind{23}int ncurves, int spdimen, \\
\ind{23}int pitch, const float *ctlpoints, \\
\ind{23}float t, float *cpoints );}
\begin{sloppypar}
Procedura \texttt{mbs\_multideBoorf} jest implementacj"a
algorytmu de~Boora
obliczania punk\-tu na krzywej B-sklejanej. Dane dla procedury to
\texttt{ncurves} krzywych B-sklejanych stopnia \texttt{degree}, po"lo"ronych
w~przestrzeni o wymiarze \texttt{spdimen}. Ka"rda z tych krzywych jest
okre"slona za pomoc"a tego samego niemalej"acego ci"agu \texttt{lastknot+1}
w"ez"l"ow przechowywanych w tablicy \texttt{knots}.
\end{sloppypar}

"Lamane kontrolne s"a przekazywane w tablicy \texttt{ctlpoints}; ka"rda z
nich jest opisana przez \texttt{(lastknot-degree)*spdimen} liczb
zmiennopozycyjnych, przy czym pocz"atek opisu kolejnej "lamanej jest
\texttt{pitch} miejsc po poprzedniej.

Parametr \texttt{t} ma warto"s"c parametru, dla kt"orego nale"ry obliczy"c
punkt na ka"rdej z krzywych. Wsp"o"lrz"edne tych punkt"ow s"a wpisywane do
tablicy \texttt{cpoints}, kt"ora musi mie"c d"lugo"s"c co najmniej
\texttt{ncurves*spdimen}.

Warto"sci"a procedury jest liczba $n-r$, tj.\ r"o"rnica stopnia krzywych
i~krotno"sci liczby~$t$ w~ci"agu w"ez"l"ow. Je"sli liczba ta jest nieujemna,
to okre"sla minimaln"a klas"e ci"ag"lo"sci krzywych w~otoczeniu~$t$.

\vspace{\bigskipamount}
\cprog{%
\#define mbs\_deBoorC1f(degree,lastknot,knots,coeff,t,value) \bsl \\
\ind{2}mbs\_multideBoorf(degree,lastknot,knots,1,1,0,coeff,t,value) \\
\#define mbs\_deBoorC2f(degree,lastknot,knots,coeff,t,value) \bsl \\
\ind{2}mbs\_multideBoorf(degree,lastknot,knots,1,2,0,coeff,t,value) \\
\#define mbs\_deBoorC3f(degree,lastknot,knots,coeff,t,value) ... \\
\#define mbs\_deBoorC4f(degree,lastknot,knots,coeff,t,value) ...}
Cztery makra, kt"ore wywo"luj"a procedur"e \texttt{mbs\_multideBoorf} w celu
obliczenia warto"sci \emph{jednej} skalarnej funkcji sklejanej lub
B-sklejanej krzywej p"laskiej, przestrzennej lub czterowymiarowej. Parametry
makra powinny spe"lnia"c warunki podane w~opisie procedury
\texttt{mbs\_multideBoorf}.

\vspace{\bigskipamount}
\cprog{%
void mbs\_deBoorC2Rf ( int degree, \\
\ind{22}int lastknot, const float *knots, \\
\ind{22}const point3f *ctlpoints, float t, \\
\ind{22}point2f *cpoint );}
\begin{sloppypar}
Procedura \texttt{mbs\_deBoorC2Rf} oblicza punkt wymiernej B-sklejanej
krzywej p"las\-kiej stopnia \texttt{degree}, okre"slonej dla
niemalej"acego ci"agu \texttt{lastknot+1} w"ez"l"ow podanych w~tablicy
\texttt{knots}. W~tablicy \texttt{ctlpoints} nale"ry poda"c punkty kontrolne
\emph{krzywej jednorodnej} (po"lo"ronej w $\R^3$).

Argument krzywej jest warto"sci"a parametru \texttt{t} i~musi nale"re"c
do przedzia"lu
$[\mbox{\texttt{knots[degree]}},\mbox{\texttt{knots[lastknot-degree]}}]$.
Wsp"o"l\-rz"ed\-ne obliczonego punktu
krzywej s"a przekazywane za pomoc"a parametru \texttt{cpoint}.
\end{sloppypar}

W"la"sciwe obliczenie wykonuje procedura \texttt{mbs\_multideBoorf}.

\vspace{\bigskipamount}
\cprog{%
void mbs\_deBoorC3Rf ( int degree, \\
\ind{22}int lastknot, const float *knots, \\
\ind{22}const point4f *ctlpoints, float t, \\
\ind{22}point3f *cpoint );}
\begin{sloppypar}
Procedura \texttt{mbs\_deBoorC3Rf} oblicza punkt wymiernej krzywej
B-sklejanej stopnia \texttt{degree} w przestrzeni tr"ojwymiarowej,
okre"slonej dla niemalej"acego ci"agu \texttt{lastknot+1} w"ez"l"ow
podanych w tablicy \texttt{knots}.

W tablicy \texttt{ctlpoints}
nale"ry poda"c punkty kon\-trol\-ne \emph{krzywej jednorodnej}
(po"lo"ronej w $\R^4$).
Argument krzywej jest warto"sci"a parametru \texttt{t} i musi nale"re"c
do przedzia"lu
$[\mbox{\texttt{knots[degree]}},\mbox{\texttt{knots[lastknot-degree]}}]$.
Wsp"o"lrz"edne obliczonego punktu
krzywej s"a przekazywane za pomoc"a parametru \texttt{cpoint}.
\end{sloppypar}

W"la"sciwe obliczenie wykonuje procedura \texttt{mbs\_multideBoorf}.


\vspace{\bigskipamount}
\cprog{%
void mbs\_deBoorP3f ( int degreeu, \\
\ind{21}int lastknotu, const float *knotsu, \\
\ind{21}int degreev, \\
\ind{21}int lastknotv, const float *knotsv, \\
\ind{21}int pitch, \\
\ind{21}const point3f *ctlpoints, \\
\ind{21}float u, float v, point3f *ppoint );}
Procedura \texttt{mbs\_deBoorP3f} oblicza punkt p"lata B-sklejanego
w~przestrzeni tr"ojwymiarowej. Stopie"n p"lata ze wzgl"edu na parametry
$u$~i~$v$ s"a r"owne odpowiednio \texttt{degreeu} i~\texttt{degreev}.
Ci"ag w"ez"l"ow ,,$u$'' o d"lugo"sci \texttt{lastknotu+1} jest podany
w~tablicy \texttt{knotsu}, podobnie ci"ag w"ez"l"ow ,,$v$'' o d"lugo"sci
\texttt{lastknotv+1} jest podany w~tablicy \texttt{knotsv}.

Punkty kontrolne p"lata s"a podane w tablicy \texttt{ctlpoints}, przy czym
ich kolejno"s"c jest nast"epuj"aca: najpierw s"a punkty z~pierwszej kolumny
siatki, potem z drugiej itd, przy czym kolumna sk"lada si"e z
\texttt{lastknotv-degreev} punkt"ow.

Parametry \texttt{u} i \texttt{v} okre"slaj"a punkt w dziedzinie, dla
kt"orego nale"ry obliczy"c punkt p"lata. Wsp"o"lrz"edne tego punktu s"a
zwracane za pomoc"a parametru \texttt{ppoint}.

W"la"sciwe obliczenie polega na wywo"laniu procedury
\texttt{mbs\_multideBoorf}.

\vspace{\bigskipamount}
\cprog{%
void mbs\_deBoorP3Rf ( int degreeu, \\
\ind{22}int lastknotu, const float *knotsu, \\
\ind{22}int degreev, \\
\ind{22}int lastknotv, const float *knotsv, \\
\ind{22}int pitch, \\
\ind{22}const point4f *ctlpoints, \\
\ind{22}float u, float v, point3f *ppoint );}
\begin{sloppypar}
Procedura \texttt{mbs\_deBoorP3Rf} wyznacza punkt wymiernego p"lata
B-sklejanego
w~przes\-trze\-ni tr"ojwymiarowej. Stopie"n p"lata ze wzgl"edu na parametry
$u$~i~$v$ s"a r"owne odpowiednio \texttt{degreeu} i~\texttt{degreev}.
Ci"ag w"ez"l"ow ,,$u$'' o d"lugo"sci \texttt{lastknotu+1} jest podany
w~tablicy \texttt{knotsu}, podobnie ci"ag w"ez"l"ow ,,$v$'' o d"lugo"sci
\texttt{lastknotv+1} jest podany w~tablicy \texttt{knotsv}.
\end{sloppypar}

W tablicy \texttt{ctlpoints} nale"ry poda"c punkty kontrolne \emph{p"lata
jednorodnego}, przy czym ich kolejno"s"c jest nast"epuj"aca: najpierw s"a
punkty z~pierwszej kolumny siatki, potem z drugiej itd, przy czym kolumna
sk"lada si"e z \texttt{lastknotv-degreev} punkt"ow.

Parametry \texttt{u} i \texttt{v} okre"slaj"a punkt w dziedzinie, dla
kt"orego nale"ry obliczy"c punkt p"lata. Wsp"o"lrz"edne tego punktu s"a
zwracane za pomoc"a parametru \texttt{ppoint}.

W"la"sciwe obliczenie polega na wywo"laniu procedury
\texttt{mbs\_multideBoorf}.


\vspace{\bigskipamount}
\cprog{%
void mbs\_deBoorP4f ( int degreeu, \\
\ind{21}int lastknotu, const float *knotsu, \\
\ind{21}int degreev, \\
\ind{21}int lastknotv, const float *knotsv, \\
\ind{21}int pitch, \\
\ind{21}const point4f *ctlpoints, \\
\ind{21}float u, float v, point4f *ppoint );}
Procedura \texttt{mbs\_deBoorP4f} oblicza punkt p"lata B-sklejanego
w~przestrzeni czterowymiarowej. Stopie"n p"lata ze wzgl"edu na parametry
$u$~i~$v$ s"a r"owne odpowiednio \texttt{degreeu} i~\texttt{degreev}.
Ci"ag w"ez"l"ow ,,$u$'' o d"lugo"sci \texttt{lastknotu+1} jest podany
w~tablicy \texttt{knotsu}, podobnie ci"ag w"ez"l"ow ,,$v$'' o d"lugo"sci
\texttt{lastknotv+1} jest podany w~tablicy \texttt{knotsv}.

Punkty kontrolne p"lata s"a podane w tablicy \texttt{ctlpoints}, przy czym
ich kolejno"s"c jest nast"epuj"aca: najpierw s"a punkty z~pierwszej kolumny
siatki, potem z drugiej itd, przy czym kolumna sk"lada si"e z
\texttt{lastknotv-degreev} punkt"ow.

Parametry \texttt{u} i \texttt{v} okre"slaj"a punkt w dziedzinie, dla
kt"orego nale"ry obliczy"c punkt p"lata. Wsp"o"lrz"edne tego punktu s"a
zwracane za pomoc"a parametru \texttt{ppoint}.

W"la"sciwe obliczenie polega na wywo"laniu procedury
\texttt{mbs\_multideBoorf}.


\vspace{\bigskipamount}
Pochodna krzywej sklejanej $\bm{s}$ danej wzorem~(\ref{eq:BScurve:def})
w~punkcie $t$ wyra"ra si"e wzorem
\begin{align}
  \bm{s}'(t) = \frac{n}{u_{k+1}-u_k}(\bm{d}^{(n-r-1)}_{k-r} -
  \bm{d}^{(n-r-1)}_{k-r-1}),
\end{align}
w~kt"orym wyst"epuj"a punkty $\bm{d}^{(n-r-1)}_{k-r}$
i~$\bm{d}^{(n-r-1)}_{k-r-1}$ obliczane w~algorytmie de~Boora. Opisane ni"rej
procedury realizuj"a ten algorytm uzupe"lniony o~obliczenie pochodnej.

\vspace{\bigskipamount}
\cprog{%
int mbs\_multideBoorDerf ( int degree, int lastknot, \\
\ind{26}const float *knots, \\
\ind{26}int ncurves, int spdimen, \\
\ind{26}int pitch, \\
\ind{26}const float *ctlpoints, \\
\ind{26}float t, float *cpoints, \\
\ind{26}float *dervect );}
\begin{sloppypar}
Procedura \texttt{mbs\_multideBoorDerf} s"lu"ry do obliczenia punkt"ow na
\texttt{ncurves} krzywych B-sklejanych stopnia \texttt{degree}, po"lo"ronych
w~przestrzeni o~wymiarze \texttt{spdimen}. Dodatkowo procedura oblicza wektory
pochodnych tych krzywych dla podanego argumentu~\texttt{t}.
\end{sloppypar}

Spos"ob reprezentowania krzywych jest taki sam jak spos"ob reprezentowania
krzywych dla potrzeb procedury \texttt{mbs\_multideBoorf}. Obliczone
wsp"o"lrz"edne punkt"ow krzywych s"a wpisywane (w taki sam spos"ob) do
tabliy \texttt{cpoints}, a wsp"o"lrz"edne wektor"ow pochodnych do tablicy
\texttt{dervect}.

\begin{sloppypar}
Je"sli warto"s"c $t$ parametru \texttt{t} jest r"owna w"ez"lowi o krotno"sci
\texttt{degree} lub wi"ekszej, to procedura oblicza warto"sci pochodnych
prawostronnych krzywych w punkcie $t$, z~wyj"atkiem przypadku, gdy liczba
$t$ okre"sla koniec dziedziny krzywych (to znaczy gdy
$t=\mbox{\texttt{knots[lastknot-degree]}}$). W~tym przypadku obliczane s"a
pochodne lewostronne. Wartosci"a procedury jest r"o"znica stopnia krzywych
i~krotno"sci liczby~$t$ w~ci"agu w"ez"l"ow, kt"ora okre"sla klas"e
ci"ag"lo"sci krzywych w~otoczeniu punktu~$t$.
\end{sloppypar}

\vspace{\bigskipamount}
\cprog{%
\#define mbs\_deBoorDerC1f(degree,lastknot,knots,ctlpoints,t,\bsl \\
\ind{4}cpoint,cder)\bsl \\
\ind{2}mbs\_multideBoorDerf(degree,lastknot,knots,1,1,0,ctlpoints,t,\bsl\\
\ind{4}cpoint,cder) \\
\#define mbs\_deBoorDerC2f(degree,lastknot,knots,ctlpoints,t,\bsl \\
\ind{4}cpoint,cder)\bsl \\
\ind{2}mbs\_multideBoorDerf(degree,lastknot,knots,1,2,0,ctlpoints,t,\bsl\\
\ind{4}cpoint,cder) \\
\#define mbs\_deBoorDerC3f(degree,lastknot,knots,ctlpoints,t,\bsl \\
\ind{4}cpoint,cder)\ ... \\
\#define mbs\_deBoorDerC4f(degree,lastknot,knots,ctlpoints,t,\bsl \\
\ind{4}cpoint,cder)\ ...}
\begin{sloppypar}
Cztery makra wywo"luj"ace procedur"e \texttt{mbs\_multideBoorDerf} w celu
obliczenia warto"sci \emph{jednej} funkcji sklejanej lub punktu krzywej
B-sklejanej wraz pochodn"a w punkcie \texttt{t}. Parametry musz"a spe"lnia"c
warunki podane w opisie procedury \texttt{mbs\_multideBoorDerf}.
\end{sloppypar}

\vspace{\bigskipamount}
\cprog{%
int mbs\_multideBoorDer2f ( int degree, \\
\ind{23}int lastknot, const float *knots, \\
\ind{23}int ncurves, int spdimen, \\
\ind{23}int pitch, const float *ctlpoints, \\
\ind{23}float t, float *p, float *d1, float *d2 );}
\begin{sloppypar}
Procedura \texttt{mbs\_multideBoorDer2f} oblicza punkty $\bm{s}_i(t)$ oraz
wektory $\bm{s}'_i(t)$ i~$\bm{s}''_i(t)$ krzywych B-sklejanych $\bm{s}_i$
stopnia~$n$ dla ustalonego~$t$.
\end{sloppypar}

Parametry wej"sciowe: \texttt{degree} --- stopie"n~$n$,
\texttt{lastknot} --- numer~$N$ ostatniego w"ez"la,
\texttt{knots} --- tablica w"ez"l"ow, \texttt{ncurves}
--- liczba krzywych, \texttt{pitch} --- podzia"lka tablicy punkt"ow
kontrolnych, \texttt{ctlpoints} --- tablica punkt"ow kontrolnych,
\texttt{t} --- liczba~$t$.

Parametry wyj"sciowe: \texttt{p} --- tablica, do kt"orej procedura wstawia
punkty $\bm{s}_i(t)$, \texttt{d1} --- tablica, do~kt"orej procedura wstawia
wektory $\bm{s}'_i(t)$, \texttt{d2} --- tablica, do~kt"orej procedura
wstawia wektory $\bm{s}''_i(t)$. Podzia"lka wszystkich tych tablic jest
r"owna wymiarowi przestrzeni, \texttt{spdimen}.

Warto"sci"a procedury jest liczba $n-r$, gdzie $r$ oznacza krotno"s"c
(liczb"e wyst"apie"n) liczby~$t$ w~ci"agu w"ez"l"ow. Okre"sla ona klas"e
ci"ag"lo"sci krzywych $\bm{s}_i$ w~otoczeniu punktu~$t$.

Je"sli krzywa lub kt"ora"s pochodna jest nieci"ag"la w~otoczeniu~$t$, to
obliczony punkt lub wektor jest r"owny granicy lewostronnej (np.\
$\lim_{x\searrow t}\bm{s}'(x)$).

\vspace{\bigskipamount}
\cprog{%
\#define mbs\_deBoorDer2C1f(degree,lastknot,knots,coeff,t,p,d1,d2) \bsl \\
\ind{2}mbs\_multideBoorDer2f(degree,lastknot,knots,1,1,0,coeff,t,p,d1,d2) \\
\#define mbs\_deBoorDer2C2f(degree,lastknot,knots,ctlpoints,t, \bsl \\
\ind{4}p,d1,d2) \bsl \\
\ind{2}mbs\_multideBoorDer2f(degree,lastknot,knots,1,2,0, \bsl \\
\ind{4}(float*)ctlpoints,t,(float*)p,(float*)d1,(float*)d2) \\
\#define mbs\_deBoorDer2C3f(degree,lastknot,knots,ctlpoints,t, \bsl \\
\ind{4}p,d1,d2)\ ... \\
\#define mbs\_deBoorDer2C4f(degree,lastknot,knots,ctlpoints,t, \bsl \\
\ind{4}p,d1,d2)\ ...}
\begin{sloppypar}
Powy"rsze makra s"lu"r"a do wywo"lywania procedury
\texttt{mbs\_multideBoorDer2f} w~sytuacji, gdy nale"ry obliczy"c punkt
i~pochodne rz"edu $1$ i $2$ jednej krzywej \mbox{B-sklejanej} po"lo"ronej
w~przestrzeni o~wymiarze $1$, $2$, $3$ lub $4$.
\end{sloppypar}

\newpage
%\vspace{\bigskipamount}
\cprog{%
int mbs\_multideBoorDer3f ( int degree, \\
\ind{23}int lastknot, const float *knots, \\
\ind{23}int ncurves, int spdimen, \\
\ind{23}int pitch, const float *ctlpoints, float t, \\
\ind{23}float *p, float *d1, float *d2, float *d3 );}
\begin{sloppypar}
Procedura \texttt{mbs\_multideBoorDer3f} oblicza punkty $\bm{s}_i(t)$ oraz
wektory $\bm{s}'_i(t)$, $\bm{s}''_i(t)$ i~$\bm{s}'''_i(t)$ krzywych
B-sklejanych $\bm{s}_i$ stopnia~$n$ dla ustalonego~$t$.
\end{sloppypar}

Parametry wej"sciowe: \texttt{degree} --- stopie"n~$n$,
\texttt{lastknot} --- numer~$N$ ostatniego w"ez"la,
\texttt{knots} --- tablica w"ez"l"ow, \texttt{ncurves}
--- liczba krzywych, \texttt{pitch} --- podzia"lka tablicy punkt"ow
kontrolnych, \texttt{ctlpoints} --- tablica punkt"ow kontrolnych,
\texttt{t} --- liczba~$t$.

Parametry wyj"sciowe: \texttt{p} --- tablica, do kt"orej procedura wstawia
punkty $\bm{s}_i(t)$, \texttt{d1} --- tablica, do~kt"orej procedura wstawia
wektory $\bm{s}'_i(t)$, \texttt{d2} --- tablica, do~kt"orej procedura
wstawia wektory $\bm{s}''_i(t)$, \texttt{d3} --- tablica, do~kt"orej procedura
wstawia wektory $\bm{s}'''_i(t)$. Podzia"lka wszystkich tych tablic jest
r"owna wymiarowi przestrzeni, \texttt{spdimen}.

Warto"sci"a procedury jest liczba $n-r$, gdzie $r$ oznacza krotno"s"c
(liczb"e wyst"apie"n) liczby~$t$ w~ci"agu w"ez"l"ow. Okre"sla ona klas"e
ci"ag"lo"sci krzywych $\bm{s}_i$ w~otoczeniu punktu~$t$.

Je"sli krzywa lub kt"ora"s pochodna jest nieci"ag"la w~otoczeniu~$t$, to
obliczony punkt lub wektor jest r"owny granicy lewostronnej (np.\
$\lim_{x\searrow t}\bm{s}'(x)$).

\vspace{\bigskipamount}
\cprog{%
\#define mbs\_deBoorDer3C1f(degree,lastknot,knots,coeff,t, \bsl \\
\ind{4}p,d1,d2,d3) \bsl \\
\ind{2}mbs\_multideBoorDer3f(degree,lastknot,knots,1,1,0,coeff,t, \bsl \\
\ind{4}p,d1,d2,d3) \\
\#define mbs\_deBoorDer3C2f(degree,lastknot,knots,ctlpoints,t, \bsl \\
\ind{4}p,d1,d2,d3) \bsl \\
\ind{2}mbs\_multideBoorDer3f(degree,lastknot,knots,1,2,0, \bsl \\
\ind{4}(float*)ctlpoints,t,(float*)p,(float*)d1,(float*)d2,(float*)d3) \\
\#define mbs\_deBoorDer3C3f(degree,lastknot,knots,ctlpoints,t, \bsl \\
\ind{4}p,d1,d2,d3)\ ... \\
\#define mbs\_deBoorDer3C4f(degree,lastknot,knots,ctlpoints,t, \bsl \\
\ind{4}p,d1,d2,d3)\ ...}
\begin{sloppypar}
Powy"rsze makra s"lu"r"a do wywo"lywania procedury
\texttt{mbs\_multideBoorDer3f} w~sytuacji, gdy nale"ry obliczy"c punkt
i~pochodne rz"edu $1$, $2$ i~$3$ jednej krzywej \mbox{B-sklejanej}
po"lo"ronej w~przestrzeni o~wymiarze $1$, $2$, $3$ lub $4$.
\end{sloppypar}

\begin{figure}[ht]
  \centerline{\epsfig{file=bsder123.ps}}
  \caption{Wektory pochodnych rz"edu $1$, $2$ i~$3$ krzywej B-sklejanej}
  \centerline{pi"atego stopnia, obliczone przez procedury
    \texttt{mbs\_multideBoorDerf},}
  \centerline{\texttt{mbs\_multideBoorDer2f} i~\texttt{mbs\_multideBoorDer3f}}
\end{figure}

\vspace{\bigskipamount}
\cprog{%
char mbs\_deBoorDerPf ( int degreeu, int lastknotu, \\
\ind{20}const float *knotsu, \\
\ind{20}int degreev, int lastknotv, \\
\ind{20}const float *knotsv, \\
\ind{20}int spdimen, int pitch, const float *ctlpoints, \\
\ind{20}float u, float v, \\
\ind{20}float *ppoint, \\
\ind{20}float *uder, float *vder );}
Procedura \texttt{mbs\_deBoorDerPf} s"lu"ry do obliczenia punktu p"lata
B-sklejanego, odpowiadaj"acego punktowi
dziedziny o wsp"o"lrz"ednych \texttt{u} i \texttt{v}, oraz wektor"ow
pochodnych cz"astkowych w tym punkcie.

\vspace{\bigskipamount}
\cprog{%
char mbs\_deBoorDer2Pf ( int degreeu, int lastknotu, \\
\ind{20}const float *knotsu, \\
\ind{20}int degreev, int lastknotv, \\
\ind{20}const float *knotsv, \\
\ind{20}int spdimen, int pitch, const float *ctlpoints, \\
\ind{20}float u, float v, \\
\ind{20}float *ppoint, \\
\ind{20}float *uder, float *vder, \\
\ind{20}float *uuder, float *uvder, float *vvder );}
Procedura \texttt{mbs\_deBoorDer2Pf} s"lu"ry do obliczenia punktu p"lata
B-sklejanego, odpowiadaj"acego punktowi
dziedziny o wsp"o"lrz"ednych \texttt{u} i \texttt{v}, oraz wektor"ow
pochodnych cz"astkowych pierwszego i~drugiego rz"edu w~tym punkcie.

\vspace{\bigskipamount}
\cprog{%
char mbs\_deBoorDer3Pf ( int degreeu, int lastknotu, \\
\ind{6}const float *knotsu, \\
\ind{6}int degreev, int lastknotv, \\
\ind{6}const float *knotsv, \\
\ind{6}int spdimen, int pitch, const float *ctlpoints, \\
\ind{6}float u, float v, \\
\ind{6}float *ppoint, \\
\ind{6}float *uder, float *vder, \\
\ind{6}float *uuder, float *uvder, float *vvder, \\
\ind{6}float *uuuder, float *uuvder, float *uvvder, float *vvvder );}
Procedura \texttt{mbs\_deBoorDer3Pf} s"lu"ry do obliczenia punktu p"lata
B-sklejanego, odpowiadaj"acego punktowi
dziedziny o wsp"o"lrz"ednych \texttt{u} i \texttt{v}, oraz wektor"ow
pochodnych cz"astkowych pierwszego, drugiego i~trzeciego rz"edu w~tym punkcie.



\subsection{Schemat Hornera dla krzywych i p"lat"ow B\'{e}ziera}

Schemat Hornera jest algorytmem obliczania warto"sci wielomianu (lub punktu
krzywej) o~z"lo"rono"sci proporcjonalnej do stopnia (algorytmy de~Casteljau
i~de~Boora maj"a z"lo"rono"s"c proporcjonaln"a do kwadratu stopnia). Aby go
stosowa"c do krzywych B-sklejanych (co op"laca si"e wtedy, gdy obliczamy
wiele punkt"ow), trzeba przej"s"c do reprezentacji kawa"lkami B\'{e}ziera,
przy u"ryciu procedury \texttt{mbs\_multiMaxKnotInsf} (zobacz
p.~\ref{ssect:max:knot:ins}).

\vspace{\bigskipamount}
\cprog{%
void mbs\_multiBCHornerf ( int degree, int ncurves, \\
\ind{26}int spdimen, int pitch, \\
\ind{26}const float *ctlpoints, \\
\ind{26}float t, float *cpoints );}
Procedura \texttt{mbs\_multiBCHornerf} oblicza punkty na \texttt{ncurves}
krzywych B\'{e}ziera stopnia \texttt{degree} w przestrzeni o wymiarze
\texttt{spdimen}. Punkty kontrolne tych krzywych s"a podane w tablicy
\texttt{ctlpoints}. Wsp"o"lrz"edne punkt"ow kontrolnych ka"rdej krzywej
s"a upakowane w tej tablicy, przy czym odleg"lo"s"c mi"edzy pocz"atkami
reprezentacji kolejnych dw"och krzywych jest warto"sci"a parametru
\texttt{pitch}.
Parametr \texttt{t} okre"sla parametr krzywych,
dla kt"orego maj"a by"c obliczone punkty. Obliczone punkty s"a wstawiane do
tablicy \texttt{cpoints}, kt"ora musi mie"c d"lugo"s"c co najmniej
\texttt{ncurves*spdimen}.

\vspace{\bigskipamount}
\cprog{%
\#define mbs\_BCHornerC1f(degree,coeff,t,value) \bsl \\
\ind{2}mbs\_multiBCHornerf ( degree, 1, 1, 0, coeff, t, value ) \\
\#define mbs\_BCHornerC2f(degree,coeff,t,value) \bsl \\
\ind{2}mbs\_multiBCHornerf ( degree, 1, 2, 0, coeff, t, value ) \\
\#define mbs\_BCHornerC3f(degree,coeff,t,value) ... \\
\#define mbs\_BCHornerC4f(degree,coeff,t,value) ...}
Cztery makra, kt"ore wywo"luj"a procedur"e \texttt{mbs\_multiBCHornerf}
w~celu obliczenia warto"sci wielomianu danego w bazie Bernsteina albo punktu
na jednej krzywej B\'{e}ziera w przestrzeni dwu-, tr"oj- i czterowymiarowej.

\vspace{\bigskipamount}
\cprog{%
void mbs\_BCHornerC2Rf ( int degree, \\
\ind{24}const point3f *ctlpoints, \\
\ind{24}float t, point2f *cpoint ); \\
void mbs\_BCHornerC3Rf ( int degree, \\
\ind{24}const point4f *ctlpoints, \\
\ind{24}float t, point3f *cpoint );}
Powy"rsze procedury obliczaj"a punkt p"laskiej lub tr"ojwymiarowej wymiernej
krzywej B\'{e}ziera, za pomoc"a schematu Hornera (zastosowanego do
reprezentacji jednorodnej).

\vspace{\bigskipamount}
\cprog{%
void mbs\_FindBezPatchDiagFormf ( int degreeu, int degreev, \\
\ind{33}int spdimen, const float *cpoints, \\
\ind{33}int k, int l, float u, float v, \\
\ind{33}float *dfcp );}
Procedura \texttt{mbs\_FindBezPatchDiagFormf} wyznacza form"e diagonaln"a
stopnia $(k,l)$ p"lata B\'{e}ziera $\bm{p}$ stopnia $(n,m)$ w~punkcie $(u,v)$;
jest to prostok"atny p"lat B\'{e}ziera stopnia $(k,l)$, kt"ory mo"rna otrzyma"c
wykonuj"ac $n-k$ krok"ow algorytmu de~Casteljau na wierszach i~$m-l$ krok"ow
na kolumnach; p"lat ten umo"rliwia obliczenie punktu p"lata $\bm{p}$ i~jego
pochodnych rz"edu $1,\ldots,k$ ze wzgl"edu na $u$ i~$1,\ldots,l$ ze wzgl"edu
na $v$. Zamiast algorytmu de~Casteljau procedura u"rywa szybszego
schematu Hornera (procedury \texttt{mbs\_multiBCHornerf}).

Wszystkie procedury obliczania punktu i~pochodnych (a~tak"re krzywizn)
p"lata B\'{e}ziera opisane dalej w~tym punkcie powinny by"c zrealizowane
przy u"ryciu tej procedury, ale na razie tylko procedura
\texttt{mbs\_BCHornerDer3Pf} jest taka (i~trzeba j"a jeszcze dopracowa"c).


\vspace{\bigskipamount}
\cprog{%
void mbs\_BCHornerPf ( int degreeu, int degreev, int spdimen, \\
\ind{22}const float *ctlpoints, \\
\ind{22}float u, float v, float *ppoint );}
\begin{sloppypar}
Procedura \texttt{mbs\_BCHornerPf} oblicza punkt $\bm{p}(u,v)$ p"lata
B\'{e}ziera $\bm{p}$ stopnia $(n,m)$, po"lo"ronego w~przestrzeni
o~wymiarze~$d$. Parametry \texttt{degreeu} i~\texttt{degreev} okre"slaj"a
stopie"n p"lata (ich warto"sciami s"a liczby $n$ i $m$). Parametr
\texttt{spdimen} okre"sla wymiar $d$ przestrzeni, w~kt"orej le"ry p"lat.
W~tablicy \texttt{ctlpoints} nale"ry poda"c punkty kontrolne ($(n+1)(m+1)d$
liczb).%
\end{sloppypar}

Procedura umieszcza obliczony wynik (wsp"o"lrz"edne punktu p"lata) w~tablicy
\texttt{ppoint}, kt"ora musi mie"c d"lugo"s"c co najmniej \texttt{spdimen}.

\vspace{\bigskipamount}
\cprog{%
\#define mbs\_BCHornerP1f(degreeu,degreev,coeff,u,v,ppoint) \bsl \\
\ind{2}mbs\_BCHornerPf ( degreeu, degreev, 1, coeff, u, v, ppoint ) \\
\#define mbs\_BCHornerP2f(degreeu,degreev,ctlpoints,u,v,ppoint ) \bsl \\
\ind{2}mbs\_BCHornerPf ( degreeu, degreev, 2, (float*)ctlpoints, \bsl \\
\ind{4}u, v, (float*)ppoint ) \\
\#define mbs\_BCHornerP3f(degreeu,degreev,ctlpoints,u,v,ppoint ) \ldots \\
\#define mbs\_BCHornerP4f(degreeu,degreev,ctlpoints,u,v,ppoint ) \ldots}
\begin{sloppypar}
Powy"rsze makra s"lu"r"a do obliczania punktu p"lata B\'{e}ziera
w~przestrzeni o~wymiarze $1$, $2$, $3$ i~$4$, za pomoc"a procedury
\texttt{mbs\_BCHornerPf}.
\end{sloppypar}

\vspace{\bigskipamount}
\cprog{%
void mbs\_BCHornerP3Rf ( int degreeu, int degreev, \\
\ind{24}const point4f *ctlpoints, float u, float v, \\
\ind{24}point3f *p );}
Procedura \texttt{mbs\_BCHornerP3Rf} oblicza punkt wymiernego p"lata
B\'{e}ziera w przes\-trze\-ni tr"ojwymiarowej, reprezentowanego w~postaci
jednorodnej.

\vspace{\bigskipamount}
\cprog{%
void mbs\_multiBCHornerDerf ( int degree, int ncurves, \\
\ind{29}int spdimen, int pitch, \\
\ind{29}const float *ctlpoints, \\
\ind{29}float t, float *p, float *d );}
\begin{sloppypar}
Procedura \texttt{mbs\_multiBCHornerDerf} oblicza za pomoc"a schematu
Hornera punk\-ty $\bm{c}_i(t)$ i~wektory pochodnej $\bm{c}_i'(t)$ krzywych
B\'{e}ziera $\bm{c}_i$ po"lo"ronych w~przestrzeni o~wymiarze~$d$.
\end{sloppypar}

Parametry: \texttt{degree} --- okre"sla stopie"n krzywej, \texttt{ncurves}
--- liczb"e krzywych, \texttt{spdimen} --- wymiar $d$ przestrzeni,
\texttt{pitch} --- podzia"lk"e tablicy \texttt{ctlpoints} zawieraj"acej
punkty kontrolne. Parametr \texttt{t} ma warto"s"c~$t$.

Wsp"o"lrz"edne punkt"ow $\bm{c}_i(t)$ i~wektor"ow $\bm{c}_i'(t)$ procedura
wpisuje odpowiednio do tablic \texttt{p} i~\texttt{d}, kt"ore musz"a mie"c
d"lugo"s"c co najmniej \texttt{ncurves*spdimen}.

\vspace{\bigskipamount}
\cprog{%
\#define mbs\_BCHornerDerC1f(degree,coeff,t,p,d) \bsl \\
\ind{2}mbs\_multiBCHornerDerf ( degree, 1, 1, 0, coeff, t, p, d ) \\
\#define mbs\_BCHornerDerC2f(degree,ctlpoints,t,p,d) \bsl \\
\ind{2}mbs\_multiBCHornerDerf ( degree, 1, 2, 0, (float*)ctlpoints, t, \bsl \\
\ind{4}(float*)p, (float*)d ) \\
\#define mbs\_BCHornerDerC3f(degree,ctlpoints,t,p,d) \ldots \\
\#define mbs\_BCHornerDerC4f(degree,ctlpoints,t,p,d) \ldots}
Powy"rsze makra s"lu"r"a do obliczenia punktu i~pochodnej w~punkcie~$t$
jednej krzywej B\'{e}ziera w~przestrzeni o~wymiarze $1$, $2$, $3$ lub $4$,
przez wywo"lanie procedury \texttt{mbs\_multiBCHornerDerf}.

\vspace{\bigskipamount}
\cprog{%
void mbs\_BCHornerDerC2Rf ( int degree, const point3f *ctlpoints, \\
\ind{27}float t, point2f *p, vector2f *d ); \\
void mbs\_BCHornerDerC3Rf ( int degree, const point4f *ctlpoints, \\
\ind{27}float t, point3f *p, vector3f *d );}
Procedury \texttt{mbs\_BCHornerDerC2Rf} i~\texttt{mbs\_BCHornerDerC3Rf}
obliczaj"a punkt $\bm{p}(t)$ i~wektor $\bm{p}'(t)$ wymiernej krzywej
B\'{e}ziera $\bm{p}$ odpowiednio w~przestrzeni dwu- i~tr"ojwymiarowej.

Parametry: \texttt{degree} --- stopie"n krzywej, \texttt{ctlpoints} ---
tablica punkt"ow kon\-trol\-nych krzywej jednorodnej, \texttt{t} ---
liczba~$t$.
Procedury przypisuj"a obliczony punkt $\bm{p}(t)$ parametrowi~\texttt{p},
a~wektor~$\bm{p}'(t)$ parametrowi~\texttt{d}.

\vspace{\bigskipamount}
\cprog{%
void mbs\_BCHornerDerPf ( int degreeu, int degreev, int spdimen, \\
\ind{25}const float *ctlpoints, \\
\ind{25}float u, float v, \\
\ind{25}float *p, float *du, float *dv );}
Procedura \texttt{mbs\_BCHornerDerPf} oblicza punkt $\bm{p}(u,v)$ oraz
pochodne cz"astkowe $\frac{\partial}{\partial u}\bm{p}(u,v)$
i~$\frac{\partial}{\partial v}\bm{p}(u,v)$ p"lata B\'{e}ziera~$\bm{p}$
stopnia $(n,m)$, po"lo"ronego w~przestrzeni o~wymiarze~$d$.

Parametry: \texttt{degreeu}, \texttt{degreev} --- okre"slaj"a stopie"n
p"lata (liczby $n$~i~$m$). Parametr \texttt{spdimen} okre"sla wymiar~$d$
przestrzeni, tablica \texttt{ctlpoints} zawiera wsp"o"lrz"edne punkt"ow
kontrolnych.

Wyniki (punkt p"lata i~wektory pochodnych) procedura wpisuje do tablic
\texttt{p}, \texttt{du} i~\texttt{dv}, kt"ore musz"a mie"c d"lugo"s"c co
najmniej \texttt{spdimen}.

\vspace{\bigskipamount}
\cprog{%
\#define mbs\_BCHornerDerP1f(degreeu,degreev,coeff,u,v,p,du,dv) \bsl \\
\ind{2}mbs\_BCHornerDerPf ( degreeu, degreev, 1, coeff, u, v, p, du, dv ) \\
\#define mbs\_BCHornerDerP2f(degreeu,degreev,ctlpoints,u,v,p,du,dv) \bsl \\
\ind{2}mbs\_BCHornerDerPf ( degreeu,degreev,2,(float*)ctlpoints,u,v, \bsl \\
\ind{4}(float*)p, (float*)du, (float*)dv ) \\
\#define mbs\_BCHornerDerP3f(degreeu,degreev,ctlpoints,u,v,p,du,dv) \bsl \\
\ind{2}\ldots \\
\#define mbs\_BCHornerDerP4f(degreeu,degreev,ctlpoints,u,v,p,du,dv) \bsl \\
\ind{2}\ldots }
Powy"rsze makra s"lu"r"a do obliczania punkt"ow i~pochodnych cz"astkowych
p"lat"ow B\'{e}ziera po"lo"ronych w~przestrzeni o~wymiarze $1$, $2$, $3$
lub~$4$.

\vspace{\bigskipamount}
\cprog{%
void mbs\_BCHornerDerP3Rf ( int degreeu, int degreev, \\
\ind{26}const point4f *ctlpoints, \\
\ind{26}float u, float v, \\
\ind{26}point3f *p, vector3f *du, vector3f *dv );}
Procedura \texttt{mbs\_BCHornerDerP3Rf} oblicza punkt $\bm{p}(u,v)$
i~wektory pochodnych cz"astkowych wymiernego p"lata B\'{e}ziera po"lo"ronego
w~przestrzeni tr"ojwymiarowej.

Parametry: \texttt{degreeu}, \texttt{degreev} --- stopie"n p"lata ze
wzgl"edu na parametry $u$~i~$v$, \texttt{ctlpoints} --- tablica punkt"ow
kontrolnych p"lata \emph{jednorodnego}, \texttt{u}, \texttt{v} --- liczby
$u$~i~$v$, \texttt{*p}, \texttt{*du}, \texttt{*dv} ---struktury, do~kt"orych
procedura wpisuje wyniki.

\vspace{\bigskipamount}
\cprog{%
void mbs\_multiBCHornerDer2f ( int degree, int ncurves, \\
\ind{30}int spdimen, int pitch, \\
\ind{30}const float *ctlpoints, float t, \\
\ind{30}float *p, float *d1, float *d2 );}
\begin{sloppypar}
Procedura \texttt{mbs\_multiBCHornerDer2f} oblicza za pomoc"a schematu
Hornera punk\-ty $\bm{c}_i(t)$ i~wektory pochodnych $\bm{c}_i'(t)$
i~$\bm{c}_i''(t)$ krzywych
B\'{e}ziera $\bm{c}_i$ po"lo"ronych w~przestrzeni o~wymiarze~$d$.
\end{sloppypar}

Parametry: \texttt{degree} --- okre"sla stopie"n krzywej, \texttt{ncurves}
--- liczb"e krzywych, \texttt{spdimen} --- wymiar $d$ przestrzeni,
\texttt{pitch} --- podzia"lk"e tablicy \texttt{ctlpoints} zawieraj"acej
punkty kontrolne. Parametr \texttt{t} ma warto"s"c~$t$.

Obliczone wsp"o"lrz"edne punkt"ow $\bm{c}_i(t)$ i~wektor"ow $\bm{c}_i'(t)$
i~$\bm{c}_i''(t)$ procedura
wpisuje odpowiednio do tablic \texttt{p}, \texttt{d1} i~\texttt{d2},
kt"ore musz"a mie"c d"lugo"s"c co najmniej \texttt{ncurves*spdimen}.

\vspace{\bigskipamount}
\cprog{%
\#define mbs\_BCHornerDer2C1f(degree,coeff,t,p,d1,d2) \bsl \\
\ind{2}mbs\_multiBCHornerDer2f ( degree, 1, 1, 0, coeff, t, p, d1, d2 ) \\
\#define mbs\_BCHornerDer2C2f(degree,ctlpoints,t,p,d1,d2) \bsl \\
\ind{2}mbs\_multiBCHornerDer2f ( degree, 1, 2, 0, (float*)ctlpoints, \bsl \\
\ind{4}t, (float*)p, (float*)d1, (float*)d2 ) \\
\#define mbs\_BCHornerDer2C3f(degree,ctlpoints,t,p,d1,d2) \ldots \\
\#define mbs\_BCHornerDer2C4f(degree,ctlpoints,t,p,d1,d2) \ldots}
Powy"rsze makra s"lu"r"a do obliczania punkt"ow i~wektor"ow pochodnych
pierwszego i~drugiego rz"edu jednej krzywej B\'{e}ziera po"lo"ronych
w~przestrzeni o~wymiarze $1$, $2$, $3$ lub $4$, za pomoc"a procedury
\texttt{mbs\_multiBCHornerDer2f}.

\vspace{\bigskipamount}
\cprog{%
void mbs\_BCHornerDer2C2Rf ( int degree, const point3f *ctlpoints, \\
\ind{17}float t, point2f *p, vector2f *d1, vector2f *d2 ); \\
void mbs\_BCHornerDer2C3Rf ( int degree, const point4f *ctlpoints, \\
\ind{17}float t, point3f *p, vector3f *d1, vector3f *d2 );}
Procedury \texttt{mbs\_BCHornerDer2C2Rf} i~\texttt{mbs\_BCHornerDer2C3Rf}
obliczaj"a punkt $\bm{p}(t)$ wymiernej krzywej B\'{e}ziera odpowiednio
w~przestrzeni dwu- i~tr"ojwymiarowej i~jej pochodne pierwszego i~drugiego
rz"edu w~punkcie~$t$.

Parametry: \texttt{degree} --- stopie"n krzywej, \texttt{ctlpoints} ---
tablica punkt"ow kontrolnych wielomianowej krzywej jednorodnej, \texttt{t}
--- liczba~$t$, \texttt{*p}, \texttt{*d1} i~\texttt{*d2} --- struktury,
do~kt"orych procedura przypisze odpowiednio punkt~$\bm{p}(t)$
i~wektory~$\bm{p}'(t)$ i~$\bm{p}''(t)$.

\vspace{\bigskipamount}
\cprog{%
void mbs\_BCHornerDer2Pf ( int degreeu, int degreev, int spdimen, \\
\ind{26}const float *ctlpoints, \\
\ind{26}float u, float v, \\
\ind{26}float *p, float *du, float *dv, \\
\ind{26}float *duu, float *duv, float *dvv );}
Procedura \texttt{mbs\_BCHornerDer2Pf} s"lu"ry do obliczania punktu
$\bm{p}(u,v)$ i~wektor"ow pochodnych cz"astkowych rz"edu $1$ i~$2$ p"lata
B\'{e}ziera $\bm{p}$ po"lo"ronego w~przestrzeni o~wymiarze~$d$.

Parametry: \texttt{degreeu}, \texttt{degreev} --- okre"slaj"a stopie"n
p"lata (liczby $n$~i~$m$). Parametr \texttt{spdimen} okre"sla wymiar~$d$
przestrzeni, tablica \texttt{ctlpoints} zawiera wsp"o"lrz"edne punkt"ow
kontrolnych.

Wyniki (punkt p"lata i~wektory pochodnych) procedura wpisuje do tablic
\texttt{p}, \texttt{du}, \texttt{dv}, \texttt{duu}, \texttt{duv}
i~\texttt{dvv}, kt"ore musz"a mie"c d"lugo"s"c co najmniej \texttt{spdimen}.

\vspace{\bigskipamount}
\cprog{%
\#define mbs\_BCHornerDer2P1f(degreeu,degreev,coeff,u,v, \bsl \\
\ind{4}p,du,dv,duu,duv,dvv) \bsl \\
\ind{2}mbs\_BCHornerDer2Pf ( degreeu, degreev, 1, coeff, u, v, \bsl \\
\ind{4}p, du, dv, duu, duv, dvv ) \\
\#define mbs\_BCHornerDer2P2f(degreeu,degreev,ctlpoints, \bsl \\
\ind{4}u,v,p,du,dv,duu,duv,dvv) \bsl \\
\ind{2}mbs\_BCHornerDer2Pf ( degreeu, degreev, 2, (float*)ctlpoints, \bsl \\
\ind{4}u, v, (float*)p, (float*)du, (float*)dv, \bsl \\
\ind{4}(float*)duu, (float*)duv, (float*)dvv ) \\
\#define mbs\_BCHornerDer2P3f(degreeu,degreev,ctlpoints,u,v, \bsl \\
\ind{4}p,du,dv,duu,duv,dvv) \ldots \\
\#define mbs\_BCHornerDer2P4f(degreeu,degreev,ctlpoints,u,v, \bsl \\
\ind{4}p,du,dv,duu,duv,dvv) \ldots}
Powy"rsze makra s"lu"r"a do obliczania punkt"ow i~wektor"ow pochodnych
rz"edu $1$~i~$2$ p"lat"ow B\'{e}ziera po"lo"ronych w~przestrzeniach
o~wymiarach $1$, $2$, $3$ i~$4$, za pomoc"a procedury
\texttt{mbs\_BCHornerDer2Pf}.

\vspace{\bigskipamount}
\cprog{%
void mbs\_BCHornerDer2P3Rf ( int degreeu, int degreev, \\
\ind{28}const point4f *ctlpoints, \\
\ind{28}float u, float v, \\
\ind{28}point3f *p, vector3f *du, vector3f *dv, \\
\ind{28}vector3f *duu, vector3f *duv, \\
\ind{28}vector3f *dvv );}
Procedura \texttt{mbs\_BCHornerDer2P3Rf} oblicza punkt wymiernego p"lata
B\'{e}ziera $\bm{p}$, po"lo"ronego w~przestrzeni tr"ojwymiarowej,
oraz wektory pochodnych rz"edu $1$~i~$2$.

Parametry: \texttt{degreeu}, \texttt{degreev} --- stopie"n p"lata ze
wzgl"edu na parametry $u$~i~$v$, \texttt{ctlpoints} --- tablica punkt"ow
kontrolnych p"lata \emph{jednorodnego}, \texttt{u}, \texttt{v} --- liczby
$u$~i~$v$, \texttt{*p}, \texttt{*du}, \texttt{*dv}, \texttt{*duu}, \texttt{*duv},
\texttt{*dvv} --- struktury, do kt"orych procedura ma wpisa"c wyniki,
odpowiednio punkt $\bm{p}(u,v)$, $\frac{\partial}{\partial u}\bm{p}(u,v)$,
$\frac{\partial}{\partial v}\bm{p}(u,v)$,
$\frac{\partial^2}{\partial u^2}\bm{p}(u,v)$,
$\frac{\partial^2}{\partial u\partial v}\bm{p}(u,v)$
i~$\frac{\partial^2}{\partial v^2}\bm{p}(u,v)$.%
\begin{figure}[ht]
  \centerline{\epsfig{file=patchder.ps}}
  \caption{P"lat B\'{e}ziera i jego wektory pochodnych cz"astkowych}
  \centerline{pierwszego i~drugiego rz"edu.}
\end{figure}

\vspace{\bigskipamount}
\cprog{%
void mbs\_multiBCHornerDer3f ( int degree, int ncurves, \\
\ind{23}int spdimen, int pitch, \\
\ind{23}const float *ctlpoints, float t, \\
\ind{23}float *p, float *d1, float *d2, float *d3 );}
\begin{sloppypar}
Procedura \texttt{mbs\_multiBCHornerDer3f} oblicza za pomoc"a schematu
Hornera punk\-ty $\bm{c}_i(t)$ i~wektory pochodnych $\bm{c}_i'(t)$,
$\bm{c}_i''(t)$ i~$\bm{c}_i'''(t)$ krzywych
B\'{e}ziera $\bm{c}_i$ po"lo"ronych w~przestrzeni o~wymiarze~$d$.
\end{sloppypar}

Parametry: \texttt{degree} --- okre"sla stopie"n krzywej, \texttt{ncurves}
--- liczb"e krzywych, \texttt{spdimen} --- wymiar $d$ przestrzeni,
\texttt{pitch} --- podzia"lk"e tablicy \texttt{ctlpoints} zawieraj"acej
punkty kontrolne. Parametr \texttt{t} ma warto"s"c~$t$.

Obliczone wsp"o"lrz"edne punkt"ow $\bm{c}_i(t)$ i~wektor"ow $\bm{c}_i'(t)$,
$\bm{c}_i''(t)$ i~$\bm{c}_i'''(t)$ procedura
wpisuje odpowiednio do tablic \texttt{p}, \texttt{d1}, \texttt{d2}
i~\texttt{d3}, kt"ore musz"a mie"c d"lugo"s"c co najmniej
\texttt{ncurves*spdimen}.

\vspace{\bigskipamount}
\cprog{%
\#define mbs\_BCHornerDer3C1f(degree,coeff,t,p,d1,d2,d3) \bsl \\
\ind{2}mbs\_multiBCHornerDer3f ( degree, 1, 1, 0, coeff, t, \bsl \\
\ind{4}p, d1, d2, d3 ) \\
\#define mbs\_BCHornerDer3C2f(degree,ctlpoints,t,p,d1,d2,d3) \bsl \\
\ind{2}mbs\_multiBCHornerDer3f ( degree, 1, 2, 0, (float*)ctlpoints, \bsl \\
\ind{4}t, (float*)p, (float*)d1, (float*)d2, (float*)d3 ) \\
\#define mbs\_BCHornerDer3C3f(degree,ctlpoints,t,p,d1,d2,d3) \ldots \\
\#define mbs\_BCHornerDer3C4f(degree,ctlpoints,t,p,d1,d2,d3) \ldots}
Powy"rsze makra s"lu"r"a do obliczania punkt"ow i~wektor"ow pochodnych
pierwszego, drugiego i~trzeciego rz"edu jednej krzywej B\'{e}ziera
po"lo"ronych w~przestrzeni o~wymiarze $1$, $2$, $3$ lub $4$, za pomoc"a
procedury \texttt{mbs\_multiBCHornerDer3f}.

\vspace{\bigskipamount}
\cprog{%
void mbs\_BCHornerDer3Pf ( int degreeu, int degreev, int spdimen, \\
\ind{26}const float *ctlpoints, \\
\ind{26}float u, float v, \\
\ind{26}float *p, float *pu, float *pv, \\
\ind{26}float *puu, float *puv, float *pvv, \\
\ind{26}float *puuu, float *puuv, float *puvv, \\
\ind{26}float *pvvv );}
Procedura \texttt{mbs\_BCHornerDer3Pf} oblicza punkt $\bm{p}(u,v)$
p"lata B\'{e}ziera $\bm{p}$ po"lo"ronego w~przestrzeni o~wymiarze
\texttt{spdimen} i~jego pochodne rz"edu $1,\ldots,3$ w~tym punkcie. Obecna
wersja procedury dzia"la przy za"lo"reniu, "re stopie"n p"lata ze wzgl"edu na
oba parametry jest nie mniejszy ni"r~$3$ --- zaimplementowanie obs"lugi
pozosta"lych przypadk"ow pozostaje do zrobienia.

\vspace{\bigskipamount}
\cprog{%
\#define mbs\_BCHornerDer3P1f(degreeu,degreev,coeff,u,v, \bsl \\
\ind{4}p,pu,pv,puu,puv,pvv,puuu,puuv,puvv,pvvv) \bsl \\
\ind{2}mbs\_BCHornerDer3Pf ( degreeu, degreev, 1, coeff, u, v, \bsl \\
\ind{4}p, pu, pv, puu, puv, pvv, puuu, puuv, puvv, pvvv ) \\
\#define mbs\_BCHornerDer3P2f(degreeu,degreev,ctlpoints,u,v, \bsl \\
\ind{4}p,pu,pv,puu,puv,pvv,puuu,puuv,puvv,pvvv) \bsl \\
\ind{2}mbs\_BCHornerDer3Pf ( degreeu, degreev, 2, (float*)ctlpoints, \bsl \\
\ind{4}u, v, (float*)p, (float*)pu, (float*)pv, (float*)puu, \bsl \\
\ind{4}(float*)puv, (float*)pvv, (float*)puuu, (float*)puuv, \bsl \\
\ind{4}(float*)puvv, (float*)pvvv ) \\
\#define mbs\_BCHornerDer3P3f(degreeu,degreev,ctlpoints,u,v, \bsl \\
\ind{4}p,pu,pv,puu,puv,pvv,puuu,puuv,puvv,pvvv) \ldots \\
\#define mbs\_BCHornerDer3P4f(degreeu,degreev,ctlpoints,u,v, \bsl \\
\ind{4}p,pu,pv,puu,puv,pvv,puuu,puuv,puvv,pvvv) \ldots}
Podane wy"rej makra s"lu"r"a do wywo"lywania procedury
\texttt{mbs\_BCHormerDer3Pf} w~celu obliczenia punktu i~pochodnych p"lat"ow
po"lo"ronych w~przestrzeniach o~wymiarach odpowiednio $1,\ldots,4$.


\newpage
\subsection{Obliczanie krzywizn i uk"ladu Freneta krzywej}

Obliczanie krzywizn i wektor"ow uk"ladu Freneta jest oprogramowane tylko dla
krzywych B\'{e}ziera. Chc"ac obliczy"c krzywizn"e krzywej B-sklejanej trzeba
dokona"c maksymalnego wstawienia w"ez"l"ow (np.\ za pomoc"a procedury
\texttt{mbs\_multiMaxKnotInsf}) w~celu otrzymania reprezentacji B\'{e}ziera
poszczeg"olnych "luk"ow. Zwykle krzy\-wiz\-na b"edzie tablicowana, a~zatem
wykonamy jedn"a tak"a konwersj"e, a~potem wiele oblicze"n krzywizny
w~r"o"rnych punktach. Dlatego nie ma procedur obliczaj"acych bezpo"srednio
krzywizny krzywych B-sklejanych. Procedury opisane poni"rej wykonuj"a
obliczenia za pomoc"a procedury \texttt{mbs\_multiBCHornerf}.

\vspace{\bigskipamount}
\cprog{%
void mbs\_BCFrenetC2f ( int degree, const point2f *ctlpoints, \\
\ind{23}float t, point2f *cpoint, \\
\ind{23}vector2f *fframe, float *curvature );}
Procedura \texttt{mbs\_BCFrenetC2f} oblicza krzywizn"e i~wektory
styczny~$\bm{t}$ i~normalny~$\bm{n}$ uk"ladu Freneta p"laskiej krzywej
B\'{e}ziera stopnia \texttt{degree}, kt"orej punkty kontrolne s"a podane
w~tablicy \texttt{ctlpoints}. Parametr krzywej jest r"owny \texttt{t}.
Tablica \texttt{fframe} musi pomie"sci"c dwa wektory. Ponadto procedura
oblicza punkt krzywej i~przypisuje go do parametru \texttt{*cpoint}.

\vspace{\bigskipamount}
\cprog{%
void mbs\_BCFrenetC2Rf ( int degree, const point3f *ctlpoints, \\
\ind{24}float t, point2f *cpoint, \\
\ind{24}vector2f *fframe, float *curvature );}
Procedura \texttt{mbs\_BCFrenetC2f} oblicza krzywizn"e i~wektory
styczny~$\bm{t}$ i~normalny~$\bm{n}$ uk"ladu Freneta p"laskiej wymiernej
krzywej B\'{e}ziera stopnia \texttt{degree}, kt"orej punkty kontrolne
(w~reprezentacji jednorodnej) s"a podane w~tablicy \texttt{ctlpoints}.
Parametr krzywej jest r"owny \texttt{t}.
Tablica \texttt{fframe} musi pomie"sci"c dwa wektory. Ponadto procedura
oblicza punkt krzywej i~przypisuje go do parametru \texttt{*cpoint}.

\vspace{\bigskipamount}
\cprog{%
void mbs\_BCFrenetC3f ( int degree, const point3f *ctlpoints, \\
\ind{23}float t, point3f *cpoint, \\
\ind{23}vector3f *fframe, float *curvatures );}
Procedura \texttt{mbs\_BCFrenetC3f} oblicza krzywizn"e i skr"ecenie
wielomianowej krzywej B\'{e}ziera stopnia \texttt{degree} oraz wektory
uk"ladu Freneta: styczny $\bm{t}$, normalny $\bm{n}$ i binormalny $\bm{b}$
w~punkcie odpowiadaj"acym danemu parametrowi \texttt{t}. Tablica
\texttt{ctlpoints} zawiera punkty kontrolne krzywej. Krzywizna i
skr"ecenie s"a wpisywane do tablicy \texttt{curvatures}, a wektory do
tablicy \texttt{fframe}. Ponadto procedura oblicza punkt krzywej
i~przypisuje do parametru \texttt{*cpoint}.

\vspace{\bigskipamount}
\cprog{%
void mbs\_BCFrenetC3Rf ( int degree, const point4f *ctlpoints, \\
\ind{24}float t, point3f *cpoint, \\
\ind{24}vector3f *fframe, float *curvatures );}
Procedura \texttt{mbs\_BCFrenetC3f} oblicza krzywizn"e i skr"ecenie
wymiernej krzywej B\'{e}ziera stopnia \texttt{degree} oraz wektory
uk"ladu Freneta: styczny $\bm{t}$, normalny $\bm{n}$ i binormalny $\bm{b}$
w~punkcie odpowiadaj"acym danemu parametrowi \texttt{t}. Tablica
\texttt{ctlpoints} zawiera punkty kontrolne krzywej w reprezentacji
jednorodnej. Krzywizna i~skr"ecenie s"a wpisywane do tablicy
\texttt{curvatures}, a wektory do
tablicy \texttt{fframe}. Ponadto procedura
oblicza punkt krzywej i przypisuje go do parametru \texttt{*cpoint}.


\subsection{Obliczanie wektora normalnego p"lata}

\cprog{%
void mbs\_BCHornerNvP3f ( int degreeu, int degreev, \\
\ind{25}const point3f *ctlpoints, \\
\ind{25}float u, float v, \\
\ind{25}point3f *p, vector3f *nv );}

\hspace*{\parindent}
Procedura \texttt{mbs\_BCHornerNvP3f} s"lu"ry do obliczenia punktu
p"lata B\'{e}ziera po"lo"ronego w~przestrzeni tr"ojwymiarowej i~jego wektora
normalnego w~tym
punkcie. Obliczony wektor jest iloczynem pochodnych cz"astkowych i~mo"re
by"c wektorem zerowym je"sli w~danym punkcie jest osobliwo"s"c, nawet je"sli
w~tym punkcie i~jego otoczeniu p"laszczyzna styczna do p"lata jest
jednoznacznie okre"slona.

\vspace{\bigskipamount}
\cprog{%
void mbs\_BCHornerNvP3Rf ( int degreeu, int degreev, \\
\ind{26}const point4f *ctlpoints, \\
\ind{26}float u, float v, \\
\ind{26}point3f *p, vector3f *nv );}
Procedura \texttt{mbs\_BCHornerNvP3Rf} s"lu"ry do obliczania punktu
wymiernego p"lata B\'{e}ziera po"lo"ronego w~przestrzeni tr"ojwymiarowej
i~jego wektora normalnego w tym punkcie. Wsp"o"lrz"edne obliczonego wektora
normalnego to pierwsze trzy wsp"o"lrz"edne iloczynu wektorowego
$\bm{P}\wedge\bm{P}_u\wedge\bm{P}_v$ (iloczynu punktu p"lata jednorodnego
i~jego pochodnych cz"astkowych). Wektor ten mo"re by"c wektorem zerowym,
je"sli p"lat ma osobliwo"s"c, nawet je"sli p"laszczyzna styczna w~danym
punkcie jest okre"slona jednoznacznie.


\subsection{Obliczanie form podstawowych i krzywizn p"lat"ow}

Obliczanie form podstawowych i~krzywizn jest oprogramowane tylko dla
p"lat"ow B\'{e}ziera; powody, dla kt"orych nie ma procedur bezpo"srednio
obliczaj"acych te obiekty dla p"lat"ow B-sklejanych s"a takie same jak
powody, dla kt"orych w~bibliotece s"a tylko procedury obliczania krzywizn
i~uk"ladu Freneta krzywych B\'{e}ziera.

\vspace{\bigskipamount}
\cprog{%
void mbs\_FundFormsBP3f ( int degreeu, int degreev, \\
\ind{25}const point3f *ctlpoints, \\
\ind{25}float u, float v, \\
\ind{25}float *firstform, float *secondform );}
Procedura \texttt{mbs\_FundFormsBP3f} oblicza wsp"o"lczynniki macierzy
pierwszej i~drugiej formy podstawowej wielomianowego p"lata B\'{e}ziera
w~przestrzeni tr"ojwymiarowej.

Parametry: \texttt{degreeu}, \texttt{degreev} --- stopie"n p"lata ze
wzgl"edu na parametry $u$ i~$v$, \texttt{ctlpoints} --- tablica punkt"ow
kontrolnych (spakowana, tj.\ bez obszar"ow nieu"rywanych mi"edzy danymi
opisuj"acymi kolejne kolumny siatki kontrolnej). Parametry \texttt{u}
i~\texttt{v} okre"slaj"a punkt, w~kt"orym procedura ma obliczy"c formy.

Parametry \texttt{firstform} i~\texttt{secondform} s"a wska"znikami tablic
o~d"lugo"sci co najmniej~$3$. Procedura wpisuje do tych tablic
wsp"o"lczynniki macierzy form, odpowiednio
$g_{11}=\scp{\bm{p}_u}{\bm{p}_u}$, $g_{12}=g_{21}=\scp{\bm{p}_u}{\bm{p}_v}$
i~$g_{22}=\scp{\bm{p}_v}{\bm{p}_v}$, oraz
$b_{11}=\scp{\bm{n}}{\bm{p}_{uu}}$, $b_{12}=b_{21}=\scp{\bm{n}}{\bm{p}_{uv}}$
i~$b_{22}=\scp{\bm{n}}{\bm{p}_{vv}}$ (gdzie $\bm{n}$ oznacza jednostkowy
wektor normalny p"lata w~punkcie $(u,v)$).


\vspace{\bigskipamount}
\cprog{%
void mbs\_GMCurvaturesBP3f ( int degreeu, int degreev, \\
\ind{28}const point3f *ctlpoints, \\
\ind{28}float u, float v, \\
\ind{28}float *gaussian, float *mean );}
Procedura \texttt{mbs\_GMCurvaturesBP3f} oblicza krzywizn"e gaussowsk"a
i~"sredni"a wielomianowego p"lata B\'{e}ziera w~$\R^3$. Parametry
\texttt{degreeu}, \texttt{degreev}, \texttt{ctlpoints}, \texttt{u}
i~\texttt{v} s"a identyczne jak odpowiednie parametry poprzedniej procedury.

Parametry \texttt{*gaussian} i~\texttt{*mean} s"lu"r"a do wyprowadzenia
wynik"ow; procedura przypisuje im odpowiednio obliczon"a warto"s"c krzywizny
gaussowskiej i~"sredniej.


\vspace{\bigskipamount}
\cprog{%
void mbs\_PrincipalDirectionsBP3f ( int degreeu, int degreev, \\
\ind{35}const point3f *ctlpoints, \\
\ind{35}float u, float v, \\
\ind{35}float *k1, vector2f *v1, \\
\ind{35}float *k2, vector2f *v2 );}
Procedura \texttt{mbs\_PrincipalDirectionsBP3f} oblicza krzywizny i~kierunki
g"l"owne wielomianowego p"lata B\'{e}ziera w~przestrzeni tr"ojwymiarowej. 
Parametry \texttt{degreeu}, \texttt{degreev}, \texttt{ctlpoints}, \texttt{u}
i~\texttt{v} s"a identyczne jak odpowiednie parametry poprzednich dw"och
procedur.

Parametrom \texttt{*k1} i~\texttt{*k2} procedura przypisuje warto"sci
krzywizn g"l"ownych p"lata, a~odpowiednie kierunki g"l"owne (w~przestrzeni
stycznej do dziedziny p"lata) s"a przypisywane parametrom \texttt{*v1}
i~\texttt{*v2}.


\begin{figure}[ht]
  \centerline{\epsfig{file=patchpdir.ps}}
  \caption{Wektory odpowiadaj"ace kierunkom g"l"ownym w pewnym punkcie
    p"lata B\'{e}ziera}
\end{figure}
\vspace{\bigskipamount}
\ucprog{%
void mbs\_FundFormsBP3Rf ( int degreeu, int degreev, \\
\ind{26}const point4f *ctlpoints, \\
\ind{26}float u, float v, \\
\ind{26}float *firstform, float *secondform );}

\dcprog{%
void mbs\_GMCurvaturesBP3Rf ( int degreeu, int degreev, \\
\ind{29}const point4f *ctlpoints, \\
\ind{29}float u, float v, \\
\ind{29}float *gaussian, float *mean ); \\
void mbs\_PrincipalDirectionsBP3Rf ( int degreeu, int degreev, \\
\ind{36}const point4f *ctlpoints, \\
\ind{36}float u, float v, \\
\ind{36}float *k1, vector2f *v1, \\
\ind{36}float *k2, vector2f *v2 );}
\begin{sloppypar}
Powy"rsze procedury obliczaj"a odpowiednio wsp"o"lczynniki
macierzy pierwszej i~drugiej formy podstawowej, krzywizny gaussowsk"a
i~"sredni"a oraz krzywizny i~kierunki g"l"owne wymiernego p"lata B\'{e}ziera
$\bm{p}$. Procedury te s"a dok"ladnymi odpowiednikami procedur
\texttt{mbs\_FundFormsBP3f}, \texttt{mbs\_GMCurvaturesBP3f}
i~\texttt{mbs\_GMCurvaturesBP3f} i~maj"a takie same parametry, z~wyj"atkiem
\texttt{ctlpoints}, kt"ory jest tablic"a punkt"ow kontrolnych p"lata
\emph{jednorodnego} w~przestrzeni~$\R^4$.
\end{sloppypar}

\newpage
\section{Tablicowanie krzywych}

Opisane ni"rej procedury obliczaj"a ci"ag punkt"ow krzywej B\'{e}ziera
lub B-sklejanej oraz pochodne rz"edu $1$ i~$2$ albo~$1$, $2$ i~$3$
dla ci"agu warto"sci parametru $t_0,\ldots,t_{k-1}$, wywo"luj"ac
w~p"etli odpowiednie procedury opisane wcze"sniej. G"l"ownym ich
zastosowaniem jest wykorzystanie do tablicowania p"lat"ow Coonsa
(przez procedury opisane w~p.~\ref{sect:Coons:patch:procedures}).

\vspace{\bigskipamount}
\cprog{%
void mbs\_TabBezCurveDer2f ( int spdimen, int degree, \\
\ind{28}const float *cp, \\
\ind{28}int nkn, const float *kn, \\
\ind{28}int ppitch, \\
\ind{28}float *p, float *dp, float *ddp );}
Procedura \texttt{mbs\_TabBezCurveDer2f} tablicuje krzyw"a B\'{e}ziera
i~jej pochodne rz"e\-du $1$ i~$2$ przy u"ryciu procedury
\texttt{mbs\_multiBCHornerDer2f}.

Parametry: \texttt{spdimen} --- wymiar
przestrzeni, \texttt{degree} --- stopie"n krzywej, \texttt{cp} --- tablica
punkt"ow kontrolnych, \texttt{nkn} --- liczba~$k$, \texttt{kn} --- tablica
zawieraj"aca $k$~liczb (warto"sci parametru), \texttt{ppitch} --- podzia"lka
tablic~\texttt{p}, \texttt{dp} i~\texttt{ddp}, do kt"orych maj"a by"c wpisane
wsp"o"lrz"edne punkt"ow krzywej, wektor"ow pochodnej i~wektor"ow pochodnej
drugiego rz"edu. Pierwsze wsp"o"lrz"edne kolejnych punkt"ow i~wektor"ow
s"a wpisywane do tablic na miejsca odleg"le od siebie o~warto"s"c parametru
\texttt{ppitch}.

\vspace{\bigskipamount}
\cprog{%
void mbs\_TabBezCurveDer3f ( int spdimen, int degree, \\
\ind{20}const float *cp, \\
\ind{20}int nkn, const float *kn, \\
\ind{20}int ppitch, \\
\ind{20}float *p, float *dp, float *ddp, float *dddp );}
Procedura \texttt{mbs\_TabBezCurveDer3f} tablicuje krzyw"a B\'{e}ziera
i~jej pochodne rz"e\-du $1$, $2$ i~$3$ przy u"ryciu procedury
\texttt{mbs\_multiBCHornerDer3f}.

Parametry: \texttt{spdimen} --- wymiar
przestrzeni, \texttt{degree} --- stopie"n krzywej, \texttt{cp} --- tablica
punkt"ow kontrolnych, \texttt{nkn} --- liczba~$k$, \texttt{kn} --- tablica
zawieraj"aca $k$~liczb (warto"sci parametru), \texttt{ppitch} --- podzia"lka
tablic~\texttt{p}, \texttt{dp}, ]texttt{ddp} i~\texttt{dddp}, do kt"orych
maj"a by"c wpisane odpowiednio wsp"o"lrz"edne punkt"ow krzywej, i~wektor"ow
pochodnej rz"edu~$1$, $2$ i~$3$. Pierwsze wsp"o"lrz"edne kolejnych punkt"ow
i~wektor"ow s"a wpisywane do tablic na miejsca odleg"le od siebie
o~warto"s"c parametru
\texttt{ppitch}.

%\newpage
\vspace{\bigskipamount}
\cprog{%
void mbs\_TabBSCurveDer2f ( int spdimen, int degree, int lastknot, \\
\ind{27}const float *knots, const float *cp, \\
\ind{27}int nkn, const float *kn, int ppitch, \\
\ind{27}float *p, float *dp, float *ddp );}
Procedura \texttt{mbs\_TabBSCurveDer2f} tablicuje krzyw"a B-sklejan"a
i~jej pochodne rz"e\-du $1$, i~$2$ przy u"ryciu procedury
\texttt{mbs\_multideBoorDer2f}.

\begin{sloppypar}
Parametry: \texttt{spdimen} --- wymiar
przestrzeni, \texttt{degree} --- stopie"n krzywej,
\texttt{lastknot} --- numer ostatniego w"ez"la,
\texttt{knots} --- tablica w"ez"l"ow, \texttt{cp} --- tablica
punkt"ow kontrolnych, \texttt{nkn} --- liczba~$k$, \texttt{kn} --- tablica
zawieraj"aca $k$~liczb (warto"sci parametru), \texttt{ppitch} --- podzia"lka
tablic~\texttt{p}, \texttt{dp} i~\texttt{ddp}, do kt"orych maj"a by"c wpisane
odpowiednio wsp"o"lrz"edne punkt"ow krzywej, i~wektor"ow pochodnej
rz"edu~$1$, i~$2$. Pierwsze wsp"o"lrz"edne kolejnych punkt"ow i~wektor"ow
s"a wpisywane do tablic na miejsca odleg"le od siebie o~warto"s"c parametru
\texttt{ppitch}.%
\end{sloppypar}

\vspace{\bigskipamount}
\cprog{%
void mbs\_TabBSCurveDer3f ( int spdimen, int degree, int lastknot, \\
\ind{20}const float *knots, const float *cp, \\
\ind{20}int nkn, const float *kn, int ppitch, \\
\ind{20}float *p, float *dp, float *ddp, float *dddp );}
Procedura \texttt{mbs\_TabBSCurveDer3f} tablicuje krzyw"a B-sklejan"a
i~jej pochodne rz"e\-du $1$, $2$ i~$3$ przy u"ryciu procedury
\texttt{mbs\_multideBoorDer3f}.

\begin{sloppypar}
Parametry: \texttt{spdimen} --- wymiar
przestrzeni, \texttt{degree} --- stopie"n krzywej,
\texttt{lastknot} --- numer ostatniego w"ez"la,
\texttt{knots} --- tablica w"ez"l"ow, \texttt{cp} --- tablica
punkt"ow kontrolnych, \texttt{nkn} --- liczba~$k$, \texttt{kn} --- tablica
zawieraj"aca $k$~liczb (warto"sci parametru), \texttt{ppitch} --- podzia"lka
tablic~\texttt{p}, \texttt{dp}, \texttt{ddp} i~\texttt{dddp}, do kt"orych
maj"a by"c wpisane odpowiednio wsp"o"lrz"edne punkt"ow krzywej, i~wektor"ow
pochodnej rz"edu~$1$, $2$ i~$3$. Pierwsze wsp"o"lrz"edne kolejnych punkt"ow
i~wektor"ow s"a wpisywane do tablic na miejsca odleg"le od siebie o~warto"s"c
parametru \texttt{ppitch}.%
\end{sloppypar}


\newpage
\section{Znajdowanie reprezentacji pochodnych}

Czym innym jest obliczanie wektora pochodnej krzywej w~ustalonym punkcie,
a~czym innym znalezienie krzywej reprezentuj"acej pochodn"a krzywej danej.
Opisane ni"rej procedury obliczaj"a takie reprezentacje na podstawie
wzor"ow
\begin{align}
  \frac{\mathrm{d}}{\mathrm{d}t}\sum_{i=0}^n\bm{p}_iB^n_i(t) =
  \sum_{i=0}^{n-1}n(\bm{p}_{i+1}-\bm{p}_i)B^{n-1}_{i+1}(t),
\end{align}
dla krzywych B\'{e}ziera, oraz
\begin{align}
  \frac{\mathrm{d}}{\mathrm{d}t}\sum_{i=0}^{N-n-1}\bm{d}_iN^n_i(t) =
  \sum_{i=0}^{N-n-2}\frac{n}{u_{i+n+1}-u_{i+1}}(\bm{d}_{i+1}-\bm{d}_i)
  N^{n-1}_{i+1}(t),
\end{align}
dla krzywych B-sklejanych. Funkcje B-sklejane $N^n_i$ oraz $N^{n-1}_i$ s"a
okre"slone dla tego samego ci"agu w"ez"l"ow.

\vspace{\bigskipamount}
\cprog{%
void mbs\_multiFindBezDerivativef ( int degree, \\
\ind{33}int ncurves, int spdimen, \\
\ind{33}int pitch, const float *ctlpoints, \\
\ind{33}int dpitch, float *dctlpoints );}
Procedura \texttt{mbs\_multiFindBezDerivativef} oblicza punkty kontrolne
krzywych B\'{e}ziera stopnia $n-1$ opisuj"acych pochodne danych krzywych
B\'{e}ziera stopnia~$n$.

\begin{sloppypar}\hyphenpenalty=200
Parametry wej"sciowe: \texttt{degree} --- stopie"n $n$ krzywych danych (musi
by"c dodatni),
\texttt{ncurves} --- liczba krzywych danych, \texttt{spdimen} --- wymiar
przestrzeni, w~kt"orej le"r"a krzywe, \texttt{pitch} --- podzia"lka
tablicy \texttt{ctlpoints} okre"slaj"aca odleg"lo"s"c pocz"atk"ow
reprezentacji krzywych, \texttt{ctlpoints} --- tablica zawieraj"aca
wsp"o"lrz"edne punkt"ow kontrolnych krzywych danych.
\end{sloppypar}

Parametr \texttt{dpitch} opisuje podzia"lk"e tablicy \texttt{dctlpoints}, do
kt"orej procedura wpisuje obliczone punkty kontrolne krzywych opisj"acych
pochodne.

\vspace{\bigskipamount}
\cprog{%
\#define mbs\_FindBezDerivativeC1f(degree,coeff,dcoeff) \bsl \\
\ind{2}mbs\_multiFindBezDerivativef ( degree, 1, 1, 0, coeff, 0, dcoeff ) \\
\#define mbs\_FindBezDerivativeC2f(degree,ctlpoints,dctlpoints) \bsl \\
\ind{2}mbs\_multiFindBezDerivativef ( degree, 1, 2, 0, \bsl \\
\ind{4}(float*)ctlpoints, 0, (float*)dctlpoints ) \\
\#define mbs\_FindBezDerivativeC3f(degree,ctlpoints,dctlpoints) \bsl \\
\ind{2}mbs\_multiFindBezDerivativef ( degree, 1, 3, 0, \bsl \\
\ind{4}(float*)ctlpoints, 0, (float*)dctlpoints ) \\
\#define mbs\_FindBezDerivativeC4f(degree,ctlpoints,dctlpoints) \bsl \\
\ind{2}mbs\_multiFindBezDerivativef ( degree, 1, 4, 0, \bsl \\
\ind{4}(float*)ctlpoints, 0, (float*)dctlpoints )}
Powy"rsze makra wywo"luj"a procedur"e \texttt{mbs\_multiFindBezDerivativef}
w~celu obliczenia punkt"ow kontrolnych jednej krzywej B\'{e}ziera stopnia~$n$
po"lo"ronej w~przestrzeni o~wymiarze $1$, $2$, $3$, $4$.

\vspace{\bigskipamount}
\cprog{%
void mbs\_multiFindBSDerivativef ( int degree, int lastknot, \\
\ind{33}const float *knots, \\
\ind{33}int ncurves, int spdimen, \\
\ind{33}int pitch, const float *ctlpoints, \\
\ind{33}int *lastdknot, float *dknots, \\
\ind{33}int dpitch, float *dctlpoints );}
Procedura \texttt{mbs\_multiFindBSDerivativef} oblicza punkty kontrolne
krzywych B-sklejanych stopnia $n-1$ opisuj"acych pochodne danych krzywych
B-sklejanych stopnia~$n$.

Parametry wej"sciowe: \texttt{degree} --- stopie"n $n$ krzywych danych,
\texttt{lastknot} --- indeks $N$ ostatniego w"ez"la, \texttt{knots} ---
tablica w"ez"l"ow $u_0,\ldots,u_N$, \texttt{ncurves} --- liczba krzywych
danych, \texttt{spdimen} --- wymiar przestrzeni, w~kt"orej le"r"a krzywe,
\texttt{pitch} --- podzia"lka tablicy \texttt{ctlpoints} okre"slaj"aca
odleg"lo"s"c pocz"atk"ow reprezentacji krzywych, \texttt{ctlpoints} ---
tablica zawieraj"aca wsp"o"lrz"edne punkt"ow kontrolnych krzywych danych.

Parametr wyj"sciowy \texttt{*lastdknot} otrzymuje warto"s"c $N-2$, za"s do
tablicy \texttt{dknots} procedura przepisuje w"ez"ly $u_1,\ldots,u_{N-1}$.
Parametry \texttt{lastdknot} i~\texttt{dknots} mog"a mie"c warto"s"c
\texttt{NULL} i~wtedy s"a ignorowane.

Parametr \texttt{dpitch} opisuje podzia"lk"e tablicy \texttt{dctlpoints}, do
kt"orej procedura wpisuje obliczone punkty kontrolne krzywych opisj"acych
pochodne.

\vspace{\bigskipamount}
\cprog{%
\#define mbs\_FindBSDerivativeC1f(degree,lastknot,knots,coeff, \bsl \\
\ind{4}lastdknot,dknots,dcoeff) \bsl \\
\ind{2}mbs\_multiFindBSDerivativef ( degree, lastknot, knots, 1, 1, 0, \bsl \\
\ind{4}coeff, lastdknot, dknots, 0, dcoeff ) \\
\#define mbs\_FindBSDerivativeC2f(degree,lastknot,knots,ctlpoints, \bsl \\
\ind{4}lastdknot,dknots,dctlpoints) \bsl \\
\ind{2}mbs\_multiFindBSDerivativef ( degree, lastknot, knots, 1, 2, 0, \bsl \\
\ind{4}(float*)ctlpoints, lastdknot, dknots, 0, (float*)dctlpoints ) \\
\#define mbs\_FindBSDerivativeC3f(degree,lastknot,knots,ctlpoints, \bsl \\
\ind{4}lastdknot,dknots,dctlpoints)\ ... \\
\#define mbs\_FindBSDerivativeC4f(degree,lastknot,knots,ctlpoints, \bsl \\
\ind{4}lastdknot,dknots,dctlpoints)\ ...}
\begin{sloppypar}
Powy"rsze makra wywo"luj"a procedur"e
\texttt{mbs\_multiFindBSDerivativef} w~celu znalezienia reprezentacji
pochodnej jednej krzywej B-sklejanej po"lo"ronej w~przes\-trze\-ni o~wymiarze
$1$, $2$, $3$ lub $4$.
\end{sloppypar}

