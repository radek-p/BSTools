
%/* //////////////////////////////////////////////////// */
%/* This file is a part of the BSTools procedure package */
%/* written by Przemyslaw Kiciak.                        */
%/* //////////////////////////////////////////////////// */

\newpage
\section{Rysowanie p"lat"ow obci"etych}

\subsection{\label{ssect:trimpatch:bound}Reprezentacja dziedziny}

Dziedzina obci"etego p"lata B-sklejanego stopnia $(n,m)$, o~w"ez"lach
$u_0,\ldots,u_N$ oraz $v_0,\ldots,v_M$ jest podzbiorem prostok"ata
$[u_n,u_{N-n}]\times[v_m,v_{M-m}]$. W~szczeg"olno"sci jest to zawsze zbi"or
ograniczony.
Brzeg dziedziny powierzchni obci"etej jest sum"a p"laskich krzywoliniowych
"lamanych zamkni"etych. Ka"rda taka "lamana sk"lada si"e z
\begin{itemize}
  \item "lamanych (ci"ag"ow odcink"ow),
  \item krzywych B\'{e}ziera,
  \item krzywych B-sklejanych,
\end{itemize}
zwanych dalej elementami brzegu,
przy czym punkty (wierzcho"lki "lamanej i~punkty kontrolne) mog"a by"c dane
za pomoc"a wsp"o"lrz"ednych kartezja"nskich (wtedy s"a typu
\texttt{point2f}) albo jednorodnych (wtedy s"a typu \texttt{vector3f}).

Dane opisuj"ace ka"rd"a tak"a "laman"a musz"a spe"lnia"c nast"epuj"acy
warunek: krzywe B-sklejane musz"a by"c ci"ag"le, a~ponadto
punkt ko"ncowy ka"rdego elementu ("lamanej lub krzywej) jest
punktem pocz"atkowym elementu nast"epnego (przy czym za element nast"epny
elementu ostatniego uwa"ra si"e element pierwszy). Je"sli ten warunek nie
jest spe"lniony, to procedury rysowania p"lat"ow obci"etych wstawi"a
odpowiednie odcinki.

Drugi warunek to brak punkt"ow niew"la"sciwych. Wystarczy, aby wszystkie
wsp"o"lrz"edne wagowe by"ly dodatnie, cho"c nie jest to konieczne. Natomiast
niespe"lnienie tego warunku, czyli podanie brzegu, kt"ory jest
nieograniczony, mo"re spowodowa"c b"l"ad wykonania programu. Co~wi"ecej,
wszystkie "lamane i~krzywe musz"a le"re"c w~prostok"acie, kt"ory jest
dziedzin"a p"lata nieobci"etego.


\vspace{\medskipamount}
Brzeg dziedziny b"edzie reprezentowany za pomoc"a tablicy zawieraj"acej
struktury typu \texttt{polycurvef}.

\vspace{\bigskipamount}
\cprog{%
typedef struct\{ \\
\ind{4}boolean closing; \\
\ind{4}byte\ind{4}spdimen; \\
\ind{4}short\ind{3}degree; \\
\ind{4}int\ind{5}lastknot; \\
\ind{4}float\ind{3}*knots; \\
\ind{4}float\ind{3}*points; \\
\ind{2}\} polycurvef;
}
Pole \texttt{closing} okre"sla, z~czym "l"aczy si"e koniec elementu. Je"sli
warto"sci"a tego pola jest \texttt{false}, to z~nast"epnym elementem (tj.\
z~elementem opisanym przez nast"epny element tablicy). Je"sli \texttt{true},
to z pocz"atkiem elementu pierwszego w~tablicy lub ostatniego
z~element"ow poprzedzaj"acych dany, kt"orego poprzednik ma pole
\texttt{closing} o~warto"sci \texttt{true} (czyli z~pocz"atkiem
ostatniego elementu wyznaczaj"acego pocz"atek nowego zamkni"etego
fragmentu brzegu).

\begin{sloppypar}
Pole \texttt{spdimen} mo"re mie"c warto"s"c \texttt{2} lub \texttt{3}.
W~pierwszym przypadku pole \texttt{points} wskazuje tablic"e struktur
\texttt{point2f} --- zawieraj"a one wsp"o"lrz"edne kartezja"nskie punkt"ow
na p"laszczy"znie. W~drugim przypadku tablica wskazywana przez pole
\texttt{points} zawiera struktury \texttt{vector3f}, kt"ore zawieraj"a
wsp"o"lrz"edne jednorodne punkt"ow (co oznacza, "re krzywa, kt"orej to s"a
punkty kontrolne, jest wymierna, w~reprezentacji jednorodnej).
\end{sloppypar}

Pole \texttt{degree} okre"sla stopie"n $n$ krzywej, kt"ory musi by"c wi"ekszy
lub r"owny $1$.

Pole \texttt{lastknot} okre"sla indeks~$N$ ostatniego wierzcho"lka "lamanej
albo ostatniego w"ez"la krzywej B-sklejanej.

Pole \texttt{knots} jest wska"znikiem do tablicy w"ez"l"ow krzywej
B-sklejanej, o~d"lugo"sci $N+1$.

Pole \texttt{points} jest wska"znikiem do tablicy zawieraj"acej wierzcho"lki
"lamanej, tablica ta zawiera pary lub tr"ojki liczb typu \texttt{float},
zale"rnie od warto"sci pola \texttt{spdimen}.

Aby okre"sli"c \textbf{"laman"a} z"lo"ron"a z $N$ odcink"ow,
nale"ry polom strukury nada"c warto"sci
\texttt{degree=1}, \texttt{lastknot=$N$}, \texttt{knots=NULL}. W tablicy
wskazywanej przez pole \texttt{points} ma by"c \texttt{spdimen*$(N+1)$}
liczb typu \texttt{float} ($N+1$ struktur \texttt{point2f} lub
\texttt{vector3f}).

Aby okre"sli"c \textbf{krzyw"a B\'{e}ziera} stopnia $n>0$, nale"ry polom
struktury nada"c warto"sci
\texttt{degree=$n$}, \texttt{lastknot=-1}, \texttt{knots=NULL}. W tablicy
wskazywanej przez pole \texttt{points} ma by"c \texttt{spdimen*$(n+1)$}
liczb typu \texttt{float} ($n+1$ struktur \texttt{point2f} lub
\texttt{vector3f}).

Aby okre"sli"c \textbf{krzyw"a B-sklejan"a} stopnia $n>0$, nale"ry polom
struktury nada"c warto"sci
\texttt{degree=$n$}, \texttt{lastknot=$N$}. Pole \texttt{knots} ma
wskazywa"c tablic"e $N+1$ liczb typu \texttt{float}, zawieraj"ac"a ci"ag
w"ez"l"ow, a~pole \texttt{points} tablic"e
\texttt{spdimen*$(N-n)$} liczb typu \texttt{float}, zawieraj"ac"a
wsp"o"lrz"edne punkt"ow kontrolnych.

Nie ma wymagania, aby reprezentacja krzywej B-sklejanej wchodz"acej
w~sk"lad brzegu by"la o~ko"ncach zaczepionych, ale krzywa taka musi
"l"aczy"c si"e z s"asiednimi elementami brzegu. W~szczeg"olno"sci mo"rna
okre"sli"c sp"ojny fragment brzegu dziedziny p"lata jako jedn"a
zamkni"et"a krzyw"a B-sklejan"a.

\vspace{\medskipamount}
Brzeg p"lata obci"etego mo"re (ale nie musi) by"c zorientowany. Mo"rna
przyj"a"c konwencj"e, "re podczas poruszania si"e wzd"lu"r wszystkich
"lamanych i~krzywych opisuj"acych brzeg zgodnie z~naturaln"a
parametryzacj"a, mamy wn"etrze dziedziny po lewej stronie (albo po prawej).
Procedury rysowania p"lata powinny by"c tak zaimplementowane, aby
odpowiednia informacja by"la wyprowadzana. Ponadto brzeg mo"re mie"c
samoprzeci"ecia, co nie powinno powodowa"c b"l"ed"ow wykonania programu.

\vspace{\medskipamount}\noindent
\textbf{Przyk"lad.} Brzeg dziedziny p"lata na
rys.~\ref{fig:trimpatch} ma nast"epuj"acy opis:%
\begin{figure}[ht]
  \centerline{\epsfig{file=trimpatch.ps}}
  \caption{\label{fig:trimpatch}Obraz dziedziny i B-sklejanego p"lata
    obci"etego}
\end{figure}

\vspace{\medskipamount}
\noindent{\ttfamily
\#define n  3 \\
\#define NNt1a 10 \\
float ut1a[NNt1a+1] = \\
\ind{2}\{-0.5, 0.0, 0.0, 0.0, 1.4, 2.8, 4.2, 5.6, 5.6, 5.6, 6.1\}; \\
point2f cpt1a[NNt1a-n] = \\
\ind{2}\{\{4.0,0.4\},\{3.5,0.4\},\{2.8,0.8\},\{2.6,2.0\},\{2.8,3.2\},\{3.5,3.6\}, \\
\ind{3}\{4.0,3.6\}\}; \\
point2f cpd1a[3] = \{\{4.0,3.6\},\{4.0,4.0\},\{1.2,4.0\}\}; \\
point3f cpt1b[4] = \{\{1.2,4.0,1.0\},\{1.5,3.0,1.0\},\{0.5,2.5,0.75\},\\
\ind{20}\{0.0,2.5,1.0\}\}; \\
point2f cpd1b[4] = \{\{0.0,2.5\},\{0.0,0.0\},\{4.0,0.0\},\{4.0,0.4\}\}; \\
point2f cpd1c[5] = \{\{0.3,1.1\},\{1.6,1.6\},\{2.1,1.1\},\{1.6,0.6\}, \\
\ind{20}\{0.3,1.1\}\}; \\
point3f cpt1c[4] = \{\{2.0,3.3,1.0\},\{0.6,1.65,0.5\},\{0.6,1.25,0.5\}, \\
\ind{20}\{2.0,2.5,1.0\}\}; \\
point3f cpt1d[4] = \{\{2.0,2.5,1.0\},\{1.25,1.25,0.5\},\{1.25,1.5,0.5\}, \\
\ind{20}\{2.0,3.0,1.0\}\}; \\
point2f cpd1e[2] = \{\{2.0,3.0\},\{2.0,3.3\}\}; \\
polycurvef boundary1[8] = \\
\ind{2}\{\{false,2,n,NNt1a,\&ut1a[0],(float*)\&cpt1a[0]\}, /* B-sklejana */ \\
\ind{3}\{false,2, 1, \ 2, NULL,(float*)\&cpd1a[0]\}, \ /* "lamana */ \\
\ind{3}\{false,3, 3, -1, NULL,(float*)\&cpt1b[0]\}, \ /* krzywa B\'{e}ziera */ \\
\ind{3}\{true, 2, 1, \ 3, NULL,(float*)\&cpd1b[0]\}, \ /* "lamana */ \\
\ind{3}\{true, 2, 1, \ 4, NULL,(float*)\&cpd1c[0]\}, \ /* "lamana */ \\
\ind{3}\{false,3, 3, -1, NULL,(float*)\&cpt1c[0]\}, \ /* krzywa B\'{e}ziera */ \\
\ind{3}\{false,3, 3, -1, NULL,(float*)\&cpt1d[0]\}, \ /* krzywa B\'{e}ziera */ \\
\ind{3}\{true, 2, 1, \ 1, NULL,(float*)\&cpd1e[0]\}\}; /* odcinek */
}\vspace{\medskipamount}

Brzeg w~przyk"ladzie sk"lada si"e z~trzech zamkni"etych krzywych. Pierwsza
z~nich jest ,,obrysem zewn"etrznym'' i~sk"lada si"e z~czterech element"ow:
krzywej B-sklejanej, "lamanej z"lo"ronej z~dw"och odcink"ow, wymiernej krzywej
B\'{e}ziera i~"lamanej z"lo"ronej z~trzech odcink"ow. Druga krzywa to
jedna "lamana zamkni"eta z"lo"rona z~czterech odcink"ow, a~trzecia krzywa
sk"lada si"e z~dw"och p"o"lokr"eg"ow (reprezentowanych jako wymierne krzywe
B\'{e}ziera trzeciego stopnia) i~"lamanej sk"ladaj"acej si"e z~jednego
odcinka.

\textbf{Indeksem} dowolnego punktu na p"laszczy"znie nazwiemy liczb"e
okr"a"re"n (w~kierunku przeciwnym do zegara) wykonanych wok"o"l tego punktu
podczas obchodzenia brzegu zgodnie z~jego orientacj"a. Pierwsze dwie
z~trzech krzywych zamkni"etych opisuj"acych brzeg p"lata w~przyk"ladzie s"a
zorientowane tak, "re poruszaj"ac si"e wzd"lu"r nich ma si"e wn"etrze
dziedziny po lewej stronie. Trzecia krzywa jest zorientowana odwrotnie.
Dlatego indeks punkt"ow na zewn"atrz ,,obrysu'' (pierwszej krzywej)
i~wewn"atrz wielok"ata, kt"orego brzegiem jest druga krzywa jest r"owny~$0$,
za"s punkty w~dw"och p"o"lkolach ograniczonych przez trzeci"a krzyw"a maj"a
indeks r"owny~$2$. Indeks punkt"ow z~wn"etrza dziedziny jest r"owny~$1$.


\subsection{Kompilacja brzegu dziedziny}

Rysowanie p"lata obci"etego wymaga wielokrotnego obliczania punkt"ow
przeci"ecia prostych z~brzegiem dziedziny p"lata. Dla oszcz"edno"sci czasu
reprezentacja opisana w~poprzednim punkcie jest t"lumaczona na kod
zawieraj"acy opis "lamanych i~krzywych B\'{e}ziera, z~kt"orych sk"lada si"e
brzeg.

\vspace{\bigskipamount}
\cprog{%
int  mbs\_TrimCVBoundSizef ( int nelem, const polycurvef *bound );}
Warto"sci"a procedury \texttt{mbs\_TrimCVBoundSizef} jest d"lugo"s"c
(w~bajtach) kodu opisuj"acego brzeg dziedziny p"lata obci"etego. Brzeg jest
reprezentowany zgodnie z~opisem w~poprzednim punkcie; poszczeg"olne cz"e"sci
brzegu s"a "lamanymi, krzywymi B\'{e}ziera lub krzywymi B-sklejanymi
opisanymi przez kolejne elementy tablicy \texttt{bound} o~d"lugo"sci
\texttt{nelem}.

Procedury tej mo"rna u"ry"c w~celu zarezerwowania odpowiedniej tablicy do
przechowania kodu.

\vspace{\bigskipamount}
\cprog{%
void *mbs\_CompileTrimPatchBoundf ( int nelem, \\
\ind{35}const polycurvef *bound, \\
\ind{35}void *buffer );}
Procedura \texttt{mbs\_CompileTrimPatchBoundf} dokonuje ,,kompilacji'' opisu
brzegu dziedziny obci"etego p"lata B-sklejanego, tj.\ generuje kod
reprezentuj"acy "lamane i~krzywe B\'{e}ziera, z~kt"orych sk"lada si"e brzeg
(krzywe B-sklejane s"a zast"epowane ci"agami odpowiednich krzywych
B\'{e}ziera).

Parametr \texttt{nelem} okre"sla d"lugo"s"c tablicy \texttt{bound}, kt"orej
elementy opisuj"a brzeg dziedziny, za"s parametr \texttt{buffer} jest
tablic"a, w~kt"orej ma by"c umieszczony kod. Zak"lada si"e, "re tablica ta
jest dostatecznie d"luga (do obliczenia potrzebnej jej d"lugo"sci s"lu"ry
procedura \texttt{mbs\_TrimCVBoundSizef}). Je"sli warto"s"c parametru
\texttt{buffer} jest wska"znikiem pustym (\texttt{NULL}),
to procedura rezerwuje w~pami"eci pomocniczej (za pomoc"a
\texttt{pkv\_GetScratchMem}) tablic"e o~d"lugo"sci obliczonej
za pomoc"a procedury \texttt{mbs\_TrimCVBoundSizef}.

Warto"sci"a procedury jest wska"znik tablicy zawieraj"acej utworzony kod
(czyli pocz"atkowa warto"s"c parametru \texttt{buffer} lub adres tablicy
zarezerwowanej przez procedur"e), albo \texttt{NULL}, je"sli wyst"api"l
b"l"ad (np.\ brak pami"eci).


\subsection{Wykonywanie obrazk"ow kreskowych}

\begin{sloppypar}\hyphenpenalty=400
Obrazek kreskowy p"lata lub jego dziedziny sk"lada si"e z~krzywych
b"ed"acych obrazami brzegu oraz linii sta"lego pierwszego i~drugiego
parametru. Aby narysowa"c taki obrazek nale"ry wyznazy"c te linie, czyli
wyznaczy"c cz"e"sci wsp"olne odpowiednich prostych z~dziedzin"a p"lata. To
zadanie wykonuje opisana ni"rej procedura
\texttt{mbs\_FindBoundLineIntersectionsf}. Bardziej ,,wysokopoziomowa''
procedura \texttt{mbs\_DrawTrimBSPatchDomf} generuje zbi"or prostych
i~wyznacza ich cz"e"sci wsp"olne z~dziedzin"a. Dla ka"rdej takiej cz"e"sci
(odcinka) procedura wywo"luje podan"a jako parametr \textbf{procedur"e
wyj"sciow"a}, kt"ora w~odpowiedni spos"ob wy"swietla ten odcinek
w~dziedzinie albo jego obraz (fragment krzywej sta"lego parametru) na
p"lacie.
\end{sloppypar}

\vspace{\bigskipamount}
\cprog{%
typedef struct \{ \\
\ind{4}float t; \\
\ind{4}char \ sign1, sign2; \\
\ind{2}\} signpoint1f;}
Struktura typu \texttt{signpoint1f} s"lu"ry do opisania punktu przeci"ecia
prostej z~brzegiem p"lata obci"etego. Prosta jest dana w~postaci
parametrycznej i~dzieli p"laszczyzn"e (w~kt"orej le"ry dziedzina) na dwie
p"o"lp"laszczyzny. Pole~\texttt{t} struktury s"lu"ry do przechowania
warto"sci parametru prostej odpowiadaj"acego punktowi przeci"ecia
z~brzegiem, za"s pola \texttt{sign1} i~\texttt{sign2} opisuj"a spos"ob,
w~jaki brzeg przecina si"e z~prost"a. Mo"rliwe warto"sci tych p"ol to $0$,
$-1$ i~$+1$, kt"ore odpowiadaj"a przypadkom, gdy punkt pocz"atkowy
(\texttt{sign1}) albo ko"ncowy (\texttt{sign2}) przecinaj"acego si"e
z~prost"a ma"lego fragmentu brzegu le"ry na tej prostej albo we wn"etrzu
jednej z~dw"och p"o"lp"laszczyzn.

\vspace{\bigskipamount}
\cprog{%
void mbs\_FindBoundLineIntersectionsf ( const void *bound, \\
\ind{39}const point2f *p0, float t0, \\
\ind{39}const point2f* p1, float t1, \\
\ind{39}signpoint1f *inters, \\
\ind{39}int *ninters );}
\begin{sloppypar}
Procedura \texttt{mbs\_FindBoundLineIntersectionsf} oblicza punkty
prze\-ci"e\-cia pros\-tej przechodz"acej przez punkty \texttt{p0} i~\texttt{p1}
z~brzegiem dziedziny p"lata obci"etego, reprezentowanego przez kod w~tablicy
\texttt{bound} (otrzymany za pomoc"a procedury
\texttt{mbs\_CompileTrimPatchBoundf}). Znaleziome punkty przeci"ecia s"a
wstawiane do tablicy \texttt{inters}. Je"sli brzeg ma z~prost"a wsp"olny
odcinek, to~w~tablicy \texttt{inters} jest on reprezentowany przez dwa elementy,
odpowiadaj"ace pocz"atkowi i~ko"ncowi tego odcinka, przy czym w~takim
przypadku pola \texttt{sign1} i~\texttt{sign2} tych element"ow maj"a
warto"s"c~$0$.
\end{sloppypar}

Liczby \texttt{t0} i~\texttt{t1} s"a parametrami prostej przyporz"adkowanymi
odpowiednio punktom \texttt{p0} i~\texttt{p1}, przy czym zar"owno punkty te,
jak i~odpowiadaj"ace im parametry musz"a by"c r"o"rne.

Pocz"atkowa warto"s"c parametru \texttt{*ninters} okre"sla d"lugo"s"c
(pojemno"s"c) tablicy \texttt{inters}, czyli maksymaln"a liczb"e punkt"ow
przeci"ecia, jak"a program spodziewa si"e znale"z"c. Na~wyj"sciu parametr
ten otrzymuje warto"s"c r"own"a liczbie znalezionych przeci"e"c. Je"sli
wyst"api"l b"l"ad (np.\ w~kodzie), albo przepe"lnienie
tablicy~\texttt{inters}, to parametr \texttt{inters} na wyj"sciu z~procedury
ma warto"s"c ujemn"a.

Tablica \texttt{inters} po znalezieniu wszystkich przeci"e"c jest sortowana
w~kolejno"sci rosn"acych warto"sci p"ol~\texttt{t}.

\vspace{\bigskipamount}
\cprog{%
void mbs\_DrawTrimBSPatchDomf ( int degu, int lastuknot, \\
\ind{20}const float *uknots, \\
\ind{20}int degv, int lastvknot, \\
\ind{20}const float *vknots, \\
\ind{20}int nelem, const polycurvef *bound, \\
\ind{20}int nu, float au, float bu, \\
\ind{20}int nv, float av, float bv, \\
\ind{20}int maxinters, \\
\ind{20}void (*NotifyLine)(char,int,point2f*,point2f*), \\
\ind{20}void (*DrawLine)(point2f*,point2f*,int), \\
\ind{20}void (*DrawCurve)(int,int,const float*) );}
Procedura \texttt{mbs\_DrawTrimBSPatchDomf} mo"re by"c u"ryta do utworzenia
obrazu kreskowego dziedziny obci"etego p"lata B-sklejanego, albo samego
p"lata. Celem procedury jest wyznaczenie odcink"ow le"r"acych w~dziedzinie
takiego p"lata i~wywo"lanie dla ka"rdego takiego odcinka procedury
wyj"sciowej, kt"ora wykonuje rysowanie. Ca"lo"s"c wiedzy o~sposobie dalszego
przetwarzania, w~tym rysowania odcinka (np.\ na ekranie lub w~pliku
postscriptowym) jest odizolowana od procedury
\texttt{mbs\_DrawTrimBSPatchDomf}.

Pierwsze $8$ parametr"ow procedury sk"lada si"e na opis brzegu dziedziny
obci"etego p"lata B-sklejanego. S"a to kolejno: stopie"n~$n$ p"lata ze
wzgl"edu na parametr~$u$ (\texttt{degu}), indeks~$N$ ostatniego w"ez"la
w~ci"agu $u_0,\ldots,u_N$ (\texttt{lastuknot}), tablica z~tymi w"ez"lami
(\texttt{uknots}), stopie"n~$m$ p"lata ze wzgl"edu na parametr~$v$
(\texttt{degv}), indeks~$M$ ostatniego elementu ci"agu w"ez"l"ow
$v_0,\ldots,v_M$, tablica z~tymi w"ez"lami (\texttt{vknots}), liczba
element"ow brzegu (\texttt{nelem}) i~tablica \texttt{bound}, kt"orej
elementy reprezentuj"a brzeg dziedziny p"lata zgodnie z~opisem
w~p.~\ref{ssect:trimpatch:bound}.

Nast"epne $6$ parametr"ow procedury okre"sla siatk"e prostych, kt"orych
przeci"ecia z~dziedzin"a p"lata maj"a by"c wyznaczone. Siatka sk"lada si"e
z~prostych ,,pionowych'' (linii sta"lego parametru~$u$) i~,,poziomych''
(linii sta"lego parametru~$v$).

Linie ,,pionowe'' odpowiadaj"a w"ez"lom $u_n,\ldots,u_{N-n}$
(a~zatem maj"a niepuste przeci"ecia z~dziedzin"a p"lata nieobci"etego)
i~dodatkowo liczbom dziel"acym ka"rdy z~przedzia"l"ow $[u_i,u_{i+1}]$,
$i=n,\ldots,N-n-1$, na podprzedzia"ly o~r"ownych d"lugo"sciach.
Domy"slna liczba tych podprzedzia"l"ow jest r"owna warto"sci parametru
\texttt{nu}, ale jest ona dobierana tak, aby d"lugo"s"c podprzedzia"l"ow
by"la nie mniejsza ni"r warto"s"c parametru \texttt{au} i~nie wi"eksza ni"r
warto"s"c parametru \texttt{bu}.

W~podobny spos"ob parametry \texttt{nv}, \texttt{av} i~\texttt{bv}
okre"slaj"a zbi"or prostych ,,poziomych'' (tj.\ linii sta"lego
parametru~$v$) generowany przez procedur"e.

Parametr \texttt{maxinters} okre"sla maksymaln"a spodziewan"a liczb"e
przeci"e"c prostej z~brzegiem dziedziny p"lata obci"etego. Stosownie do
warto"sci tego parametru procedura rezerwuje tablic"e na przeci"ecia
i~w~razie jej przepe"lnienia mo"re zawie"s"c.

Ostatnie trzy parametry to procedury wyj"sciowe. Ka"rdy z~nich mo"re mie"c
warto"s"c \texttt{NULL}, co oznacza, "re odpowiednie wyniki nie b"ed"a przez
procedur"e wyprowadzane.

Pierwsza z~procedur, \texttt{NotifyLine}, jest wywo"lywana dla ka"rdej nowej
prostej ,,pionowej'' lub ,,poziomej'' z~wygenerowanej przez procedur"e
siatki. Pierwszy parametr (typu \texttt{char}) tej procedury ma
warto"s"c~$1$ je"sli prosta jest pionowa, albo~$2$ je"sli pozioma.
Drugi parametr okre"sla numer odpowiedniego przedzia"lu mi"edzy w"ez\-"la\-mi,
a~kolejne dwa parametry to punkty ko"ncowe odcinka b"ed"acego przeci"eciem
prostej z~dziedzin"a p"lata nieobci"etego. Na~przyk"lad je"sli pierwszy
parametr ma warto"s"c~$1$, a~drugi~$k$, to prosta jest ,,pionowa'', tj.\
jest lini"a sta"lego parametru~$u$, kt"ory jest liczb"a z~przedzia"lu
$[u_k,u_{k+1})$. Liczba ta jest te"r warto"sci"a wsp"o"lrz"ednej~$x$
punkt"ow przekazanych jako trzeci i~czwarty parametr.

Procedura wyj"sciowa \texttt{DrawLine} jest wywo"lywana po znalezieniu
przeci"e"c brzegu dziedziny p"lata obci"etego z~prost"a, dla \emph{ka"rdej}
pary kolejnych punkt"ow przeci"ecia. Punkty te s"a przekazywane jako
pierwsze dwa parametry. Trzeci parametr ma warto"s"c, kt"ora jest indeksem
punkt"ow wewn"atrz odcinka (zobacz p.~\ref{ssect:trimpatch:bound}). Je"sli
brzeg jest zorientowany w~ten spos"ob, "re podczas jego obchodzenia mamy
wn"etrze dziedziny po lewej stronie, to indeks ten zawsze b"edzie mia"l
warto"s"c~$1$ (co~oznacza, "re odcinek le"ry w dziedzinie) albo~$0$ (co
oznacza, "re odcinek le"ry poza dziedzin"a). W~og"olno"sci orientacja
poszczeg"olnych krzywych zamkni"etych, z~kt"orych sk"lada si"e brzeg, mo"re
by"c inna (tak jest np.\ w~przyk"ladzie w~p.~\ref{ssect:trimpatch:bound}).
Okre"slenie kt"ore odcinki le"r"a w~dziedzinie zale"ry od procedury
\texttt{DrawLine} (mo"re ona np.\ by"c oparta o~regu"l"e parzysto"sci:
w~dziedzinie le"r"a te odcinki, kt"orych punkty maj"a indeks nieparzysty).

Procedura \texttt{DrawCurve} jest wywo"lywana w~celu narysowania element"ow
brzegu dziedziny p"lata obci"etego. Pierwszy jej parametr ma warto"s"c~$d=2$
albo~$3$, co oznacza odpowiednio, "re p"laska krzywa B\'{e}ziera jest
wielomianowa albo wymierna (w~reprezentacji jednorodnej). Drugi parametr
okre"sla stopie"n $n$ krzywej (je"sli ma warto"s"c $1$, to krzywa jest
odcinkiem, co mo"rna wykorzysta"c). Trzeci parametr jest tablic"a punkt"ow
kontrolnych, czyli $(n+1)d$ liczb zmiennopozycyjnych, kt"ore s"a
wsp"o"lrz"ednymi tych punkt"ow.


\subsection*{Przyk"lad --- procedury wyj"sciowe dla obrazk"ow kreskowych}

Poni"rej s"a opisane procedury przyk"ladowe, kt"orych zadaniem jest
wykonanie obraz"ow na rys.~\ref{fig:trimpatch} w~j"ezyku PostScript.

Obrazek z~lewej strony przedstawia dziedzin"e p"lata B-sklejanego,
tj.\ przeci"ecia \mbox{linii} sta"lego parametru odpowiadaj"acych w"ez"lom
p"lata
z~dziedzin"a oraz jej brzeg. Odcinki linii sta"lego parametru s"a rysowane
przez podan"a ni"rej procedur"e \texttt{DrawLine1}, kt"ora u"rywa procedury
\texttt{MapPoint} do odpowiedniego odwzorowania (przeskalowania
i~przesuni"ecia) ko"nc"ow odcink"ow.

\vspace{\medskipamount}
\noindent{\ttfamily
void DrawLine1 ( point2f *p0, point2f *p1, int index ) \\
\{ \\
\ind{2}point2f q0, q1; \\
\ind{2}if ( index == 1 ) \{ \\
\ind{4}ps\_Set\_Line\_Width ( 2.0 ); \\
\ind{4}MapPoint ( frame, p0, \&q0 ); \\
\ind{4}MapPoint ( frame, p1, \&q1 ); \\
\ind{4}ps\_Draw\_Line ( q0.x, q0.y, q1.x, q1.y ); \\
\ind{2}\} \\
\} /*DrawLine1*/
}\vspace{\medskipamount}

Brzeg dziedziny zosta"l narysowany przez procedur"e
\texttt{DrawCurve1}, kt"orej skr"ocona wersja jest taka (pe"lna wersja jest
w~pliku \texttt{trimpatch.c}):

\vspace{\medskipamount}
\noindent{\ttfamily
void DrawCurve1 ( int dim, int degree, const float *cp ) \\
\{ \\
\#define DENS 50 \\
\ind{2}int i, size; \\
\ind{2}float t; \\
\ind{2}point2f *c, p; \\
\ind{2}ps\_Set\_Line\_Width ( 6.0 ); \\
\ind{2}if ( degree == 1 ) \{ \\
\ind{4}/* Krzywa B\'{e}ziera stopnia $1$ jest odcinkiem, wi"ec ten */ \\
\ind{4}/* przypadek jest traktowany osobno.\ Tablica cp zawiera 4 */ \\
\ind{4}/* lub 6 liczb, tj.\ wsp"o"lrz"edne kartezjanskie lub jednorodne */ \\
\ind{4}/* (zale"rnie od warto"sci parametru dim) ko"nc"ow odcinka.\ */ \\
\ind{4}... \\
\ind{2}\} \\
\ind{2}else /* degree > 1, rysujemy "laman"a */ \{ \\
\ind{4}if ( c = pkv\_GetScratchMem ( size=(DENS+1)*sizeof(point2f) ) ) \{ \\
\ind{6}if ( dim == 2 ) \{ \\
\ind{8}for ( i = 0; i <= DENS; i++ ) \{ \\
\ind{10}t = (float)i/(float)DENS; \\
\ind{10}mbs\_BCHornerC2f ( degree, cp, t, \&p ); \\
\ind{10}MapPoint ( frame, \&p, \&c[i] ); \\
\ind{8}\} \\
\ind{6}\} \\
\ind{6}else if ( dim == 3 ) \{ \\
\ind{8}for ( i = 0; i <= DENS; i++ ) \{ \\
\ind{10}t = (float)i/(float)DENS; \\
\ind{10}mbs\_BCHornerC2Rf ( degree, (point3f*)cp, t, \&p ); \\
\ind{10}MapPoint ( frame, \&p, \&c[i] ); \\
\ind{8}\} \\
\ind{6}\} \\
\ind{6}else goto out; \\
\ind{6}ps\_Draw\_Polyline ( c, DENS ); \\
out: \\
\ind{6}pkv\_FreeScratchMem ( size ); \\
\ind{4}\} \\
\ind{2}\} \\
\#undef DENS \\
\} /*DrawCurve1*/
}\vspace{\medskipamount}

Wywo"lanie procedury \texttt{mbs\_DrawTrimBSPatchDomf}, kt"ore spowodowa"lo
powstanie tego rysunku, ma posta"c

\vspace{\medskipamount}
\noindent{\ttfamily
\ind{2}mbs\_DrawTrimBSPatchDomf ( n1, NN1, u1, m1, MM1, v1, 8, boundary1, \\
\ind{28}1, 2.0, 2.0, 1, 2.0, 2.0, \\
\ind{28}20, NULL, DrawLine1, DrawCurve1 );
}\vspace{\medskipamount}

Pierwsze $6$ parametr"ow opisuje stopie"n i~w"ez"ly, czyli w~szczeg"olno"sci
dziedzin"e p"lata nieobci"etego, zgodnie z~wcze"sniejszym opisem.
Dziedzina ta jest prostok"atem $[0,4]\times[0,4]$, a~d"lugo"sci
przedzia"l"ow mi"edzy w"ez"lami s"a mi"edzy $1$ i~$1.5$.
Dlatego warto"sci parametr"ow \texttt{nu}, \texttt{au}, \texttt{bu},
\texttt{nv}, \texttt{av}, \texttt{bv} zapewniaj"a rysowanie tylko linii
sta"lego parametru odpowiadaj"ace w"ez"lom p"lata.

Wykonanie takiego obrazka p"lata obci"etego jak na rys.~\ref{fig:trimpatch}
z~prawej strony wymaga odwzorowania odpowiednich linii w~dziedzinie na
p"lat, a~nast"epnie ich zrzutowanie. Procedura \texttt{DrawLine2}, kt"ora
zosta"la u"ryta w~tym przypadku, ma posta"c

\vspace{\medskipamount}
\noindent{\ttfamily
void DrawLine2 ( point2f *p0, point2f *p1, int index ) \\
\{ \\
\#define LGT 0.05 \\
\ind{2}void     *sp; \\
\ind{2}int      i, k; \\
\ind{2}float    t, d; \\
\ind{2}vector2f v; \\
\ind{2}point2f  q, *c; \\
\ind{2}point3f  p, r; \\
\ind{2}if ( index == 1 ) \{ \\
\ind{4}ps\_Set\_Line\_Width ( 2.0 ); \\
\ind{4}SubtractPoints2f ( p1, p0, \&v ); \\
\ind{4}d = sqrt ( DotProduct2f(\&v,\&v) ); \\
\ind{4}k = (int)(d/LGT+0.5); \\
\ind{4}sp = pkv\_GetScratchMemTop (); \\
\ind{4}c = (point2f*)pkv\_GetScratchMem ( (k+1)*sizeof(point2f) ); \\
\ind{4}for ( i = 0; i <= k; i++ ) \{ \\
\ind{6}t = (float)i/(float)k; \\
\ind{6}InterPoint2f ( p0, p1, t, \&q ); \\
\ind{6}mbs\_deBoorP3f ( n1, NN1, u1, m1, MM1, v1, 3*(MM1-m1), \\
\ind{22}\&cp1[0][0], q.x, q.y, \&p ); \\
\ind{6}PhotoPointUDf ( \&CPos, \&p, \&r ); \\
\ind{6}c[i].x = r.x;  c[i].y = r.y; \\
\ind{4}\} \\
\ind{4}ps\_Draw\_Polyline ( c, k ); \\
\ind{4}pkv\_SetScratchMemTop ( sp ); \\
\ind{2}\} \\
\#undef LGT \\
\} /*DrawLine2*/
}\vspace{\medskipamount}
 
Kilka s"l"ow wyja"snienia: zamiast krzywej procedura rysuje "laman"a. Liczba
jej odcink"ow jest zale"rna od d"lugo"sci odcinka w~dziedzinie p"lata,
kt"orego obraz na p"lacie jest rysowany (mo"rna by te"r wzi"a"c pod uwag"e
kszta"lt p"lata, ale tak jest najpro"sciej). Procedura \texttt{DrawLine2} ma
dost"ep do reprezentacji p"lata (tj.\ w"ez"l"ow i~punkt"ow kontrolnych)
poprzez zmienne globalne. Punkty p"lata s"a obliczane za pomoc"a algorytmu
de~Boora (przez wywo"lanie \texttt{mbs\_deBoorP3f}). Mo"rna zmniejszy"c
koszt obliczania tych punkt"ow, podaj"ac jako parametr
\texttt{NotifyLine} procedur"e, kt"orej zadaniem by"loby wyznaczenie
reprezentacji B-sklejanej (ewentualnie kawa"lkami B\'{e}ziera) krzywej
sta"lego parametru $u$ albo $v$. Wywo"lania procedury podanej jako
parametr \texttt{DrawLine} po wywo"laniu \texttt{NotifyLine} maj"a na celu
narysowanie "luk"ow tej krzywej sta"lego parametru.

