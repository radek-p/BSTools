
%/* //////////////////////////////////////////////////// */
%/* This file is a part of the BSTools procedure package */
%/* written by Przemyslaw Kiciak.                        */
%/* //////////////////////////////////////////////////// */

\newpage
\section{\label{sect:Coons:patch:procedures}Przetwarzanie p"lat"ow Coonsa}

\subsection{P"laty wielomianowe}

Wielomianowe p"laty Coonsa s"a reprezentowane za pomoc"a krzywych B\'{e}ziera,
kt"orych stopnie mog"a by"c r"o"rne. Dziedzin"a p"lata jest kwadrat $[0,1]^2$,
zatem liczby $a,b,c,d$ opisane w~p.~\ref{ssect:Coons:patch:def} s"a r"owne
odpowiednio $0,1,0,1$.

\vspace{\bigskipamount}
\cprog{%
void mbs\_BezC1CoonsFindCornersf ( int spdimen, \\
\ind{34}int degc00, const float *c00, \\
\ind{34}int degc01, const float *c01, \\
\ind{34}int degc10, const float *c10, \\
\ind{34}int degc11, const float *c11, \\
\ind{34}float *pcorners );}
\begin{sloppypar}
Procedura \texttt{mbs\_BezC1CoonsFindCornersf} wyznacza macierz~$\bm{P}$
o~wymiarach $4\times\nobreak 4$, kt"orej elementami s"a odpowiednie punkty
krzywych $\bm{c}_{00},\bm{c}_{10},\bm{c}_{01},\bm{c}_{11}$ i~wektory ich
pochodnych.%
\end{sloppypar}

Parametry: \texttt{spdimen} --- wymiar~$d$ przestrzeni, w~kt"orej le"r"a
krzywe i~reprezentowany przez nie wielomianowy bikubiczny (klasy~$C^1$)
p"lat Coonsa. Ka"rda para parametr"ow \texttt{degc??} i~\texttt{c??}
opisuje jedn"a z~krzywych, stopie"n i~tablic"e punkt"ow kontrolnych.

Parametr \texttt{pcorners} jest wska"znikiem tablicy, w~kt"orej ma by"c
umieszczony wynik; tablica ta musi mie"c d"lugo"s"c~$16d$.

\vspace{\bigskipamount}
\cprog{%
boolean mbs\_BezC1CoonsToBezf ( int spdimen, \\
\ind{31}int degc00, const float *c00, \\
\ind{31}int degc01, const float *c01, \\
\ind{31}int degc10, const float *c10, \\
\ind{31}int degc11, const float *c11, \\
\ind{31}int degd00, const float *d00, \\
\ind{31}int degd01, const float *d01, \\
\ind{31}int degd10, const float *d10, \\
\ind{31}int degd11, const float *d11, \\
\ind{31}int *n, int *m, float *p );}
Procedura \texttt{mbs\_BezC1CoonsToBezf} wyznacza reprezentacj"e B\'{e}ziera
bikubicznego p"lata Coonsa (klasy~$C^1$) okre"slonego przez dane krzywe
wielomianowe. Warto"sci"a procedury jest \texttt{true}, je"sli obliczenie
zako"nczy"lo si"e sukcesem, a~\texttt{false} w~przeciwnym razie (przyczyn"a
niepowodzenia mo"re by"c brak pami"eci na stosie pami"eci pomocniczej).

Warto"s"c parametru \texttt{spdimen} jest wymiarem~$d$ przestrzeni, w~kt"orej
le"r"a krzywe i~okre"slony przez nie p"lat. Ka"rda para parametr"ow
\texttt{degc??} i~\texttt{c??} opisuje odpowiedni"a krzyw"a z~rodziny
$\bm{c}_{00},\bm{c}_{01},\bm{c}_{10},\bm{c}_{11}$, przez podanie stopnia
i~tablicy punkt"ow kontrolnych. Kolejne pary parametr"ow \texttt{degd??}
i~\texttt{d??} opisuj"a w~ten sam spos"ob krzywe z~rodziny
$\bm{d}_{00},\bm{d}_{01},\bm{d}_{10},\bm{d}_{11}$. Krzywe te musz"a
spe"lnia"c (z~dok"ladno"sci"a do b"l"ed"ow zaokr"agle"n) warunki
zgodno"sci~(\ref{eq:Coons:compat:cond}), co \emph{nie jest} sprawdzane.

Zmienne \texttt{*n} i~\texttt{*m} otrzymuj"a warto"sci opisuj"ace stopie"n
reprezentacji B\'{e}ziera p"lata. Warto"sci"a~$n$ zmiennej \texttt{*n} jest
najwi"eksza z~warto"sci parametr"ow \texttt{degc??} lub~$3$ (je"sli liczba~$3$
jest wi"eksza). Podobnie, warto"s"c~$m$ zmiennej \texttt{*m} jest najwi"eksz"a
z~warto"sci parametr"ow \texttt{degd??} lub~$3$. W~tablicy wskazywanej przez
parametr \texttt{p} procedura umieszcza punkty kontrolne B\'{e}ziera p"lata;
tablica ta musi by"c dostatecznie pojemna (musi mie"c d"lugo"s"c co najmniej
$(n+1)(m+1)d$).

\vspace{\bigskipamount}
\cprog{%
void mbs\_TabCubicHFuncDer2f ( float a, float b, \\
\ind{20}int nkn, const float *kn, \\
\ind{20}float *hfunc, float *dhfunc, float *ddhfunc );}
Procedura \texttt{mbs\_TabCubicHFuncDer2f} tablicuje wielomiany
$\tilde{H}_{00},\tilde{H}_{10},\tilde{H}_{01}$ i~$\tilde{H}_{11}$,
bed"ace podstaw"a definicji bikubicznego p"lata Coonsa, oraz ich pochodne
rz"edu~$1$ i~$2$. Wyniki tego obliczenia mog"a by"c u"ryte do stablicowania
bikubicznego p"lata Coonsa na siatce prostok"atnej, za pomoc"a procedury
\texttt{mbs\_TabBezC1CoonsDer2f} (p"lat wielomianowy) lub
\texttt{mbs\_TabBSC1CoonsDer2f} (p"lat sklejany).

Parametry \texttt{a} i~\texttt{b} opisuj"a ko"nce odcinka przyj"etego
za dziedzin"e krzywych $\bm{c}_{ij}$ lub $\bm{d}_{ij}$; dla wielomianowych
p"lat"ow Coonsa ich warto"sciami powinny by"c liczby $0$ i~$1$.

Parametr \texttt{nkn} okre"sla liczb"e~$k$ punkt"ow $u_m\in[a,b]$, w~kt"orych
nale"ry obliczy"c warto"sci wielomian"ow; punkty te (tj.\ liczby
zmiennopozycyjne) s"a podane w~tablicy \texttt{kn}.

Warto"sci wielomian"ow oraz ich pochodnych rz"edu~$1$ i~$2$ s"a wpisywane
odpowiednio do tablic \texttt{hfunc}, \texttt{dhfunc} i~\texttt{ddhfunc}.
Tablice te musz"a mie"c d"lugo"s"c co najmniej~$4k$; na ka"rdych kolejnych
czterech pozycjach w~tablicy s"a wpisywane warto"sci czterech wielomian"ow
lub ich pochodnych w~kolejnym punkcie $u_m$.

\vspace{\bigskipamount}
\cprog{%
void mbs\_TabCubicHFuncDer3f ( float a, float b, int nkn, \\
\ind{20}const float *kn, \\
\ind{20}float *hfunc, float *dhfunc, float *ddhfunc, \\
\ind{20}float *dddhfunc );}

\vspace{\bigskipamount}
\cprog{%
boolean mbs\_TabBezC1CoonsDer2f ( int spdimen, \\
\ind{8}int nknu, const float *knu, const float *hfuncu, \\
\ind{8}const float *dhfuncu, const float *ddhfuncu, \\
\ind{8}int nknv, const float *knv, const float *hfuncv, \\
\ind{8}const float *dhfuncv, const float *ddhfuncv, \\
\ind{8}int degc00, const float *c00, \\
\ind{8}int degc01, const float *c01, \\
\ind{8}int degc10, const float *c10, \\
\ind{8}int degc11, const float *c11, \\
\ind{8}int degd00, const float *d00, \\
\ind{8}int degd01, const float *d01, \\
\ind{8}int degd10, const float *d10, \\
\ind{8}int degd11, const float *d11, \\
\ind{8}float *p, float *pu, float *pv, \\
\ind{8}float *puu, float *puv, float *pvv );}
Procedura \texttt{mbs\_TabBezC1CoonsDer2f} s"lu"ry do szybkiego stablicowania
bikubicznego wielomianowego p"lata Coonsa, razem z~pochodnymi cz"astkowymi
rz"edu~$1$ i~$2$, dla punkt"ow $(u_i,v_j)$, gdzie $i\in\{0,\ldots,k-1\}$,
$j\in\{0,\ldots,l-1\}$.

Parametr~\texttt{spdimen} okre"sla wymiar~$d$ przestrzeni, w~kt"orej jest
p"lat. Parametr~\texttt{nknu} okre"sla liczb"e~$k$, tablica~\texttt{knu}
zawiera liczby $t_0,\ldots,t_{k-1}$. Tablice \texttt{hfuncu},
\texttt{dhfuncu},
\texttt{ddhfuncu} zawieraj"a odpowiednio warto"sci wielomian"ow
$H_{00},H_{10},H_{01},H_{11}$ i~ich pochodnych rz"edu~$1$ i~$2$ w~punktach
$u_0,\ldots,u_{k-1}$; warto"sci te najpro"sciej jest obliczy"c
wywo"luj"ac zawczasu procedur"e \texttt{mbs\_TabCubicHFuncDer2f}
(z~parametrami \texttt{a}${}=0$, \texttt{b}${}=1$).

Ci"ag liczb $v_j$ okre"slaj"acych drugie wsp"o"lrz"edne punkt"ow
tablicowania p"lata jest w~analogiczny spos"ob opisany za pomoc"a
parametr"ow \texttt{nknv} i~\texttt{knv}, tablice \texttt{hfuncv},
\texttt{dhfuncv} i~\texttt{ddhfuncv} zawieraj"a warto"sci funkcji $H_{ij}$
i~ich pochodnych dla tych liczb.

Pary parametr"ow \texttt{degc??} i~\texttt{c??} oraz \texttt{degd??}
i~\texttt{d??} opisuj"a krzywe B\'{e}ziera okre"slaj"ace p"lat.
Krzywe te musz"a spe"lnia"c (z~dok"ladno"sci"a do b"l"ed"ow zaokr"agle"n)
warunki zgodno"sci~(\ref{eq:Coons:compat:cond}).

Do tablic wskazywanych przez parametry \texttt{p}, \texttt{pu}, \texttt{pv},
\texttt{puu}, \texttt{puv}, \texttt{pvv} procedura wpisuje obliczone punkty
p"lata i~wektory pochodnych cz"astkowych rz"edu~$1$ i~$2$; je"sli kt"ory"s
z~tych parametr"ow ma warto"s"c \texttt{NULL}, to odpowiednia pochodna
nie jest tablicowana. W~przeciwnym razie wskazywana tablica musi mie"c
d"lugo"s"c co najmniej~$k^2d$.

Warto"sci"a procedury jest \texttt{true} w~razie sukcesu i~\texttt{false}
w~razie niepowodzenia oblicze"n (z~powodu braku miejsca na stosie
pami"eci pomocniczej).

\vspace{\bigskipamount}
\cprog{%
boolean mbs\_TabBezC1CoonsDer3f ( int spdimen, \\
\ind{8}int nknu, const float *knu, const float *hfuncu, \\
\ind{8}const float *dhfuncu, const float *ddhfuncu, \\
\ind{8}const float *dddhfuncu, \\
\ind{8}int nknv, const float *knv, const float *hfuncv, \\
\ind{8}const float *dhfuncv, const float *ddhfuncv, \\
\ind{8}const float *dddhfuncv, \\
\ind{8}int degc00, const float *c00, \\
\ind{8}int degc01, const float *c01, \\
\ind{8}int degc10, const float *c10, \\
\ind{8}int degc11, const float *c11, \\
\ind{8}int degd00, const float *d00, \\
\ind{8}int degd01, const float *d01, \\
\ind{8}int degd10, const float *d10, \\
\ind{8}int degd11, const float *d11, \\
\ind{8}float *p, float *pu, float *pv, \\
\ind{8}float *puu, float *puv, float *pvv, \\
\ind{8}float *puuu, float *puuv, float *puvv, float *pvvv );}

\vspace{\bigskipamount}
\cprog{%
boolean mbs\_TabBezC1Coons0Der2f ( int spdimen, \\
\ind{8}int nknu, const float *knu, const float *hfuncu, \\
\ind{8}const float *dhfuncu, const float *ddhfuncu, \\
\ind{8}int nknv, const float *knv, const float *hfuncv, \\
\ind{8}const float *dhfuncv, const float *ddhfuncv, \\
\ind{8}int degc00, const float *c00, \\
\ind{8}int degc01, const float *c01, \\
\ind{8}int degd00, const float *d00, \\
\ind{8}int degd01, const float *d01, \\
\ind{8}float *p, float *pu, float *pv, \\
\ind{8}float *puu, float *puv, float *pvv );}
\begin{sloppypar}
Procedura \texttt{mbs\_TabBezC1Coons0Der2f} jest nieco uproszczon"a wersj"a
procedury \texttt{mbs\_TabBezC1CoonsDer2f} dla przypadku, gdy krzywe
$\bm{c}_{10},\bm{c}_{11},\bm{d}_{10}$ i~$\bm{d}_{11}$ s"a zerowe
(tj.\ gdy wszystkie ich punkty kontrolne maj"a zerowe wszystkie
wsp"o"lrz"edne). Stablicowanie p"lata okre"slonego przez takie krzywe mo"re
by"c wykonane w~kr"otszym czasie; z~procedury tej korzysta biblioteka
\texttt{libg1hole}.%
\end{sloppypar}

Parametry procedury \texttt{mbs\_TabBezC1Coons0Der2f} s"a takie same,
jak parametry proceury \texttt{mbs\_TabBezC1CoonsDer2f} o~tych samych
nazwach.

\vspace{\bigskipamount}
\cprog{%
boolean mbs\_TabBezC1Coons0Der3f ( int spdimen, \\
\ind{8}int nknu, const float *knu, const float *hfuncu, \\   
\ind{8}const float *dhfuncu, const float *ddhfuncu, \\
\ind{8}const float *dddhfuncu, \\
\ind{8}int nknv, const float *knv, const float *hfuncv, \\
\ind{8}const float *dhfuncv, const float *ddhfuncv, \\
\ind{8}const float *dddhfuncv, \\
\ind{8}int degc00, const float *c00, \\
\ind{8}int degc01, const float *c01, \\
\ind{8}int degd00, const float *d00, \\
\ind{8}int degd01, const float *d01, \\
\ind{8}float *p, float *pu, float *pv, \\
\ind{8}float *puu, float *puv, float *pvv, \\
\ind{8}float *puuu, float *puuv, float *puvv, float *pvvv );}


\vspace{\bigskipamount}
\cprog{%
void mbs\_BezC2CoonsFindCornersf ( int spdimen, \\
\ind{34}int degc00, const float *c00, \\
\ind{34}int degc01, const float *c01, \\
\ind{34}int degc02, const float *c02, \\
\ind{34}int degc10, const float *c10, \\
\ind{34}int degc11, const float *c11, \\
\ind{34}int degc12, const float *c12, \\
\ind{34}float *pcorners );}
\begin{sloppypar}
Procedura \texttt{mbs\_BezC2CoonsFindCornersf} wyznacza macierz~$\bm{P}$  
o~wymiarach $6\times\nobreak 6$, kt"orej elementami s"a odpowiednie punkty
krzywych
$\bm{c}_{00},\bm{c}_{10},\bm{c}_{01},\bm{c}_{11},\bm{c}_{02},\bm{c}_{12}$
i~wektory ich pochodnych.%   
\end{sloppypar}

Parametry: \texttt{spdimen} --- wymiar~$d$ przestrzeni, w~kt"orej le"r"a
krzywe i~reprezentowany przez nie wielomianowy dwupi"etny (klasy~$C^2$)
p"lat Coonsa. Ka"rda para parametr"ow \texttt{degc??} i~\texttt{c??}
opisuje jedn"a z~krzywych, stopie"n i~tablic"e punkt"ow kontrolnych.

Parametr \texttt{pcorners} jest wska"znikiem tablicy, w~kt"orej ma by"c
umieszczony wynik; tablica ta musi mie"c d"lugo"s"c~$36d$.


\vspace{\bigskipamount}
\cprog{%
boolean mbs\_BezC2CoonsToBezf ( int spdimen, \\
\ind{31}int degc00, const float *c00, \\
\ind{31}int degc01, const float *c01, \\
\ind{31}int degc02, const float *c02, \\
\ind{31}int degc10, const float *c10, \\
\ind{31}int degc11, const float *c11, \\
\ind{31}int degc12, const float *c12, \\
\ind{31}int degd00, const float *d00, \\
\ind{31}int degd01, const float *d01, \\
\ind{31}int degd02, const float *d02, \\
\ind{31}int degd10, const float *d10, \\
\ind{31}int degd11, const float *d11, \\
\ind{31}int degd12, const float *d12, \\
\ind{31}int *n, int *m, float *p );}
Procedura \texttt{mbs\_BezC2CoonsToBezf} dokonuje konwersji dwupi"etnego
p"lata Coonsa do postaci B\'{e}ziera. P"lat jest dany za pomoc"a
$12$~krzywych wielomianowych, opisuj"acych brzeg (krzywe $c_{00}$, $c_{10}$,
$d_{00}$, $d_{10}$) i~pochodne poprzeczne rz"edu~$1$ (krzywe $c_{01}$,
$c_{11}$, $d_{01}$, $d_{11}$) i~$2$ (krzywe $c_{02}$, $c_{12}$, $d_{02}$,
$d_{12}$). Wszystkie te krzywe s"a dane w~postaci B\'{e}ziera, ich stopnie
s"a okre"slone za pomoc"a pa\-ra\-met\-r"ow
\texttt{degc00}, \ldots, \texttt{degd12},
punkty kontrolne (w~przestrzeni o~wymiarze \texttt{spdimen}) s"a dane
w~tablicach \texttt{c00}, \ldots, \texttt{d12}.

Parametry wyj"sciowe to \texttt{*n} i~\texttt{*m}, kt"ore otrzymuj"a
warto"sci okre"slaj"ace stopie"n, oraz tablica \texttt{p}, do kt"orej
procedura wpisuje wsp"o"lrz"edne punkt"ow kontrolnych B\'{e}ziera p"lata.

\vspace{\bigskipamount}
\cprog{%
void mbs\_TabQuinticHFuncDer3f (  float a, float b, \\
\ind{32}int nkn, const float *kn, \\
\ind{32}float *hfunc, float *dhfunc, \\
\ind{32}float *ddhfunc, float *dddhfunc );}
\begin{sloppypar}
Procedura \texttt{mbs\_TabQuinticHFuncDer3f} tablicuje wielomiany   
$\tilde{H}_{00},\tilde{H}_{10},\tilde{H}_{01}$, $\tilde{H}_{11},\tilde{H}_{02}$
i~$\tilde{H}_{12}$, bed"ace podstaw"a definicji dwupi"etnego p"lata Coonsa,
oraz ich pochodne rz"edu~$1$, $2$ i~$3$. Wyniki tego obliczenia mog"a by"c
u"ryte do stablicowania dwupi"etnego p"lata Coonsa na siatce prostok"atnej,
za pomoc"a procedury \texttt{mbs\_TabBezC2CoonsDer3f} (p"lat wielomianowy) lub
\texttt{mbs\_TabBSC2CoonsDer3f} (p"lat sklejany).%
\end{sloppypar}

Parametry \texttt{a} i~\texttt{b} opisuj"a ko"nce odcinka przyj"etego
za dziedzin"e krzywych $\bm{c}_{ij}$ lub $\bm{d}_{ij}$; dla wielomianowych
p"lat"ow Coonsa ich warto"sciami powinny by"c liczby $0$ i~$1$.

Parametr \texttt{nkn} okre"sla liczb"e~$k$ punkt"ow $u_m\in[a,b]$,
w~kt"orych nale"ry obliczy"c warto"sci wielomian"ow; punkty te (tj.\ liczby
zmiennopozycyjne) s"a podane w~tablicy \texttt{kn}.

Warto"sci wielomian"ow oraz ich pochodnych rz"edu~$1$, $2$ i~$3$ s"a wpisywane 
odpowiednio do tablic \texttt{hfunc}, \texttt{dhfunc}, \texttt{ddhfunc}
i~\texttt{dddhfunc}. Tablice te musz"a mie"c d"lugo"s"c co najmniej~$6k$;
na ka"rdych kolejnych sze"sciu pozycjach w~tablicy s"a wpisywane warto"sci
sze"sciu wielomian"ow lub ich pochodnych w~kolejnym punkcie $u_m$.

\vspace{\bigskipamount}
\cprog{%
boolean mbs\_TabBezC2CoonsDer3f ( int spdimen, \\
\ind{8}int nknu, const float *knu, const float *hfuncu, \\
\ind{8}const float *dhfuncu, const float *ddhfuncu, \\
\ind{8}const float *dddhfuncu, \\
\ind{8}int nknv, const float *knv, const float *hfuncv, \\
\ind{8}const float *dhfuncv, const float *ddhfuncv, \\
\ind{8}const float *dddhfuncv, \\
\ind{8}int degc00, const float *c00, \\
\ind{8}int degc01, const float *c01, \\
\ind{8}int degc02, const float *c02, \\
\ind{8}int degc10, const float *c10, \\
\ind{8}int degc11, const float *c11, \\
\ind{8}int degc12, const float *c12, \\
\ind{8}int degd00, const float *d00, \\
\ind{8}int degd01, const float *d01, \\
\ind{8}int degd02, const float *d02, \\
\ind{8}int degd10, const float *d10, \\
\ind{8}int degd11, const float *d11, \\
\ind{8}int degd12, const float *d12, \\
\ind{8}float *p, float *pu, float *pv, float *puu, \\
\ind{8}float *puv, float *pvv, \\
\ind{8}float *puuu, float *puuv, float *puvv, float *pvvv );}
Procedura \texttt{mbs\_TabBezC2CoonsDer3f} s"lu"ry do szybkiego stablicowania
dwupi"etnego wielomianowego p"lata Coonsa, razem z~pochodnymi cz"astkowymi
rz"edu~$1$, $2$ i~$3$, dla punkt"ow $(u_i,v_j)$, gdzie
$i\in\{0,\ldots,k-1\}$, $j\in\{0,\ldots,l-1\}$.

\begin{sloppypar}
Parametr~\texttt{spdimen} okre"sla wymiar~$d$ przestrzeni, w~kt"orej jest
p"lat. Parametr~\texttt{nknu} okre"sla liczb"e~$k$, tablica~\texttt{knu}
zawiera liczby $u_0,\ldots,u_{k-1}$. Tablice \texttt{hfuncu},
\texttt{dhfuncu},
\texttt{ddhfuncu} i~\texttt{dddhfuncu} zawieraj"a odpowiednio warto"sci
wielomian"ow $H_{00},H_{10},H_{01},H_{11},H_{02},H_{12}$ i~ich pochodnych
rz"edu~$1$, $2$ i~$3$ w~punktach $t_0,\ldots,t_{k-1}$; warto"sci te
najpro"sciej jest obliczy"c wywo"luj"ac zawczasu procedur"e
\texttt{mbs\_TabQuinticHFuncDer3f} (z~parametrami \texttt{a}${}=0$,
\texttt{b}${}=1$).%
\end{sloppypar}

Parametry \texttt{nknv}, \texttt{knv}, \texttt{hfuncv}, \texttt{dhfuncv},
\texttt{ddhfuncv} i~\texttt{dddhfuncv} w~analogiczny spos"ob reprezentuj"a
ci"ag $v_0,\ldots,v_{l-1}$ i~warto"sci i~pochodne funkcji $H_{ij}$
w~punktach z~tego ci"agu.

Pary parametr"ow \texttt{degc??} i~\texttt{c??} oraz \texttt{degd??}
i~\texttt{d??} opisuj"a krzywe B\'{e}ziera okre"slaj"ace p"lat.
Krzywe te musz"a spe"lnia"c (z~dok"ladno"sci"a do b"l"ed"ow zaokr"agle"n)
warunki zgodno"sci~(\ref{eq:Coons:compat:cond}).

Do tablic wskazywanych przez parametry \texttt{p}, \texttt{pu}, \texttt{pv},
\texttt{puu}, \texttt{puv}, \texttt{pvv}, \texttt{puuu}, \texttt{puuv},
\texttt{puvv} i~\texttt{pvvv} procedura wpisuje obliczone punkty
p"lata i~wektory pochodnych cz"astkowych rz"edu~$1$, $2$ i~$3$; je"sli kt"ory"s
z~tych parametr"ow ma warto"s"c \texttt{NULL}, to odpowiednia pochodna
nie jest tablicowana. W~przeciwnym razie wskazywana tablica musi mie"c
d"lugo"s"c co najmniej~$k^2d$.

Warto"sci"a procedury jest \texttt{true} w~razie sukcesu i~\texttt{false}
w~razie niepowodzenia oblicze"n (z~powodu braku miejsca na stosie
pami"eci pomocniczej).


\vspace{\bigskipamount}
\cprog{%
boolean mbs\_TabBezC2Coons0Der3f ( int spdimen, \\
\ind{8}int nknu, const float *knu, const float *hfuncu, \\
\ind{8}const float *dhfuncu, const float *ddhfuncu, \\
\ind{8}const float *dddhfuncu, \\
\ind{8}int nknv, const float *knv, const float *hfuncv, \\
\ind{8}const float *dhfuncv, const float *ddhfuncv, \\
\ind{8}const float *dddhfuncv, \\
\ind{8}int degc00, const float *c00, \\
\ind{8}int degc01, const float *c01, \\
\ind{8}int degc02, const float *c02, \\
\ind{8}int degd00, const float *d00, \\
\ind{8}int degd01, const float *d01, \\
\ind{8}int degd02, const float *d02, \\
\ind{8}float *p, float *pu, float *pv, float *puu, float *puv, \\
\ind{8}float *pvv, \\
\ind{8}float *puuu, float *puuv, float *puvv, float *pvvv );}
\begin{sloppypar}
Procedura \texttt{mbs\_TabBezC2Coons0Der3f} jest nieco uproszczon"a wersj"a
procedury \texttt{mbs\_TabBezC2CoonsDer3f} dla przypadku, gdy krzywe
$\bm{c}_{10},\bm{c}_{11},\bm{c}_{12},\bm{d}_{10},\bm{d}_{11}$, i~$\bm{d}_{12}$
s"a zerowe (tj.\ gdy wszystkie ich punkty kontrolne maj"a zerowe wszystkie
wsp"o"lrz"edne). Stablicowanie p"lata okre"slonego przez takie krzywe mo"re
by"c wykonane w~kr"otszym czasie; z~procedury tej korzysta biblioteka
\texttt{libg1hole}.%
\end{sloppypar}

Parametry procedury \texttt{mbs\_TabBezC2Coons0Der3f} s"a takie same,   
jak parametry proceury \texttt{mbs\_TabBezC2CoonsDer3f} o~tych samych
nazwach.


\newpage
\subsection{P"laty sklejane}

Sklejane p"laty Coonsa s"a okre"slone za pomoc"a krzywych B-sklejanych;
zar"owno stopnie poszczeg"olnych krzywych, jak i~ci"agi w"ez"l"ow u"ryte do
ich reprezentowania, mog"a by"c r"o"rne; krzywe $\bm{c}_{ij}$ musz"a mie"c
tylko wsp"oln"a dziedzin"e (wyznaczon"a przez w"ez"ly brzegowe w~ich
ci"agach w"ez"l"ow) i~to samo dotyczy krzywych $\bm{d}_{ij}$.

\vspace{\bigskipamount}
\cprog{%
void mbs\_BSC1CoonsFindCornersf ( int spdimen, \\
\ind{8}int degc00, int lastknotc00, const float *knotsc00, \\
\ind{8}const float *c00, \\
\ind{8}int degc01, int lastknotc01, const float *knotsc01, \\
\ind{8}const float *c01, \\
\ind{8}int degc10, int lastknotc10, const float *knotsc10, \\
\ind{8}const float *c10, \\
\ind{8}int degc11, int lastknotc11, const float *knotsc11, \\
\ind{8}const float *c11, \\
\ind{8}float *pcorners );}
\begin{sloppypar}
Procedura \texttt{mbs\_BSC1CoonsFindCornersf} wyznacza macierz~$\bm{P}$
o~wymiarach $4\times\nobreak 4$, kt"orej elementami s"a odpowiednie punkty
krzywych $\bm{c}_{00},\bm{c}_{10},\bm{c}_{01},\bm{c}_{11}$ i~wektory ich  
pochodnych.%
\end{sloppypar}

Parametry: \texttt{spdimen} --- wymiar~$d$ przestrzeni, w~kt"orej le"r"a
krzywe i~reprezentowany przez nie sklejany bikubiczny (klasy~$C^1$) 
p"lat Coonsa. Ka"rda czw"orka parametr"ow \texttt{degc??},
\texttt{lastknotc??}, \texttt{knotsc??} i~\texttt{c??}    
opisuje jedn"a z~krzywych, stopie"n, numer ostatniego w"ez"la, tablic"e
w"ez"l"ow i~tablic"e punkt"ow kontrolnych.    

Parametr \texttt{pcorners} jest wska"znikiem tablicy, w~kt"orej ma by"c
umieszczony wynik; tablica ta musi mie"c d"lugo"s"c~$16d$.


\vspace{\bigskipamount}
\cprog{%
boolean mbs\_BSC1CoonsToBSf ( int spdimen, \\
\ind{8}int degc00, int lastknotc00, const float *knotsc00, \\
\ind{8}const float *c00, \\
\ind{8}int degc01, int lastknotc01, const float *knotsc01, \\
\ind{8}const float *c01, \\
\ind{8}int degc10, int lastknotc10, const float *knotsc10, \\
\ind{8}const float *c10, \\
\ind{8}int degc11, int lastknotc11, const float *knotsc11, \\
\ind{8}const float *c11, \\
\ind{8}int degd00, int lastknotd00, const float *knotsd00, \\
\ind{8}const float *d00, \\
\ind{8}int degd01, int lastknotd01, const float *knotsd01, \\
\ind{8}const float *d01, \\
\ind{8}int degd10, int lastknotd10, const float *knotsd10, \\
\ind{8}const float *d10, \\
\ind{8}int degd11, int lastknotd11, const float *knotsd11, \\
\ind{8}const float *d11, \\
\ind{8}int *degreeu, int *lastuknot, float *uknots, \\
\ind{8}int *degreev, int *lastvknot, float *vknots, float *p );}
Procedura \texttt{mbs\_BSC1CoonsToBSf} wyznacza reprezentacj"e B-sklejan"a
bikubicznego p"lata Coonsa (klasy~$C^1$) okre"slonego przez dane krzywe
sklejane. Warto"sci"a procedury jest \texttt{true}, je"sli obliczenie
zako"nczy"lo si"e sukcesem, a~\texttt{false} w~przeciwnym razie (przyczyn"a
niepowodzenia mo"re by"c brak pami"eci na stosie pami"eci pomocniczej
lub niepoprawne ci"agi w"ez"l"ow w~reprezentacjach krzywych).

Warto"s"c parametru \texttt{spdimen} jest wymiarem~$d$ przestrzeni, w~kt"orej
le"r"a krzywe i~okre"slony przez nie p"lat. Ka"rda czw"orka parametr"ow
\texttt{degc??}, \texttt{lastknotc??}, \texttt{knotsc??} i~\texttt{c??}
opisuje odpowiedni"a krzyw"a z~rodziny
$\bm{c}_{00},\bm{c}_{01},\bm{c}_{10},\bm{c}_{11}$, przez podanie stopnia
numeru ostatniego w"ez"la, ci"agu w"ez"l"ow i~tablicy punkt"ow kontrolnych.
Kolejne pary parametr"ow \texttt{degd??}, \texttt{lastknotd??},
\texttt{knotsd??} i~\texttt{d??} opisuj"a w~ten sam spos"ob krzywe z~rodziny
$\bm{d}_{00},\bm{d}_{01},\bm{d}_{10},\bm{d}_{11}$. Krzywe te musz"a
spe"lnia"c (z~dok"ladno"sci"a do b"l"ed"ow zaokr"agle"n) warunki
zgodno"sci~(\ref{eq:Coons:compat:cond}), co \emph{nie jest} sprawdzane.

Zmienne \texttt{*n} i~\texttt{*m} otrzymuj"a warto"sci opisuj"ace stopie"n
reprezentacji B-sklejanej p"lata. Warto"s"c~$n$ zmiennej \texttt{*n} jest
najwi"eksza z~warto"sci parametr"ow \texttt{degc??} lub~$3$ (je"sli liczba~$3$
jest wi"eksza). Podobnie, warto"s"c~$m$ zmiennej \texttt{*m} jest najwi"eksz"a
z~warto"sci parametr"ow \texttt{degd??} lub~$3$. Parametry
\texttt{lastuknot}, \texttt{uknots}, \texttt{lastvknot} i~\texttt{vknots}
s"lu"r"a do wyprowadzenia ci"ag"ow w"ez"l"ow skonstruowanej reprezentacji
B-sklejanej p"lata Coonsa. W~tablicy wskazywanej przez
parametr \texttt{p} procedura umieszcza punkty kontrolne.

Tablice \texttt{unkots}, \texttt{vknots} i~\texttt{p} musz"a by"c
dostatecznie pojemne; aby zarezerwowa"c te tablice, mo"rna pos"lu"ry"c si"e
procedur"a \texttt{mbs\_FindBSCommonKnotSequencef} dla rodzin krzywych
$\bm{c}_{ij}$ oraz $\bm{d}_{ij}$; zmienne wskazywane przez parametr
\texttt{lastknot} tej procedury powinny mie"c warto"sci pocz"atkowe~$3$.


\vspace{\bigskipamount}
\cprog{%
boolean mbs\_TabBSC1CoonsDer2f ( int spdimen, \\
\ind{8}int nknu, const float *knu, const float *hfuncu, \\
\ind{8}const float *dhfuncu, const float *ddhfuncu, \\
\ind{8}int nknv, const float *knv, const float *hfuncv, \\
\ind{8}const float *dhfuncv, const float *ddhfuncv, \\
\ind{8}int degc00, int lastknotc00, const float *knotsc00, \\
\ind{8}const float *c00, \\
\ind{8}int degc01, int lastknotc01, const float *knotsc01, \\
\ind{8}const float *c01, \\
\ind{8}int degc10, int lastknotc10, const float *knotsc10, \\
\ind{8}const float *c10, \\
\ind{8}int degc11, int lastknotc11, const float *knotsc11, \\
\ind{8}const float *c11, \\
\ind{8}int degd00, int lastknotd00, const float *knotsd00, \\
\ind{8}const float *d00, \\
\ind{8}int degd01, int lastknotd01, const float *knotsd01, \\
\ind{8}const float *d01, \\
\ind{8}int degd10, int lastknotd10, const float *knotsd10, \\
\ind{8}const float *d10, \\
\ind{8}int degd11, int lastknotd11, const float *knotsd11, \\
\ind{8}const float *d11, \\
\ind{8}float *p, float *pu, float *pv, \\
\ind{8}float *puu, float *puv, float *pvv );}
Procedura \texttt{mbs\_TabBSC1CoonsDer2f} s"lu"ry do szybkiego stablicowania
bikubicznego sklejanego p"lata Coonsa, razem z~pochodnymi cz"astkowymi
rz"edu~$1$ i~$2$, dla punkt"ow $(u_i,v_j)$, gdzie $i\in\{0,\ldots,k_u-1\}$,
$j\in\{0,\ldots,k_v-1\}$.

Parametr~\texttt{spdimen} okre"sla wymiar~$d$ przestrzeni, w~kt"orej jest
p"lat. Pa\-ra\-met\-ry~\texttt{nknu} i~\texttt{nknv} okre"slaj"a liczby~$k_u$
i~$k_v$, tablice~\texttt{knu} i~\texttt{knv}
zawierj"a odpowiednio liczby $u_0,\ldots,u_{k_u-1}$ i~$v_0,\ldots,v_{k_v-1}$.
Tablice \texttt{hfuncu}, \texttt{dhfuncu},
\texttt{ddhfuncu} zawieraj"a odpowiednio warto"sci wielomian"ow
$\tilde{H}_{00},\tilde{H}_{10},\tilde{H}_{01},\tilde{H}_{11}$
i~ich pochodnych rz"edu~$1$ i~$2$ w~punktach
$u_0,\ldots,u_{k_u-1}$; warto"sci te najpro"sciej jest obliczy"c
wywo"luj"ac zawczasu procedur"e \texttt{mbs\_TabCubicHFuncDer2f}
z~parametrami \texttt{a}, \texttt{b} o~warto"sciach b"ed"acych ko"ncami
przedzia"lu zmienno"sci parametru~$u$ p"lata. Podobnie, tablice
\texttt{hfuncv}, \texttt{dhfuncv}, \texttt{ddhfuncv} zawieraj"a
warto"sci wielomian"ow $\hat{H}_{00},\hat{H}_{10},\hat{H}_{01},\hat{H}_{11}$  
i~ich pochodnych rz"edu~$1$ i~$2$ w~punktach $v_0,\ldots,v_{k_v-1}$

Czw"orki parametr"ow \texttt{degc??}, \texttt{lastknotc??}, \texttt{knotsc??}
i~\texttt{c??} oraz \texttt{degd??}, \texttt{lastknotd??}, \texttt{knotsd??}
i~\texttt{d??} opisuj"a krzywe B-sklejane okre"slaj"ace p"lat.
Krzywe te musz"a spe"lnia"c (z~dok"ladno"sci"a do b"l"ed"ow zaokr"agle"n)
warunki zgodno"sci~(\ref{eq:Coons:compat:cond}).

Do tablic wskazywanych przez parametry \texttt{p}, \texttt{pu}, \texttt{pv},
\texttt{puu}, \texttt{puv}, \texttt{pvv} procedura wpisuje obliczone punkty
p"lata i~wektory pochodnych cz"astkowych rz"edu~$1$ i~$2$; je"sli kt"ory"s
z~tych parametr"ow ma warto"s"c \texttt{NULL}, to odpowiednia pochodna
nie jest tablicowana. W~przeciwnym razie wskazywana tablica musi mie"c
d"lugo"s"c co najmniej~$k_uk_vd$.

Warto"sci"a procedury jest \texttt{true} w~razie sukcesu i~\texttt{false}
w~razie niepowodzenia oblicze"n (z~powodu braku miejsca na stosie
pami"eci pomocniczej).


\vspace{\bigskipamount}
\cprog{%
boolean mbs\_TabBSC1Coons0Der2f ( int spdimen, \\
\ind{8}int nknu, const float *knu, const float *hfuncu, \\
\ind{8}const float *dhfuncu, const float *ddhfuncu, \\
\ind{8}int nknv, const float *knv, const float *hfuncv, \\
\ind{8}const float *dhfuncv, const float *ddhfuncv, \\
\ind{8}int degc00, int lastknotc00, const float *knotsc00, \\
\ind{8}const float *c00, \\
\ind{8}int degc01, int lastknotc01, const float *knotsc01, \\
\ind{8}const float *c01, \\
\ind{8}int degd00, int lastknotd00, const float *knotsd00, \\
\ind{8}const float *d00, \\
\ind{8}int degd01, int lastknotd01, const float *knotsd01, \\
\ind{8}const float *d01, \\
\ind{8}float *p, float *pu, float *pv, \\
\ind{8}float *puu, float *puv, float *pvv );}
\begin{sloppypar}
Procedura \texttt{mbs\_TabBSC1Coons0Der2f} jest nieco uproszczon"a wersj"a
procedury \texttt{mbs\_TabBSC1CoonsDer2f} dla przypadku, gdy krzywe
$\bm{c}_{10},\bm{c}_{11},\bm{d}_{10}$ i~$\bm{d}_{11}$ s"a zerowe
(tj.\ gdy wszystkie ich punkty kontrolne maj"a zerowe wszystkie
wsp"o"lrz"edne). Stablicowanie p"lata okre"slonego przez takie krzywe mo"re
by"c wykonane w~kr"otszym czasie.%
\end{sloppypar}

Parametry procedury \texttt{mbs\_TabBSC1Coons0Der2f} s"a takie same,
jak parametry proceury \texttt{mbs\_TabBSC1CoonsDer2f} o~tych samych
nazwach.


\vspace{\bigskipamount}
\cprog{%
void mbs\_BSC2CoonsFindCornersf ( int spdimen, \\
\ind{8}int degc00, int lastknotc00, const float *knotsc00, \\
\ind{8}const float *c00, \\
\ind{8}int degc01, int lastknotc01, const float *knotsc01, \\
\ind{8}const float *c01, \\
\ind{8}int degc02, int lastknotc02, const float *knotsc02, \\
\ind{8}const float *c02, \\
\ind{8}int degc10, int lastknotc10, const float *knotsc10, \\
\ind{8}const float *c10, \\
\ind{8}int degc11, int lastknotc11, const float *knotsc11, \\
\ind{8}const float *c11, \\
\ind{8}int degc12, int lastknotc12, const float *knotsc12, \\
\ind{8}const float *c12, \\
\ind{8}float *pcorners );}
\begin{sloppypar}
Procedura \texttt{mbs\_BSC2CoonsFindCornersf} wyznacza macierz~$\bm{P}$
o~wymiarach $6\times\nobreak 6$, kt"orej elementami s"a odpowiednie punkty
krzywych
$\bm{c}_{00},\bm{c}_{10},\bm{c}_{01},\bm{c}_{11},\bm{c}_{02},\bm{c}_{12}$
i~wektory ich pochodnych.%
\end{sloppypar}

Parametry: \texttt{spdimen} --- wymiar~$d$ przestrzeni, w~kt"orej le"r"a
krzywe i~reprezentowany przez nie sklejany dwupi"etny (klasy~$C^2$) 
p"lat Coonsa. Ka"rda czw"orka parametr"ow \texttt{degc??},
\texttt{lastknotc??}, \texttt{knotsc??} i~\texttt{c??}    
opisuje jedn"a z~krzywych, stopie"n, numer ostatniego w"ez"la, tablic"e
w"ez"l"ow i~tablic"e punkt"ow kontrolnych.    

Parametr \texttt{pcorners} jest wska"znikiem tablicy, w~kt"orej ma by"c
umieszczony wynik; tablica ta musi mie"c d"lugo"s"c~$36d$.


\vspace{\bigskipamount}
\cprog{%
boolean mbs\_BSC2CoonsToBSf ( int spdimen, \\
\ind{8}int degc00, int lastknotc00, const float *knotsc00, \\
\ind{8}const float *c00, \\
\ind{8}int degc01, int lastknotc01, const float *knotsc01, \\
\ind{8}const float *c01, \\
\ind{8}Aint degc02, int lastknotc02, const float *knotsc02, \\
\ind{8}const float *c02, \\
\ind{8}int degc10, int lastknotc10, const float *knotsc10, \\
\ind{8}const float *c10, \\
\ind{8}int degc11, int lastknotc11, const float *knotsc11, \\
\ind{8}const float *c11, \\
\ind{8}int degc12, int lastknotc12, const float *knotsc12, \\
\ind{8}const float *c12, \\
\ind{8}int degd00, int lastknotd00, const float *knotsd00, \\
\ind{8}const float *d00, \\
\ind{8}int degd01, int lastknotd01, const float *knotsd01, \\
\ind{8}const float *d01, \\
\ind{8}int degd02, int lastknotd02, const float *knotsd02, \\
\ind{8}const float *d02, \\
\ind{8}int degd10, int lastknotd10, const float *knotsd10, \\
\ind{8}const float *d10, \\
\ind{8}int degd11, int lastknotd11, const float *knotsd11, \\
\ind{8}const float *d11, \\
\ind{8}int degd12, int lastknotd12, const float *knotsd12, \\
\ind{8}const float *d12, \\
\ind{8}int *degreeu, int *lastuknot, float *uknots, \\
\ind{8}int *degreev, int *lastvknot, float *vknots, float *p );}
Procedura \texttt{mbs\_BSC2CoonsToBSf} wyznacza reprezentacj"e B-sklejan"a
dwupi"etnego p"lata Coonsa (klasy~$C^2$) okre"slonego przez dane krzywe
sklejane. Warto"sci"a procedury jest \texttt{true}, je"sli obliczenie
zako"nczy"lo si"e sukcesem, a~\texttt{false} w~przeciwnym razie (przyczyn"a
niepowodzenia mo"re by"c brak pami"eci na stosie pami"eci pomocniczej
lub niepoprawne ci"agi w"ez"l"ow w~reprezentacjach krzywych).

Warto"s"c parametru \texttt{spdimen} jest wymiarem~$d$ przestrzeni, w~kt"orej
le"r"a krzywe i~okre"slony przez nie p"lat. Ka"rda czw"orka parametr"ow
\texttt{degc??}, \texttt{lastknotc??}, \texttt{knotsc??} i~\texttt{c??}
opisuje odpowiedni"a krzyw"a z~rodziny
$\bm{c}_{00},\bm{c}_{01},\bm{c}_{02},\bm{c}_{10},\bm{c}_{11},\bm{c}_{12}$,
przez podanie stopnia
numeru ostatniego w"ez"la, ci"agu w"ez"l"ow i~tablicy punkt"ow kontrolnych.
Kolejne pary parametr"ow \texttt{degd??}, \texttt{lastknotd??},
\texttt{knotsd??} i~\texttt{d??} opisuj"a w~ten sam spos"ob krzywe z~rodziny
$\bm{d}_{00},\bm{d}_{01},\bm{d}_{02},\bm{d}_{10},\bm{d}_{11},\bm{d}_{12}$.
Krzywe te musz"a spe"lnia"c (z~dok"ladno"sci"a do b"l"ed"ow zaokr"agle"n) warunki
zgodno"sci~(\ref{eq:Coons:compat:cond}), co \emph{nie jest} sprawdzane.

Zmienne \texttt{*n} i~\texttt{*m} otrzymuj"a warto"sci opisuj"ace stopie"n
reprezentacji B-sklejanej p"lata. Warto"s"c~$n$ zmiennej \texttt{*n} jest
najwi"eksza z~warto"sci parametr"ow \texttt{degc??} lub~$5$ (je"sli
liczba~$5$ jest wi"eksza). Podobnie, warto"s"c~$m$ zmiennej \texttt{*m}
jest najwi"eksz"a z~warto"sci parametr"ow \texttt{degd??} lub~$5$. Parametry
\texttt{lastuknot}, \texttt{uknots}, \texttt{lastvknot} i~\texttt{vknots}
s"lu"r"a do wyprowadzenia ci"ag"ow w"ez"l"ow skonstruowanej reprezentacji
B-sklejanej p"lata Coonsa. W~tablicy wskazywanej przez
parametr \texttt{p} procedura umieszcza punkty kontrolne.

Tablice \texttt{unkots}, \texttt{vknots} i~\texttt{p} musz"a by"c
dostatecznie pojemne; aby zarezerwowa"c te tablice, mo"rna pos"lu"ry"c si"e
procedur"a \texttt{mbs\_FindBSCommonKnotSequencef} dla rodzin krzywych
$\bm{c}_{ij}$ oraz $\bm{d}_{ij}$; zmienne wskazywane przez parametr
\texttt{lastknot} tej procedury powinny mie"c warto"sci pocz"atkowe~$5$.


\vspace{\bigskipamount}
\cprog{%
boolean mbs\_TabBSC2CoonsDer3f ( int spdimen, \\
\ind{8}int nknu, const float *knu, const float *hfuncu, \\
\ind{8}const float *dhfuncu, const float *ddhfuncu, \\
\ind{8}const float *dddhfuncu, \\
\ind{8}int nknv, const float *knv, const float *hfuncv, \\
\ind{8}const float *dhfuncv, const float *ddhfuncv, \\
\ind{8}const float *dddhfuncv, \\
\ind{8}int degc00, int lastknotc00, const float *knotsc00, \\
\ind{8}const float *c00, \\
\ind{8}int degc01, int lastknotc01, const float *knotsc01, \\
\ind{8}const float *c01, \\
\ind{8}int degc02, int lastknotc02, const float *knotsc02, \\
\ind{8}const float *c02, \\
\ind{8}int degc10, int lastknotc10, const float *knotsc10, \\
\ind{8}const float *c10, \\
\ind{8}int degc11, int lastknotc11, const float *knotsc11, \\
\ind{8}const float *c11, \\
\ind{8}int degc12, int lastknotc12, const float *knotsc12, \\
\ind{8}const float *c12, \\
\ind{8}int degd00, int lastknotd00, const float *knotsd00, \\
\ind{8}const float *d00, \\
\ind{8}int degd01, int lastknotd01, const float *knotsd01, \\
\ind{8}const float *d01, \\
\ind{8}int degd02, int lastknotd02, const float *knotsd02, \\
\ind{8}const float *d02, \\
\ind{8}int degd10, int lastknotd10, const float *knotsd10, \\
\ind{8}const float *d10, \\
\ind{8}int degd11, int lastknotd11, const float *knotsd11, \\
\ind{8}const float *d11, \\
\ind{8}int degd12, int lastknotd12, const float *knotsd12, \\
\ind{8}const float *d12, \\
\ind{8}float *p, float *pu, float *pv, \\
\ind{8}float *puu, float *puv, float *pvv, \\
\ind{8}float *puuu, float *puuv, float *puvv, float *pvvv );}
Procedura \texttt{mbs\_TabBSC2CoonsDer3f} s"lu"ry do szybkiego stablicowania
dwupi"etnego sklejanego p"lata Coonsa, razem z~pochodnymi cz"astkowymi
rz"edu~$1$, $2$ i~$3$, dla punkt"ow $(u_i,v_j)$, gdzie $i\in\{0,\ldots,k_u-1\}$,
$j\in\{0,\ldots,k_v-1\}$.

Parametr~\texttt{spdimen} okre"sla wymiar~$d$ przestrzeni, w~kt"orej jest
p"lat. Pa\-ra\-met\-ry~\texttt{nknu} i~\texttt{nknv} okre"slaj"a liczby~$k_u$
i~$k_v$, tablice~\texttt{knu} i~\texttt{knv}
zawierj"a odpowiednio liczby $u_0,\ldots,u_{k_u-1}$ i~$v_0,\ldots,v_{k_v-1}$.
Tablice \texttt{hfuncu}, \texttt{dhfuncu}, \texttt{ddhfuncu}, \texttt{dddhfuncu}
zawieraj"a odpowiednio warto"sci wielomian"ow
$\tilde{H}_{00},\tilde{H}_{10},\tilde{H}_{01},\tilde{H}_{11},%
\tilde{H}_{02},\tilde{H}_{12}$ i~ich pochodnych rz"edu~$1$, $2$ i~$3$
w~punktach $u_0,\ldots,u_{k_u-1}$; warto"sci te najpro"sciej jest obliczy"c
wywo"luj"ac zawczasu procedur"e \texttt{mbs\_TabQuinticHFuncDer3f}
z~parametrami \texttt{a}, \texttt{b} o~warto"sciach b"ed"acych ko"ncami
przedzia"lu zmienno"sci parametru~$u$ p"lata. Podobnie, tablice
\texttt{hfuncv}, \texttt{dhfuncv}, \texttt{ddhfuncv}, \texttt{dddhfuncv}
zawieraj"a warto"sci wielomian"ow $\hat{H}_{00},\hat{H}_{10},%
\hat{H}_{01},\hat{H}_{11},\hat{H}_{02},\hat{H}_{12}$
i~ich pochodnych rz"edu~$1$, $2$ i~$3$ w~punktach $v_0,\ldots,v_{k_v-1}$

Czw"orki parametr"ow \texttt{degc??}, \texttt{lastknotc??}, \texttt{knotsc??}
i~\texttt{c??} oraz \texttt{degd??}, \texttt{lastknotd??}, \texttt{knotsd??}
i~\texttt{d??} opisuj"a krzywe B-sklejane okre"slaj"ace p"lat.
Krzywe te musz"a spe"lnia"c (z~dok"ladno"sci"a do b"l"ed"ow zaokr"agle"n)
warunki zgodno"sci~(\ref{eq:Coons:compat:cond}).

Do tablic wskazywanych przez parametry \texttt{p}, \texttt{pu}, \texttt{pv},
\texttt{puu}, \texttt{puv}, \texttt{pvv}, \texttt{puuu}, \texttt{puuv},
\texttt{puvv}, \texttt{pvvv} procedura wpisuje obliczone punkty
p"lata i~wektory pochodnych cz"astkowych rz"edu~$1$, $2$ i~$3$; je"sli kt"ory"s
z~tych parametr"ow ma warto"s"c \texttt{NULL}, to odpowiednia pochodna
nie jest tablicowana. W~przeciwnym razie wskazywana tablica musi mie"c
d"lugo"s"c co najmniej~$k_uk_vd$.

Warto"sci"a procedury jest \texttt{true} w~razie sukcesu i~\texttt{false}
w~razie niepowodzenia oblicze"n (z~powodu braku miejsca na stosie
pami"eci pomocniczej).


\vspace{\bigskipamount}
\cprog{%
boolean mbs\_TabBSC2Coons0Der3f ( int spdimen, \\
\ind{8}int nknu, const float *knu, const float *hfuncu, \\
\ind{8}const float *dhfuncu, const float *ddhfuncu, \\
\ind{8}const float *dddhfuncu, \\
\ind{8}int nknv, const float *knv, const float *hfuncv, \\
\ind{8}const float *dhfuncv, const float *ddhfuncv, \\
\ind{8}const float *dddhfuncv, \\
\ind{8}int degc00, int lastknotc00, const float *knotsc00, \\
\ind{8}const float *c00, \\
\ind{8}int degc01, int lastknotc01, const float *knotsc01, \\
\ind{8}const float *c01, \\
\ind{8}int degc02, int lastknotc02, const float *knotsc02, \\
\ind{8}const float *c02, \\
\ind{8}int degd00, int lastknotd00, const float *knotsd00, \\
\ind{8}const float *d00, \\
\ind{8}int degd01, int lastknotd01, const float *knotsd01, \\
\ind{8}const float *d01, \\
\ind{8}int degd02, int lastknotd02, const float *knotsd02, \\
\ind{8}const float *d02, \\
\ind{8}float *p, float *pu, float *pv, \\
\ind{8}float *puu, float *puv, float *pvv, \\
\ind{8}float *puuu, float *puuv, float *puvv, float *pvvv );}
\begin{sloppypar}
Procedura \texttt{mbs\_TabBSC2Coons0Der3f} jest nieco uproszczon"a wersj"a
procedury \texttt{mbs\_TabBSC2CoonsDer3f} dla przypadku, gdy krzywe
$\bm{c}_{10},\bm{c}_{11},\bm{c}_{12},\bm{d}_{10},\bm{d}_{11}$ i~$\bm{d}_{12}$
s"a zerowe (tj.\ gdy wszystkie ich punkty kontrolne maj"a zerowe wszystkie
wsp"o"lrz"edne). Stablicowanie p"lata okre"slonego przez takie krzywe mo"re
by"c wykonane w~kr"otszym czasie.%
\end{sloppypar}

Parametry procedury \texttt{mbs\_TabBSC2Coons0Der3f} s"a takie same,
jak parametry proceury \texttt{mbs\_TabBSC2CoonsDer3f} o~tych samych
nazwach.


\newpage
\section{Produkt sferyczny}

\begin{sloppypar}
Produkt sferyczny p"laskich krzywych parametrycznych,
$\bm{p}(t)=[x_{\bm{p}}(t),y_{\bm{p}}(t)]^T$
i~$\bm{q}(t)=[x_{\bm{q}}(t),y_{\bm{q}}(t)]^T$,
jest powierzchni"a parametryczn"a w~$\R^3$, dan"a wzorem
\begin{align*}
  \bm{s}(u,v) = \left[\begin{array}{c}
    x_{\bm{p}}(u)x_{\bm{q}}(v) \\
    y_{\bm{p}}(u)x_{\bm{q}}(v) \\
    y_{\bm{q}}(v)
  \end{array}\right].
\end{align*}
Krzywe~$\bm{p}$ i~$\bm{q}$ s"a zwane odpowiednio r"ownikiem (\textsl{equator})
i~po"ludnikiem (\textsl{meridian}). Procedury opisane ni"rej obliczaj"a
punkty kontrolne reprezentacji B-sklejanej produktu sferycznego p"laskich
krzywych B-sklejanych, odpowiednio kawa"lkami wielomianowej i~wymiernej.%
\end{sloppypar}

Ci"ag w"ez"l"ow r"ownika jest ci"agiem ,,$u$'' produktu sferycznego,
a~ci"ag w"ez"l"ow po"ludnika jest ci"agiem ,,$v$''.

\vspace{\bigskipamount}
\cprog{%
void mbs\_SphericalProductf ( \\
\ind{6}int degree\_eq, int lastknot\_eq, const point2f *cpoints\_eq, \\
\ind{6}int degree\_mer, int lastknot\_mer, const point2f *cpoints\_mer, \\
\ind{6}int pitch, point3f *spr\_cp );}

\vspace{\bigskipamount}
\cprog{%
void mbs\_SphericalProductRf ( \\
\ind{6}int degree\_eq, int lastknot\_eq, const point3f *cpoints\_eq, \\
\ind{6}int degree\_mer, int lastknot\_mer, const point3f *cpoints\_mer, \\
\ind{6}int pitch, point4f *spr\_cp );}

