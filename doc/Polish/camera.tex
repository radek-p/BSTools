
%/* //////////////////////////////////////////////////// */
%/* This file is a part of the BSTools procedure package */
%/* written by Przemyslaw Kiciak.                        */
%/* //////////////////////////////////////////////////// */

\chapter{Biblioteka \texttt{libcamera}}

Biblioteka \texttt{libcamera} zawiera procedury obs"luguj"ace kamery, czyli
obiekty, kt"ore s"lu"r"a do odwzorowania (rzutowania) punkt"ow przestrzeni
tr"ojwymiarowej na p"laszczyzn"e, oczywi"scie w~celu wykonywania obrazk"ow.
S"a dwa rodzaje rzut"ow: perspektywiczne i r"ownoleg"le. Te pierwsze s"lu"r"a
do wykonywania ,,fotografii'', a~te drugie do ,,rysunk"ow technicznych''.


\section{Kamera}

\subsection{Opis kamery i algorytm rzutowania}

Struktura danych i~nag"l"owki procedur obs"luguj"acych rzutowanie
s"a opisane w~plikach \texttt{cameraf.h} i~\texttt{camerad.h}.
Oba pliki mo"rna w"l"aczy"c do programu za po"srednictwem pliku
\texttt{camera.h}.

\vspace{\bigskipamount}
\cprog{%
typedef struct CameraRecf \{ \\
\ind{2}boolean parallel, upside, c\_fixed; \\
\ind{2}byte magnification; \\
\ind{2}short xmin, ymin, width, height; \\
\ind{2}float aspect; \\
\ind{2}point3f position; \\
\ind{2}float psi, theta, phi; \\
\ind{2}point3f g\_centre, c\_centre; \\
\ind{2}float xscale, yscale; \\
\ind{2}trans3f CTr, CTrInv; \\
\ind{2}vector4f cplane[6]; \\
\ind{2}union \{ \\
\ind{4}struct \{ \\
\ind{6}float f; \\
\ind{6}float xi0, eta0; \\
\ind{6}float dxi0, deta0; \\
\ind{4}\} persp; \\
\ind{4}struct \{ \\
\ind{6}float   wdt, hgh, diag; \\
\ind{6}boolean dim\_case; \\
\ind{4}\} para; \\
\ind{2}\} vd; \\
\} CameraRecf;}
Struktura \texttt{CameraRecf} opisuje kamer"e, tj.\ obiekt,
kt"ory okre"s\-la rzutowanie perspektywiczne lub r"ownoleg"le na
p"laszczyzn"e.%
\begin{figure}[ht]
  \centerline{\epsfig{file=camera.ps}}
  \caption{Uk"lad wsp"o"lrz"ednych kamery i klatka dla rzutu perspektywicznego}
\end{figure}

Pola struktury maj"a nast"epuj"ace znaczenie:
\begin{mydescription}
  \item[]\texttt{parallel} --- je"sli pole to ma warto"s"c \texttt{false}
    (\texttt{0}), to rzutowanie jest perspektywiczne, a~w~przeciwnym razie
    r"ownoleg"le.
  \item[]\texttt{upside} --- je"sli pole to ma warto"s"c \texttt{false},
    to o"s~$y$ uk"ladu wsp"o"lrz"ednych na obrazie jest zorientowana
    do do"lu (tak jak w~oknach systemu XWindow), a~w~przeciwnym razie --- do
    g"ory (tak jak w~bibliotece OpenGL i~w~domy"slnym uk"ladzie w~j"ezyku
    PostScript).
  \item[]\texttt{c\_fixed} --- parametr okre"slaj"acy zmiany "srodka obrotu
    kamery przy przesuwaniu. Warto"s"c \texttt{false} oznacza, "re punkt ten
    jest nieruchomy w~uk"ladzie globalnym, za"s \texttt{true} oznacza, "re
    jest nieruchomy w~uk"ladzie kamery.
  \item[]\texttt{magnification} --- domy"slnie pole to ma warto"s"c~$1$,
    co oznacza, "re jednostki osi w~uk"ladzie klatki s"a r"owne szeroko"sci
    i~wysoko"sci piksela. Je"sli pole to ma wi"eksz"a warto"s"c, to
    jednostki s"a o~ten czynnik kr"otsze, co mo"re by"c u"ryteczne,
    je"sli obraz ma by"c wykonany z~nadpr"obkowaniem.
  \item[]\texttt{xmin}, \texttt{ymin}, \texttt{width}, \texttt{height} ---
    wsp"o"lrz"edne g"ornego lewego rogu klatki oraz jej szeroko"s"c
    i~wysoko"s"c w~pikselach.
  \item[]\texttt{aspect} --- wsp"o"lczynnik aspekt rastra, tj.\ iloraz
    szeroko"sci i~wysoko"sci piksela.
  \item[]\texttt{position} --- po"lo"renie obserwatora (wsp"o"lrz"edne
    w~uk"ladzie globalnym).
  \item[]\texttt{psi}, \texttt{theta},  \texttt{phi} --- k"aty Eulera
    $\psi$, $\vartheta$, $\varphi$ okre"slaj"ace po"lo"renie kamery (tj.\
    kierunek).
  \item[]\texttt{g\_centre}, \texttt{c\_centre} --- wsp"o"lrz"edne "srodka
    obrotu kamery, tj.\ punktu, kt"ory le"ry na osi obrotu kamery podczas
    obracania jej.
  \item[]\texttt{xscale}, \texttt{yscale} --- wsp"o"lczynniki skalowania osi
    $x$~i~$y$ uk"ladu kamery.
  \item[]\texttt{Ctr}, \texttt{CTrInv} --- przekszta"lcenia opisuj"ace
    przej"scia od uk"ladu globalnego do uk"ladu kamery i~odwrotne.
  \item[]\texttt{cplane} --- reprezentacje czterech p"o"lprzestrzeni,
    kt"orych przeci"eciem jest ostros"lup widzenia. P"o"lprzestrze"n
    $ax+by+cz+d>0$ jest reprezentowana przez wektor
    o~wsp"o"lrz"ednych $a$, $b$, $c$, $d$.

    Na razie s"a oprogramowane cztery p"o"lprzestrzenie, docelowo ma
    by"c~$6$.
  \item[]\texttt{vd} --- unia, kt"ora zawiera dane specyficzne dla rodzaju
    rzutowania. Struktura \texttt{vd.persp} zawiera dane dla rzut"ow
    perspektywicznych, \texttt{vd.para} dla r"ownoleg"lych.
  \item[]\texttt{vd.persp.f} --- d"lugo"s"c ogniskowej kamery, mierzona
    w~jednostkach, w~kt"orych przek"atna klatki ma d"lugo"s"c~$1$.
    D"lugo"s"c~$1$ ogniskowej odpowiada obiektywowi standardowemu.
  \item[]\texttt{vd.persp.xi0}, \texttt{vd.persp.eta0} --- przesuni"ecie
    pikseli po rzutowaniu perspektywicznym.
  \item[]\texttt{vd.persp.dxi0}, \texttt{vd.persp.deta0} --- wsp"o"lrz"edne
    $x$, $y$ (w~uk"ladzie kamery) "srodka klatki. Domy"slnie maj"a
    warto"s"c~$0$, co oznacza, "re "srodek klatki le"ry na ,,osi optycznej''
    kamery. Nadanie innej warto"sci jest potrzebne w~programie obs"luguj"acym
    par"e kamer w~celu tworzenia obraz"ow stereoskopowych.
  \item[]\texttt{vd.para.wdt}, \texttt{vd.para.hgh}, \texttt{vd.para.diag}
    --- wymiary (szeroko"s"c, wysoko"s"c i~d"lugo"s"c przek"atnej) klatki
    mierzona w~jednostkach uk"ladu globalnego.
  \item[]\texttt{vd.para.dim\_case} --- parametr okre"slaj"acy, kt"ory
    z~powy"zszych trzech wymiar"ow klatki jest podany przez u"rytkownika;
    $0$ --- d"lugo"s"c przek"atnej, $1$ --- szeroko"s"c, $2$ --- wysoko"s"c.
    Na podstawie jednego z~tych wymiar"ow podczas obliczania macierzy
    przekszta"lcenia zostan"a obliczone pozosta"le dwa.

    Domy"slnie pole to otrzymuje warto"s"c~$0$, a~pole \texttt{vd.para.diag}
    warto"s"c~$1$.
\end{mydescription}

\subsubsection*{Algorytm rzutowania:}

Obraz punktu $\bm{p}$ reprezentowanego za pomoc"a wsp"o"lrz"ednych
w~uk"ladzie globalnym jest obliczany tak:
\begin{enumerate}
  \item\begin{sloppypar}
    Punkt $\bm{p}$ jest poddawany przekszta"lceniu afinicznemu
    reprezentowanemu przez pole \texttt{CTr}. Krok ten jest przej"sciem do
    uk"ladu wsp"o"lrz"ednych kamery.%
    \end{sloppypar}
  \item Je"sli rzutowanie jest perspektywiczne, to obliczone wsp"o"lrz"edne
    $x$ i $y$ w uk"ladzie kamery s"a dzielone przez wsp"o"lrz"edn"a $z$, co
    jest w"la"sciwym rzutowaniem perspektywicznym, a~nast"epnie obliczane
    s"a wsp"o"lrz"edne $x$, $y$ punktu na obrazie, przez dodanie
    warto"sci p"ol \texttt{xi0} i~\texttt{eta0}.

    Je"sli rzutowanie jest r"ownoleg"le, to ten krok oblicze"n jest pomijany.
  \item Je"sli o"s~$y$ na obrazie jest zorientowana do g"ory (pole
   \texttt{upside} ma warto"s"c niezerow"a), to wsp"o"lrz"edna~$y$
   jest zast"epowana przez $2y_{\mathrm{min}}+h-y$, gdzie $y_{\mathrm{min}}$
   jest warto"sci"a pola \texttt{ymin}, a~$h$ jest warto"sci"a pola
   \texttt{height}.
\end{enumerate}


\subsubsection*{Okre"slanie przej"scia do uk"ladu kamery:}

Przekszta"lcenie, kt"ore opisuje przej"scie od globalnego uk"ladu
wsp"o"lrz"ednych do uk"ladu kamery (wykonywane w~pierwszym kroku rzutowania
opisanego wy"rej) jest z"lo"reniem trzech przekszta"lce"n afinicznych:
\begin{enumerate}
  \item Skalowania osi $x$ i~$y$ o~czynniki b"ed"ace warto"sciami
    odpowiednio p"ol \texttt{xscale} i~\texttt{yscale}.
  \item Obrotu opisanego za pomoc"a k"at"ow Eulera, b"ed"acych warto"sciami
    p"ol \texttt{psi}, \texttt{theta}, \texttt{phi}.
  \item Przesuni"ecia, kt"ore umieszcza pocz"atek uk"ladu w~punkcie
    okre"slonym przez pole \texttt{position}.
\end{enumerate}
Nadawanie warto"sci powy"rszych p"ol powinno odbywa"c si"e w~zasadzie
wy"l"acznie za~pomoc"a opisanych dalej procedur, kt"ore w~szczeg"olno"sci
obliczaj"a po"lo"renie kamery jako wynik ci"agu jej przemieszcze"n od
domy"slnego po"lo"renia pocz"atkowego.


\subsection{Procedury obs"lugi kamery}

\cprog{%
void CameraInitFramef ( CameraRecf *CPos, \\
\ind{17}boolean parallel, boolean upside, \\
\ind{17}short width, short height, short xmin, short ymin, \\
\ind{17}float aspect );}
\hspace*{\parindent}Procedura \texttt{CameraInitFramef} inicjalizuje
w~strukturze \texttt{*CPos} pola opisuj"ace spos"ob rzutowania
i~wielko"s"c klatki kamery (w~pikselach) oraz aspekt rastra (iloraz
szeroko"sci i~wysoko"sci piksela).

\begin{sloppypar}
Parametr \texttt{parallel} o~warto"sci \texttt{false} okre"sla rzutowanie
perspektywiczne, a~warto"s"c \texttt{true} spowoduje okre"slenie
rzutowania r"ownoleg"lego.%
\end{sloppypar}

\begin{sloppypar}
Parametr \texttt{upside} o~warto"sci \texttt{false} spowoduje przyj"ecie
na obrazie uk"ladu wsp"o"lrz"ednych z~osi"a~$y$ skierowan"a do do"lu,
a~\texttt{true} --- do g"ory.%
\end{sloppypar}

Parametry \texttt{width} i~\texttt{height} opisuj"a szeroko"s"c
i~wysoko"s"c klatki, za"s parametry \texttt{xmin} i~\texttt{ymin}
po"lo"renie g"ornego lewego rogu klatki.

Wywo"lanie tej procedury powinno poprzedzi"c wszystkie dalsze akcje
z~u"ryciem kamery, ale \emph{nie wystarczy} do okre"slenia rzutowania. To
powinno by"c zrobione przez wywo"lanie \texttt{CameraInitPosf} i~ewentualnie
pewnej liczby wywo"la"n procedur zmieniaj"acych po"lo"renie kamery. Je"sli
trzeba zmieni"c wielko"s"c klatki (np.\ w~celu dostosowania jej do nowych
wymiar"ow okna zmienionego przez u"rytkownika programu na ekranie), bez
zmieniania po"lo"renia obserwatora, to po wywo"laniu
\texttt{CameraInitFramef} nale"ry wywo"la"c procedur"e
\texttt{CameraSetMappingf}.

\vspace{\bigskipamount}
\cprog{%
void CameraSetMagf ( CameraRecf *CPos, byte mag );}
Procedura \texttt{CameraSetMagf} s"lu"ry do okre"slenia powi"ekszenia
rozdzielczo"sci rastra; domy"slnie jednostkami osi uk"ladu kamery s"a
szeroko"s"c i~wysoko"s"c jednego piksela. Wywo"luj"ac t"e procedur"e
z~parametrem \texttt{mag}${}=n$ (gdzie $n$ jest ca"lkowit"a liczb"a dodatni"a)
zmniejszamy te jednostki $n$~razy, co mo"re si"e przyda"c podczas tworzenia
obrazu z~nadpr"obkowaniem (ang.\ \textsl{supersampling}).

\vspace{\bigskipamount}
\cprog{%
void CameraSetMappingf ( CameraRecf *CPos );}
Procedura \texttt{CameraSetMappingf} ma na celu obliczenie macierzy
wykorzystywanych przez procedury rzutowania. Procedura ta jest wywo"lywana
przez wszystkie procedury nadaj"ace lub zmieniaj"ace po"lo"renie kamery, tak
wi"ec jej wywo"lywanie przez aplikacje jest zb"edne, z~wyj"atkiem sytuacji,
gdy wielko"s"c klatki zosta"la zmieniona (za pomoc"a
\texttt{CameraInitFramef}) i~okre"slone wcze"sniej po"lo"renie kamery
ma pozosta"c niezmienione.

\vspace{\bigskipamount}
\cprog{%
void CameraProjectPoint3f ( CameraRecf *CPos, const point3f *p, \\
\ind{28}point3f *q );}
Procedura \texttt{CameraProjectPoint3f} s"lu"ry do obliczenia obrazu punktu
\texttt{*p} w~rzucie perspektywicznym lub r"ownoleg"lym.
Wsp"o"lrz"edne tego obrazu to wsp"o"lrz"edne $x$ i $y$ punktu \texttt{*q}.
Jego wsp"o"lrz"edna $z$ okre"sla g"l"eboko"s"c punktu, tj.\ jego odleg"lo"s"c
(ze znakiem) od p"laszczyzny r"ownoleg"lej do rzutni, zawieraj"acej
po"lo"renie obserwatora. Mo"re ona si"e przyda"c do rozstrzygania
widoczno"sci w~algorytmach linii lub powierzchni zas"loni"etej.

\vspace{\bigskipamount}
\cprog{%
void CameraUnProjectPoint3f ( CameraRecf *CPos, const point3f *p, \\
\ind{30}point3f *q );}
Procedura \texttt{CameraUnProjectPoint3f} oblicza przeciwobraz
punktu~\texttt{p} w~rzucie. Wsp"o"lrz"edne $x$, $y$ punktu~\texttt{*q}
s"a podane w~uk"ladzie obrazu, wsp"o"lrz"edna~$z$ jest g"l"eboko"sci"a
(w~uk"ladzie kamery). Jest to wi"ec przekszta"lcenie odwrotne do
przekszta"lcenia dokonywanego przez procedur"e
\texttt{CameraProjectPoint3f}.

Wsp"o"lrz"edne $x$, $y$, $z$ przeciwobrazu s"a przypisywane do odpowiednich
p"ol parametru \texttt{*q}.

\vspace{\bigskipamount}
\cprog{%
void CameraProjectPoint2f ( CameraRecf *CPos, const point2f *p, \\
\ind{28}point2f *q );}
\begin{sloppypar}
Procedura \texttt{CameraProjectPoint2f} dokonuje rzutowania
punktu~$\bm{p}$, kt"orego wsp"o"lrz"edne $x$, $y$ s"a dane za pomoc"a
parametru~\texttt{p}, a~wsp"o"lrz"edna~$z$ jest r"owna~$0$.%
\end{sloppypar}

Wsp"o"lrz"edne $x$, $y$ rzutu s"a przypisywane polom parametru~\texttt{*q}.

W~zasadzie stosowanie tej procedury ma sens tylko dla rzutowania
r"ownoleg"lego.

\newpage
%\vspace{\bigskipamount}
\cprog{%
void CameraUnProjectPoint2f ( CameraRecf *CPos, const point2f *p, \\
\ind{30}point2f *q );}
Procedura \texttt{CameraUnProjectPoint2f} znajduje przeciwobraz
punktu~$\bm{p}$, kt"orego wsp"o"lrz"edne $x$, $y$ (w~uk"ladzie obrazu)
s"a dane za pomoc"a parametru~\texttt{p}, a~wsp"o"lrz"edna~$z$
(w~uk"ladzie kamery) jest r"owna~$0$.

Wsp"o"lrz"edne przeciwobrazu s"a przypisywane polom parametru~\texttt{*q}.

Procedur"e t"e mo"rna stosowa"c tylko dla rzutowania r"ownoleg"lego.

\vspace{\bigskipamount}
\cprog{%
void CameraRayOfPixelf ( CameraRecf *CPos, float xi, float eta, \\
\ind{25}ray3f *ray );}
\begin{sloppypar}
Procedura \texttt{CameraRayOfPixel} dla punktu o~wsp"o"lrz"ednych na obrazie
$x={}$\texttt{xi}, $y={}$\texttt{eta}, znajduje reprezentacj"e promienia,
tj.\ p"o"lprostej, kt"orej pocz"atek (dla rzutowania perspektywicznego)
jest po"lo"reniem obserwatora, i~kt"ora przechodzi przez podany punkt rzutni.
Dla rzutowania r"ownoleg"lego pocz"atek promienia jest punktem rzutni,
a~jego kierunek jest kierunkiem rzutowania.%
\end{sloppypar}

\begin{sloppypar}
Pole~\texttt{p} struktury \texttt{*ray} otrzymuje warto"s"c opisuj"ac"a
pocz"atek promienia (we wsp"o"lrz"ednych w~uk"ladzie globalnym),
a~przypisana przez procedur"e warto"s"c pola~\texttt{v} to jednostkowy
wektor kierunkowy promienia. Procedura jest napisana na g"l"ownie potrzeby
"sledzenia promieni.%
\end{sloppypar}

\vspace{\bigskipamount}
\cprog{%
void CameraInitPosf ( CameraRecf *CPos );}
Procedura \texttt{CameraInitPosf} przypisuje kamerze domy"slne po"lo"renie,
w~kt"orym osie $x$, $y$ i~$z$ uk"ladu kamery pokrywaj"a si"e z osiami $x$,
$y$, $z$ uk"ladu globalnego. D"lugo"s"c ogniskowej kamery otrzymuje
warto"s"c $1$. Przed wywo"laniem tej procedury nale"ry okre"sli"c wielko"s"c
klatki i~aspekt obrazu, za pomoc"a procedury \texttt{CameraInitFramef}.

Po wywo"laniu procedury \texttt{CameraInitPosf} kamera jest gotowa do
rzutowania punkt"ow, a~tak"re do zmieniania po"lo"renia i~ogniskowej.

\vspace{\bigskipamount}
\cprog{%
void CameraSetRotCentref ( CameraRecf *CPos, point3f *centre, \\
\ind{22}boolean global\_coord, boolean global\_fixed );}
Procedura \texttt{CameraSetRotCentref} s"lu"ry do okre"slania punktu, przez
kt"ory przechodz"a osie nast"epnie wykonywanych obrot"ow kamery. Parametr
\texttt{*centre} okre"sla ten punkt, parametr \texttt{global\_coord}
okre"sla, czy wsp"o"lrz"edne tego punktu s"a podane w~uk"ladzie globalnym
(je"sli ma warto"s"c \texttt{true}), czy w~uk"ladzie kamery (je"sli
\texttt{false}). Parametr \texttt{global\_fixed} okre"sla, czy przy
przesuni"eciach kamery punkt ten jest nieruchomy w uk"ladzie globalnym
(je"sli \texttt{true}), czy w~uk"ladzie kamery (je"sli \texttt{false}).

\vspace{\bigskipamount}
\cprog{%
void CameraMoveToGf ( CameraRecf *CPos, point3f *pos );}
Procedura \texttt{CameraMoveToGf} przesuwa (bez obrotu) kamer"e do
po"lo"renia \texttt{*pos}, wyspecyfikowanego w~uk"ladzie globalnym.

\vspace{\bigskipamount}
\cprog{%
void CameraTurnGf ( CameraRecf *CPos, \\
\ind{20}float psi, float theta, float phi );}
Procedura \texttt{CameraTurnGf} nadaje kamerze po"lo"renie k"atowe
okre"slone przez podanie k"at"ow Eulera (precesji,
\texttt{psi}, nutacji, \texttt{theta}, i~obrotu w"la"sciwego, \texttt{phi}),
w~globalnym uk"ladzie wsp"o"lrz"ednych.

\vspace{\medskipamount}
\begin{sloppypar}\noindent
\textbf{Uwaga:} Przewiduj"e zmian"e sposobu okre"slania po"lo"renia
k"atowego kamery, w~zwi"azku z~czym bezpo"srednie u"rywanie tej procedury
jest \emph{niewskazane}.
\end{sloppypar}

\vspace{\bigskipamount}
\cprog{%
void CameraMoveGf ( CameraRecf *CPos, vector3f *v );}
Procedura \texttt{CameraMoveGf} poddaje kamer"e przesuni"eciu o~wektor
\texttt{v}, podany w~globalnym uk"ladzie wsp"o"lrz"ednych.

\vspace{\bigskipamount}
\cprog{%
void CameraMoveCf ( CameraRecf *CPos, vector3f *v );}
Procedura \texttt{CameraMoveGf} poddaje kamer"e przesuni"eciu o~wektor
\texttt{v}, podany w~uk"ladzie wsp"o"lrz"ednych kamery.

\vspace{\bigskipamount}
\cprog{%
void CameraRotGf ( CameraRecf *CPos, \\
\ind{19}float psi, float theta, float phi );}
Procedura \texttt{CameraRotGf} dokonuje obrotu kamery. Obr"ot jest
okre"slony za~pomoc"a k"at"ow Eulera w~uk"ladzie globalnym.
O"s obrotu przechodzi przez pocz"atek globalnego uk"ladu wsp"o"lrz"ednych
lub przez punkt okre"slony za pomoc"a procedury~\texttt{SetCameraRotCentref}.

\vspace{\bigskipamount}
\cprog{%
\#define CameraRotXGf(Camera,angle) \bsl \\
\ind{2}CameraRotGf(Camera, 0.0, angle, 0.0) \\
\#define CameraRotYGf(Camera,angle) \bsl \\
\ind{2}CameraRotGf(Camera, 0.5 * PI, angle, -0.5 * PI) \\
\#define CameraRotZGf(Camera,angle) \bsl \\
\ind{2}CameraRotGf(Camera, angle, 0.0, 0.0)}
Trzy makra dokonuj"a obrot"ow kamery wok"o"l trzech osi uk"ladu
globalnego.

\vspace{\bigskipamount}
\cprog{%
void CameraRotVGf ( CameraRecf *CPos, vector3f *v, float angle );}
Procedura \texttt{CameraRotVGf} dokonuje obrotu kamery wok"o"l osi
o~kierunku wektora \texttt{v} i~k"at~\texttt{angle}. Wsp"o"lrz"edne
wektora~\texttt{v} s"a dane w~globalnym uk"ladzie wsp"o"lrz"ednych.

\vspace{\bigskipamount}
\cprog{%
void CameraRotCf ( CameraRecf *CPos, \\
\ind{19}float psi, float theta, float phi );}
Procedura \texttt{CameraRotCf} dokonuje obrotu kamery. Obr"ot jest
okre"slony za pomoc"a k"at"ow Eulera w~uk"ladzie kamery. O"s obrotu
przechodzi przez pocz"atek \emph{globalnego} uk"ladu wsp"o"lrz"ednych lub
przez punkt okre"slony za pomoc"a procedury~\texttt{SetCameraRotCentref}.

\vspace{\bigskipamount}
\cprog{%
\#define CameraRotXCf(Camera,angle) \bsl \\
\ind{2}CameraRotCf ( Camera, 0.0, angle, 0.0 ) \\
\#define CameraRotYCf(Camera,angle) \bsl \\
\ind{2}CameraRotCf ( Camera, 0.5 * PI, angle, -0.5 * PI ) \\
\#define CameraRotZCf(Camera,angle) \bsl \\
\ind{2}CameraRotCf ( Camera, angle, 0.0, 0.0 )}
Trzy makra dokonuj"a obrot"ow kamery wok"o"l trzech osi uk"ladu kamery.

\vspace{\bigskipamount}
\cprog{%
void CameraRotVCf ( CameraRecf *CPos, vector3f *v, float angle );}
Procedura \texttt{CameraRotVCf} dokonuje obrotu kamery wok"o"l osi
o~kierunku wektora \texttt{v} i~k"at~\texttt{angle}. Wsp"o"lrz"edne
wektora~\texttt{v} s"a dane w~uk"ladzie wsp"o"lrz"ednych kamery.

\vspace{\bigskipamount}
\cprog{%
void CameraSetFf ( CameraRecf *CPos, float f );}
Procedura \texttt{CameraSetFf} ustawia ,,d"lugo"s"c ogniskowej'' kamery.

\vspace{\bigskipamount}
\cprog{%
void CameraZoomf ( CameraRecf *CPos, float fchange );}
Procedura \texttt{CameraZoomf} zmienia ,,d"lugo"s"c ogniskowej'' kamery,
mno"r"ac d"lugo"s"c dotychczasow"a przez parametr \texttt{fchange}, kt"ory
musi by"c dodatni.

\vspace{\bigskipamount}
\cprog{%
boolean CameraClipPoint3f ( CameraRecf *CPos, \\
\ind{28}point3f *p, point3f *q );}
Procedura \texttt{CameraClipPoint3f} sprawdza, czy obraz punktu~\texttt{p}
mie"sci si"e w~klatce i~je"sli tak, to dokonuje rzutowania tego punktu.
Wsp"o"lrz"edne obrazu punktu s"a przekazywane za pomoc"a parametru~\texttt{q}.
Warto"s"c \texttt{true} oznacza, "re obraz punktu jest widoczny, \texttt{false},
"re nie.

\vspace{\bigskipamount}
\cprog{%
boolean CameraClipLine3f ( CameraRecf *CPos, \\
\ind{22}point3f *p0, float t0, point3f *p1, float t1, \\
\ind{22}point3f *q0, point3f *q1 );}
Procedura \texttt{CameraClipLine3f} obcina do ostros"lupa widoczno"sci
odcinek $\{\,(1-t)\bm{p}_0+t\bm{p}_1\colon t\in[t_0,t_1]\,\}$.
Je"sli przeci"ecie odcinka z~ostros"lupem widoczno"sci jest niepuste,
to ko"nce tej cz"e"sci s"a rzutowane i~wyprowadzane za pomoc"a parametr"ow
\texttt{q0} i~\texttt{q1}. Warto"sci"a procedury jest wtedy \texttt{true}.

Procedura jest implementacj"a algorytmu Lianga-Barsky'ego.

\vspace{\bigskipamount}
\cprog{%
boolean CameraClipPolygon3f ( CameraRecf *CPos, \\
\ind{30}int n, const point3f *p, \\
\ind{30}void (*output)(int n, point3f *p) );}
Procedura \texttt{CameraClipPolygon3f} dokonuje obcinania wielok"ata
do ostros"lupa widzenia, przy u"ryciu algorytmu Sutherlanda-Hodgmana.
Parametr~\texttt{n} okre"sla liczb"e wierzcho"lk"ow wielok"ata
w przestrzeni, podanych w tablicy~\texttt{p}; brzeg tego wielok"ata
jest jedn"a "laman"a zamkni"et"a.

\begin{sloppypar}
Parametr \texttt{output} wskazuje procedur"e, kt"ora zostanie wywo"lana,
je"sli cz"e"s"c wsp"ol\-na wielok"ata z~ostros"lupem widzenia jest niepusta.
Parametr~\texttt{n} tej procedury okre"sla liczb"e wierzcho"lk"ow
obci"etego wielok"ata. W~tablicy~\texttt{p} s"a podane rzuty tych
wierzcho"lk"ow.%
\end{sloppypar}


\newpage
\section{Para kamer do obraz"ow stereoskopowych}

Aby wykona"c par"e obraz"ow stereoskopowych trzeba umie"sci"c w~przestrzeni
dwie kamery, a~nast"epnie wykona"c obrazy, dokonuj"ac rzutowania za pomoc"a
ka"rdej z~nich. Opisane ni"rej procedury u"latwiaj"a manipulowanie tak"a
par"a kamer; ka"rda z~nich odpowiada pewnej procedurze manipulacji jedn"a
kamer"a. Procedur tych u"rywa si"e \emph{zamiast} opisanych wy"rej procedur
ustawiania kamer w~przestrzeni. Natomiast podczas tworzenia obraz"ow
b"ed"a u"rywane procedury rzutowania lub generowania promieni, dla ka"rdej
kamery niezale"rnie od drugiej.
\begin{figure}[ht]
  \centerline{\epsfig{file=stereo.ps}}
  \caption{Para kamer stereo i ich wsp"olna klatka}
\end{figure}%

Struktura danych i nag"l"owki procedur obs"luguj"acych par"e kamer
stereoskopowych s"a opisane w~pliku nag"l"owkowym \texttt{stereo.h}.

\vspace{\bigskipamount}
\cprog{%
typedef struct StereoRecf \{ \\
\ind{2}point3f\ind{4}position; \\
\ind{2}float\ind{6}d; \\
\ind{2}float\ind{6}l; \\
\ind{2}CameraRecf left, right; \\
\ind{2}trans3f\ind{4}STr, STrInv; \\
\} StereoRecf;}
Struktura typu \texttt{StereoRecf} zawiera opisy dw"och kamer (lewej
i~prawej).

\vspace{\bigskipamount}
\cprog{%
void StereoInitFramef ( StereoRecf *Stereo, boolean upside, \\
\ind{17}short width, short height, short xmin, short ymin, \\
\ind{17}float aspect );}
\begin{sloppypar}
Procedura \texttt{StereoInitFramef} s"lu"ry do okre"slenia wymiar"ow klatek
kamer w~pikselach i~wsp"o"lczynnika aspect. W~tym celu procedura wywo"luje
dla ka"rdej z~kamer procedur"e \texttt{CameraInitFramef} z~tymi samymi
parametrami. Procedur"e t"e nale"ry wywo"la"c jako pierwsz"a w~procesie
inicjalizacji pary kamer. Po jej wywo"laniu jeszcze procedura \emph{nie jest}
gotowa do u"rycia.%
\end{sloppypar}

Parametr \texttt{upside} okre"sla zwrot osi~$y$ w~uk"ladzie wsp"olrz"ednych
obrazu --- zobacz opis procedury \texttt{CameraInitFramef}.

\vspace{\bigskipamount}
\cprog{%
void StereoSetDimf ( StereoRecf *Stereo, \\
\ind{21}float f, float d, float l );}
Procedura \texttt{StereoSetDimf} s"lu"ry do okre"slenia wymiar"ow pary kamer
w~jednostkach d"lugo"sci zwi"azanych z~podstawowym uk"ladem
wsp"o"lrz"ednych, w~kt"orym opisuje si"e obiekty do narysowania.
Parametr~\texttt{f} okre"sla bezwymiarow"a d"lugo"s"c ogniskowej, tj.\ iloraz
odleg"lo"sci p"laszczyzny klatki od "srodka rzutowania i~przek"atnej klatki.
Parametr~\texttt{d} okre"sla odleg"lo"s"c mi"edzy "zrenicami oczu
obserwatora (tj.\ mi"edzy "srodkami rzutowania kamer), a~parametr~\texttt{l}
okre"sla odleg"lo"s"c "srodk"ow rzutowania od p"laszczyzny klatki (tj.\
ekranu). Przek"atna klatki ma zatem d"lugo"s"c $l/f$ w~jednostkach
zwi"azanych ze stosowanym uk"ladem. D"lugo"s"c przek"atnej klatki w~calach
zale"ry od monitora.

\vspace{\medskipamount}\noindent
\textbf{Uwaga:} Procedura \texttt{StereoInitPosf} przypisuje atrybuty $f=1$,
$d=0$ i~$l=1$, kt"ore nie s"a zbyt u"ryteczne. Trzeba im zatem nada"c
bardziej sensowne warto"sci, w"la"snie przez wywo"lanie procedury
\texttt{StereoSetDimf}.

\vspace{\bigskipamount}
\cprog{%
void StereoSetMagf ( StereoRecf *Stereo, char mag );}
Procedura \texttt{SetStereoMagf} okre"sla wsp"o"lczynnik powi"ekszenia
rozdzielczo"sci kamer (np.\ dla cel"ow antyaliasowania) przez wywo"lanie
procedury \texttt{CameraSetMagf} dla obu kamer. Domy"slna warto"s"c tego
wsp"o"lczynnika (bez wywo"lywania tej procedury) jest r"owna~$1$.

\vspace{\bigskipamount}
\cprog{%
void StereoSetMappingf ( StereoRecf *Stereo );}
Procedura \texttt{StereoSetMappingf} okre"sla po"lo"renie "srodk"ow rzutowania
kamer i~przygotowuje je do u"rycia (tj.\ rzutowania punkt"ow itd.), przez
wywo"lanie procedury \texttt{CameraSetMappingf}. Przed wywo"laniem tej procedury
nale"ry wywo"la"c \texttt{StereoInitFramef}, \texttt{StereoInitPosf}
i~\texttt{StereoSetDimf}.

Opisane ni"rej procedury ustawiaj"ace i~zmieniaj"ace pozycj"e pary kamer
wywo"luj"a t"e procedur"e, a~wi"ec typowe aplikacje nie musz"a jej
wywo"lywa"c bezpo"srednio.

\vspace{\bigskipamount}
\cprog{%
void StereoInitPosf ( StereoRecf *Stereo );}
Procedura \texttt{StereoInitPosf} ustawia par"e kamer w~domy"slnym
po"lo"reniu wyj"s\-cio\-wym. Obie kamery s"a ustawiane jednakowo, w~takiej
pozycji, jak przez procedur"e \texttt{CameraInitPosf}.

\newpage
%\vspace{\bigskipamount}
\cprog{%
void StereoSetRotCentref ( StereoRecf *Stereo, \\
\ind{22}point3f *centre, \\
\ind{22}boolean global\_coord, boolean global\_fixed );}
Procedura okre"sla punkt, przez kt"ory przechodzi o"s obrotu pary kamer
podczas zmian jej po"lo"renia wykonywanego przez procedury opisane ni"rej.
Spos"ob jego okre"slania i~przetwarzania jest identyczny jak w~przypadku
jednej kamery.

\vspace{\bigskipamount}
\cprog{%
void StereoMoveGf ( StereoRecf *Stereo, vector3f *v );}
Procedura \texttt{StereoMoveGf} dokonuje przemieszczenia pary kamer (bez
zmiany kierunku) o~wektor~$\bm{v}$, okre"slony w~uk"ladzie globalnym.

\vspace{\bigskipamount}
\cprog{%
void StereoMoveCf ( StereoRecf *Stereo, vector3f *v );}
Procedura \texttt{StereoMoveCf} dokonuje przemieszczenia pary kamer (bez
zmiany kierunku) o~wektor~$\bm{v}$, okre"slony w~uk"ladzie kamer.

\vspace{\bigskipamount}
\cprog{%
void StereoRotGf ( StereoRecf *Stereo, \\
\ind{19}float psi, float theta, float phi );}
Procedura \texttt{StereoRotGf} dokonuje obrotu pary kamer, okre"slonego
za~pomoc"a k"at"ow Eulera $\psi$, $\vartheta$, $\varphi$ w~uk"ladzie globalnym.

\vspace{\bigskipamount}
\cprog{%
\#define StereoRotXGf(Stereo,angle) \bsl \\                         
\ind{2}StereoRotGf ( Stereo, 0.0, angle, 0.0 ) \\
\#define StereoRotYGf(Stereo,angle) \bsl \\
\ind{2}StereoRotGf ( Stereo, 0.5*PI, angle, -0.5*PI ) \\
\#define StereoRotZGf(Stereo,angle) \bsl \\
\ind{2}StereoRotGf ( Stereo, angle, 0.0, 0.0 )}
Powy"rsze makra dokonuj"a obrotu pary kamer wok"o"l osi r"ownoleg"lych
odpowiednio do osi $x$, $y$ i~$z$ globalnego uk"ladu wsp"o"lrz"ednych.

\vspace{\bigskipamount}
\cprog{%
void StereoRotVGf ( StereoRecf *Stereo, vector3f *v, float angle );}
Procedura \texttt{StereoRotVGf} dokonuje obrotu pary kamer wok"o"l osi
o~kierunku wektora~$\bm{v}$, okre"slonego w~uk"ladzie globalnym.

\vspace{\bigskipamount}
\cprog{%
void StereoRotCf ( StereoRecf *Stereo, \\
\ind{19}float psi, float theta, float phi );}
Procedura \texttt{StereoRotCf} dokonuje obrotu pary kamer okre"slonego przez
k"aty Eulera $\psi$, $\vartheta$, $\varphi$ w~uk"ladzie wp"o"lrz"ednych pary
kamer.

\newpage
%\vspace{\bigskipamount}
\cprog{%
\#define StereoRotXCf(Stereo,angle) \bsl \\
\ind{2}StereoRotCf ( Stereo, 0.0, angle, 0.0 ) \\
\#define StereoRotYCf(Stereo,angle) \bsl \\
\ind{2}StereoRotCf ( Stereo, 0.5*PI, angle, -0.5*PI ) \\
\#define StereoRotZCf(Stereo,angle) \bsl \\
\ind{2}StereoRotCf ( Stereo, angle, 0.0, 0.0 )}
Powy"rsze makra dokonuj"a obrotu pary kamer wok"o"l osi r"ownoleg"lych
odpowiednio do osi $x$, $y$ i~$z$ uk"ladu wsp"o"lrz"ednych pary kamer.

\vspace{\bigskipamount}
\cprog{%
void StereoRotVCf ( StereoRecf *Stereo, vector3f *v, float angle );}
Procedura \texttt{StereoRotVCf} dokonuje obrotu pary kamer wok"o"l osi
o~kierunku wektora~$\bm{v}$, okre"slonego w~uk"ladzie wsp"o"lrz"ednych pary
kamer.

\vspace{\bigskipamount}
\cprog{%
void StereoZoomf ( StereoRecf *Stereo, float fchange );}
Procedura \texttt{StereoZoomf} mno"ry d"lugo"s"c ogniskowej kamer przez
warto"s"c parametru \texttt{fchange}. Lepiej jej nie u"rywa"c.


