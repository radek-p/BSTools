
%/* //////////////////////////////////////////////////// */
%/* This file is a part of the BSTools procedure package */
%/* written by Przemyslaw Kiciak.                        */
%/* //////////////////////////////////////////////////// */

\chapter{Biblioteka \texttt{libmultibs}}

\section[Podstawowe definicje i sposoby reprezentowania krzywych i p"lat"ow]%
{Podstawowe definicje i sposoby reprezentowania \\ krzywych i p"lat"ow}

\subsection{Krzywe B\'{e}ziera}
\unl{Krzywa B\'{e}ziera} jest okre"slona wzorem
\begin{align}\label{eq:Bezc:def}
  \bm{p}(t) = \sum_{i=0}^n \bm{p}_iB^n_i(t),
\end{align}
w kt"orym wyst"epuj"a \unl{punkty kontrolne} $\bm{p}_0,\ldots,\bm{p}_n$
oraz \unl{wielomiany Bernsteina}
\begin{align}
  B^n_i(t) \stackrel{\mathrm{def}}{=} \binom{n}{i}t^i(1-t)^{n-i},\quad
  i=0,\ldots,n.
\end{align}

"Lamana, kt"orej kolejnymi wierzcho"lkami s"a
punkty $\bm{p}_0,\ldots,\bm{p}_n$, nazywa si"e
\unl{"la-} \unl{man"a kontroln"a krzywej}.
Ka"rdy punkt kontrolny ma $d$ wsp"o"lrz"ednych i~wtedy krzywa le"ry
w~przestrzeni $d$-wymiarowej. W~szczeg"olno"sci dla $d=1$
wz"or~(\ref{eq:Bezc:def}) okre"sla wielomian zmiennej~$t$
stopnia co najwy"rej~$n$.

Reprezentacja krzywej B\'{e}ziera sk"lada si"e z~liczby~$n$ i~ci"agu $n+1$
punkt"ow kontrolnych, kt"orych wsp"o"lrz"edne (w~sumie $(n+1)d$ liczb
zmiennopozycyjnych) nale"ry umie"sci"c w~tablicy ,,po kolei''
(tj.\ najpierw $d$~wsp"o"lrz"ednych punktu $\bm{p}_0$, potem $\bm{p}_1$
itd.).

\vspace{\medskipamount}
\unl{Wymierna krzywa B\'{e}ziera} jest okre"slona wzorem
\begin{align}
  \bm{p}(t) =
  \frac{\sum_{i=0}^n w_i\bm{p}_i B^n_i(t)}{\sum_{i=0}^n w_iB^n_i(t)},
\end{align}
w kt"orym wyst"epuj"a wielomiany Bernsteina, punkty kontrolne
$\bm{p}_0,\ldots,\bm{p}_n$ i~\unl{wagi} $w_0,\ldots,w_n$. Krzywa taka
le"ry w~tej samej przestrzeni, w~kt"orej le"r"a punkty kontrolne.

Je"sli $w_i=0$ dla pewnego $i$, to wyra"renie $w_i\bm{p}_i$ mo"remy
zast"api"c przez dowolny wektor $\bm{v}_i$, rozszerzaj"ac definicj"e
krzywej, ale przynajmniej jedna waga musi by"c r"o"rna od zera.

Punkty kontrolne $\bm{p}_i$ krzywej wymiernej s"a wygodne dla u"rytkownika
programu, kt"ory mo"re interakcyjnie je dobiera"c, ale procedury w~tej
bibliotece przetwarzaj"a tzw.\ \unl{reprezentacj"e jednorodn"a}.
Dla krzywej w~przestrzeni $d$-wymiarowej jest ni"a krzywa wielomianowa
po"lo"rona w~przestrzeni o wymiarze $d+1$:
\begin{align}
  \bm{P}(t) = \sum_{i=0}^n\bm{P}_iB^n_i(t),
\end{align}
kt"orej punkty kontrolne $\bm{P}_i$ s"a dane wzorem
\begin{align}
  \bm{P}_i = \left[\begin{array}{c}w_i\bm{p}_i \\ w_i \end{array}\right].
\end{align}
Ostatnia (tj.\ $d+{}$pierwsza) wsp"o"lrz"edna jednorodna jest wi"ec
odpowiedni"a wsp"o"lrz"edn"a wagow"a. Je"sli $w_i=0$ to
\begin{align}
  \bm{P}_i = \left[\begin{array}{c}\bm{v}_i \\ 0 \end{array}\right].
\end{align}

Wsp"o"lrz"edne kartezja"nskie punktu $\bm{p}(t)$ krzywej wymiernej otrzymuje
si"e przez podzielenie pierwszych $d$ wsp"o"lrz"ednych punktu $\bm{P}(t)$
przez jego ostatni"a wsp"o"lrz"edn"a jednorodn"a.

Reprezentacja krzywej wymiernej np.\ w~przestrzeni tr"ojwymiarowej sk"lada
si"e z~liczby~$n$ (kt"ora okre"sla stopie"n reprezentacji) i~tablicy
$4(n+1)$ liczb zmiennopozycyjnych, b"ed"acych wsp"o"lrz"ednymi kolejnych
punkt"ow $\bm{P}_i$.
Poniewa"r krzywe jednorodne s"a zwyk"lymi krzywymi wielomianowymi, wi"ec
do ich przetwarzania w~wi"ekszo"sci przypadk"ow s"lu"r"a te same procedury,
co do przetwarzania wielomianowych krzywych B\'{e}ziera.


\subsection{Prostok"atne p"laty B\'{e}ziera}

\begin{sloppypar}
\unl{Prostok"atny p"lat B\'{e}ziera} jest okre"slony wzorem
\begin{align}
  \bm{p}(u,v) = \sum_{i=0}^n\sum_{j=0}^m \bm{p}_{ij}B^n_i(u)B^m_j(v),
\end{align}
w kt"orym wyst"epuj"a wielomiany Bernsteina $B^n_i$ i $B^m_j$ stopni
odpowiednio $n$~i~$m$ oraz punkty kontrolne $\bm{p}_{ij}$. Dla ustalenia
uwagi \unl{wierszem siatki kontrolnej} p"lata nazywamy "laman"a
o~wierzcho"lkach
w~punktach $\bm{p}_{0j},\ldots,\bm{p}_{nj}$ (dla $j\in\{0,\ldots,m\}$),
a~\unl{kolumna} jest to "lamana o~wierzcho"lkach
$\bm{p}_{i0},\ldots,\bm{p}_{im}$ (dla ka"rdego $i\in\{0,\ldots,n\}$).
\end{sloppypar}

Reprezentacja p"lata sk"lada si"e zatem z~liczb dodatnich $n$~i~$m$ oraz
$(n+1)(m+1)$ punkt"ow kontrolnych, czyli $(n+1)(m+1)d$ liczb
zmiennopozycyjnych,
umieszczonych w~tablicy w kolejno"sci nast"epuj"acej: najpierw
$d$~wsp"o"lrz"ednych punktu $\bm{p}_{00}$, potem $d$ wsp"o"lrz"ednych
punktu $\bm{p}_{01}$ itd. Po wsp"o"lrz"ednych punktu $\bm{p}_{0m}$ nale"ry
poda"c wsp"o"lrz"edne punktu $\bm{p}_{10}$ i~tak dalej, a"r do punktu
$\bm{p}_{nm}$. Inaczej m"owi"ac, w~tablicy s"a umieszczone kolejne kolumny
siatki kontrolnej.

Na opisan"a wy"rej tablic"e mo"rna patrze"c na wiele r"o"rnych sposob"ow.
Na przyk"lad, aby podda"c p"lat ustalonemu przekszta"lceniu afinicznemu
wystarczy zastosowa"c to przekszta"lcenie do wszystkich jego punkt"ow
kontrolnych. B"edziemy wi"ec wtedy widzie"c t"e tablic"e jako jednowymiarow"a
tablic"e punkt"ow.

Mo"remy te"r dokona"c podzia"lu p"lata za pomoc"a
algorytmu de~Casteljau na dwie cz"e"sci, dziel"ac na po"lowy przedzia"l
zmienno"sci parametru $u$~lub~$v$. W~tym ostatnim przypadku wystarczy
zastosowa"c ten algorytm do wszystkich kolumn tak jak gdyby by"ly to "lamane
kontrolne krzywych B\'{e}ziera. Tablica zawiera zatem reprezentacje $n+1$
krzywych B\'{e}ziera stopnia $m$ i~pierwsza wsp"o"lrz"edna pierwszego punktu
nast"epnej krzywej znajduje si"e o~$(m+1)d$ miejsc za pocz"atkiem
reprezentacji krzywej danej. Zatem przyjmiemy, "re podzia"lka tej tablicy
jest r"owna $d(m+1)$ (gdzie $d$~jest wymiarem przestrzeni, w kt"orej le"ry
p"lat).

Aby podzieli"c przedzia"l zmienno"sci parametru~$u$, nale"ry zastosowa"c
algorytm de~Casteljau do wszystkich wierszy statki kontrolnej p"lata.
Okazuje si"e, "re reprezentacj"e
p"lata mo"rna potraktowa"c jak reprezentacj"e krzywej B\'{e}ziera
stopnia~$n$, po"lo"ronej w~przestrzeni o~wymiarze $(m+1)d$ (ka"rda kolumna
siatki kontrolnej jest jednym punktem tej przestrzeni). W~tym przypadku
przetwarzamy \emph{jedn"a krzyw"a} B\'{e}ziera stopnia~$n$ w~przestrzeni
$(m+1)d$-wymiarowej, a~podzia"lka tablicy jest nieistotna, poniewa"r krzywa
jest tylko jedna.

\vspace{\medskipamount}
\unl{Wymierny p"lat B\'{e}ziera} jest okre"slony wzorem
\begin{align*}
  \bm{p}(u,v) =
    \frac{\sum_{i=0}^n\sum_{j=0}^m w_{ij}\bm{p}_{ij}B^n_i(u)B^m_j(v)}%
         {\sum_{i=0}^n\sum_{j=0}^m w_{ij}B^n_i(u)B^m_j(v)},
\end{align*}
w~kt"orym opr"ocz wielomian"ow Bernsteina i~punkt"ow kontrolnych
wyst"epuj"a wagi $w_{ij}$. Procedury przetwarzaj"ace wymierne p"laty
B\'{e}ziera w~bibliotece \texttt{libmultibs} przetwarzaj"a reprezentacj"e
jednorodn"a, czyli tablic"e punkt"ow kontrolnych $\bm{P}_{ij}$
wielomianowego p"lata B\'{e}ziera
\begin{align}
  \bm{P}(u,v) = \sum_{i=0}^n\sum_{j=0}^m\bm{P}_{ij}B^n_i(u)B^m_j(v),
\end{align}
po"lo"ronego w przestrzeni o wymiarze $d+1$. Zwi"azek punkt"ow kontrolnych
$\bm{p}_{ij}$ i~wag $w_{ij}$ p"lata wymiernego z~punktami $\bm{P}_{ij}$ jest
taki sam jak w~przypadku wymiernych krzywych B\'{e}ziera. Spos"ob
przechowywania w~tablicy punkt"ow kontrolnych $\bm{P}_{ij}$ p"lata
jednorodnego jest taki sam jak w przypadku p"lata wielomianowego (tylko
wymiar przestrzeni, w kt"orej le"ry p"lat jednorodny i jego punkty kontrolne
jest o $1$ wi"ekszy).


\subsection{\label{ssect:BSC}Krzywe B-sklejane}

Przyjmijmy $n\geq 0$ i~niemalej"acy ci"ag \unl{w"ez"l"ow} (liczb
rzeczywistych) $u_0,\ldots,u_N$, taki "re $N>2n$ i~$u_n<u_{N-n}$.
\unl{Krzywa B-sklejana} stopnia $n$ oparta na tym ci"agu w"ez"l"ow jest
okre"slona wzorem
\begin{align}\label{eq:BScurve:def}
  \bm{s}(t) = \sum_{i=0}^{N-n-1} \bm{d}_iN^n_i(t),
\end{align}
w kt"orym wyst"epuj"a punkty kontrolne $\bm{d}_0,\ldots,\bm{d}_{N-n-1}$ oraz
\unl{funkcje B-sklejane} $N^n_0,\ldots,N^n_{N-n-1}$. Funkcje te maj"a
kilka r"ownowa"rnych definicji, m.in.\ mo"rna je
zdefiniowa"c rekurencyjnym wzorem Mansfielda-deBoora-Coxa:
\begin{align}\label{eq:BS:basis0}
  N^0_i(t) &{}= \left\{\begin{array}{ll}1 & \mbox{dla $u_i\leq t<u_{i+1}$,} \\
    0 & \mbox{w przeciwnym razie,} \end{array}\right. \\
  \label{eq:BS:basisn}
  N^j_i(t) &{}= \frac{t-u_i}{u_{i+j}-u_i} N^{j-1}_i(t) +
    \frac{u_{i+j+1}-t}{u_{i+j+1}-u_{i+1}} N^{j-1}_{i+1}(t)\quad
    \mbox{dla $j=1,\ldots,n$}.
\end{align}

Dziedzin"a krzywej B-sklejanej jest przedzia"l $[u_n,u_{N-n-1})$. W~ka"rdym
przedziale $[u_k,u_{k+1})$ (dla $n\leq k<N-n$) krzywa \mbox{B-sklejana} jest
"lukiem wielomianowym stopnia co najwy"rej $n$.

Reprezentacja krzywej B-sklejanej sk"lada si"e z liczb ca"lkowitych
$n$~i~$N$, okre"slaj"acych odpowiednio stopie"n i~indeks ostatniego w"ez"la,
ci"agu w"ez"l"ow (tablicy liczb zmiennopozycyjnych) $u_0,\ldots,u_N$ oraz
ci"agu punkt"ow kontrolnych $\bm{d}_0,\ldots,\bm{d}_{N-n-1}$, po"lo"ronych
w~tej samej przestrzeni co krzywa --- je"sli wymiarem tej przestrzeni
jest~$d$, to w~odpowiedniej tablicy nale"ry poda"c $(N-n)d$ liczb
zmiennopozycyjnych.

\vspace{\medskipamount}
\begin{sloppypar}\hyphenpenalty=200
Dla u"latwienia wyja"sniania niuans"ow dzia"lania procedur przyj"a"lem
nast"epuj"ac"a terminologi"e: \unl{w"ez"ly brzegowe} to te, kt"ore
wyznaczaj"a brzeg dziedziny krzywej, czyli $u_n$, $u_{N-n}$ i~wszystkie
w"ez\-"ly r"owne jednemu z~tych dw"och; w"ez\-"ly brzegowe dzielimy na
\unl{lewe} i~\unl{prawe}. \unl{W"ez"ly wewn"etrzne} to
wszystkie w"ez"ly nale"r"ace do przedzia"lu $(u_n,u_{N-n})$; w"ez"lom tym
odpowiadaj"a punkty po"l"aczenia "luk"ow wielomianowych krzywej. Opr"ocz
tego mamy \unl{w"ez"ly zewn"etrzne}, kt"ore nie nale"r"a do
przedzia"lu $[u_n,u_{N-n}]$.
Niezale"rnie od tego w"ez"ly $u_0$ i~$u_N$ b"ed"a nazywane \unl{w"ez"lami
skrajnymi}.
Na przyk"lad, je"sli $n=3$, $N=15$ i
\begin{align*}
  u_0<u_1=u_2<u_3=u_4=u_5<u_6\leq\cdots\leq u_{11}<u_{12}=u_{13}=u_{14}=u_{15},
\end{align*}
to w"ez"ly $u_0$, $u_1$ i~$u_2$ s"a zewn"etrzne, w"ez"ly $u_3$, $u_4$
i~$u_{12},\ldots,u_{15}$ s"a brzegowe, a~pozosta"le w"ez"ly s"a wewn"etrzne.
W"ez"ly skrajne to $u_0$ i $u_{15}$.
\end{sloppypar}

W"ez"ly skrajne s"a potrzebne w definicji funkcji
B-sklejanych $N^n_0$ oraz $N^n_{N-n-1}$, ale nie maj"a one wp"lywu na
warto"sci tych funkcji w~przedziale $[u_n,u_{N-n})$, a~wi"ec tak"re na
kszta"lt krzywej. W~r"o"rnych pakietach oprogramowania w"ez"ly te s"a
wymagane lub nie; w~bibliotece \texttt{libmultibs} w"ez"ly te nale"ry
podawa"c (wystarczy, "re spe"lniaj"a warunki $u_0\leq u_1$ oraz
$u_{N-1}\leq u_N$).

\vspace{\medskipamount}
Ustalona krzywa sklejana mo"re mie"c r"o"rne reprezentacje; mog"a one
r"o"rni"c si"e stopniem lub ci"agiem w"ez"l"ow. Konstruowanie reprezentacji,
kt"orej ci"ag w"ez"l"ow zawiera dodatkowe liczby nazywa si"e \unl{wstawianiem
w"ez"l"ow}. W szczeg"olno"sci reprezentacja, w~kt"orej wszystkie w"ez"ly
maj"a krotno"s"c (liczb"e wyst"apie"n) $n+1$ (tj.\ jest
$u_0=\cdots=u_n$, $u_{n+1}=\cdots=u_{2n+1}$, $u_{2n+2}=\cdots=u_{3n+2}$
itd.), jest reprezentacj"a krzywej kawa"lkami B\'{e}ziera.

Je"sli ostatni (o~najwi"ekszym indeksie) lewy w"eze"l brzegowy ma indeks
$k>n$, to pocz"atkowe $k-n$ w"ez"l"ow (zaczynaj"ac od skrajnego lewego)
i~pocz"atkowe $k-n$ punkt"ow jest w~reprezentacji krzywej zb"ednych i~mo"rna
(dla pewnych cel"ow trzeba) je pomin"a"c. Podobnie, je"sli pierwszy
(o~najmniejszym indeksie) prawy w"eze"l brzegowy ma indeks $k<N-n$, to
ostatnie $N-n-k$ w"ez"l"ow i~punkt"ow kontrolnych jest zb"ednych.
Reprezentacje ze zb"ednymi w"ez"lami i~punktami kontrolnymi mog"a powsta"c
w~wyniku wstawiania w"ez"l"ow (np.\ podczas konwersji do postaci kawa"lkami
B\'{e}ziera), oraz w~wyniku wyznaczania B-sklejanej reprezentacji pochodnej
krzywej.

Krzyw"a B-sklejan"a stopnia~$n$, kt"orej w"ez"ly brzegowe maj"a krotno"s"c
wi"eksz"a lub r"own"a~$n$, nazywamy \unl{krzyw"a o ko"ncach zaczepionych}.
Je"sli ostatni (o najwi"ekszym indeksie) lewy w"eze"l brzegowy ma indeks~$k$
(poniewa"r w"ez"ly liczymy od $0$, wi"ec oczywi"scie $k\geq n$), to punkt
kontrolny $\bm{d}_{k-n}$ jest punktem krzywej odpowiadaj"acym parametrowi
$u_k$, tj.\ lewemu ko"ncowi dziedziny. Je"sli $k=n$, to jest to punkt
$\bm{d}_0$; w~przeciwnym razie punkty $\bm{d}_0,\ldots,\bm{d}_{k-n-1}$ nie
maj"a wp"lywu na kszta"lt krzywej (i~wraz z~w"ez"lami $u_0,\ldots,u_{k-n-1}$
s"a zb"edne). Podobna regu"la dotyczy prawego w"ez"la
brzegowego o najmniejszym indeksie --- je"sli jest to w"eze"l $u_{N-n}$
o~krotno"sci $n$ lub $n+1$, to punkt kontrolny $\bm{d}_{N-n-1}$ jest punktem
ko"ncowym krzywej (odpowiada on parametrowi $u_{N-n}$).

Krzywa, kt"orej w"ez"ly brzegowe maj"a krotno"s"c mniejsz"a ni"r stopie"n,
nazywa si"e \unl{krzyw"a o ko"ncach swobodnych}. Ka"rdy koniec krzywej
mo"re by"c zaczepiony lub swobodny niezale"rnie od drugiego ko"nca.

\vspace{\medskipamount}
\unl{Zamkni"ete krzywe B-sklejane} s"a reprezentowane tak samo
jak wszystkie krzywe B-sklejane. Aby krzywa B-sklejana stopnia~$n$ by"la
zamkni"eta, musz"a by"c spe"lnione nast"epuj"ace warunki: ci"ag w"ez"l"ow
$u_1,\ldots,u_{N-1}$ musi sk"lada"c si"e z~kolejnych element"ow
niesko"nczonego ci"agu liczb, takiego "re ci"ag r"o"rnic jest nieujemny
i~okresowy, o~okresie
\begin{align*}
  K=N-2n,
\end{align*}
przy czym musi by"c spe"lniony warunek $N>3n$.
Ci"ag w"ez"l"ow musi by"c niemalej"acy i~musi istnie"c
liczba dodatnia~$T$, taka "re
\begin{align*}
  u_{k+K}-u_k=T\qquad\mbox{dla $k=1,\ldots,2n-1$.}
\end{align*}
W"ez"ly $u_0$ i~$u_N$ s"a nieistotne dla kszta"ltu krzywej, ale musz"a
spe"lnia"c warunki $u_0\leq u_1$ i~$u_{N-1}\leq u_N$.

Ci"ag $\bm{d}_0,\ldots,\bm{d}_{N-n-1}$ musi sk"lada"c si"e
z~kolejnych element"ow niesko"nczonego okresowego ci"agu punkt"ow o~okresie
$N-2n$, a~zatem musz"a by"c spe"lnione r"owno"sci
\begin{align*}
  \bm{d}_{k+K} = \bm{d}_k\qquad\mbox{dla $k=0,\ldots,n-1$.}
\end{align*}

Aplikacja mo"re wykorzystywa"c ,,oszcz"edn"a'' reprezentacj"e zamkni"etej
krzywej B-sklejanej, w~kt"orej w"ez"ly $u_{K+1},\ldots,u_{N-1}$ oraz
punkty kontrolne $\bm{d}_{K},\ldots,\allowbreak\bm{d}_{N-n-1}$, mo"rliwe
do odtworzenia na podstawie powy"rszych warunk"ow, s"a nieobecne. Jednak aby
u"ry"c procedur z~biblioteki \texttt{libmultibs} trzeba utworzy"c
,,robocz"a'' reprezentacj"e krzywej, sk"ladaj"ac"a si"e z tablic
zawieraj"acych wszystkie w"ez"ly i~punkty kontrolne.

Obliczanie punktu i~wiele innych oblicze"n dla krzywych zamkni"etych mo"re
by"c wykonywane przez procedury biblioteki \texttt{libmultibs} przeznaczone
do przetwarzania ,,zwyk"lych'' krzywych o~ko"ncach swobodnych. Zmiany
reprezentacji takie jak wstawianie i~usuwanie w"ez"l"ow oraz podwy"rszanie
stopnia wymaga u"rycia procedur, kt"ore zapewni"a spe"lnienie podanych
wy"rej warunk"ow przez now"a reprezentacj"e krzywej. Procedury takie maj"a
w~nazwie s"lowo ,,\texttt{Closed}'', przy czym w~wi"ekszo"sci przypadk"ow
czekaj"a dopiero na napisanie.


\subsection{P"laty B-sklejane}

\unl{P"lat B-sklejany} jest okre"slony wzorem
\begin{align}\label{eq:BSpatch:def}
  \bm{s}(u,v) =
  \sum_{i=0}^{N-n-1}\sum_{j=0}^{M-m-1}\bm{d}_{ij}N^n_i(u) N^m_j(v),
\end{align}
w kt"orym wyst"epuj"a dwa uk"lady funkcji B-sklejanych stopni (w og"olno"sci
r"o"rnych) $n$ i $m$, oparte odpowiednio na (w~og"olno"sci r"o"rnych,
nawet je"sli $n=m$) ci"agach w"ez"l"ow $u_0,\ldots,u_N$ i $v_0,\ldots,v_M$.
Ka"rdy z~tych ci"ag"ow musi by"c niemalej"acy i~dostatecznie d"lugi (ma by"c
$N>2n$, $M>2m$, $u_n<u_{N-n}$ i~$v_m<v_{M-m}$). Do ka"rdego z~tych ci"ag"ow
stosuje si"e terminologia i~wszystkie uwagi podane w~poprzednim punkcie.

Tablica punkt"ow kontrolnych $\bm{d}_{ij}$, kt"ore razem z~w"ez"lami
stanowi"a reprezentacj"e p"lata, zawiera kolejno wsp"o"lrz"edne punkt"ow
$\bm{d}_{00},\bm{d}_{01},\ldots,\bm{d}_{0,N-n-1}$, a nast"epnie
$\bm{d}_{10},\bm{d}_{11},\ldots,\bm{d}_{1,N-n-1}$ itd., czyli kolejne
kolumny siatki kontrolnej p"lata.

Mi"edzy tymi kolumnami mog"a wyst"epowa"c
w~tablicy obszary nieu"rywane, kt"ore umo"rliwiaj"a np.\ wstawianie
w"ez"la do ci"agu $(v_i)$ w"ez"l"ow pocz"atkowej reprezentacji p"lata.
Odbywa si"e to tak, jakby w"eze"l by"l wstawiany do reprezentacji wielu
krzywych B-sklejanych, kt"orych "lamanymi kontrolnymi s"a kolumny siatki
kontrolnej p"lata. Po wstawieniu w"ez"la d"lugo"s"c ka"rdego obszaru
nieu"rywanego zmniejsza si"e o~$d$~miejsc na liczby zmiennopozycyjne (gdzie
$d$~jest wymiarem przestrzeni, w~kt"orej le"ry p"lat). Podzia"lk"a tablicy
w~tym przypadku jest odleg"lo"s"c pocz"atk"ow ci"ag"ow wsp"o"lrz"ednych
kolejnych kolumn.

Odpowiednikiem krzywych o ko"ncach zaczepionych i~krzywych o~ko"ncach
swobodnych s"a \unl{p"laty o~brzegach zaczepionych} i~\unl{swobodnych}.
Na przyk"lad, je"sli ci"ag w"ez"l"ow~,,$u$'' spe"lnia warunek
$u_1=\cdots=u_n<u_{n+1}$, to krzywa sta"lego parametru $u=u_n$ (jedna
z~czterech krzywych brzegowych p"lata) jest krzyw"a B-sklejan"a stopnia $m$,
opart"a na ci"agu w"ez"l"ow ,,$v$'', kt"orej "lamana kontrolna jest pierwsz"a
kolumn"a siatki kontrolnej p"lata. Oczywi"scie, ka"rdy z~czterech brzeg"ow
p"lata mo"re by"c zaczepiony albo swobodny niezale"rnie od pozosta"lych.

Odpowiednikiem krzywych zamkni"etych s"a p"laty zamkni"ete, kt"ore mog"a
by"c rurkami lub torusami. Jeden lub oba ci"agi w"ez"l"ow, a~tak"re ci"ag
wierszy lub kolumn siatki kontrolnej (traktowanych jako punkty) musz"a
spe"lnia"c warunki dla zamkni"etych krzywych B-sklejanych.


\subsection{Krzywe i p"laty NURBS}

\unl{Krzywe i p"laty NURBS} (ang.\ \textsl{non-uniform rational B-splines})
s"a to krzywe i~p"la\-ty powierzchni kawa"lkami wymiernych, kt"orych zwi"azek
z~krzywymi i~p"latami B-sklejanymi jest taki sam, jak zwi"azek wymiernych
krzywych i~p"lat"ow B\'{e}ziera z~wielomianowymi krzywymi i~p"latami
B\'{e}ziera. Zatem, mo"rna wybra"c jeden lub dwa ci"agi w"ez"l"ow oraz
uk"lad punkt"ow kontrolnych $\bm{d}_i$ lub $\bm{d}_{ij}$ w~przestrzeni
$d$-wymiarowej i~ka"rdemu punktowi przyporz"adkowa"c wag"e $w_i$ albo
$w_{ij}$. Nast"epnie wystarczy okre"sli"c wektory w~przestrzeni
$d+1$-wymiarowej
\begin{align}
  \bm{D}_i =
  \left[\begin{array}{c} w_i\bm{d}_i \\ w_i \end{array}\right]
  \qquad\mbox{albo}\qquad
  \bm{D}_{ij} =
  \left[\begin{array}{c} w_{ij}\bm{d}_{ij} \\ w_{ij} \end{array}\right]
\end{align}
i~oblicza"c punkty krzywej lub p"lata B-sklejanego (kt"orego to s"a punkty
kontrolne) po"lo"ronego w~tej przestrzeni, po czym dzieli"c pierwsze
$d$~wsp"o"lrz"ednych ka"rdego takiego punktu przez wsp"o"lrz"edn"a
$d+\mathord{\mbox{pierwsz"a}}$ (tzw.\ wagow"a), co daje w wyniku
wsp"o"lrz"edne kartezja"nskie odpowiednich punkt"ow krzywej lub p"lata
wymiernego.

Procedury w bibliotece \texttt{libmultibs} przetwarzaj"a w"la"snie tak"a
reprezentacj"e jednorodn"a.


\subsection{\label{ssect:Coons:patch:def}P"laty Coonsa}

P"laty Coonsa klasy~$C^k$ s"a to p"laty tensorowe okre"slone za pomoc"a
odpowiednio g"ladkich krzywych opisuj"acych brzeg i~tzw.\ pochodne
poprzeczne rz"edu $1,\ldots,k$. Dla dowolnego~$k\in\N$ p"lat Coonsa jest
okre"slony wzorem
\begin{align}\label{eq:Coons:patch:def}
  \bm{p}(u,v) = \bm{p}_1(u,v)+\bm{p}_2(u,v)-\bm{p}_3(u,v),
\end{align}
w~kt"orym
\begin{align*}
  \bm{p}_1(u,v) &{}= \bm{C}(u)\hat{H}(v)^T, \quad
  \bm{p}_2(u,v) = \tilde{H}(u)\bm{D}(v)^T, \quad
  \bm{p}_3(u,v) = \tilde{H}(u)\bm{P}\hat{H}(v)^T,
\end{align*}
przy czym
\begin{align*}
  \bm{C}(u) &{}= [\bm{c}_{00}(u),\bm{c}_{10}(u),\bm{c}_{01}(u),\bm{c}_{11}(u),
               \ldots,\bm{c}_{0k}(u),\bm{c}_{1k}(u)], \\
  \bm{D}(v) &{}= [\bm{d}_{00}(v),\bm{d}_{10}(v),\bm{d}_{01}(v),\bm{d}_{11}(v),
               \ldots,\bm{d}_{0k}(v),\bm{d}_{1k}(v)], \\
  \tilde{H}(u) &{}= [\tilde{H}_{00}(u),\tilde{H}_{10}(u),
                     \tilde{H}_{01}(u),\tilde{H}_{11}(u),\ldots,
                     \tilde{H}_{0k}(u),\tilde{H}_{1k}(u)], \\
  \hat{H}(v) &{}= [\hat{H}_{00}(v),\hat{H}_{10}(v),
                     \hat{H}_{01}(v),\hat{H}_{11}(v),\ldots,
                     \hat{H}_{0k}(v),\hat{H}_{1k}(v)].
\end{align*}
Krzywe $\bm{c}_{00},\ldots,\bm{c}_{1k}$ opisuj"a dwa przeciwleg"le brzegi
p"lata i~pochodne poprzeczne na tych brzegach. Krzywe te musz"a mie"c
identyczn"a dziedzin"e, kt"or"a oznaczymy~$[a,b]$. Podobnie, krzywe
$\bm{d}_{00},\ldots,\bm{d}_{1k}$ opisuj"a drug"a par"e przeciwleg"lych
brzeg"ow i~pochodne poprzeczne i~r"ownie"r musz"a mie"c identyczn"a dziedzin"e,
$[c,d]$. Dziedzin"a p"lata Coonsa jest prostok"at $[a,b]\times[c,d]$.

Macierz~$\bm{P}$ ma wymiary $(2k+2)\times(2k+2)$ i~sk"lada si"e z~punkt"ow
danych krzywych oraz wektor"ow ich pochodnych rz"edu~$1,\ldots,k$:
\begin{align}\label{eq:Coons:compat:cond}
  \bm{P} ={}& \left[\begin{array}{ccccc}
    \bm{c}_{00}(a) & \bm{c}_{10}(a) & \ldots & \bm{c}_{0k}(a) & \bm{c}_{1k}(a) \\
    \bm{c}_{00}(b) & \bm{c}_{10}(b) & \ldots & \bm{c}_{0k}(b) & \bm{c}_{1k}(b) \\
    \vdots & \vdots & & \vdots & \vdots \\
    \bm{c}^{(k)}_{00}(a) & \bm{c}^{(k)}_{10}(a) & \ldots &
    \bm{c}^{(k)}_{0k}(a) & \bm{c}^{(k)}_{1k}(a) \\
    \bm{c}^{(k)}_{00}(b) & \bm{c}^{(k)}_{10}(b) & \ldots &
    \bm{c}^{(k)}_{0k}(b) & \bm{c}^{(k)}_{1k}(b)
  \end{array}\right] = \nonumber \\
 &\left[\begin{array}{ccccc}
    \bm{d}_{00}(c) & \bm{d}_{00}(d) & \ldots & \bm{d}_{0k}(c) & \bm{d}_{0k}(d) \\
    \bm{d}_{10}(c) & \bm{d}_{10}(d) & \ldots & \bm{d}_{1k}(c) & \bm{d}_{1k}(d) \\
    \vdots & \vdots & & \vdots & \vdots \\
    \bm{d}^{(k)}_{00}(c) & \bm{d}^{(k)}_{00}(d) & \ldots &
    \bm{d}^{(k)}_{0k}(c) & \bm{d}^{(k)}_{0k}(d) \\
    \bm{d}^{(k)}_{10}(c) & \bm{d}^{(k)}_{10}(d) & \ldots &
    \bm{d}^{(k)}_{1k}(c) & \bm{d}^{(k)}_{1k}(d)
  \end{array}\right].
\end{align}
Krzywe okre"slaj"ace p"lat musz"a spe"lnia"c warunki zgodno"sci, przedstawione
wy"rej w~postaci r"owno"sci macierzy.

Funkcje $\tilde{H}_{mj}(u)$ i~$\hat{H}_{mj}(v)$ tworz"a tzw.\ lokalne bazy
Hermite'a. Jest przyj"ete za"lo"renie,
"re funkcje te s"a wielomianami stopnia $2k+1$ dla $k\in\{1,2\}$, cho"c mo"rna
by u"ry"c zamiast nich innych funkcji klasy~$C^k$, np.\ funkcji sklejanych stopnia
$k+1$. Funkcje te s"a okre"slone wzorem
\begin{align*}
  \tilde{H}_{mj}(u) = (b-a)^jH_{mj}\Bigl(\frac{u-a}{b-a}\Bigr),\qquad
  \hat{H}_{mj}(v) = (d-c)^jH_{mj}\Bigl(\frac{v-c}{d-c}\Bigr).
\end{align*}

Dla $k=1$ jest
\begin{alignat*}{2}
  H_{00}(t) &{}= B^3_0(t)+B^3_1(t),\qquad &
  H_{10}(t) &{}= B^3_2(t)+B^3_3(t),\\
  H_{01}(t) &{}= \frac{1}{3}B^3_1(t),\qquad &
  H_{11}(t) &{}= -\frac{1}{3}B^3_2(t).
\end{alignat*}
Poniewa"r wielomiany u"ryte do okre"slenia p"lata, tj.\ interpolacji
w~obu kierunkach mi"edzy zadanymi krzywymi, s"a dla $k=1$ trzeciego stopnia,
wi"ec p"laty Coonsa klasy~$C^1$ s"a nazywane \textbf{p"latami bikubicznymi
Coonsa}, cho"c p"lat taki mo"re by"c okre"slony przez krzywe dowolnego
stopnia.

Dla $k=2$ p"lat jest okre"slony przy u"ryciu wielomian"ow pi"atego stopnia,
\begin{alignat*}{2}
  H_{00}(t) &{}= B^5_0(t)+B^5_1(t)+B^5_2(t),\qquad &
  H_{10}(t) &{}= B^5_3(t)+B^5_4(t)+B^5_5(t), \\
  H_{01}(t) &{}= \frac{1}{5}B^5_1(t)+\frac{2}{5}B^5_2(t), \qquad &
  H_{11}(t) &{}= -\frac{2}{5}B^5_3(t)-\frac{1}{5}B^5_4(t), \\
  H_{02}(t) &{}= \frac{1}{20}B^5_2(t), \qquad &
  H_{12}(t) &{}= \frac{1}{20}B^5_3(t),
\end{alignat*}
w~zwi"azku z~czym p"laty Coonsa klasy~$C^2$ s"a nazywane \textbf{dwupi"etnymi
p"latami Coonsa}.

P"laty Coonsa (bikubiczne i~dwupi"etne) mog"a by"c okre"slone za pomoc"a
krzywych wielomianowych albo sklejanych. W~pierwszym przypadku krzywe te s"a
krzywymi B\'{e}ziera, a~za dziedzin"e p"lata przyjmuje si"e kwadrat $[0,1]^2$.
Poszczeg"olne krzywe mog"a mie"c r"o"rne stopnie.

Dziedzina p"lat"ow okre"slonych za pomoc"a krzywych sklejanych (o~reprezentacji
B-sklejanej) mo"re by"c dowolnym prostok"atem $[a,b]\times[c,d]$. Krzywe
okre"slaj"ace p"lat mog"a mie"c reprezentacje r"o"rnych stopni, reprezentacje
te mog"a r"ownie"r mie"c r"o"rne ci"agi w"ez"l"ow (ale musz"a mie"c identyczne
dziedziny, wyznaczone przez w"ez"ly brzegowe).

Biblioteka \texttt{libmultibs} zawiera procedury wyznaczaj"ace reprezentacj"e
B\'{e}ziera lub B-sklejan"a p"lata Coonsa na podstawie okre"slaj"acych go
krzywych, a~tak"re procedury szybkiego tablicowania p"lat"ow Coonsa razem
z~pochodnymi; procedury te znalaz"ly zastosowanie w~bibliotekach
\texttt{libg1hole} i~\texttt{libg2hole}.


\subsection{Nazwy procedur i parametr"ow}

Jako u"latwienie dla u"rytkownika jest pomy"slany spos"ob nazywania procedur
(kt"orych nazwy maj"a u"latwi"c odgadni"ecie spe"lnianej funkcji)
i ich parametr"ow formalnych (kt"ore, je"sli maj"a tak"a sam"a nazw"e
w~r"o"rnych procedurach, to oznaczaj"a to samo).

Ka"rda procedura i~makro przeznaczone do wywo"lywania jako procedura
w~bibliotece \texttt{libmultibs} ma nazw"e zaczynaj"ac"a si"e od prefiksu
\texttt{mbs\_}.

Je"sli bezpo"srednio po prefiksie wyst"epuje ci"ag znak"ow \texttt{multi},
to procedura s"lu"ry do przetwarzania jednej lub wielu krzywych (liczba
krzywych jest warto"sci"a parametru, kt"ory ma nazw"e \texttt{ncurves}).

Ko"nc"owka nazwy sk"lada si"e z~dw"och cz"e"sci. Pierwsza cz"e"s"c mo"re
by"c pusta, je"sli po prefiksie jest ,,\texttt{multi}'', albo
okre"sla rodzaj krzywej lub p"lata przetwarzanego przez dan"a procedur"e lub
makro. Litera \texttt{C} oznacza krzyw"a za"s \texttt{P} p"lat. Cyfra
okre"s\-la wymiar przestrzeni (np.\ \texttt{2} oznacza p"laszczyzn"e),
w~kt"orej le"ry krzywa lub p"lat. Litera \texttt{R} po cyfrze oznacza
krzyw"a lub p"lat wymierny w reprezentacji jednorodnej. \textbf{Uwaga:} punkty
kontrolne maj"a wtedy o~jedn"a wsp"o"lrz"edn"a wi"ecej. Druga cz"e"s"c
ko"nc"owki jest liter"a \texttt{f} albo \texttt{d} i~wskazuje odpowiednio
precyzj"e (pojedyncz"a, \texttt{float}, albo podw"ojn"a, \texttt{double}%
\footnote{W~,,powa"rnych'' zastosowaniach nale"ry
stosowa"c tylko podw"ojn"a precyzj"e, chyba, "re i~ona nie wystarczy.})
arytmetyki zmiennopozycyjnej, w~kt"orej procedura otrzymuje dane
i~wyprowadza wyniki.

"Srodkowa cz"e"s"c nazwy okre"sla funkcj"e wykonywan"a przez procedur"e.
Makra i~procedury o~tej samej nazwie r"o"rni"a si"e zastosowaniem ---
uniwersalnym (je"sli jest przedrostek ,,\texttt{multi}'') albo do krzywych
lub p"lat"ow w~przestrzeni o~wymiarze okre"slonym przez prefiks. Lista tych
nazw zawiera m.in.
\begin{mydescription}
  \item\texttt{deBoor} --- obliczanie punkt"ow krzywych i~p"lat"ow
    B-sklejanych za pomoc"a algorytmu de~Boora.
  \item\texttt{deBoorDer} --- obliczanie punkt"ow i~pochodnych
    krzywych i~p"lat"ow B-sklejanych za pomoc"a algorytmu de~Boora.
  \item\texttt{BCHorner} --- obliczanie punkt"ow krzywych i~p"lat"ow
    B\'{e}ziera za pomoc"a schematu Hornera.
  \item\texttt{BCHornerDer} --- obliczanie punkt"ow i~pochodnych
    krzywych i~p"lat"ow B\'{e}ziera za pomoc"a schematu Hornera.
  \item\texttt{BCFrenet} --- obliczanie krzywizn i~wektor"ow uk"ladu Freneta
    w~danym punkcie krzywej B\'{e}ziera.
  \item\texttt{BCHornerNv} --- obliczanie wektora normalnego w~danym punkcie
    p"lata B\'{e}ziera.
  \item\texttt{KnotIns} --- wstawianie (jednego) w"ez"la przy u"ryciu
    algorytmu Boehma.
  \item\texttt{KnotRemove} --- usuwanie (jednego) w"ez"la.
  \item\texttt{*Oslo*} --- procedury zwi"azane z~wstawianiem i~usuwaniem
    (w~jednym kroku) wielu w"ez"l"ow przy u"ryciu algorytmu Oslo.
  \item\texttt{MaxKnotIns} --- wstawianie w"ez"l"ow w~celu
    otrzymania reprezentacji \mbox{B-sklejanej} z~wszystkimi w"ez"lami
    wewn"etrznymi o~krotno"sci $n+1$ i~brzegowymi o~krotno"sci~$n$ lub
    $n+1$, czyli reprezentacji kawa"lkami B\'{e}ziera.
  \item\texttt{BisectB} --- podzia"l krzywych B\'{e}ziera na "luki
    zwi"azany z~podzia"lem dziedziny na dwa odcinki o~tej samej d"lugo"sci,
    algorytmem de~Casteljau.
  \item\texttt{DivideB} --- podzia"l krzywych B\'{e}ziera na "luki
    zwi"azany z~podzia"lem dziedziny na dwa odcinki o~dowolnych d"lugo"sciach,
    algorytmem de~Casteljau.
  \item\texttt{BCDegElev} --- podwy"rszanie stopnia krzywych i~p"lat"ow
    B\'{e}ziera.
  \item\texttt{BSDegElev} --- podwy"rszanie stopnia krzywych i~p"lat"ow
    B-sklejanych.
  \item\texttt{MultBez} --- mno"renie wielomian"ow i~krzywych B\'{e}ziera.
  \item\texttt{MultBS} --- mno"renie funkcji i~krzywych sklejanych.
  \item\texttt{BezNormal} --- wyznaczanie p"lata B\'{e}ziera
    opisuj"acego wektor normalny danego p"lata B\'{e}ziera.
  \item\texttt{BSCubicInterp} --- konstrukcja B-sklejanych interpolacyjnych
    krzywych trzeciego stopnia.
  \item\texttt{ConstructApproxBS} --- konstrukcja B-sklejanych krzywych
    aproksymacyjnych.
  \item\texttt{Closed} --- procedura, w~kt"orej nazwie wyst"epuje to s"lowo
    s"lu"ry do przetwarzania krzywych zamkni"etych.
\end{mydescription}

\underline{Parametry formalne} procedur i~makr maj"a nast"epuj"ace nazwy:
\begin{mydescription}
  \item\texttt{spdimen} --- okre"sla wymiar $d$ przestrzeni, w~kt"orej le"r"a
    krzywe, czyli liczb"e wsp"o"lrz"ednych ka"rdego punktu tej przestrzeni.
    Je"sli ko"nc"owka nazwy procedury lub makra zawiera liter"e \texttt{R},
    kt"ora wskazuje na obiekt wymierny, to punkty kontrolne maj"a o~$1$
    wi"ecej wsp"o"lrz"edn"a.
  \item\texttt{degree} ---okre"sla stopie"n reprezentacji krzywej. Stopie"n
    p"lata ze wzgl"e\-du na ka"rdy z~jego parametr"ow jest okre"slany za
    pomoc"a parametr"ow, kt"ore maj"a nazwy \texttt{degreeu}
    i~\texttt{degreev}.
  \item\texttt{lastknot} --- okre"sla liczb"e $N$, kt"ora jest indeksem
    ostatniego w"ez"la, a~zatem ci"ag w"ez"l"ow sk"lada si"e z~$N+1$
    w"ez"l"ow.
  \item\texttt{knots} --- tablica liczb zmiennopozycyjnych, w~kt"orej podaje
    si"e w"ez"ly. Dwie tablice z~ci"agami w"ez"l"ow sk"ladaj"acymi si"e na
    reprezentacj"e p"lata B-sklejanego przekazuje si"e przy u"ryciu
    parametr"ow o~nazwach \texttt{knotsu} i~\texttt{knotsv}.
  \item\texttt{ctlpoints} --- tablica punkt"ow kontrolnych. W~przypadku, gdy
    wymiar~$d$ przes\-trze\-ni jest r"owny~$1$ (procedura lub makro s"lu"ry do
    przetwarzania funkcji skalarnych), parametr poprzez kt"ory przekazuje
    si"e odpowiedni"a tablic"e nazywa si"e \texttt{coeff}.
  \item\texttt{pitch} --- podzia"lka tablicy punkt"ow kontrolnych, tj.\
    r"o"rnica indeks"ow pierwszych wsp"o"lrz"ednych pierwszych punkt"ow
    kolejnych "lamanych kontrolnych lub kolumn siatki kontrolnej w~tablicy
    \texttt{ctlpoints}. Zawsze tablice takie s"a traktowane jako tablice
    \emph{liczb zmiennopozycyjnych}, a~zatem jednostka podzia"lki jest to
    d"lugo"s"c reprezentacji zmiennopozycyjnej jednej liczby (nawet je"sli
    parametr formalny \texttt{ctlpoints} jest typu np.\ \texttt{point3f}).
\end{mydescription}
Je"sli parametry procedury s"lu"r"a do przekazania dw"och reprezentacji,
np.\ procedura na podstawie reprezentacji wej"sciowej konstruuje
reprezentacj"e wynikow"a, to odpowiednie parametry maj"a nazwy rozszerzone
o~fragment \texttt{in} albo \texttt{out}. Parametry s"lu"r"ace do
przekazania danych wej"sciowych wyst"epuj"a w~listach parametr"ow
\emph{przed} parametrami opisuj"acymi dane wyj"sciowe.



