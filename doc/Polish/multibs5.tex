
%/* //////////////////////////////////////////////////// */
%/* This file is a part of the BSTools procedure package */
%/* written by Przemyslaw Kiciak.                        */
%/* //////////////////////////////////////////////////// */

\newpage
\section[Dzia"lania algebraiczne na funkcjach i krzywych sklejanych]%
{Dzia"lania algebraiczne na funkcjach i krzywych \\ sklejanych}

\begin{sloppypar}
Zadaniem opisanych w~tym punkcie procedur jest wyznaczanie
B-sklejanej reprezentacji sumy (wektorowych) krzywych B-sklejanych oraz
iloczynu (skalarnych) funkcji i~krzywych (wektorowych). Dzia"lania te mog"a
by"c potrzebne w~r"o"rnych konstrukcjach, na przyk"lad powierzchni
wykazuj"acych ci"ag"lo"s"c geometryczn"a.
\end{sloppypar}


\subsection{Dodawanie funkcji i~krzywych sklejanych}

Dodanie, tj.\ wyznaczenie reprezentacji sumy, krzywych B-sklejanych wymaga
okre"s\-le\-nia stopnia i~ci"agu w"ez"l"ow tej reprezentacji. Stopie"n
reprezentacji sumy jest najwi"ekszym stopniem sk"ladnik"ow. Ci"ag w"ez"l"ow
jest okre"slony przez ci"agi w"ez\-"l"ow sk"ladnik"ow, kt"ore musz"a
wyznacza"c t"e sam"a dziedzin"e.
Aby doda"c krzywe (kt"ore musz"a le"re"c w~przestrzeni o~tym samym
wymiarze), trzeba znale"z"c reprezentacje tych krzywych, o~stopniu i~ci"agu
w"ez"l"ow, kt"ore b"ed"a u"rywane do reprezentowania sumy (i~to zadanie
pomocnicze jest najbardziej skomplikowanym i~kosztownym elementem algorytmu
dodawania). Ostatnim i~najprostszym elementem procedury jest sumowanie
wsp"o"lczynnik"ow (tj.\ punkt"ow kontrolnych) sk"ladnik"ow.

\vspace{\bigskipamount}
\cprog{%
boolean mbs\_FindBSCommonKnotSequencef ( int *degree, int *lastknot, \\
\ind{30}float **knots, int nsequences, ...\ );}
Procedura \texttt{mbs\_FindBSCommonKnotSequencef} otrzymuje $k$~ci"ag"ow
w"ez"l"ow, b"ed"acych podstaw"a reprezentacji krzywych B-sklejanych
podanych stopni. Zadaniem procedury jest znalezienie minimalnego
stopnia i~ci"agu w"ez"l"ow, wystarczaj"acego do reprezentowania sumy
tych krzywych. Znaleziony stopie"n~$n$ jest najwi"ekszym spo"sr"od
podanych stopni krzywych, lub pocz"atkow"a warto"sci"a
zmiennej~\texttt{*degree}, je"sli jest wi"eksza (mo"rna zatem wymusi"c
poszukiwanie reprezentacji wy"rszego stopnia). Znaleziony ci"ag w"ez"l"ow
ma nast"epuj"ace w"lasno"sci:
\begin{itemize}
\item W"ez"ly brzegowe maj"a krotno"s"c $n+1$ (zatem w"ez"ly zewn"etrzne,
  w~tym skrajne, pokrywaj"a si"e z~w"ez"lami brzegowymi).
\item Wyst"epuj"a w nim wszystkie w"ez"ly wewn"etrzne ci"ag"ow w"ez"l"ow
  danych.
\item Krotno"sci w"ez"l"ow wewn"etrznych s"a dobrane tak, aby po
  podwy"rszeniu stopnia ka"rdej krzywej do~$n$ mo"rna j"a by"lo
  reprezentowa"c z~tym ci"agiem w"ez"l"ow.
\end{itemize}

Parametry \texttt{degree}, \texttt{lastknot} i~\texttt{knots} s"lu"r"a do
wyprowadzenia wynik"ow (z~uwagi na sk"ladni"e~C wyst"epuj"a one na pocz"atku
listy parametr"ow, co jest odst"epstwem od konwencji przyj"etej w~ca"lym
pakiecie BSTools). Zmienne wskazywane przez te parametry otrzymuj"a
warto"sci, kt"orymi s"a odpowiednio stopie"n reprezentacji, numer ostatniego
w"ez"la w~znalezionym ci"agu i~wska"znik do tablicy z~tymi w"ez"lami.

\vspace{\medskipamount}
\noindent
\textbf{Uwaga:} Procedura rezerwuje t"e tablic"e na stosie pami"eci
pomocniczej, a~zatem wywo"luj"acy j"a podprogram jest odpowiedzialny za
zwolnienie tej rezerwacji (za pomoc"a procedury \texttt{pkv\_FreeScratchMem}
albo \texttt{pkv\_SetScratchMemTop}).

\vspace{\medskipamount}
Parametr \texttt{nsequences} okre"sla liczb"e~$k$ danych ci"ag"ow w"ez"l"ow
(musi by"c $k\geq 1$). W~wywo"laniu procedury nale"ry po nim poda"c
$3k$~parametr"ow. Kolejne tr"ojki parametr"ow opisuj"a kolejne ci"agi.
Pierwszym elementem tr"ojki jest stopie"n $n_i$ krzywej (typu \texttt{int}),
drugim elementem tr"ojki jest indeks $N_i$ ostatniego w"ez"la (typu
\texttt{int}), a~trzeci element tr"ojki jest wska"znikiem tablicy
z~$N_i+1$ liczbami zmiennopozycyjnymi reprezentuj"acymi w"ez"ly (parametr
ten ma typ \texttt{float*}).

We wszystkich ci"agach danych w"eze"l o~numerze $n_i$ musi by"c identyczny;
to samo dotyczy w"ez"la o~numerze $N_i-n_i$.

Warto"sci"a procedury jest \texttt{true}, je"sli obliczenie zako"nczy"lo
si"e sukcesem, albo \texttt{false} w~przeciwnym razie. Przyczyn"a
niepowodzenia mog"a by"c b"l"edne dane lub brak miejsca na stosie pami"eci
pomocniczej.

\vspace{\bigskipamount}
\cprog{%
boolean mbs\_multiAdjustBSCRepf ( int ncurves, int spdimen, \\
\ind{12}int indegree, int inlastknot, const float *inknots, \\
\ind{12}int inpitch, const float *inctlpoints, \\
\ind{12}int outdegree, int outlastknot, const float *outknots, \\
\ind{12}int outpitch, float *outctlpoints );}
\begin{sloppypar}
Procedura \texttt{mbs\_multiAdjustBSCRepf} ,,uzgadnia'' reprezentacj"e
krzywych B-sklejanych, tj.\ wyznacza reprezentacj"e potrzebnego stopnia,
opart"a na wskazanym ci"agu w"ez"l"ow. Polega to na podwy"rszeniu stopnia,
je"sli stopie"n pocz"atkowy jest za ma"ly, a~nast"epnie wstawieniu
,,brakuj"acych'' w"ez"l"ow (za pomoc"a algorytmu Oslo). Aby doda"c
$k$~krzywych B-sklejanych o~r"o"rnych reprezentacjach (ale o~tej samej
dziedzinie), nale"ry wyznaczy"c stopie"n i~ci"ag w"ez"l"ow ich sumy
(za pomoc"a \texttt{mbs\_FindBSCommonKnotSequencef}), a~nast"epnie
wyznaczy"c odpowiedni"a reprezentacj"e ka"rdego sk"ladnika, wywo"luj"ac
procedur"e \texttt{mbs\_multiAdjustBSCRepf}.%
\end{sloppypar}

Parametry: \texttt{ncurves} --- liczba krzywych, \texttt{spdimen} --- wymiar
przestrzeni, w~kt"orej one le"r"a, \texttt{indegree}, \texttt{inlastknot},
\texttt{inknots} --- stopie"n, indeks ostatniego w"ez"la i~tablica w"ez"l"ow
reprezentacji danej, \texttt{inpitch} --- podzia"lka tablicy
\texttt{inctlpoints}, w~kt"orej s"a podane punkty kontrolne reprezentacji
danej.

Parametry \texttt{outdegree}, \texttt{outlastknot} i~\texttt{outknots}
opisuj"a stopie"n i~ci"ag w"ez"l"ow reprezentacji, kt"or"a procedura ma
wyznaczy"c. Punkty kontrolne tej reprezentacji s"a wpisywane do tablicy
\texttt{outctlpoints} o~podzia"lce \texttt{outpitch}.

Warto"sci"a procedury jest \texttt{true} w~razie sukcesu oraz \texttt{false}
w~razie pora"zki (spowodowanej b"l"ednymi danymi lub brakiem miejsca na
stosie pami"eci pomocniczej).

\vspace{\bigskipamount}
\cprog{%
\#define mbs\_AdjustBSCRepC1f(indegree,inlastknot,inknots, \bsl \\
\ind{4}inctlpoints,outdegree,outlastknot,outknots,outctlpoints) \bsl \\
\ind{2}mbs\_multiAdjustBSCRepf (1,1,indegree,inlastknot,inknots,0, \bsl \\
\ind{4}inctlpoints,outdegree,outlastknot,outknots,0,outctlpoints) \\
\#define mbs\_AdjustBSCRepC2f(indegree,inlastknot,inknots, \bsl \\
\ind{4}inctlpoints,outdegree,outlastknot,outknots,outctlpoints) \bsl \\
\ind{2}mbs\_multiAdjustBSCRepf (1,2,indegree,inlastknot,inknots,0, \bsl \\
\ind{4}(float*)inctlpoints,outdegree,outlastknot,outknots,0, \bsl \\
\ind{4}(float*)outctlpoints) \\
\#define mbs\_AdjustBSCRepC3f(indegree,inlastknot,inknots, \bsl \\
\ind{4}inctlpoints,outdegree,outlastknot,outknots,outctlpoints) ... \\
\#define mbs\_AdjustBSCRepC4f(indegree,inlastknot,inknots, \bsl \\
\ind{4}inctlpoints,outdegree,outlastknot,outknots,outctlpoints) ...}

\vspace{\bigskipamount}
\cprog{%
void mbs\_multiAddBSCurvesf ( int ncurves, int spdimen, \\
\ind{19}int degree1, int lastknot1, const float *knots1, \\
\ind{19}int pitch1, const float *ctlpoints1, \\
\ind{19}int degree2, int lastknot2, const float *knots2, \\
\ind{19}int pitch2, const float *ctlpoints2, \\
\ind{19}int *sumdeg, int *sumlastknot, float *sumknots, \\
\ind{19}int sumpitch, float *sumctlpoints );}
Procedura \texttt{mbs\_multiAddBSCurvesf} oblicza sumy \texttt{ncurves}
par krzywych B-sklejanych w~przestrzeni o~wymiarze \texttt{spdimen}.

Pierwsza krzywa w~parze jest okre"slona za~pomoc"a parametr"ow
\texttt{degree1} (stopie"n), \texttt{lastknot1} (indeks ostatniego w"ez"la),
\texttt{knots1} (tablica w"ez"l"ow) i~\texttt{ctlpoints1} (tablica punkt"ow
kontrolnych o~podzia"lce \texttt{pitch1}).

Druga krzywa w~parze jest podobnie okre"slona za pomoc"a parametr"ow
\texttt{degree2}, \texttt{lastknot2}, \texttt{knots2}, \texttt{pitch2}
i~\texttt{ctlpoints2}.

\begin{sloppypar}\hyphenpenalty=200
Parametry wyj"sciowe to \texttt{*sumdeg} (otrzymuje warto"s"c stopnia
reprezentacji sumy), \texttt{*sumlastknot} (indeks ostatniego w"ez"la
reprezentacji sumy), \texttt{sumknots} (tablica, do kt"orej procedura
wstawia w"ez"ly reprezentacji sumy), \texttt{sumctlpoints} (tablica
o~podzia"lce~\texttt{sumpitch}, do~kt"orej procedura wstawia punkty
kontrolne sum).
\end{sloppypar}

\vspace{\bigskipamount}
\cprog{%
\#define mbs\_AddBSCurvesC1f(degree1,lastknot1,knots1,ctlpoints1, \bsl \\
\ind{4}degree2,lastknot2,knots2,ctlpoints2, \bsl \\
\ind{4}sumdeg,sumlastknot,sumknots,sumctlpoints) \bsl \\
\ind{2}mbs\_multiAddBSCurvesf (1,1,degree1,lastknot1,knots1,0, \bsl \\
\ind{4}ctlpoints1,degree2,lastknot2,knots2,0,ctlpoints2, \bsl \\
\ind{4}sumdeg,sumlastknot,sumknots,0,sumctlpoints) \\
\#define mbs\_AddBSCurvesC2f(degree1,lastknot1,knots1,ctlpoints1, \bsl \\
\ind{4}degree2,lastknot2,knots2,ctlpoints2, \bsl \\
\ind{4}sumdeg,sumlastknot,sumknots,sumctlpoints) \bsl \\
\ind{2}mbs\_multiAddBSCurvesf (1,2,degree1,lastknot1,knots1,0, \bsl \\
\ind{4}(float*)ctlpoints1, \bsl \\
\ind{4}degree2,lastknot2,knots2,0,(float*)ctlpoints2, \bsl \\
\ind{4}sumdeg,sumlastknot,sumknots,0,(float*)sumctlpoints) \\
\#define mbs\_AddBSCurvesC3f(degree1,lastknot1,knots1,ctlpoints1, \bsl \\
\ind{4}degree2,lastknot2,knots2,ctlpoints2, \bsl \\
\ind{4}sumdeg,sumlastknot,sumknots,sumctlpoints) ... \\
\#define mbs\_AddBSCurvesC4f(degree1,lastknot1,knots1,ctlpoints1, \bsl \\
\ind{4}degree2,lastknot2,knots2,ctlpoints2, \bsl \\
\ind{4}sumdeg,sumlastknot,sumknots,sumctlpoints) ...}

\begin{figure}[ht]
\centerline{\epsfig{file=addspl.ps}}
\caption{Funkcje sklejane stopni $3$ i~$4$ i~ich suma}
\end{figure}


\vspace{\bigskipamount}
\cprog{%
void mbs\_multiSubtractBSCurvesf ( int ncurves, int spdimen, \\
\ind{19}int degree1, int lastknot1, const float *knots1, \\
\ind{19}int pitch1, const float *ctlpoints1, \\
\ind{19}int degree2, int lastknot2, const float *knots2, \\
\ind{19}int pitch2, const float *ctlpoints2, \\
\ind{19}int *sumdeg, int *sumlastknot, float *sumknots, \\
\ind{19}int sumpitch, float *sumctlpoints );}

\vspace{\bigskipamount}
\cprog{%
\#define mbs\_SubtractBSCurvesC1f(degree1,lastknot1,knots1, \bsl \\
\ind{4}ctlpoints1,degree2,lastknot2,knots2,ctlpoints2, \bsl \\
\ind{4}sumdeg,sumlastknot,sumknots,sumctlpoints) \bsl \\
\ind{2}mbs\_multiSubtractBSCurvesf (1,1,degree1,lastknot1,knots1,0, \bsl \\
\ind{4}ctlpoints1,degree2,lastknot2,knots2,0,ctlpoints2, \bsl \\
\ind{4}sumdeg,sumlastknot,sumknots,0,sumctlpoints) \\
\#define mbs\_SubtractBSCurvesC2f(degree1,lastknot1,knots1, \bsl \\
\ind{4}ctlpoints1,degree2,lastknot2,knots2,ctlpoints2, \bsl \\
\ind{4}sumdeg,sumlastknot,sumknots,sumctlpoints) \bsl \\
\ind{2}mbs\_multiSubtractBSCurvesf (1,2,degree1,lastknot1,knots1,0, \bsl \\
\ind{4}(float*)ctlpoints1,degree2,lastknot2,knots2,0, \bsl \\
\ind{4}(float*)ctlpoints2,sumdeg,sumlastknot,sumknots,0, \bsl \\
\ind{4}(float*)sumctlpoints) \\
\#define mbs\_SubtractBSCurvesC3f(degree1,lastknot1,knots1, \bsl \\
\ind{4}ctlpoints1,degree2,lastknot2,knots2,ctlpoints2, \bsl \\
\ind{4}sumdeg,sumlastknot,sumknots,sumctlpoints) ... \\
\#define mbs\_SubtractBSCurvesC4f(degree1,lastknot1,knots1, \bsl \\
\ind{4}ctlpoints1,degree2,lastknot2,knots2,ctlpoints2, \bsl \\
\ind{4}sumdeg,sumlastknot,sumknots,sumctlpoints) ...}


\newpage
\subsection{Przej"scie mi"edzy bazami Bernsteina i skalowanymi}

W~tym punkcie s"a opisane procedury pomocnicze u"rywane do mno"renia funkcji
i~krzywych sklejanych.

Aby pomno"ry"c wielomiany dane za pomoc"a wsp"o"lczynnik"ow w bazach
Bernsteina, wygodnie jest przej"s"c do tzw.\ \emph{baz skalowanych}. Baza
skalowana stopnia $n$ sk"lada si"e z wielomian"ow
\begin{align}
  b^n_i(t) \stackrel{\mathrm{def}}{=} \frac{1}{\binom{n}{i}}B^n_i(t) =
  t^i(1-t)^{n-i}.
\end{align}
Zatem wsp"o"lczynniki wielomianu w bazie skalowanej otrzymamy mno"r"ac
wsp"o"lczynniki tego wielomianu w bazie Bernsteina przez liczby
$\binom{n}{i}$.

Wynik mno"renia wielomian"ow reprezentowanych w bazach skalowanych stopni
$n$~i~$m$ otrzymujemy w~postaci uk"ladu wsp"o"lczynnik"ow w bazie skalowanej
stopnia $n+m$, po czym mo"remy przej"s"c do bazy Bernsteina stopnia $n+m$
wykonuj"ac odpowiednie dzielenia.

\vspace{\bigskipamount}
\cprog{%
void mbs\_multiBezScalef ( int degree, int narcs, \\
\ind{26}int ncurves, int spdimen, \\
\ind{26}int pitch, float *ctlpoints );}
Procedura \texttt{mbs\_multiBezScalef} otrzymuje tablic"e krzywych
B\'{e}ziera stopnia \texttt{degree} w~przestrzeni o~wymiarze \texttt{spdimen}
i~oblicza wsp"o"lczynniki tych krzywych w~bazie skalowanej. Przyj"ete jest
za"lo"renie, "re krzywe te to kolejne "luki krzywych B-sklejanych, kt"ore
zosta"ly otrzymane przez odpowiednie wstawienie w"ez"l"ow (np.\ za pomoc"a
procedury \texttt{mbs\_multiMaxKnotInsf}).

\begin{sloppypar}
Parametry: \texttt{degree} --- stopie"n krzywych, \texttt{narcs} --- liczba
"luk"ow B\'{e}ziera wchodz"acych w~sk"lad ka"rdej krzywej B-sklejanej,
\texttt{ncurves} --- liczba krzywych B-sklejanych, \texttt{spdimen} ---
wymiar~$d$ przestrzeni, w~kt"orej le"r"a krzywe.
\end{sloppypar}

Parametr \texttt{pitch} okre"sla podzia"lk"e tablicy \texttt{ctlpoints},
w~kt"orej przed wywo"laniem procedury s"a podane punkty kontrolne krzywych
(tj.\ ich wsp"o"lczynniki w~bazach Bernsteina stopnia $n=$\texttt{degree}),
a~po jej zako"nczeniu wsp"o"lczynniki w~bazach skalowanych.
Warto"s"c parametru \texttt{pitch} okre"sla odleg"lo"s"c pocz"atk"ow
pierwszych punkt"ow kontrolnych kolejnych krzywych \emph{B-sklejanych}.
Reprezentacje kolejnych krzywych B\'{e}ziera zajmuj"a zawsze $(n+1)d$
miejsc i~nie ma mi"edzy nimi wolnych miejsc. Podzia"lka tablicy nie mo"re
by"c mniejsza ni"r $(n+1)d$\texttt{*narcs}.

\vspace{\bigskipamount}
\cprog{%
void mbs\_multiBezUnscalef ( int degree, int narcs, \\
\ind{28}int ncurves, int spdimen, \\
\ind{28}int pitch, float *ctlpoints );}
\begin{sloppypar}
Procedura \texttt{mbs\_multiBezUnscalef} otrzymuje tablic"e reprezentacji
krzywych wielomianowych w~bazie skalowanej i~dokonuje przej"scia do bazy
Bernsteina, tj.\ do reprezentacji B\'{e}ziera. Parametry tej procedury
(z~wyjatkiem opisu pocz"atkowej i~ko"ncowej zawarto"sci tablicy
\texttt{ctlpoints}) s"a identyczne jak parametry procedury
\texttt{mbs\_multiBezScalef}.
\end{sloppypar}


\subsection{Mno"renie funkcji i~krzywych sklejanych}

W tym punkcie s"a opisane procedury mno"renia wielomianowych i~sklejanych
krzywych (tj.\ funkcji wektorowych) przez skalarne funkcje wielomianowe
i~sklejane. Dane dla procedur sk"ladaj"a si"e z~reprezentacji jednej lub
wi"ekszej liczby funkcji skalarnych (wielomian"ow lub funkcji sklejanych
jednej zmiennej) $s_i$ oraz jednej lub wi"ekszej liczby krzywych
(wielomianowych lub sklejanych) $\bm{v}_i$. Obie te liczby musz"a by"c
r"owne, lub jedna z~nich musi by"c r"owna $1$. Procedury obliczaj"a
reprezentacj"e B\'{e}ziera lub B-sklejan"a funkcji wektorowych
\begin{align*}
  \bm{w}_i(t) = s_i(t)\bm{v}_i(t),
\end{align*}
przy czym je"sli jest tylko jedna funkcja skalarna $s_0$ i~wi"ecej funkcji
wektorowych, to przyjmuje si"e, "re ka"rda z tych funkcji wektorowych
b"edzie pomno"rona przez t"e jedn"a funkcj"e $s_0$ i podobnie, je"sli jest
wiele funkcji skalarnych $s_i$ i~jedna funkcja wektorowa $\bm{v}_0$, to
obliczane s"a iloczyny tej jednej funkcji wektorowej i~wszystkich funkcji
skalarnych.

Opisane tu procedury mog"a znale"z"c zastosowanie w r"o"rnych zaawansowanych
konstrukcjach. Najprostsza z~nich to podwy"rszenie stopnia krzywej, przez
pomno"renie jej przez odpowiednio reprezentowan"a funkcj"e sta"l"a
$s_0(t)=1$ (stopie"n reprezentacji funkcji $s_0$ jest r"o"rnic"a stopni
reprezentacji wynikowej i~danej przetwarzanej funkcji wektorowej). W~tym
przypadku lepiej jest jednak u"ry"c procedury podwy"rszania
stopnia (np.\ \texttt{mbs\_multiBSDegElevf}), kt"ora realizuje zr"eczniejszy
algorytm.%
\begin{figure}[ht]
  \centerline{\epsfig{file=splmult.ps}}
  \caption{Mno"renie p"laskiej krzywej wektorowej przez funkcj"e sklejan"a}
\end{figure}

\vspace{\bigskipamount}
\cprog{%
int mbs\_BSProdRepSizef ( int degree1, int lastknot1, \\
\ind{25}const float *knots1, \\
\ind{25}int degree2, int lastknot2, \\
\ind{25}const float *knots2 );}
Procedura \texttt{mbs\_BSProdRepSizef} otrzymuje \emph{dwa} ci"agi w"ez"l"ow,
\texttt{knots1} o d"lugo"sci \texttt{lastknot1+1} oraz \texttt{knots2}
o~d"lugo"sci \texttt{lastknot2+1}. Pierwszy z tych ci"ag"ow s"lu"ry do
okre"slenia funkcji sklejanej stopnia \texttt{degree1}, a drugi --- funkcji
sklejanej stopnia \texttt{degree2}. Ci"agi te powinny okre"sla"c tak"a sam"a
dziedzin"e krzywych sklejanych. Warto"sci"a procedury jest indeks ostatniego
elementu najkr"otszego ci"agu w"ez"l"ow wystarczaj"acego do reprezentowania
iloczynu dowolnych funkcji sklejanych okre"slonych dla podanych ci"ag"ow
w"ez"l"ow.

\vspace{\bigskipamount}
\cprog{%
void mbs\_SetBSProdKnotsf ( int degree1, int lastknot1, \\
\ind{27}const float *knots1, \\
\ind{27}int degree2, int lastknot2, \\
\ind{27}const float *knots2, \\
\ind{27}int *degree, int *lastknot, \\
\ind{27}float *knots );}
Procedura \texttt{mbs\_SetBSProdKnotsf} otrzymuje dwa ci"agi w"ez"l"ow i tworzy
nowy ci"ag, kt"ory jest minimalnym ci"agiem wystarczaj"acym do
reprezentowania iloczynu dowolnych funkcji sklejanych okre"slonych dla
ci"ag"ow danych.

\vspace{\bigskipamount}
\cprog{%
void mbs\_multiMultBezCf ( int nscf, int degscf, int scfpitch, \\
\ind{26}const float *scfcoeff, \\
\ind{26}int spdimen, \\
\ind{26}int nvecf, int degvecf, int vecfpitch, \\
\ind{26}const float *vecfcp, \\
\ind{26}int *degprod, int prodpitch, \\
\ind{26}float *prodcp );}
Procedura \texttt{mbs\_multiMultBezCf} wykonuje mno"renie wielomian"ow
reprezentowanych w~bazie wielomian"ow Bernsteina stopnia \texttt{degscf}
i~wielomianowych funkcji wektorowych (krzywych B\'{e}ziera) stopnia
\texttt{degvecf}. Parametr \texttt{spdimen} okre"sla wymiar przestrzeni,
w~kt"orej le"r"a te krzywe. Liczba funkcji skalarnych jest okre"slona przez
parametr \texttt{nscf}, a~liczba krzywych jest warto"sci"a parametru
\texttt{nvecf}. Liczba obliczonych iloczyn"ow jest r"owna wi"ekszej z tych
liczb, zobacz uwagi na pocz"atku tego punktu.

Tablica \texttt{scfcoeff} zawiera wsp"o"lczynniki wielomian"ow w~bazie
Bernsteina, przy czym wsp"o"lczynniki ka"rdego wielomianu zajmuj"a kolejne
miejsca w~tablicy, a~jej podzia"lka (czyli r"o"rnica indeks"ow pierwszych
wsp"o"lczynnik"ow kolejnych dw"och wielomian"ow) jest podana jako warto"s"c
parametru \texttt{scfpitch}. Podobnie, parametr \texttt{vecfpitch} okre"sla
podzia"lk"e tablicy \texttt{vecfcp}, kt"ora zawiera wektorowe
wsp"o"lczynniki krzywych (ka"rdy wsp"o"lczynnik sk"lada si"e z~kolejnych
\texttt{spdimen} liczb).

Iloczyny s"a reprezentowane w~bazie Bernsteina stopnia r"ownego sumie stopni
reprezentacji argument"ow (tj.\
$\mathord{\mbox{\texttt{degscf}}}+\mathord{\mbox{\texttt{degvecf}}}$),
kt"ory to stopie"n jest zwracany poprzez parametr \texttt{degprod}.
Reprezentacje poszczeg"olnych iloczyn"ow sk"ladaj"a si"e z ci"ag"ow
$\mathord{\mbox{\texttt{spdimen}}}*(\mathord{\mbox{stopie"n}}+1)$ liczb,
kt"ore procedura wpisuje do tablicy \texttt{prodcp} z~podzia"lk"a
\texttt{prodpitch}.

\vspace{\bigskipamount}
\cprog{%
void mbs\_multiMultBSCf ( int nscf, int degscf, \\
\ind{25}int scflastknot, const float *scfknots, \\
\ind{25}int scfpitch, const float *scfcoeff, \\
\ind{25}int spdimen, \\
\ind{25}int nvecf, int degvecf, \\
\ind{25}int vecflastknot, const float *vecfknots, \\
\ind{25}int vecfpitch, const float *vecfcp, \\
\ind{25}int *degprod, int *prodlastknot, \\
\ind{25}float *prodknots, \\
\ind{25}int prodpitch, float *prodcp );}
Procedura \texttt{mbs\_multiMultBSCf} oblicza reprezentacje iloczyn"ow
\texttt{nscf} skalarnych funkcji sklejanych $s_i$ i~\texttt{nvecf}
wektorowych funkcji sklejanych $\bm{v}_i$, przy czym liczby te mog"a by"c
r"o"rne i wtedy jedna z nich musi by"c r"owna $1$ (zobacz uwagi na pocz"atku
tego punktu).

Funkcje skalarne s"a okre"slone za pomoc"a parametr"ow \texttt{degscf}
(stopie"n reprezentacji), \texttt{scflastknot} i~\texttt{scfknots} (indeks
ostatniego w"ez"la i~tablica z~tymi w"ez"lami), \texttt{scfcoeff}
i~\texttt{scfpitch} (tablica wsp"o"lczynnik"ow w~bazie B-sklejanej
poszczeg"olnych funkcji i~podzia"lka tej tablicy).

Funkcje wektorowe, w~przestrzeni o~wymiarze \texttt{spdimen}, s"a podobnie
reprezentowane przez parametry \texttt{degvecf}, \texttt{vecflastknot},
\texttt{vecfknots}, \texttt{vecfpitch} i~\texttt{vecfcp}.

Wynik jest wpisywany do tablic \texttt{prodknots} (w"ez"ly) i \texttt{prodcp}
(wektorowe wsp"o"lczynniki w~bazie B-sklejanej stopnia r"ownego sumie
stopni reprezentacji argument"ow, jest on zwracany poprzez parametr
\texttt{degprod}). Podzia"lka tej ostatniej tablicy jest okre"slona przez
parametr \texttt{prodpitch}. Pocz"atkowa warto"s"c parametru
\texttt{*prodlastknot} okre"sla ilo"s"c miejsca w tablicy \texttt{prodknots}
(ma by"c o $1$ wi"eksza od warto"sci tego parametru). Nale"ry j"a obliczy"c
i~utworzy"c odpowiedni"a tablic"e \emph{przed} wywo"laniem procedury
\texttt{mbs\_multiMultBSCf}, najlepiej jest u"ry"c do tego procedury
\texttt{mbs\_BSProdRepSizef}, kt"ora bada ci"agi w"ez"l"ow argument"ow
mno"renia, podane jej jako parametry.


\subsection{Wyznaczanie p"lata opisuj"acego wektory normalne}

\cprog{%
void mbs\_BezP3NormalDeg ( int degreeu, int degreev, \\
\ind{26}int *ndegu, int *ndegv ); \\
char mbs\_BezP3Normalf ( int degreeu, int degreev, \\
\ind{24}const point3f *ctlpoints, \\
\ind{24}int *ndegu, int *ndegv, vector3f *ncp );}
\hspace*{\parindent}Procedura \texttt{mbs\_BezP3Normalf} oblicza
punkty kontrolne p"lata $\bm{n}=\bm{p}_u\wedge\bm{p}_v$,
opisuj"acego wektor normalny danego wielomianowego p"lata
B\'{e}ziera $\bm{p}$ stopnia $(n,m)$ w~$\R^3$.
Parametry $\mathord{\mbox{\texttt{degreeu}}}=n$
i~$\mathord{\mbox{\texttt{degreev}}}=m$ okre"slaj"a stopie"n
p"lata danego. Jego punkty kontrolne s"a podane w tablicy
\texttt{ctlpoints}, kt"ora zawiera kolejne kolumny tego p"lata bez przerw.

Obliczony p"lat wyj"sciowy ma stopie"n \texttt{*ndegu}$=2n-1$
i~\texttt{*ndegv}$=2m-1$, a jego punkty kontrolne s"a wpisywane do tablicy
\texttt{ncp} (bez przerw mi"edzy kolumnami).

\begin{sloppypar}
Warto"s"c procedury \texttt{mbs\_BezP3Normalf} jest r"owna~$0$ w~razie
niepowodzenia (b"l"edne parametry lub brak pami"eci), albo~$1$, je"sli
obliczenie zako"nczy"lo si"e sukcesem.
\end{sloppypar}

\vspace{\smallskipamount}
Procedura \texttt{mbs\_BezP3NormalDeg} oblicza stopie"n p"lata opisuj"acego
wektor normalny. Mo"re ona by"c u"ryteczna w celu zarezerwowania
bloku pami"eci o wielko"sci odpowiedniej do przechowania punkt"ow
kontrolnych p"lata $\bm{n}$.


\vspace{\bigskipamount}
\cprog{%
void mbs\_BezP3RNormalDeg ( int degreeu, int degreev, \\
\ind{27}int *ndegu, int *ndegv ); \\
char mbs\_BezP3RNormalf ( int degreeu, int degreev, \\
\ind{25}const point4f *ctlpoints, \\
\ind{25}int *ndegu, int *ndegv, vector3f *ncp );}
Procedura \texttt{mbs\_BezP3RNormalf} oblicza
punkty kontrolne p"lata $\bm{n}$ opisuj"acego wektor normalny danego
wymiernego p"lata B\'{e}ziera $\bm{p}$ stopnia $(n,m)$ w~$\R^3$.
Punkty te powstaj"a przez odrzucenie ostatniej wsp"o"lrz"ednej punkt"ow
kontrolnych p"lata $\bm{N}=\bm{P}\wedge\bm{P}_u\wedge\bm{P}_v$ w $\R^4$.
Parametry $\mathord{\mbox{\texttt{degreeu}}}=n$
i~$\mathord{\mbox{\texttt{degreev}}}=m$ okre"slaj"a stopie"n
p"lata danego. Punkty kontrolne jego jednorodnej reprezentacji s"a podane
w~tablicy \texttt{ctlpoints}, kt"ora zawiera kolejne kolumny tego p"lata
bez przerw.

\begin{figure}[ht]
  \centerline{\raisebox{90pt}[0pt][0pt]{a)}\epsfig{file=nvtest1.ps}}
  \centerline{\raisebox{96pt}[0pt][0pt]{b)}\epsfig{file=nvtest2.ps}}
  \caption{Siatki kontrolne p"lat"ow B\'{e}ziera i ich p"lat"ow}
  \centerline{opisuj"acych wektory normalne: a) p"lat wielomianowy, b) wymierny}
\end{figure}
Obliczony p"lat wyj"sciowy ma stopie"n \texttt{*ndegu}$=3n-2$
i~\texttt{*ndegv}$=3m-2$, a jego punkty kontrolne s"a wpisywane do tablicy
\texttt{ncp} (bez przerw mi"edzy kolumnami).

Warto"s"c procedury \texttt{mbs\_BezP3RNormalf} jest r"owna~$0$ w~razie
niepowodzenia (b"l"edne parametry lub brak pami"eci), albo~$1$, je"sli
obliczenie zako"nczy"lo si"e sukcesem.

\vspace{\smallskipamount}
Procedura \texttt{mbs\_BezP3RNormalDeg} oblicza stopie"n p"lata opisuj"acego
wektor normalny. Mo"re ona by"c u"ryteczna w celu zarezerwowania
bloku pami"eci o wielko"sci odpowiedniej do przechowania punkt"ow
kontrolnych p"lata $\bm{n}$.


\clearpage
\section{Zmiana w"ez"l"ow krzywych B-sklejanych na ko"ncach}

\cprog{%
void mbs\_multiBSChangeLeftKnotsf ( int ncurves, int spdimen, \\
\ind{35}int degree, float *knots, \\
\ind{35}int pitch, float *ctlpoints, \\
\ind{35}float *newknots );}
\begin{sloppypar}
\hspace{\parindent}
Procedura \texttt{mbs\_multiBSChangeLeftKnotsf} dokonuje zmiany
reprezentacji krzywych B-sklejanych stopnia~$n$, kt"ora polega na
wymianie pocz"atkowych $n+1$ w"ez"l"ow na w"ez"ly podane w~tablicy
\texttt{newknots}. Mo"re to by"c zwi"azane z rozszerzeniem lub obci"eciem
dziedziny i~w~takim przypadku pierwszy "luk wielomianowy jest wyd"lu"rany
lub skracany.
\end{sloppypar}

Nowe w"ez"ly musz"a by"c uporz"adkowane niemalej"aco; musi by"c
$u_n<u_{n+1}$

\vspace{\bigskipamount}
\cprog{%
void mbs\_multiBSChangeRightKnotsf ( int ncurves, int spdimen, \\
\ind{36}int degree, \\
\ind{36}int lastknot, float *knots, \\
\ind{36}int pitch, float *ctlpoints, \\
\ind{36}float *newknots );}
\begin{sloppypar}
Procedura \texttt{ mbs\_multiBSChangeRightKnotsf } dokonuje zmiany
reprezentacji krzy\-wych B-sklejanych stopnia~$n$, kt"ora polega na wymianie
ko"ncowych $n+1$ w"ez\-"l"ow (tj.\ w"ez"l"ow $u_{N-n},\ldots,u_N$)
na w"ez"ly podane w~tablicy \texttt{newknots}. Mo"re to by"c zwi"azane
z~rozszerzeniem lub obci"eciem dziedziny i~w~takim przypadku ostatni
"luk wielomianowy jest wyd"lu"rany lub skracany.
\end{sloppypar}

Nowe w"ez"ly musz"a by"c uporz"adkowane niemalej"aco; musi by"c
$u_{N-n}>u_{N-n-1}$.

\vspace{\bigskipamount}
\cprog{%
\#define mbs\_BSChangeLeftKnotsC1f(degree,knots,coeff,newknots) \bsl \\
\ind{2}mbs\_multiBSChangeLeftKnotsf(1,1,degree,knots,0,coeff,newknots) \\
\#define mbs\_BSChangeLeftKnotsC2f(degree,knots,ctlpoints,newknots) \bsl \\
\ind{2}mbs\_multiBSChangeLeftKnotsf(1,2,degree,knots,0, \bsl \\
\ind{30}(float*)ctlpoints,newknots) \\
\#define mbs\_BSChangeLeftKnotsC3f(degree,knots,ctlpoints,newknots) \bsl \\
\ind{2}... \\
\#define mbs\_BSChangeLeftKnotsC4f(degree,knots,ctlpoints,newknots) \bsl \\
\ind{2}... \\
\#define mbs\_BSChangeRightKnotsC1f(degree,lastknot,knots,coeff, \bsl \\
\ind{34}newknots) ... \\
\#define mbs\_BSChangeRightKnotsC2f(degree,lastknot,knots, \bsl \\
\ind{34}ctlpoints,newknots) ... \\
\#define mbs\_BSChangeRightKnotsC3f(degree,lastknot,knots, \bsl \\
\ind{34}ctlpoints,newknots) ... \\
\#define mbs\_BSChangeRightKnotsC4f(degree,lastknot,knots, \bsl \\
\ind{34}ctlpoints,newknots)}
\begin{sloppypar}
Makra s"lu"r"a do wywo"lywania procedury
\texttt{mbs\_multiBSChangeLeftKnotsf}
albo \texttt{mbs\_multiBSChangeRightKnotsf} w~razie potrzeby zastosowania
jej do jednej krzywej B-sklejanej w~przestrzeni o~wymiarze $1,\ldots,4$.
\end{sloppypar}

\begin{figure}[ht]
  \centerline{\epsfig{file=chknots.ps}}
  \caption{Krzywa B-sklejana przed i po zmianie reprezentacji na ko"ncach}
\end{figure}


\clearpage
\section{Konstrukcje krzywych interpolacyjnych}

Konstrukcja krzywych interpolacyjnych jest czasem g"l"ownym zadaniem, ale
te"r przydaje si"e do rozwi"azania zada"n takich jak wyznaczanie
powierzchni rozpinanych i wype"lniaj"acych.


\subsection{Konstrukcja krzywych interpolacyjnych trzeciego stopnia}

W tym punkcie jest opisana procedura wyznaczania B-sklejanej reprezentacji
krzywych interpolacyjnych \emph{trzeciego stopnia}. W"ez"ly interpolacyjne
(podane jako parametr wej"sciowy) b"ed"a w"ez"lami krzywych, przy czym
skrajne w"ez"ly interpolacyjne w~wyznaczonej reprezentacji B-sklejanej
b"ed"a mia"ly krotno"s"c $3$, a~do tego dojd"a jeszcze jednokrotne w"ez"ly
dodatkowe, kt"ore s"a potrzebne w~definicji reprezentacji B-sklejanej.

Opr"ocz w"ez"l"ow i~warunk"ow interpolacyjnych nale"ry poda"c \emph{warunki
brzegowe}. Rodzaje warunk"ow brzegowych obs"lugiwanych przez procedur"e
w~obecnej wersji s"a opisane dalej.

\vspace{\bigskipamount}
\cprog{%
void mbs\_multiBSCubicInterpf ( int lastinterpknot, \\
\ind{31}float *interpknots, \\
\ind{31}int ncurves, int spdimen, \\
\ind{31}int xpitch, const float *x, \\
\ind{31}int ypitch, \\
\ind{31}char bcl, const float *ybcl, \\
\ind{31}char bcr, const float *ybcr, \\
\ind{31}int *lastbsknot, \\
\ind{31}float *bsknots, \\
\ind{31}int bspitch, \\
\ind{31}float *ctlpoints );}
\begin{sloppypar}
Procedura \texttt{mbs\_multiBSCubicInterpf} s"lu"ry do konstruowania
B-sklejanych krzywych interpolacyjnych trzeciego stopnia, klasy $C^2$.
\end{sloppypar}
\begin{sloppypar}
Parametry: \texttt{lastinterpknot} okre"sla indeks ostatniego w"ez"la
interpolacyjnego, oznacz\-my go liter"a~$N$. W"ez"ly interpolacyjne
$u_0,\ldots,u_N$, kt"ore tworz"a ci"ag "sci"sle rosn"acy, podaje si"e
w~tablicy \texttt{interpknots}.
\end{sloppypar}

\begin{figure}[ht]
  \centerline{\begin{picture}(2050,2883)
    \put(0,0){\epsfig{file=intbsc.ps}}
    \put(0,2770){a)}
    \put(1100,2770){b)}
    \put(0,1770){c)}
    \put(1100,1770){d)}
    \put(0,770){e)}
    \put(1100,770){f)}
  \end{picture}}
  \caption{B-sklejane krzywe interpolacyjne trzeciego stopnia.}
  \centerline{W"ez"lami s"a liczby $0,1,\ldots,10$. Warunki brzegowe krzywej na rysunku a)}
  \centerline{by"ly okre"slone przez podanie wektor"ow
    pochodnych na ko"ncach,}
  \centerline{na rysunku b) krzywa jest okre"slona za pomoc"a warunk"ow Bessela}
  \centerline{c) dane wektory pochodnych drugiego rz"edu, d) naturalna krzywa sklejana,}
  \centerline{e) warunek nie-w"eze"l, f) pochodna trzeciego rz"edu na
    ko"ncach r"owna $\bm{0}$}
\end{figure}%
Parametry \texttt{ncurves} i~\texttt{spdimen} okre"slaj"a liczb"e krzywych
i~wymiar przestrzeni, w~kt"orej le"r"a te krzywe. Tablica \texttt{x} zawiera
wsp"o"lrz"edne punkt"ow przez kt"ore maj"a przechodzi"c skonstruowane
krzywe; dla ka"rdej krzywej trzeba poda"c w~niej
$\mathord{\mbox{\texttt{spdimen}}}(\mathord{\mbox{\texttt{lastinterpknot}}}+1)$
liczb rzeczywistych. Podzia"lka tej tablicy (tj.\ r"o"rnica indeks"ow
pierwszej liczby w~danych dla kolejnych krzywych) jest r"owna
\texttt{xpitch}.

Parametr \texttt{ypitch} okre"sla podzia"lk"e tablic \texttt{ybcl}
i~\texttt{ybcr}, w~kt"orych nale"ry poda"c dane liczbowe okre"slaj"ace
warunki brzegowe.

Parametry \texttt{bcl} i~\texttt{bcr} s"lu"r"a do wybrania warunk"ow
brzegowych odpowiednio na lewym i prawym ko"ncu dziedziny krzywych;
wszystkie krzywe maj"a warunki tego samego rodzaju, ale na ka"rdym ko"ncu
warunek mo"re by"c inny. Dopuszczalne warto"sci tych parametr"ow s"a
zdefiniowane w~pliku \texttt{multibs.h} jako makra i~obecnie s"a
nast"epuj"ace:
\begin{mydescription}
  \item[]\texttt{BS3\_BC\_FIRST\_DER} --- warunek brzegowy jest okre"slony
    przez podanie wektora pochodnej ka"rdej krzywej w~pierwszym albo
    ostatnim w"e"zle interpolacyjnym (tj.\ $u_0$ albo $u_N$).
    Wsp"o"lrz"edne tych wektor"ow dla wszystkich krzywych nale"ry poda"c
    w~tablicy \texttt{ybcl} (je"sli w~pierwszym) albo \texttt{ybcr}
    (je"sli w~ostatnim w"e"zle). Zatem, tablice \texttt{ybcl}
    i~\texttt{ybcr} zawieraj"a dla ka"rdej krzywej \texttt{spdimen}
    liczb rzeczywistych, kt"ore s"a wsp"o"lrz"ednymi tych wektor"ow.
  \item[]\texttt{BS3\_BC\_FIRST\_DER0} --- warunek brzegowy
    okre"slony jak wy"rej, z~wektorem zerowym pochodnej w~odpowiednim
    w"e"zle interpolacyjnym. Parametr \texttt{ybcl} lub \texttt{ybcr} jest
    ignorowany, mo"re wi"ec mie"c warto"s"c \texttt{NULL}.
  \item[]\texttt{BS3\_BC\_SECOND\_DER} --- warunek brzegowy jest okre"slony
    przez podanie wektora pochodnej drugiego krzywej sklejanej rz"edu
    w~w"e"zle $u_0$ albo $u_N$. Wsp"o"lrz"edne tego wektora (albo wektor"ow,
    je"sli liczba krzywych jest wi"eksza ni"r $1$) s"a podane w~tablicy
    odpowiednio \texttt{ybcl} lub \texttt{ybcr}.
  \item[]\texttt{BS3\_BC\_SECOND\_DER0} --- warunek brzegowy jest okre"slony
    jak wy"rej, z~wektorem zerowym pochodnej drugiego rz"edu w~odpowiednim
    w"e"zle. Parametr \texttt{ybcl} lub \texttt{ybcr} jest  
    ignorowany, mo"re wi"ec mie"c warto"s"c \texttt{NULL}.

    Krzywa okre"slona z~tym warunkiem na obu ko"ncach to tzw.\
    \textbf{naturalna krzywa sklejana}.
  \item[]\texttt{BS3\_BC\_THIRD\_DER} --- warunek brzegowy jest okre"slony
    przez podanie wektor"ow pochodnej trzeciego rz"edu krzywych.
    Wsp"o"lrz"edne tych wektor"ow s"a podane w~tablicy \texttt{ybcl} albo
    \texttt{ybcr}.
  \item[]\texttt{BS3\_BC\_THIRD\_DER0} --- warunek brzegowy jest okre"slony
    przez "r"adanie, aby pochodna trzeciego rz"edu krzywych na jednym lub
    drugim ko"ncu by"la wektorem zerowym. Poniewa"r pochodna trzeciego
    rz"edu krzywej wielomianowej trzeciego stopnia jest sta"la, wi"ec
    oznacza to, "re pierwszy lub ostatni "luk wielomianowy krzywej jest
    "lukiem paraboli. Parametr \texttt{ybcl} lub \texttt{ybcr} w~przypadku
    zadania tego warunku jest ignorowany, mo"re mie"c zatem warto"s"c
    \texttt{NULL}.
  \item[]\texttt{BS3\_BC\_BESSEL} --- warunek brzegowy jest tzw.\ warunkiem
    Bessela. Wektor pochodnej krzywej w pierwszym lub ostatnim w"e"zle jest
    wektorem pochodnej wielomianowej krzywej interpolacyjnej drugiego
    stopnia, opartej na pierwszych trzech albo ostatnich trzech w"ez"lach
    interpolacyjnych konstruowanej krzywej.

    Parametr \texttt{ybcl} albo \texttt{ybcr} w przypadku okre"slenia
    warunku Bessela jest ignorowany, a~zatem mo"re mie"c warto"s"c
    \texttt{NULL}.
  \item[]\texttt{BS3\_BC\_NOT\_A\_KNOT} --- warunek nie-w"eze"l; w"eze"l
    interpolacyjny $u_1$ albo $u_{N-1}$ nie jest w"ez"lem krzywej sklejanej,
    tj.\ "luki wielomianowe krzywej "l"acz"a si"e w~tym w"e"zle
    z~ci"ag"lo"sci"a $C^\infty$. Parametr \texttt{ybcl} albo \texttt{ybcr}
    jest ignorowany, mo"re mie"c wi"ec warto"s"c \texttt{NULL}.
\end{mydescription}

Reprezentacja krzywej interpolacyjnej skonstruowanej przez procedur"e
jest opisana za pomoc"a nast"epuj"acych parametr"ow: \texttt{*lastbsknot}
--- indeks ostatniego w"ez"la krzywej sklejanej, \texttt{bsknots} ---
tablica w"ez"l"ow (to s"a w"ez"ly interpolacyjne, ale w"ez"ly $u_0$ i~$u_N$
maj"a w~tej tablicy krotno"s"c~$3$ i~s"a do"l"aczone dwa dodatkowe w"ez"ly
wymagane w definicji krzywych B-sklejanych). Parametr wej"sciowy
\texttt{bspitch}
oznacza podzia"lk"e tablicy \texttt{ctlpoints}, w~kt"orej procedura
umieszcza puknty kontrolne kolejnych krzywych interpolacyjnych.


\subsection{Konstrukcja krzywych interpolacyjnych Hermite'a}

W~tym punkcie s"a opisane procedury dokonuj"aca bardzo szczeg"olnej
konstrukcji: znajduj"a krzywe B\'{e}ziera i~B-sklejane stopnia~$n$,
spe"lniaj"ace warunki interpolacyjne Hermite'a zadane w~dw"och w"ez"lach,
$0$ i~$1$ albo $u_n$ i~$u_{N-n}$. Istnieje zastosowanie, w~kt"orym takie
w"la"snie procedury by"ly mi potrzebne (algorytm konstrukcji w~tym
przypadku jest sprawniejszy ni"r og"olny algorytm rozwi"azania zadania
interpolacyjnego Hermite'a z~krzyw"a B-sklejan"a).

\vspace{\bigskipamount}
\cprog{%
void mbs\_multiInterp2knHermiteBezf ( int ncurves, int spdimen, \\
\ind{26}int degree, \\
\ind{26}int nlbc, int lbcpitch, const float *lbc, \\
\ind{26}int nrbc, int rbcpitch, const float *rbc, \\
\ind{26}int pitch, float *ctlpoints );}
\begin{sloppypar}
Procedura \texttt{mbs\_multiInterp2knHermiteBezf} konstruuje \texttt{ncurves}
krzywych B\'{e}ziera stopnia~$n$ (stopie"n jest warto"sci"a parametru
\texttt{degree}) w~przestrzeni o~wymiarze~$d$ (jest on warto"sci"a parametru
\texttt{spdimen}).
\end{sloppypar}

Liczba warunk"ow interpolacyjnych dla ka"rdej krzywej
w~w"e"zle~$0$ jest r"owna \texttt{nlbc}, a~w~w"e"zle~$1$
\texttt{nrbc}, przy oba te parametry musz"a by"c nieujemne i~ich suma musi
by"c r"owna $n+1$ (to zapewnia, "re warunki interpolacyjne okre"slaj"a
krzywe jednoznacznie).

Warunki interpolacyjne podaje si"e w~tablicach \texttt{lbc} (dla w"ez"la~$0$)
i~\texttt{rbc} (dla w"ez"la~$1$). Pocz"atkowe $d$~liczb w~ka"rdej z~tych
tablic to odpowiedni punkt pierwszej krzywej, nast"epne $d$~liczb opisuje
wektor pochodnej, potem pochodnej drugiego rz"edu itd.
Dane opisuj"ace warunki interpolacyjne nast"epnej krzywej zaczynaj"a si"e
w~pozycji odpowiednio \texttt{lbcpitch} i~\texttt{rbcpitch}.

Punkty kontrolne skonstruowanych krzywych (czyli wynik dzia"lania procedury)
s"a wstawiane do tablicy \texttt{ctlpoints}, kt"orej podzia"lka
(odleg"lo"s"c pocz"atk"ow danych opisuj"acych kolejne krzywe) jest
warto"sci"a parametru \texttt{pitch}.

\vspace{\bigskipamount}
\cprog{%
void mbs\_multiInterp2knHermiteBSf ( int ncurves, int spdimen, \\
\ind{26}int degree, \\
\ind{26}int lastknot, const float *knots, \\
\ind{26}int nlbc, int lbcpitch, const float *lbc, \\
\ind{26}int nrbc, int rbcpitch, const float *rbc, \\
\ind{26}int pitch, float *ctlpoints );}
\begin{sloppypar}
Procedura \texttt{mbs\_multiInterp2knHermiteBSf} konstruuje \texttt{ncurves}
\mbox{B-sklejanych} krzywych stopnia~$n$ (stopie"n jest warto"sci"a parametru
\texttt{degree}) w~przestrzeni o~wymiarze~$d$ (jest on warto"sci"a parametru
\texttt{spdimen}). Krzywa jest
oparta o~ci"ag w"ez\-"l"ow o~d"lugo"sci $N+1$ podany w~tablicy \texttt{knots}
(liczba $N$ jest warto"sci"a parametru \texttt{lastknot}).
\end{sloppypar}

Liczba warunk"ow interpolacyjnych dla ka"rdej krzywej
w~w"e"zle $u_n$ jest r"owna \texttt{nlbc}, a~w~w"e"zle $u_{N-n}$
\texttt{nrbc}, przy czym "raden z~tych parametr"ow nie
mo"re by"c wi"ekszy ni"r~$n$, za"s suma ich warto"sci musi by"c r"owna
$N-n$ (to zapewnia, "re warunki interpolacyjne okre"slaj"a krzywe
jednoznacznie).

Ci"ag w"ez"l"ow w~tablicy \texttt{knots} musi spe"lnia"c warunki
$u_1=\cdots=u_n<u_{n+1}$ oraz $u_{N-n-1}<u_{N-n}=\cdots=u_{N-1}$,
kt"orych procedura nie sprawdza. Warunki interpolacyjne podaje si"e
w~tablicach \texttt{lbc} (dla w"ez"la $u_n$) i~\texttt{rbc}
(dla w"ez"la $u_{N-n}$). Pocz"atkowe $d$~liczb w~ka"rdej z~tych tablic
to odpowiedni punkt pierwszej krzywej, nast"epne $d$~liczb opisuje
wektor pochodnej, potem pochodnej drugiego rz"edu itd.
Dane opisuj"ace warunki interpolacyjne nast"epnej krzywej zaczynaj"a si"e
w~pozycji odpowiednio \texttt{lbcpitch} i~\texttt{rbcpitch}.

Punkty kontrolne skonstruowanych krzywych (czyli wynik dzia"lania procedury)
s"a wstawiane do tablicy \texttt{ctlpoints}, kt"orej podzia"lka
(odleg"lo"s"c pocz"atk"ow danych opisuj"acych kolejne krzywe) jest
warto"sci"a parametru \texttt{pitch}.


\newpage
\section{Konstrukcja krzywych aproksymacyjnych}

\begin{sloppypar}
Mo"rna na"lo"ry"c liczb"e warunk"ow interpolacyjnych na funkcj"e lub krzyw"a
sklejan"a wi"eksz"a ni"r wymiar odpowiedniej przestrzeni. Wtedy powstaje
uk"lad r"owna"n w~og"olno"sci sprzeczny. Uk"lad ten mo"rna rozwi"aza"c jako
liniowe zadanie najmniejszych kwadrat"ow i w ten spos"ob otrzyma"c funkcj"e
lub krzyw"a sklejan"a spe"lniaj"ac"a na"lo"rone warunki interpolacyjne
z~pewnm b"l"edem. W~tym punkcie s"a opisane procedury wykonuj"ace t"e
konstrukcj"e.
\end{sloppypar}

\begin{figure}[ht]
  \centerline{\epsfig{file=bsapprox.ps}}
  \caption{Przyk"lad p"laskiej aproksymacyjnej krzywej B-sklejanej}
\end{figure}
Konstrukcj"e krzywej aproksymacyjnej mo"rna przeprowadzi"c wywo"luj"ac
opisan"a dalej procedur"e \texttt{mbs\_multiConstructApproxBSCf}. Wcze"sniej
opisane procedury s"a pomocnicze i~w~typowych aplikacjach nie b"ed"a
bezpo"srednio wywo"lywane.

\vspace{\bigskipamount}
\cprog{%
boolean mbs\_ApproxBSKnotsValidf ( int degree, int lastknot, \\
\ind{28}const float *knots, \\
\ind{28}int lastiknot, const float *iknots );}
\begin{sloppypar}
Procedura \texttt{mbs\_ApproxBSKnotsValidf} dokonuje sprawdzenia, czy ci"agi
w"ez\-"l"ow interpolacyjnych i~w"ez"l"ow krzywej sklejanej spe"lniaj"a
za"lo"renia twierdzenia Schoenberga-Whitney. Je"sli tak, to konstrukcja
krzywej aproksymacyjnej jest wykonalna.
\end{sloppypar}

\vspace{\bigskipamount}
\cprog{%
int mbs\_ApproxBSBandmSizef ( int degree, const float *knots, \\
\ind{29}int lastiknot, const float *iknots );}
Procedura \texttt{mbs\_ApproxBSBandmSizef} oblicza d"lugo"s"c tablicy
potrzebnej do reprezentowania macierzy wst"egowej uk"ladu r"owna"n
rozwi"azywanego (jako liniowe zadanie najmniejszych kwadrat"ow)
w~konstrukcji B-sklejanej krzywej aproksymacyjnej.

Parametry \texttt{degree}, i~\texttt{knots} opisuj"a
przestrze"n funkcji sklejanych, kt"orej elementy opisuj"a krzyw"a
(odpowiednio stopie"n i~ci"ag w"ez"l"ow; d"lugo"s"c tego ci"agu jest
okre"slana na podstawie ci"agu w"ez"l"ow interpolacyjnych).
Parametry \texttt{lastiknot} oraz \texttt{iknots} opisuj"a ci"ag w"ez"l"ow
\emph{interpolacyjnych} krzywej --- warunki interpolacyjne okre"slone w~tych
w"ez"lach b"ed"a przez krzyw"a spe"lnione w~przybli"reniu.

Warto"sci"a procedury jest d"lugo"s"c tablicy, w~kt"orej maj"a by"c
przechowywane niezerowe wsp"o"lczynniki macierzy uk"ladu.

\vspace{\bigskipamount}
\cprog{%
boolean mbs\_ConstructApproxBSProfilef ( int degree, int lastknot, \\
\ind{28}const float *knots, \\
\ind{28}int lastiknot, const float *iknots, \\
\ind{28}bandm\_profile *prof );}
Procedura \texttt{mbs\_ConstructApproxBSProfilef} konstruuje profil macierzy
wst"egowej (zobacz p.~\ref{sect:band:matrix}) uk"ladu r"owna"n
wyst"epuj"acego w~konstrukcji aproksymacyjnej krzywej sklejanej. Parametry
\texttt{degree}, \texttt{lastknot}, \texttt{knots}, \texttt{lastiknot},
\texttt{iknots} opisuj"a w"ez"ly krzywej sklejanej i~w"ez"ly interpolacyjne
(zobacz opis parametr"ow procedury \texttt{mbs\_ApproxBSBandmSizef}).

Parametr \texttt{prof} wskazuje tablic"e o~d"lugo"sci
\texttt{lastknot}$-$\texttt{degree}$+1$. Procedura umieszcza w~niej profil
macierzy.

\vspace{\bigskipamount}
\cprog{%
boolean mbs\_ConstructApproxBSMatrixf ( int degree, int lastknot, \\
\ind{30}const float *knots, \\
\ind{30}int lastiknot, const float *iknots, \\
\ind{30}int *nrows, int *ncols, \\
\ind{30}bandm\_profile *prof, \\
\ind{30}float *a );}
\begin{sloppypar}
Procedura \texttt{mbs\_ConstructApproxBSMatrixf} oblicza wsp"o"lczynniki
macierzy uk"ladu, kt"orego rozwi"azanie "sredniokwadratowe okre"sla
sklejan"a funkcj"e lub krzyw"a aproksymacyjn"a.
\end{sloppypar}

\vspace{\bigskipamount}
\cprog{%
boolean mbs\_multiConstructApproxBSCf ( int degree, int lastknot, \\
\ind{30}const float *knots, \\
\ind{30}int lastpknot, const float *pknots, \\
\ind{30}int ncurves, int spdimen, \\
\ind{30}int ppitch, const float *ppoints, \\
\ind{30}int bcpitch, float *ctlpoints );}
\begin{sloppypar}
Procedura \texttt{mbs\_multiConstructApproxBSCf} konstruuje sklejane
funkcje lub krzy\-we aproksymacyjne przez utworzenie odpowiedniego uk"ladu
r"owna"n liniowych i~rozwi"azanie go jako liniowego zadania najmniejszych
kwadrat"ow.
\end{sloppypar}

Parametry: \texttt{degree} --- stopie"n krzywej, \texttt{lastknot} ---
indeks ostatniego w"ez"la, \texttt{knots} --- tablica w"ez"l"ow,
\texttt{lastpknot} --- indeks ostatniego w"ez"la interpolacyjnego,
\texttt{pknots} --- tablica w"ez"l"ow interpolacyjnych, \texttt{ncurves} ---
liczba konstruowanych krzywych, \texttt{spdimen} --- wymiar przestrzeni,
w~kt"orej le"r"a krzywe.

\begin{sloppypar}
Parametry \texttt{ppitch} i~\texttt{ppoints} opisuj"a warunki
interpolacyjne; \texttt{ppitch} jest podzia"lk"a tablicy \texttt{ppoints},
kt"ora zawiera punkty, przez kt"ore maj"a przechodzi"c (w~przybli"reniu)
krzywe.
\end{sloppypar}

Parametr \texttt{bcpitch} jest podzia"lk"a tablicy \texttt{ctlpoints},
w~kt"orej procedura umieszcza punkty kontrolne skonstruowanych krzywych
aproksymacyjnych.

Warto"sci"a procedury jest \texttt{true} je"sli obliczenie zako"nczy"lo si"e
sukcesem, albo \texttt{false} w~przeciwnym razie. Przyczyn"a niepowodzenia
mo"re by"c niespe"lnienie warunku regularno"sci zadania, wynikaj"acego
z~twierdzenia Schoenberga-Whitney, albo brak pami"eci.

\vspace{\bigskipamount}
\cprog{%
\#define mbs\_ConstructApproxBSC1f(degree,lastknot,knots,\bsl \\
\ind{4}lastpknot,pknots,ppoints,ctlpoints) \bsl \\
\ind{2}mbs\_multiConstructApproxBSCf (degree,lastknot,knots,lastpknot,\bsl \\
\ind{4}pknots,1,1,0,(float*)ppoints,0,(float*)ctlpoints) \\
\#define mbs\_ConstructApproxBSC2f(degree,lastknot,knots,\bsl \\
\ind{4}lastpknot,pknots,ppoints,ctlpoints) \bsl \\
\ind{2}mbs\_multiConstructApproxBSCf (degree,lastknot,knots,lastpknot,\bsl \\
\ind{4}pknots,1,2,0,(float*)ppoints,0,(float*)ctlpoints) \\
\#define mbs\_ConstructApproxBSC3f(degree,lastknot,knots,\bsl \\
\ind{4}lastpknot,pknots,ppoints,ctlpoints) ... \\
\#define mbs\_ConstructApproxBSC4f(degree,lastknot,knots,\bsl \\
\ind{4}lastpknot,pknots,ppoints,ctlpoints) ...}
Powy"rsze makra wywo"luj"a procedur"e
\texttt{mbs\_multiConstructApproxBSCf} w~celu otrzymania krzywej
aproksymacyjnej w przestrzeni jedno-, dwu-, tr"oj- lub czterowymiarowej.


\newpage
\section{Obcinanie krzywych B\'{e}ziera}

\cprog{%
boolean mbs\_FindPolynomialZerosf ( int degree, const float *coeff, \\
\ind{28}int *nzeros, float *zeros, float eps );}
\hspace*{\parindent}Procedura \texttt{mbs\_FindPolynomialZerosf} oblicza
rzeczywiste miejsca zerowe wielomianu stopnia~$n$ w~przedziale~$[0,1]$.

Parametry wej"sciowe: \texttt{degree} --- stopie"n $n$ wielomianu,
\texttt{coeff} --- wsp"o"lczynniki wielomianu w~bazie Bernsteina stopnia~$n$,
\texttt{eps} --- "r"adana dok"ladno"s"c rozwi"aza"n (musi to by"c liczba
dodatnia).

Parametry wyj"sciowe: \texttt{*nzeros} --- zmienna, kt"orej procedura
przypisze liczb"e znalezionych zer, \texttt{*zeros} --- tablica, do kt"orej
miejsca te maj"a by"c wpisane. Tablica ta ma mie"c d"lugo"s"c co najmniej~$n$.

Warto"sci"a procedury jest \texttt{true}, je"sli obliczenie zako"nczy"lo si"e
sukcesem, \texttt{false} w~przeciwnym razie, np.\ gdy zabrak"lo pami"eci
pomocniczej.

\vspace{\bigskipamount}
\cprog{%
void mbs\_ClipBC2f ( int ncplanes, const vector3f *cplanes, \\
\ind{12}int degree, const point2f *cpoints, \\
\ind{12}void (*output) (int degree, const point2f *cpoints) );}
Procedura \texttt{mbs\_ClipBC2f} obcina p"lask"a wielomianow"a krzyw"a
B\'{e}ziera do wielok"ata wypuk"lego, tj.\ oblicza i~wyprowadza "luki
krzywej po"lo"rone wewn"atrz tego wielok"ata.

Parametry: \texttt{ncplanes} --- liczba p"o"lp"laszczyzn, kt"orych
wielok"at jest przeci"eciem, \texttt{cplanes} --- tablica reprezentacji
tych p"o"lp"laszczyzn. Dla p"o"lp"laszczyzny $ax+by+c>0$ liczby
$a$, $b$, $c$ s"a wsp"o"lrz"ednymi odpowiedniego wektora
w~tab\-li\-cy~\texttt{cplanes}.

Parametry \texttt{degree} i~\texttt{cpoints} opisuj"a reprezentacj"e p"laskiej
krzywej B\'{e}ziera, tj.\ odpowiednio stopie"n i~punkty kontrolne.
Parametr \texttt{output} jest wska"znikiem procedury, kt"ora b"edzie wywo"lana
w~celu wyprowadzenia (np.\ narysowania) poszczeg"olnych "luk"ow krzywej
po"lo"ronych wewn"atrz wielok"ata.

\vspace{\bigskipamount}
\cprog{%
void mbs\_ClipBC2Rf ( int ncplanes, const vector3f *cplanes, \\
\ind{12}int degree, const point3f *cpoints, \\
\ind{12}void (*output) (int degree, const point3f *cpoints) );}
Procedura \texttt{mbs\_ClipBC2Rf} obcina p"lask"a wymiern"a krzyw"a
B\'{e}ziera do wielok"ata wypuk"lego, tj.\ oblicza i~wyprowadza "luki
krzywej po"lo"rone wewn"atrz tego wielok"ata.

Parametry: \texttt{ncplanes} --- liczba p"o"lp"laszczyzn, kt"orych
wielok"at jest przeci"eciem, \texttt{cplanes} --- tablica reprezentacji
tych p"o"lp"laszczyzn. Dla p"o"lp"laszczyzny $ax+by+c>0$ liczby
$a$, $b$, $c$ s"a wsp"o"lrz"ednymi odpowiedniego wektora
w~tab\-li\-cy~\texttt{cplanes}.

Parametry \texttt{degree} i~\texttt{cpoints} opisuj"a reprezentacj"e p"laskiej
wymiernej krzywej B\'{e}ziera, tj.\ odpowiednio stopie"n i~punkty kontrolne
krzywej jednorodnej.
Parametr \texttt{output} jest wska"znikiem procedury, kt"ora b"edzie wywo"lana
w~celu wyprowadzenia (np.\ narysowania) poszczeg"olnych "luk"ow krzywej
po"lo"ronych wewn"atrz wielok"ata.

\vspace{\bigskipamount}
\cprog{%
void mbs\_ClipBC3f ( int ncplanes, const vector4f *cplanes, \\
\ind{12}int degree, const point3f *cpoints, \\
\ind{12}void (*output) (int degree, const point3f *cpoints) );}
Procedura \texttt{mbs\_ClipBC3f} obcina wielomianow"a krzyw"a B\'{e}ziera
w~przestrzeni tr"ojwymiarowej do wielo"scianu wypuk"lego, tj.\ oblicza
i~wyprowadza "luki krzywej po"lo"rone wewn"atrz tego wielo"scianu.

Parametry: \texttt{ncplanes} --- liczba p"o"lprzestrzeni, kt"orych
wielo"scian jest przeci"eciem, \texttt{cplanes} --- tablica reprezentacji
tych p"o"lprzestrzeni. Dla p"o"lprzestrzeni $ax+by+cz+d>0$ liczby
$a$, $b$, $c$, $d$ s"a wsp"o"lrz"ednymi odpowiedniego wektora
w~tab\-li\-cy~\texttt{cplanes}.

Parametry \texttt{degree} i~\texttt{cpoints} opisuj"a reprezentacj"e
krzywej B\'{e}ziera, tj.\ odpowiednio stopie"n i~punkty kontrolne.
Parametr \texttt{output} jest wska"znikiem procedury, kt"ora b"edzie wywo"lana
w~celu wyprowadzenia (np.\ narysowania) poszczeg"olnych "luk"ow krzywej
po"lo"ronych wewn"atrz wielok"ata.

\vspace{\bigskipamount}
\cprog{%
void mbs\_ClipBC3Rf ( int ncplanes, const vector4f *cplanes, \\
\ind{12}int degree, const point4f *cpoints, \\
\ind{12}void (*output) (int degree, const point4f *cpoints) );}
Procedura \texttt{mbs\_ClipBC3Rf} obcina wymiern"a krzyw"a B\'{e}ziera
w~przestrzeni tr"ojwymiarowej do wielo"scianu wypuk"lego, tj.\ oblicza
i~wyprowadza "luki krzywej po"lo"rone wewn"atrz tego wielo"scianu.

Parametry: \texttt{ncplanes} --- liczba p"o"lprzestrzeni, kt"orych
wielo"scian jest przeci"eciem, \texttt{cplanes} --- tablica reprezentacji
tych p"o"lprzestrzeni. Dla p"o"lprzestrzeni $ax+by+cz+d>0$ liczby
$a$, $b$, $c$, $d$ s"a wsp"o"lrz"ednymi odpowiedniego wektora
w~tab\-li\-cy~\texttt{cplanes}.

Parametry \texttt{degree} i~\texttt{cpoints} opisuj"a reprezentacj"e
wymiernej krzywej B\'{e}ziera, tj.\ odpowiednio stopie"n i~punkty kontrolne
krzywej jednorodnej.
Parametr \texttt{output} jest wska"znikiem procedury, kt"ora b"edzie wywo"lana
w~celu wyprowadzenia (np.\ narysowania) poszczeg"olnych "luk"ow krzywej
po"lo"ronych wewn"atrz wielok"ata.


\newpage
\section{Badanie kszta"ltu "lamanych}

\cprog{%
boolean mbs\_MonotonicPolylinef ( int spdimen, int npoints, \\
\ind{33}int pitch, const float *points, \\
\ind{33}const float *v ); \\
boolean mbs\_MonotonicPolylineRf ( int spdimen, int npoints, \\
\ind{34}int pitch, const float *points, \\
\ind{34}const float *v );}
\begin{sloppypar}
\hspace*{\parindent}Procedury \texttt{mbs\_MonotonicPolylinef}
i~\texttt{mbs\_MonotonicPolylineRf} dokonuj"a sprawdzenia
monotoniczno"sci "lamanych ze wzgl"edu na wektor $\bm{v}$.
\end{sloppypar}

"Lamane le"r"a w~przestrzeni $\R^d$, o~wymiarze okre"slonym przez
parametr \texttt{spdimen}. Dla procedury \texttt{mbs\_MonotonicPolylinef}
musi on mie"c warto"s"c~$d$, a~dla procedury
\texttt{mbs\_MonotonicPolylineRf} $d+1$.

Parametr \texttt{npoints} okre"sla liczb"e punkt"ow. Wsp"o"lrz"edne
kartezja"nskie (dla \texttt{mbs\_MonotonicPolylinef}) albo jednorodne (dla
\texttt{mbs\_MonotonicPolylineRf}) tych punkt"ow s"a podane w~tablicy
\texttt{points}. Parametr \texttt{pitch} okre"sla odleg"lo"sci w~tablicy
pocz"atk"ow reprezentacji kolejnych punkt"ow (kt"ora mo"re by"c inna ni"r
\texttt{spdimen}).

Parametr \texttt{v} jest wska"znikiem tablicy zawieraj"acej
$d$~liczb b"ed"acych wsp"o"lrz"ednymi wektora $\bm{v}$.

Warto"sci"a ka"rdej z~tych procedur jest \texttt{true}, je"sli rzuty
kolejnych punkt"ow na prost"a o~kierunku wektora $\bm{v}$ s"a uporz"adkowane
wzd"lu"r prostej (i~wsp"o"lrz"edne wagowe wszystkich punkt"ow w~przypadku
procedury \texttt{mbs\_MonotonicPolylineRf} maj"a ten sam znak)
oraz \texttt{false} w~przeciwnym razie.

\vspace{\medskipamount}
Procedury mog"a by"c u"ryte do sprawdzania, czy "lamane kontrolne krzywych
s"a monotoniczne ze wzgl"edu na wektor $\bm{v}$, co jest warunkiem
dostatecznym monotoniczno"sci krzywych B\'{e}ziera i~B-sklejanych (przy
za"lo"reniu, dla krzywych wymiernych, "re wszystkie wagi maj"a ten sam
znak).


\newpage
\section{Rasteryzacja krzywych}

Procedury rasteryzacji krzywych korzystaj"a z~reprezentacji pikseli
w~postaci struktur \texttt{xpoint} oraz z~bufora i~makr jego obs"lugi
zdefiniowanych w~pliku \texttt{pkvaria.h}. W~szczeg"olno"sci krzywe
stopnia~$1$ s"a rasteryzowane jako odcinki, za pomoc"a procedury
\texttt{\_pkv\_DrawLine} z~biblioteki \texttt{libpkvaria}.

\vspace{\bigskipamount}
\cprog{%
void mbs\_RasterizeBC2f ( int degree, const point2f *cpoints, \\
\ind{25}void (*output)(const xpoint *buf, int n), \\
\ind{25}boolean outlast ); \\
void mbs\_RasterizeBC2Rf ( int degree, const point3f *cpoints, \\
\ind{26}void (*output)(const xpoint *buf, int n), \\
\ind{26}boolean outlast );}
Procedury \texttt{mbs\_RasterizeBC2f} i~\texttt{mbs\_RasterizeBD2Rf}
dokonuj"a rasteryzacji krzywych B\'{e}ziera, tj.\ wyznaczaj"a piksele
tworz"ace o"smiosp"ojne obrazy krzywych.

Parametry procedur: \texttt{degree} --- stopie"n krzywej, \texttt{cpoints}
--- punkty kontrolne (dla krzywej wymiernej s"a to punkty kontrolne krzywej
jednorodnej), \texttt{output} --- procedura wyj"sciowa (wyprowadzaj"aca
piksele, tj.\ na przyk"lad rysuj"aca je). Parametr \texttt{outlast}
okre"sla, czy ma by"c wyprowadzony ostatni piksel obrazu krzywej.
Rysuj"ac krzyw"a sklejan"a (z~wielu "luk"ow) albo zamkni"et"a nie nale"ry
wyprowadza"c ostatniego piksela ka"rdego "luku.

Liczba wywo"la"n procedury \texttt{output} w~trakcie dzia"lania procedur
rasteryzacji krzywej zale"ry od liczby pikseli do narysowania i~od
pojemno"sci wewn"etrznego buforu na piksele. Parametr \texttt{n} procedury
okre"sla liczb"e pikseli do wy"swietlenia.

\begin{figure}[ht]
  \centerline{\epsfig{file=rasterbc.ps}}
  \caption{Obrazy rastrowe krzywych B\'{e}ziera}
  \centerline{(wielomianowej i~wymiernej) stopnia $3$}
\end{figure}

\vspace{\bigskipamount}
\cprog{%
void mbs\_RasterizeBS2f ( int degree, int lastknot, \\
\ind{25}const float *knots, \\
\ind{25}const point2f *cpoints, \\
\ind{25}void (*output)(const xpoint *buf, int n), \\
\ind{25}boolean outlast ); \\
void mbs\_RasterizeBS2Rf ( int degree, int lastknot, \\
\ind{26}const float *knots, \\
\ind{26}const point3f *cpoints, \\
\ind{26}void (*output)(const xpoint *buf, int n), \\
\ind{26}boolean outlast );}
Procedury \texttt{mbs\_RasterizeBS2f} i~\texttt{mbs\_RasterizeBS2Rf}
dokonuj"a rasteryzacji p"laskich krzywych B-sklejanych. Parametry
\texttt{degree} (stopie"n), \texttt{lastknot} (numer ostatniego w"ez"la),
\texttt{knots} (tablica w"ez"l"ow) i~\texttt{cpoints} (tablica punkt"ow
kontrolnych) opisuj"a krzyw"a. Parametr \texttt{output} jest wska"znikiem
procedury, kt"ora b"edzie wywo"lywana w~celu wyprowadzenia (np.\
wy"swietlenia) pikseli. Parametr \texttt{outlast} okre"sla, czy ostatni
piksel ma by"c wyprowadzany, czy nie.

\vspace{\medskipamount}
\noindent
\textbf{Do zrobienia:} Obcinanie krzywych przed wy"swietlaniem. Sprawdzanie,
czy krzy\-wa nie jest tak kr"otka, "re jej obraz jest jednym pikselem.
,,Wyg"ladzanie'' obliczonego ci"agu pikseli.


\begin{figure}[ht]
  \centerline{\epsfig{file=rasterbs.ps}}
  \caption{Obrazy rastrowe krzywych B-sklejanych}
  \centerline{(wielomianowej i~wymiernej) stopnia $3$}
\end{figure}

\clearpage

