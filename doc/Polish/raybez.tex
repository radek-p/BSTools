
%/* //////////////////////////////////////////////////// */
%/* This file is a part of the BSTools procedure package */
%/* written by Przemyslaw Kiciak.                        */
%/* //////////////////////////////////////////////////// */

\chapter{Biblioteka \texttt{libraybez}}

Biblioteka \texttt{libraybez} zawiera procedury, kt"orych podstawowym
(ale nie jedynym) zadaniem jest wspomaganie "sledzenia promieni, a~dok"ladniej
wyznaczanie przeci"e"c promieni z~p"latami B\'{e}ziera. W~tym celu tworzone
jest drzewo rekurencyjnego binarnego podzia"lu p"lata, kt"ore ma na celu
przyspieszenie (przez wyeliminowanie wielokrotnego wykonywania tej samej
pracy) rozwi"azywania r"owna"n opisuj"acych przeci"ecia.

Ewentualne rozszerzenia tej biblioteki powinny obejmowa"c konstrukcj"e drzew
dla p"lat"ow B-sklejanych, tak"re obci"etych, oraz obs"lug"e drzew
z~dodatkowymi atrybutami, w~celu umo"rliwienia rozwi"azywania zada"n takich
jak wyznaczanie przeci"e"c powierzchni.

\section{Deklaracje i procedury wsp"olne}

\cprog{%
typedef struct \{ \\
\ind{2}float xmin, xmax, ymin, ymax, zmin, zmax; \\
\} Box3f;}
\hspace*{\parindent} Struktura typu \texttt{Box3f} reprezentuje prostopad"lo"scian.
W~reprezentacji fragmentu p"lata w~drzewie podzia"lu jest ona stosowana
do lokalizacji tego fragmentu w~przestrzeni (fragment le"ry wewn"atrz
odpowiedniego prostopad"lo"scianu).

\vspace{\bigskipamount}
\cprog{%
typedef struct \{ \\
\ind{2}point3f\ind{2}p; \\
\ind{2}vector3f nv; \\
\ind{2}float\ind{4}u, v, t; \\
\} RayObjectIntersf, *RayObjectIntersfp;}
\begin{sloppypar}
Struktura typu \texttt{RayObjecIntersf} reprezentuje punkt przeci"ecia
promienia z~p"latem. Struktura ta sk"lada si"e z~nast"epuj"acych p"ol:
\texttt{p} --- punkt wsp"olny p"lata i~promienia, \texttt{nv} --- wektor
normalny p"lata w~tym punkcie, \texttt{u}, \texttt{v}, \texttt{t} ---
parametry p"lata i~promienia odpowiadaj"ace punktowi przeci"ecia.
\end{sloppypar}


\section{Drzewa binarne dla wielomianowych p"lat"ow Beziera}

\cprog{%
typedef struct \_BezPatchTreeVertexf \{ \\
\ind{2}struct \_BezPatchTreeVertexf \\
\ind{12}*left, *right, *up; \\
\ind{2}point3f\ind{3}*ctlpoints; \\
\ind{2}float\ind{5}u0, u1, v0, v1; \\
\ind{2}Box3f\ind{5}bbox; \\
\ind{2}point3f\ind{3}pcent; \\
\ind{2}float\ind{5}maxder; \\
\ind{2}short int level; \\
\ind{2}char\ind{6}divdir; \\
\ind{2}char\ind{6}pad; \\
\} BezPatchTreeVertexf, *BezPatchTreeVertexfp;}
\hspace*{\parindent} Struktura typu \texttt{\_BezPatchTreeVertexf} reprezentuje
wierzcho"lek drzewa binarnego podzia"lu wielomianowego p"lata B\'{e}ziera $\bm{p}$.

\begin{sloppypar}
Pola tej struktury s"lu"r"a do przechowania nast"epuj"acych informacji:
\texttt{left}, \texttt{right}, \texttt{up} --- wska"zniki odpowiednio lewego
i~prawego poddrzewa oraz wska"znik ,,do g"ory'', tj.\ do wierzcho"lka,
kt"orego lewe lub prawe poddrzewo reprezentuje dany wierzcho"lek,
\texttt{ctlpoints} --- wska"znik tablicy punkt"ow kontrolnych 
fragmentu p"lata odpowiadaj"acego danemu wierzcho"lkowi,
\texttt{u0}, \texttt{u1}, \texttt{v0}, \texttt{v1}
--- liczby okre"slaj"ace dziedzin"e $[u_0,u_1]\times[v_0,v_1]$ fragmentu
p"lata, \texttt{bbox} --- prostopad"lo"scian zawieraj"acy fragment p"lata,
\texttt{pcent} --- punkt
$\bm{p}((u_0\nolinebreak +\nolinebreak u_1)/2,
(v_0\nolinebreak +\nolinebreak v_1)/2)$, \texttt{maxder}
--- g"orne oszacowanie d"lugo"sci wektora pochodnych cz"astkowych
fragmentu p"lata ze wzgl"edu na lokalne parametry, \texttt{level} --- poziom
wierzcho"lka w~drzewie, \texttt{divdir} --- wska"znik kierunku podzia"lu
fragmentu p"lata, \texttt{pad} --- pole nieu"rywane (wyr"ownuj"ace
wielko"s"c struktury do liczby parzystej).
\end{sloppypar}

\vspace{\bigskipamount}
\cprog{%
typedef struct \{ \\
\ind{2}unsigned char\ind{9}n, m; \\
\ind{2}unsigned int\ind{10}cpsize; \\
\ind{2}BezPatchTreeVertexfp root; \\
\} BezPatchTreef, *BezPatchTreefp;}
Struktura typu \texttt{BezPatchTreef} reprezentuje drzewo binarnego
podzia"lu wielomianowego p"lata B\'{e}ziera. Pola tej struktury s"a nast"epuj"ace:
\texttt{n}, \texttt{m} --- stopie"n p"lata ze wzgl"edu na zmienne $u$~i~$v$,
\texttt{cpsize} --- ilo"s"c miejsca potrzebnego do przechowywania punkt"ow
kontrolnych, \texttt{root} --- wska"znik korzenia drzewa.

\newpage
%\vspace{\bigskipamount}
\cprog{%
BezPatchTreefp \\
\ind{2}rbez\_NewBezPatchTreef ( unsigned char n, unsigned char m, \\
\ind{24}float u0, float u1, float v0, float v1, \\
\ind{24}point3f *ctlpoints );}
Procedura \texttt{rbez\_NewBezPatchTreef} tworzy drzewo binarnego podzia"lu
wielomianowego p"lata B\'{e}ziera i~zwraca wska"znik struktury, kt"ora
reprezentuje to drzewo. Drzewo pocz"atkowo sk"lada si"e tylko z~korzenia,
kt"ory reprezentuje ca"ly p"lat.

Parametry \texttt{n}~i~\texttt{m} okre"slaj"a stopie"n p"lata odpowiednio ze
wzgl"edu na zmienne $u$~i~$v$. Parametry \texttt{u0}, \texttt{u1},
\texttt{v0} i~\texttt{v1} okre"slaj"a dziedzin"e p"lata, tj.~prostok"at
$[u_0,u_1]\times[v_0,v_1]$ (je"sli p"lat powsta"l z~podzia"lu p"lata
B-sklejanego, to liczby te powinny by"c odpowiednimi w"ez"lami).

Parametr \texttt{ctlpoints} jest tablic"a punkt"ow kontrolnych p"lata.

Warto"sci"a procedury jest wska"znik do struktury opisuj"acej drzewo.
Obszar pami"eci na t"e struktur"e i~na struktury opisuj"ace wszystkie
wierzcho"lki drzewa jest rezerwowany za pomoc"a funkcji \texttt{malloc}.

\vspace{\bigskipamount}
\cprog{%
void rbez\_DestroyBezPatchTreef ( BezPatchTreefp tree );}
Procedura \texttt{rbez\_DestroyBezPatchTreef} zwalnia pami"e"c zajmowan"a przez
drze\-wo binarnego podzia"lu p"lata. Parametr \texttt{tree} jest wska"znikiem
struktury re\-pre\-zen\-tu\-j"a\-cej drzewo.

\vspace{\bigskipamount}
\cprog{%
BezPatchTreeVertexp \\
\ind{2}rbez\_GetBezLeftVertexf ( BezPatchTreefp tree, \\
\ind{24}BezPatchTreeVertexfp vertex ); \\
BezPatchTreeVertexfp \\
\ind{2}rbez\_GetBezRightVertexf ( BezPatchTreefp tree, \\
\ind{25}BezPatchTreeVertexfp vertex );}
\begin{sloppypar}
Procedury \texttt{rbez\_GetBezLeftVertexf} i~\texttt{rbez\_GetBezRightVertexf}
zwracaj"a wska"zniki odpowiednio lewego lub prawego poddrzewa danego
wierzcho"lka drzewa binarnego podzia"lu p"lata.%
\end{sloppypar}

Parametry: \texttt{tree} --- wska"znik struktury opisuj"acej drzewo,
\texttt{vertex} --- wska"znik jednego z~wierzcho"lk"ow tego drzewa.

Warto"sci"a procedury jest wska"znik do korzenia odpowiedniego (lewego albo
prawego) poddrzewa. Je"sli wierzcho"lek ten nie istnieje, to procedura
dokonuje podzia"lu fragmentu p"lata reprezentowanego przez wierzcho"lek
\texttt{*vertex} i~tworzy lewy i~prawy wierzcho"lek (dla ka"rdego
wierzcho"lka oba poddrzewa istniej"a albo oba nie istniej"a), a~nast"epnie
zwraca odpowiedni wska"znik.

\newpage
%\vspace{\bigskipamount}
\cprog{%
int rbez\_FindRayBezPatchIntersf ( BezPatchTreef *tree, \\
\ind{26}ray3f *ray, \\
\ind{26}int maxlevel, int maxinters, \\
\ind{26}int *ninters, RayObjectIntersf *inters );}
\begin{sloppypar}
Procedura \texttt{FindRayBezPatchIntersf} oblicza punkty przeci"ecia
promienia (p"o"lprostej) z~wielomianowym p"latem B\'{e}ziera w~$\R^3$.
\end{sloppypar}

Parametry: \texttt{tree} --- wska"znik drzewa binarnego podzia"lu p"lata;
\texttt{ray} --- wska"znik promienia (struktura \texttt{ray3f} jest
zdefiniowana w~pliku \texttt{geomf.h}; \texttt{maxlevel} --- ograniczenie
wysoko"sci drzewa binarnego podzia"lu (procedura nie b"edzie tworzy"c
wierzcho"lk"ow drzewa na wy"rszym poziomie); \texttt{maxinters} ---
d"lugo"s"c tablicy \texttt{inters}, w~kt"orej procedura ma umie"sci"c
wyniki. Tablica ta musi mie"c co najmniej tak"a d"lugo"s"c, procedura
zako"nczy dzia"lanie po znalezieniu najwy"rej tylu przeci"e"c. Warto"s"c
parametru \texttt{*ninters} na wyj"sciu jest liczb"a znalezionych
przeci"e"c.

Warto"sci"a procedury jest liczba znalezionych punkt"ow przeci"ecia.


\section{Drzewa binarne dla wymiernych p"lat"ow B\'{e}ziera}

Drzewa binarnego podzia"lu wymiernych p"lat"ow B\'{e}ziera s"a oprogramowane
w~prawie identyczny spos"ob jak drzewa binarnego podzia"lu p"lat"ow
wielomianowych. Wszystkie struktury danych i~procedury opisane w~poprzednim
punkcie maj"a tu swoje odpowiedniki.

\vspace{\bigskipamount}
\cprog{%
typedef struct \_RBezPatchTreeVertexf \{ \\
\ind{2}struct \_RBezPatchTreeVertexf \\
\ind{12}*left, *right, *up; \\
\ind{2}point4f\ind{3}*ctlpoints; \\
\ind{2}float\ind{5}u0, u1, v0, v1; \\
\ind{2}Box3f\ind{5}bbox; \\
\ind{2}point3f\ind{3}pcent; \\
\ind{2}float\ind{5}maxder; \\
\ind{2}short int level; \\
\ind{2}char\ind{6}divdir; \\
\ind{2}char\ind{6}pad; \\
\} RBezPatchTreeVertexf, *RBezPatchTreeVertexfp;}
\indent Struktura typu \texttt{\_RBezPatchTreeVertexf} reprezentuje wierzcho"lek
drzewa binarnego podzia"lu wymiernego p"lata B\'{e}ziera $\bm{p}$.

\begin{sloppypar}
Pola tej struktury s"lu"r"a do przechowania nast"epuj"acych informacji:
\texttt{left}, \texttt{right}, \texttt{up} --- wska"zniki odpowiednio lewego
i~prawego poddrzewa oraz wska"znik ,,do g"ory'', tj.\ do wierzcho"lka,
kt"orego lewe lub prawe poddrzewo reprezentuje dany wierzcho"lek,
\texttt{ctlpoints} --- wska"znik tablicy punkt"ow kontrolnych jednorodnego
p"lata B\'{e}ziera reprezentuj"acego fragment p"lata odpowiadaj"acego
danemu wierzcho"lkowi, \texttt{u0}, \texttt{u1}, \texttt{v0}, \texttt{v1}
--- liczby okre"slaj"ace dziedzin"e $[u_0,u_1]\times[v_0,v_1]$ fragmentu
p"lata, \texttt{bbox} --- prostopad"lo"scian zawieraj"acy fragment p"lata,
\texttt{pcent} --- punkt
$\bm{p}((u_0\nolinebreak +\nolinebreak u_1)/2,
(v_0\nolinebreak +\nolinebreak v_1)/2)$, \texttt{maxder}
--- g"orne oszacowanie d"lugo"sci wektora pochodnych cz"astkowych
fragmentu p"lata ze wzgl"edu na lokalne parametry, \texttt{level} --- poziom
wierzcho"lka w~drzewie, \texttt{divdir} --- wska"znik kierunku podzia"lu
fragmentu p"lata, \texttt{pad} --- pole nieu"rywane (wyr"ownuj"ace
wielko"s"c struktury do liczby pa\-rzystej).
\end{sloppypar}

\vspace{\bigskipamount}
\cprog{%
typedef struct \{ \\
\ind{2}unsigned char\ind{9}n, m; \\
\ind{2}unsigned int\ind{10}cpsize; \\
\ind{2}RBezPatchTreeVertexfp root; \\
\} RBezPatchTreef, *RBezPatchTreefp;}
Struktura typu \texttt{RBezPatchTreef} reprezentuje drzewo binarnego
podzia"lu p"lata B\'{e}ziera. Pola tej struktury s"a nast"epuj"ace:
\texttt{n}, \texttt{m} --- stopie"n p"lata ze wzgl"edu na zmienne $u$~i~$v$,
\texttt{cpsize} --- ilo"s"c miejsca potrzebnego do przechowywania punkt"ow
kontrolnych, \texttt{root} --- wska"znik korzenia drzewa.

\vspace{\bigskipamount}
\cprog{%
RBezPatchTreefp \\
\ind{2}rbez\_NewRBezPatchTreef ( unsigned char n, unsigned char m, \\
\ind{24}float u0, float u1, float v0, float v1, \\
\ind{24}point4f *ctlpoints );}
Procedura \texttt{rbez\_NewRBezPatchTreef} tworzy drzewo binarnego podzia"lu
wymiernego p"lata B\'{e}ziera i~zwraca wska"znik struktury, kt"ora
reprezentuje to drzewo. Drzewo pocz"atkowo sk"lada si"e tylko z~korzenia,
kt"ory reprezentuje ca"ly p"lat.

Parametry \texttt{n}~i~\texttt{m} okre"slaj"a stopie"n p"lata odpowiednio ze
wzgl"edu na zmienne $u$~i~$v$. Parametry \texttt{u0}, \texttt{u1},
\texttt{v0} i~\texttt{v1} okre"slaj"a dziedzin"e p"lata, tj.~prostok"at
$[u_0,u_1]\times[v_0,v_1]$ (je"sli p"lat powsta"l z~podzia"lu p"lata
B-sklejanego, to liczby te powinny by"c odpowiednimi w"ez"lami).

Parametr \texttt{ctlpoints} jest tablic"a punkt"ow kontrolnych p"lata
jednorodnego.

Warto"sci"a procedury jest wska"znik do struktury opisuj"acej drzewo.
Obszar pami"eci na t"e struktur"e i~na struktury opisuj"ace wszystkie
wierzcho"lki drzewa jest rezerwowany za pomoc"a funkcji \texttt{malloc}.

\vspace{\bigskipamount}
\cprog{%
void rbez\_DestroyRBezPatchTreef ( RBezPatchTreefp tree );}
\begin{sloppypar}
Procedura \texttt{rbez\_DestroyRBezPatchTreef} zwalnia pami"e"c zajmowan"a przez
drze\-wo binarnego podzia"lu p"lata. Parametr \texttt{tree} jest wska"znikiem
struktury reprezentuj"acej drzewo.%
\end{sloppypar}

\vspace{\bigskipamount}
\cprog{%
RBezPatchTreeVertexp \\
\ind{2}rbez\_GetRBezLeftVertexf ( RBezPatchTreefp tree, \\
\ind{24}RBezPatchTreeVertexfp vertex ); \\
RBezPatchTreeVertexfp \\
\ind{2}rbez\_GetRBezRightVertexf ( RBezPatchTreefp tree, \\
\ind{25}RBezPatchTreeVertexfp vertex );}
Procedury \texttt{rbez\_GetRBezLeftVertexf} i~\texttt{rbez\_GetRBezRightVertexf}
zwracaj"a wska"zniki odpowiednio lewego lub prawego poddrzewa danego
wierzcho"lka drzewa binarnego podzia"lu p"lata.

Parametry: \texttt{tree} --- wska"znik struktury opisuj"acej drzewo,
\texttt{vertex} --- wska"znik jednego z~wierzcho"lk"ow tego drzewa.

Warto"sci"a procedury jest wska"znik do korzenia odpowiedniego (lewego albo
prawego) poddrzewa. Je"sli wierzcho"lek ten nie istnieje, to procedura
dokonuje podzia"lu fragmentu p"lata reprezentowanego przez wierzcho"lek
\texttt{*vertex} i~tworzy lewy i~prawy wierzcho"lek (dla ka"rdego
wierzcho"lka oba poddrzewa istniej"a albo oba nie istniej"a), a~nast"epnie
zwraca odpowiedni wska"znik.

\vspace{\bigskipamount}
\cprog{%
int rbez\_FindRayRBezPatchIntersf ( RBezPatchTreef *tree, \\
\ind{26}ray3f *ray, \\
\ind{26}int maxlevel, int maxinters, \\
\ind{26}int *ninters, RayObjectIntersf *inters );}
\begin{sloppypar}
Procedura \texttt{FindRayRBezPatchIntersf} oblicza punkty przeci"ecia
promienia (p"o"lprostej) z~wymiernym p"latem B\'{e}ziera w~$\R^3$.
\end{sloppypar}

Parametry: \texttt{tree} --- wska"znik drzewa binarnego podzia"lu p"lata;
\texttt{ray} --- wska"znik promienia (struktura \texttt{ray3f} jest
zdefiniowana w~pliku \texttt{geomf.h}; \texttt{maxlevel} --- ograniczenie
wysoko"sci drzewa binarnego podzia"lu (procedura nie b"edzie tworzy"c
wierzcho"lk"ow drzewa na wy"rszym poziomie); \texttt{maxinters} ---
d"lugo"s"c tablicy \texttt{inters}, w~kt"orej procedura ma umie"sci"c
wyniki. Tablica ta musi mie"c co najmniej tak"a d"lugo"s"c, procedura
zako"nczy dzia"lanie po znalezieniu najwy"rej tylu przeci"e"c. Warto"s"c
parametru \texttt{*ninters} na wyj"sciu jest liczb"a znalezionych
przeci"e"c.

Warto"sci"a procedury jest liczba znalezionych punkt"ow przeci"ecia.

\begin{figure}[ht]
  \centerline{\epsfig{file=raybezp.ps}}
  \caption{\label{fig:raybezp}Obraz p"lata wykonany metod"a "sledzenia
    promieni za pomoc"a}
  \centerline{procedury \texttt{rbez\_FindRayRBezPatchIntersf}. Obok podzia"l}
  \centerline{dziedziny p"lata dokonany przez procedury obs"lugi drzewa
    podzia"lu}
\end{figure}
Rysunek~\ref{fig:raybezp} przedstawia obrazek wymiernego p"lata B\'{e}ziera
stopnia $(5,5)$ wykonany przy u"ryciu tej procedury. Pe"lny kod programu,
kt"ory utworzy"l ten obrazek jest w~pliku \texttt{../cpict/raybez.c}.


