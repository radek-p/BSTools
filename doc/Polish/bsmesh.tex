
%/* //////////////////////////////////////////////////// */
%/* This file is a part of the BSTools procedure package */
%/* written by Przemyslaw Kiciak.                        */
%/* //////////////////////////////////////////////////// */

\chapter{Biblioteka \texttt{libbsmesh}}

\section{Reprezentacja siatki}

Siatka jest obiektem sk"ladaj"acym si"e z~\textbf{wierzcho"lk"ow},
\textbf{kraw"edzi} i~\textbf{"scian}. Mo"re ona s"lu"ry"c m.in.\ do
reprezentowania np.\ bry"ly wielo"sciennej lub powierzchni sklejanej.
Kraw"ed"z jest odcinkiem "l"acz"acym wierzcho"lki.
"Sciana jest zamkni"et"a "laman"a zbudowan"a z~kraw"edzi. Kraw"ed"z mo"re
nale"re"c do jednej lub do dw"och "scian; w~pierwszym przypadku jest
nazywana \textbf{kraw"edzi"a brzegow"a}, a~w~drugim --- \textbf{kraw"edzi"a
wewn"etrzn"a}.

W~reprezentacji siatki przetwarzanej przez procedury z~biblioteki
\texttt{libbsmesh} kraw"edzie brzegowe i~wewn"etrzne s"a reprezentowane
odpowiednio za pomoc"a jednej lub dw"och \textbf{p"o"lkraw"edzi}.
P"o"lkraw"ed"z ma orientacj"e, tj.\ jeden z~jej wierzcho"lk"ow jest
pocz"atkiem, a~drugi ko"ncem. Orientacja drugiej p"o"lkraw"edzi w~parze
reprezentuj"acej kraw"ed"z wewn"etrzn"a jest przeciwna. Ka"rda
p"o"lkraw"ed"z jest zwi"azana tylko z~jedn"a "scian"a.

Wierzcho"lki, p"o"lkraw"edzie i~"sciany s"a przechowywane w~tablicach
(indeksowanych od~$0$), przy czym identyfikatorem wierzcho"lka,
p"o"lkraw"edzi i~"sciany jest jego indeks w~tablicy. Kompletna reprezentacja
siatki sk"lada si"e z~trzech liczb: liczby wierzcho"lk"ow~$n_v$, liczby
p"o"lkraw"edzi~$n_h$ i~liczby "scian~$n_f$, oraz sze"sciu tablic:
tablicy wierzcho"lk"ow~\texttt{v}, tablicy indeks"ow p"o"lkraw"edzi
wychodz"acych z~wierzcho"lk"ow~\texttt{vhei}, tablicy pozycji
wierzcho"lk"ow~\texttt{pos}, tablicy p"o"lkraw"edzi~\texttt{he}, tablicy
"scian~\texttt{fac} i~tablicy indeks"ow p"o"lkraw"edzi "scian~\texttt{fhei}.
Wierzcho"lki, "sciany i~p"o"lkraw"edzie s"a opisane przez nast"epuj"ace
struktury:

\vspace{\medskipamount}
\begin{listingC}
typedef struct {
    char degree;
    int  firsthalfedge;
  } BSMfacet, BSMvertex;

typedef struct {
    int v0, v1; 
    int facetnum;
    int otherhalf;
  } BSMhalfedge;  
\end{listingC}

\begin{figure}[ht]
  \centerline{\begin{picture}(3000,1700)
    \put(0,100){\epsfig{file=bsmesh.ps}}
    \put(1900,1560){\texttt{int nv = 6, nhe = 11, nf = 3;}}
    \put(1900,1440){\texttt{BSMvertex v[6] = \{\{3,0\},\{2,3\},}}
    \put(1900,1320){\texttt{ \ \ \ \{2,5\},\{2,7\},\{1,9\},\{1,10\}\};}}
    \put(1900,1200){\texttt{int vhei[11] = \{7,3,0,4,6,8,}}
    \put(1900,1080){\texttt{ \ \ \ 10,1,2,5,9\};}}
    \put(1900,960){\texttt{BSMhalfedge he[11] = \{\{0,3,1,1\},}}
    \put(1900,840){\texttt{ \ \ \{3,0,0,0\},\{3,4,1,-1\},}}
    \put(1900,720){\texttt{ \ \ \{0,1,2,4\},\{1,0,1,3\},}}
    \put(1900,600){\texttt{ \ \ \{4,1,1,-1\},\{1,5,2,-1\},}}
    \put(1900,480){\texttt{ \ \ \{0,2,0,8\},\{2,0,2,7\}}}
    \put(1900,360){\texttt{ \ \ \{5,2,2,-1\},\{2,3,0,-1\}\};}}
    \put(1900,240){\texttt{BSMfacet fac[3] = \{\{3,0\},}}
    \put(1900,120){\texttt{ \ \ \{4,3\},\{4,7\}\};}}
    \put(1300,0){\texttt{int fhei[11] = \{1,7,10,5,4,0,2,9,8,3,6\};}}
  \end{picture}}
  \caption{\label{fig:bsmesh}Przyk"lad siatki}
\end{figure}
Przyk"lad reprezentacji siatki jest pokazany na rysunku~\ref{fig:bsmesh};
tablica ze wsp"o"lrz"ednymi wierzcho"lk"ow jest pomini"eta.


\begin{listingC}
boolean bsm_CheckMeshIntegrity (
             int nv, const BSMvertex *mv, const int *mvhei,
             int nhe, const BSMhalfedge *mhe,
             int nfac, const BSMfacet *mfac, const int  *mfhei );
\end{listingC}

\begin{listingC}
void bsm_TagMesh ( int nv, BSMvertex *mv, int *mvhei,
                   int nhe, BSMhalfedge *mhe,
                   int nfac, BSMfacet *mfac, int *mfhei,
                   char *vtag, char *ftag,
                   int *vi, int *vb, int *ei, int *eb );
\end{listingC}

\newpage
\section{Procedury zag"eszczania siatki}

\begin{listingC}
boolean bsm_DoublingNum ( int inv, BSMvertex *imv, int *imvhei,
                          int inhe, BSMhalfedge *imhe,
                          int infac, BSMfacet *imfac, int *imfhei,
                          int *onv, int *onhe, int *onfac );
boolean bsm_Doublingd ( int spdimen,
                int inv, BSMvertex *imv, int *imvhei, double *iptc,
                int inhe, BSMhalfedge *imhe,
                int infac, BSMfacet *imfac, int *imfhei,
                int *onv, BSMvertex *omv, int *omvhei, double *optc,
                int *onhe, BSMhalfedge *omhe,
                int *onfac, BSMfacet *omfac, int *omfhei );
\end{listingC}

\begin{listingC}
int bsm_DoublingMatSize ( int inv, BSMvertex *imv, int *imvhei,
                          int inhe, BSMhalfedge *imhe,
                          int infac, BSMfacet *imfac, int *imfhei );
boolean bsm_DoublingMatd ( int inv, BSMvertex *imv, int *imvhei,
                           int inhe, BSMhalfedge *imhe,
                           int infac, BSMfacet *imfac, int *imfhei,
                           int *onv, BSMvertex *omv, int *omvhei,  
                           int *onhe, BSMhalfedge *omhe,
                           int *onfac, BSMfacet *omfac, int *omfhei,
                           int *ndmat, index2 *dmi, double *dmc );  
\end{listingC}

\begin{listingC}
boolean bsm_AveragingNum ( int inv, BSMvertex *imv, int *imvhei,
                           int inhe, BSMhalfedge *imhe,
                           int infac, BSMfacet *imfac, int *imfhei,
                           int *onv, int *onhe, int *onfac );
boolean bsm_Averagingd ( int spdimen,
                 int inv, BSMvertex *imv, int *imvhei, double *iptc,
                 int inhe, BSMhalfedge *imhe,
                 int infac, BSMfacet *imfac, int *imfhei,
                 int *onv, BSMvertex *omv, int *omvhei, double
                 *optc,
                 int *onhe, BSMhalfedge *omhe,
                 int *onfac, BSMfacet *omfac, int *omfhei );
\end{listingC}

\begin{listingC}
int bsm_AveragingMatSize ( int inv, BSMvertex *imv, int *imvhei,
                           int inhe, BSMhalfedge *imhe,
                           int infac, BSMfacet *imfac, int *imfhei );
boolean bsm_AveragingMatd ( int inv, BSMvertex *imv, int *imvhei,
                            int inhe, BSMhalfedge *imhe,
                            int infac, BSMfacet *imfac, int *imfhei,
                            int *onv, BSMvertex *omv, int *omvhei,  
                            int *onhe, BSMhalfedge *omhe,
                            int *onfac, BSMfacet *omfac, int *omfhei,
                            int *namat, index2 *ami, double *amc );  
\end{listingC}

\begin{listingC}
boolean bsm_RefineBSMeshd ( int spdimen, int degree,
            int inv, BSMvertex *imv, int *imvhei, double *iptc,
            int inhe, BSMhalfedge *imhe,
            int infac, BSMfacet *imfac, int *imfhei,
            int *onv, BSMvertex **omv, int **omvhei, double **optc,
            int *onhe, BSMhalfedge **omhe,
            int *onfac, BSMfacet **omfac, int **omfhei );

\end{listingC}

\begin{listingC}
boolean bsm_RefinementMatd ( int degree,
                             int inv, BSMvertex *imv, int *imvhei,
                             int inhe, BSMhalfedge *imhe,
                             int infac, BSMfacet *imfac, int *imfhei,
                             int *onv, BSMvertex **omv, int **omvhei,
                             int *onhe, BSMhalfedge **omhe,
                             int *onfac, BSMfacet **omfac, int **omfhei,
                             int *nrmat, index2 **rmi, double **rmc );  
\end{listingC}


\newpage
\section{Operacje Eulerowskie i~nie-Eulerowskie}

\begin{listingC}
void bsm_MergeMeshesd ( int spdimen,
                  int nv1, BSMvertex *mv1, int *mvhei1, double *vpc1,
                  int nhe1, BSMhalfedge *mhe1,
                  int nfac1, BSMfacet *mfac1, int *mfhei1,
                  int nv2, BSMvertex *mv2, int *mvhei2, double *vpc2,
                  int nhe2, BSMhalfedge *mhe2,
                  int nfac2, BSMfacet *mfac2, int *mfhei2,
                  int *onv, BSMvertex *omv, int *omvhei, double *ovpc,
                  int *onhe, BSMhalfedge *omhe,
                  int *onfac, BSMfacet *omfac, int *omfhei );
\end{listingC}

\begin{listingC}
boolean bsm_RemoveFacetNum ( int inv, BSMvertex *imv, int *imvhei,
                             int inhe, BSMhalfedge *imhe,
                             int infac, BSMfacet *imfac, int *imfhei,
                             int nfr,
                             int *onv, int *onhe, int *onfac );
boolean bsm_RemoveFacetd ( int spdimen,
                   int inv, BSMvertex *imv, int *imvhei, double *iptc,
                   int inhe, BSMhalfedge *imhe,
                   int infac, BSMfacet *imfac, int *imfhei,
                   int nfr,
                   int *onv, BSMvertex *omv, int *omvhei, double *optc,
                   int *onhe, BSMhalfedge *omhe,
                   int *onfac, BSMfacet *omfac, int *omfhei );
\end{listingC}

\begin{listingC}
void bsm_FacetEdgeDoublingNum ( int inv, BSMvertex *imv, int *imvhei,
                                int inhe, BSMhalfedge *imhe,
                                int infac, BSMfacet *imfac, int *imfhei,
                                int fn,
                                int *onv, int *onhe, int *onfac );
boolean bsm_FacetEdgeDoublingd ( int spdimen,
                 int inv, BSMvertex *imv, int *imvhei, double *iptc,
                 int inhe, BSMhalfedge *imhe,
                 int infac, BSMfacet *imfac, int *imfhei,
                 int fn,
                 int *onv, BSMvertex *omv, int *omvhei,
                 double *optc,
                 int *onhe, BSMhalfedge *omhe,
                 int *onfac, BSMfacet *omfac, int *omfhei );
\end{listingC}

\begin{listingC}
void bsm_RemoveVertexNum ( int inv, BSMvertex *imv, int *imvhei,
                           int inhe, BSMhalfedge *imhe,
                           int infac, BSMfacet *imfac, int *imfhei,
                           int nvr,
                           int *onv, int *onhe, int *onfac );
boolean bsm_RemoveVertexd ( int spdimen,
                int inv, BSMvertex *imv, int *imvhei, double *iptc,
                int inhe, BSMhalfedge *imhe,
                int infac, BSMfacet *imfac, int *imfhei,
                int nvr,
                int *onv, BSMvertex *omv, int *omvhei, double *optc,
                int *onhe, BSMhalfedge *omhe,
                int *onfac, BSMfacet *omfac, int *omfhei );
\end{listingC}

\begin{listingC}
void bsm_ContractEdgeNum ( int inv, BSMvertex *imv, int *imvhei,
                           int inhe, BSMhalfedge *imhe,
                           int infac, BSMfacet *imfac, int *imfhei,
                           int nche,
                           int *onv, int *onhe, int *onfac );
int bsm_ContractEdged ( int spdimen,
                int inv, BSMvertex *imv, int *imvhei, double *iptc,
                int inhe, BSMhalfedge *imhe,
                int infac, BSMfacet *imfac, int *imfhei,
                int nche,
                int *onv, BSMvertex *omv, int *omvhei, double *optc,
                int *onhe, BSMhalfedge *omhe,
                int *onfac, BSMfacet *omfac, int *omfhei );
\end{listingC}

\begin{listingC}
int bsm_HalfedgeLoopLength ( int nv, BSMvertex *mv, int *mvhei,
                             int nhe, BSMhalfedge *mhe,
                             int he );
\end{listingC}

\begin{listingC}
boolean bsm_GlueHalfedgeLoopsd ( int spdimen,
                     int inv, BSMvertex *imv, int *imvhei, double *ivc,
                     int inhe, BSMhalfedge *imhe,
                     int infac, BSMfacet *imfac, int *imfhei,
                     int he1, int he2,
                     int *onv, BSMvertex *omv, int *omvhei,
                     double *ovc,
                     int *onhe, BSMhalfedge *omhe,
                     int *onfac, BSMfacet *omfac, int *omfhei );
\end{listingC}


\newpage
\section{Wyszukiwanie podsiatek regularnych i~specjalnych}

\begin{listingC}
typedef struct {
    byte  el_type;
    byte  degree;
    byte  snet_rad;
    short snet_nvert;
    int   first_snet_vertex;
  } bsm_special_el;

typedef struct {
    int            nspecials;
    int            nspvert;
    int            nextravert;
    bsm_special_el *spel;
    int            *spvert;
  } bsm_special_elem_list;
\end{listingC}


\begin{listingC}
boolean bsm_FindRegularSubnets ( int nv, BSMvertex *mv, int *mvhei,
                         int nhe, BSMhalfedge *mhe,
                         int nfac, BSMfacet *mfac, int *mfhei,
                         int d, void *usrptr,
                         void (*output)( int d, int *vertnum, int *mtab,
                                         void *usrptr ) );
\end{listingC}

\begin{listingC}
boolean bsm_FindSpecialVSubnets ( int nv, BSMvertex *mv, int *mvhei,
                          int nhe, BSMhalfedge *mhe,
                          int nfac, BSMfacet *mfac, int *mfhei,
                          int d, void *usrptr,
                          void (*output)( int d, int k, int *vertnum,
                                          int *mtab, void *usrptr ) );
\end{listingC}

\begin{listingC}
boolean bsm_FindSpecialFSubnets ( int nv, BSMvertex *mv, int *mvhei,
                          int nhe, BSMhalfedge *mhe,
                          int nfac, BSMfacet *mfac, int *mfhei,
                          int d, void *usrptr,
                          void (*output)( int d, int k, int *vertnum,
                                          int *mtab, void *usrptr ) );
\end{listingC}

\begin{listingC}
boolean bsm_CountSpecialVSubnets ( int nv, BSMvertex *mv, int *mvhei,
                                   int nhe, BSMhalfedge *mhe,
                                   int nfac, BSMfacet *mfac, int *mfhei,
                                   byte snet_rad,
                                   int *nspecials, int *nspvert );
boolean bsm_FindSpecialVSubnetList (
                               int nv, BSMvertex *mv, int *mvhei,
                               int nhe, BSMhalfedge *mhe,
                               int nfac, BSMfacet *mfac, int *mfhei,
                               byte snet_rad,
                               boolean append,
                               bsm_special_elem_list *list );
\end{listingC}

\begin{listingC}
boolean bsm_CountSpecialFSubnets ( int nv, BSMvertex *mv, int *mvhei,
                                   int nhe, BSMhalfedge *mhe,
                                   int nfac, BSMfacet *mfac, int *mfhei,
                                   byte snet_rad,
                                   int *nspecials, int *nspvert );
boolean bsm_FindSpecialFSubnetLists (
                                int nv, BSMvertex *mv, int *mvhei,
                                int nhe, BSMhalfedge *mhe,
                                int nfac, BSMfacet *mfac, int *mfhei,
                                boolean append,
                                byte snet_rad, 
                                bsm_special_elem_list *list );
\end{listingC}


\section{Inne procedury}

\begin{listingC}
void bsm_TagBoundaryZoneVertices ( int nv, BSMvertex *mv, int *mvhei,
                                   int nhe, BSMhalfedge *mhe,
                                   char d, char *vtag );
\end{listingC}

\begin{listingC}
boolean bsm_FindVertexDistances1 ( int nv, BSMvertex *mv, int *mvhei,
                                   int nhe, BSMhalfedge *mhe,
                                   int nfac, BSMfacet *mfac, int *mfhei,
                                   int v, int *dist );
\end{listingC}

\begin{listingC}
boolean bsm_FindVertexDistances2 ( int nv, BSMvertex *mv, int *mvhei,
                                   int nhe, BSMhalfedge *mhe,
                                   int nfac, BSMfacet *mfac, int *mfhei,
                                   int v, int *dist );
\end{listingC}



