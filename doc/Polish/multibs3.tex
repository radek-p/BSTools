
%/* //////////////////////////////////////////////////// */
%/* This file is a part of the BSTools procedure package */
%/* written by Przemyslaw Kiciak.                        */
%/* //////////////////////////////////////////////////// */

\newpage
\section{Wstawianie i usuwanie w"ez"l"ow}

\subsection{\label{ssect:knot:ins}Algorytm Boehma}

Poni"rej jest opisana procedura i makra, kt"ore s"lu"r"a do wstawiania
pojedy"nczego w"ez"la do reprezentacji B-sklejanej krzywych, za pomoc"a
algorytmu Boehma. Reprezentacja krzywych jest \emph{zmieniana}, tj.\ obszar
pami"eci zajmowany pocz"atkowo przez dan"a reprezentacj"e krzywych (ci"ag
w"ez"l"ow i tablica punkt"ow kontrolnych), po wykonaniu procedury zawiera
now"a reprezentacj"e, z~dodatkowym w"ez"lem. Je"sli potrzebne s"a obie
reprezentacje, to reprezentacj"e dan"a trzeba skopiowa"c i~wykona"c
wstawianie w"ez"la do kopii.

\vspace{\bigskipamount}
\cprog{%
int mbs\_multiKnotInsf ( int degree, int *lastknot, \\
\ind{24}float *knots, \\
\ind{24}int ncurves, int spdimen, \\
\ind{24}int inpitch, int outpitch, \\
\ind{24}float *ctlpoints, float t );}
Procedura \texttt{mbs\_multiKnotInsf} wstawia w"eze"l \texttt{t} do
reprezentacji krzywych B-skle\-ja\-nych stopnia $n$=\texttt{degree}.
W ten spos"ob powstaje nowa
reprezentacja tych krzywych, przechowywana w miejscu reprezentacji
pocz"atkowej. Warto"s"c parametru \texttt{t} musi by"c liczb"a z~przedzia"lu
$[\mbox{\texttt{knots[degree]}},\mbox{\texttt{knots[lastknot-degree]}}]$.

Na wej"sciu parametr \texttt{*lastknot} okre"sla indeks $N$ ostatniego w"ez"la
w~ci"agu w"ez"l"ow reprezentacji pocz"atkowej; na wyj"sciu parametr ten jest
zwi"ekszany o \texttt{1}, co wi"a"re si"e z~wyd"lu"reniem ci"agu w"ez"l"ow o
jedn"a liczb"e --- warto"s"c parametru \texttt{t}, wstawian"a do tablicy
\texttt{knots}. Tablica ta musi mie"c zatem d"lugo"s"c co najmniej
\texttt{*lastknot+2}, aby pomie"sci"c wyd"lu"rony ci"ag w"ez"l"ow.

\begin{sloppypar}
Parametr \texttt{ncurves} okre"sla liczb"e krzywych, za"s warto"s"c~$d$
parametru \texttt{spdimen} jest wymiarem przestrzeni, w kt"orej te krzywe
s"a po"lo"rone. Ka"rda krzywa jest pocz"atkowo reprezentowana przez
$N-n$ punkt"ow w przestrzeni $d$-wymiarowej.
Wsp"o"lrz"edne tych punkt"ow ($(N-n)d$ liczb)
s"a przechowywane w tablicy \texttt{ctlpoints}. Pierwsza wsp"o"lrz"edna
pierwszego punktu kontrolnego pierwszej krzywej jest na pocz"atku tablicy.
Poniewa"r na wyj"sciu reprezentacja ka"rdej krzywej ma o jeden punkt
wi"ecej, wi"ec s"a dwa parametry opisuj"ace podzia"lk"e, czyli
odleg"lo"s"c w~tablicy mi"edzy pocz"atkami reprezentacji kolejnych krzywych:
\texttt{inpitch}
okre"sla podzia"lk"e pocz"atkow"a, co najmniej $(N-n)d$,
parametr \texttt{outpitch} okre"sla podzia"lk"e ko"ncow"a, kt"ora nie mo"re
by"c mniejsza ni"r~$(N-n+1)d$.
\end{sloppypar}

Warto"sci"a procedury jest numer $k$ przedzia"lu $[u_k,u_{k+1})$
pocz"atkowego ci"agu w"ez"l"ow, w~kt"orym le"ry nowy w"eze"l~$t$.

\vspace{\medskipamount}
\noindent
\textbf{Uwaga:} Aby wstawi"c w"eze"l do krzywej zamkni"etej, zamiast
\texttt{mbs\_multiKnotInsf} nale"ry u"ry"c
procedury~\texttt{mbs\_multiKnotInsClosedf}.


\vspace{\bigskipamount}
\cprog{%
\#define mbs\_KnotInsC1f(degree,lastknot,knots,coeff,t) \bsl \\
\ind{2}mbs\_multiKnotInsf(degree,lastknot,knots,1,1,0,0,coeff,t) \\
\#define mbs\_KnotInsC2f(degree,lastknot,knots,coeff,t) \bsl \\
\ind{2}mbs\_multiKnotInsf(degree,lastknot,knots,1,2,0,0,coeff,t) \\
\#define mbs\_KnotInsC3f(degree,lastknot,knots,coeff,t) ... \\
\#define mbs\_KnotInsC4f(degree,lastknot,knots,coeff,t) ...}
Cztery makra wywo"luj"ace procedur"e \texttt{mbs\_multiKnotInsf} w celu
wstawienia w"ez"la do reprezentacji \emph{jednej} funkcji skalarnej lub
krzywej B-sklejanej w przestrzeni dwu-, tr"oj- i czterowymiarowej.
Parametry musz"a spe"lnia"c warunki podane w opisie procedury
\texttt{mbs\_multiKnotInsf}.

\vspace{\bigskipamount}
\cprog{%
int mbs\_multiKnotInsClosedf ( int degree, int *lastknot, \\
\ind{30}float *knots, \\
\ind{30}int ncurves, int spdimen, \\
\ind{30}int inpitch, int outpitch, \\
\ind{30}float *ctlpoints, float t );}
Procedura \texttt{mbs\_multiKnotInsClosedf} wstawia w"eze"l \texttt{t} do
reprezentacji \emph{zamkni"etych} krzywych B-skle\-ja\-nych stopnia
\texttt{degree}. Mo"re to by"c wykorzystane r"ownie"r do wstawiania
w"ez"la do zamkni"etego p"lata B-sklejanego (b"ed"acego rurk"a lub torusem).
Zasadnicze obliczenie wykonuje procedura
\texttt{mbs\_multiKnotInsf}, po wykonaniu kt"orej wynik jest
,,porz"adkowany'' w~celu przywr"ocenia okresowo"sci reprezentacji krzywej.

Parametry: \texttt{degree} --- stopie"n~$n$ krzywej, \texttt{*lastknot} --- przed
wywo"laniem procedury ma warto"s"c indeksu~$N$ ostatniego w"ez"la
w~pocz"atkowym ci"agu, a~po wywo"laniu indeksu ostatniego w"ez"la w~ci"agu
wynikowym. Tablica \texttt{knots} zawiera ci"ag w"ez"l"ow, odpowiednio
pocz"atkowy i~ko"ncowy, przed i~po wykonaniu procedury. Parametr
\texttt{ncurves} okre"sla liczb"e krzywych. Parametr \texttt{spdimen}
okre"sla wymiar~$d$ przestrzeni. Parametry \texttt{inpitch} i~\texttt{outpitch}
opisuj"a podzia"lki tablicy \texttt{ctlpoints}, w~kt"orej znajduj"a si"e
punkty kontrolne, przed i~po wstawieniu w"ez"la (zobacz opis procedury
\texttt{mbs\_multiKnotInsf}). Parametr \texttt{t} okre"sla nowy w"eze"l,
kt"ory procedura ma wstawi"c.

\vspace{\bigskipamount}
\cprog{%
\#define mbs\_KnotInsClosedC1f(degree,lastknot,knots,coeff,t) \bsl \\
\ind{2}mbs\_multiKnotInsClosedf(degree,lastknot,knots,1,1,0,0,coeff,t) \\
\#define mbs\_KnotInsClosedC2f(degree,lastknot,knots,coeff,t) \bsl \\
\ind{2}mbs\_multiKnotInsClosedf(degree,lastknot,knots,1,2,0,0,coeff,t) \\
\#define mbs\_KnotInsClosedC3f(degree,lastknot,knots,coeff,t) ... \\
\#define mbs\_KnotInsClosedC4f(degree,lastknot,knots,coeff,t) ...}
Cztery makra wywo"luj"ace procedur"e \texttt{mbs\_multiKnotInsClosedf} w~celu
wstawienia w"ez"la do reprezentacji \emph{jednej} okresowej funkcji
skalarnej lub zamkni"etej krzywej B-sklejanej w przestrzeni dwu-,
tr"oj- i~czterowymiarowej.
Parametry musz"a spe"lnia"c warunki podane w opisie procedury
\texttt{mbs\_multiKnotInsClosedf}.



\subsection{Usuwanie w"ez"l"ow}

W tym punkcie jest opisana procedura usuwania pojedynczego w"ez"la
z~reprezentacji krzywych B-sklejanych i~makra, kt"ore s"lu"r"a wygodnemu
wywo"lywaniu tej procedury w przypadku jednej krzywej w przestrzeniach
o~wymiarach $1$--$4$. Procedura tworzy uk"lad r"owna"n wi"a"r"acy dwie
reprezentacje krzywej, z~macierz"a r"ownowa"rn"a zmianie reprezentacji przy
u"ryciu algorytmu Boehma wstawiania w"ez"la, a nast"epnie rozwi"azuje ten
uk"lad metod"a najmniejszych kwadrat"ow. Krzywe mog"a w~wyniku usuwania
w"ez"la ulec zmianie.

Usuwanie w"ez"la odbywa si"e ,,w~miejscu'', tj.\ obszar pami"eci zajmowany
pocz"atkowo przez dan"a reprezentacj"e, po wykonaniu procedury zawiera nowy,
kr"otszy ci"ag w"ez"l"ow i punkt"ow kontrolnych. Je"sli potrzebne s"a obie
reprezentacje, tj.\ reprezentacja pocz"atkowa i~reprezentacja z~usuni"etym
w"ez"lem, to reprezentacj"e pocz"atkow"a trzeba skopiowa"c, a~nast"epnie
wykona"c usuwanie w"ez"la na kopii.

\vspace{\bigskipamount}
\cprog{%
int mbs\_multiKnotRemovef ( int degree, int *lastknot, \\
\ind{27}float *knots, \\
\ind{27}int ncurves, int spdimen, \\
\ind{27}int inpitch, int outpitch, \\
\ind{27}float *ctlpoints, \\
\ind{27}int knotnum );}
\begin{sloppypar}
Procedura \texttt{mbs\_multiKnotRemovef} s"lu"ry do
usuwania w"ez"la z reprezentacji krzywych B-sklejanych stopnia
\texttt{degree}, po"lo"ronych w przestrzeni o wymiarze \texttt{spdimen}.
Reprezentacja krzywych jest okre"slona dla ci"agu w"ez"l"ow o d"lugo"sci
\texttt{*lastknot+1}, podanego w tablicy \texttt{knots}.
Punkty kontrolne krzywych s"a podane w tablicy \texttt{ctlpoints}.
Parametr \texttt{inpitch} okre"sla podzia"lk"e, czyli odleg"lo"s"c mi"edzy
pocz"atkami obszar"ow w tablicy \texttt{ctlpoints} zawieraj"acych
wsp"o"lrz"edne danych punkt"ow kontrolnych kolejnych krzywych. Parametr
\texttt{outpitch} okre"sla podzia"lk"e tej tablicy ustalon"a po
usuni"eciu w"ez"la.
\end{sloppypar}

W"eze"l do usuni"ecia jest okre"slony przez parametr \texttt{knotnum},
kt"ory musi mie"c warto"s"c od \texttt{degree+1} do
\texttt{lastknot-degree-1}.

Nowe reprezentacje krzywych s"a umieszczane w tablicach \texttt{knots} i
\texttt{ctlpoints}. Parametr \texttt{*lastknot} jest zmniejszany o
\texttt{1}.

\begin{figure}[b]
  \centerline{\epsfig{file=knotrem.ps}}
  \caption{\label{fig:knotrem}Przyk"lad usuwania w"ez"l"ow}
\end{figure}
Je"sli krotno"s"c usuwanego w"ez"la jest r"owna $r$ i pochodna krzywej
rz"edu $\mbox{\texttt{degree}}-r+1$ nie jest w tym w"e"zle ci"ag"la, to
krzywa po usuni"eciu w"ez"la zmieni si"e. Nowe punkty kontrolne s"a
obliczane przez rozwi"azanie liniowego zadania najmniejszych kwadrat"ow dla
uk"ladu r"owna"n liniowych opisuj"acego zmian"e bazy, co jest pewnym
sposobem rozwi"azania aproksymacyjnego (zobacz przyk"lad podany dalej).

Warto"sci"a procedury jest liczba $k$, taka "re usuni"ety w"eze"l nale"ry do
przedzia"lu $[u_k,u_{k+1})$ \emph{wynikowego} ci"agu w"ez"l"ow. Je"sli
w"eze"l o numerze \texttt{knotnum} jest mniejszy ni"r w"eze"l nast"epny, to
$k=\texttt{knotnum}-1$, ale w~og"olno"sci nie musi tak by"c.

\vspace{\medskipamount}
\noindent
\textbf{Uwaga:} Aby usun"a"c w"eze"l z~reprezentacji krzywej zamkni"etej
nale"ry u"ry"c procedury \texttt{mbs\_multiKnotRemoveClosedf}.
U"rycie procedury \texttt{mbs\_multiKnotRemovef} mo"re spowodowa"c powstanie
krzywej nie-zamkni"etej.

\vspace{\bigskipamount}
\cprog{%
\#define mbs\_KnotRemoveC1f(degree,lastknot,knots,coeff,knotnum) \bsl \\
\ind{2}mbs\_multiKnotRemovef(degree,lastknot,knots,1,1,0,0,coeff,knotnum) \\
\#define mbs\_KnotRemoveC2f(degree,lastknot,knots,ctlpoints, \bsl \\
\ind{4}knotnum) \bsl \\
\ind{2}mbs\_multiKnotRemovef(degree,lastknot,knots,1,2,0,0, \bsl \\
\ind{4}(float*)ctlpoints,knotnum) \\
\#define mbs\_KnotRemoveC3f(degree,lastknot,knots,ctlpoints, \bsl \\
\ind{4}knotnum) ... \\
\#define mbs\_KnotRemoveC4f(degree,lastknot,knots,ctlpoints, \bsl \\
\ind{4}knotnum) ...}
Cztery makra wywo"luj"ace procedur"e \texttt{mbs\_multiKnotRemovef} w celu
usuni"ecia wskazanego w"ez"la z reprezentacji \emph{jednej} funkcji
sklejanej lub krzywej B-sklejanej w~przestrzeni dwu-, tr"oj- lub
czterowymiarowej. Parametry musz"a spe"lnia"c warunki podane w opisie
procedury \texttt{mbs\_multiKnotRemovef}.

Na rysunku~\ref{fig:knotrem} jest przyk"lad usuwania w"ez"l"ow, dla
p"laskiej krzywej B-sklejanej trzeciego stopnia (zobacz program
\texttt{test/knotrem.c}). Usuwany w"eze"l ma pocz"atkow"a krotno"s"c
wi"eksz"a o~$2$ od stopnia, w zwi"azku z czym krzywa sk"lada si"e z osobnych
"luk"ow i jeden z punkt"ow kontrolnych nie ma wp"lywu na jej kszta"lt.
Usuni"ecie w"ez"la powoduje odrzucenie tego punktu, bez zmiany kszta"ltu
krzywej.

Podczas usuwania w"ez"la o krotno"sci o~$1$ wi"ekszej ni"r stopie"n "luki
musz"a zosta"c po"l"aczone --- dwa punkty kontrolne zostaj"a zast"apione
przez jeden, le"r"acy w~po"lowie odleg"lo"sci mi"edzy nimi. Dalsze usuwanie
w"ez"la polega na rozwi"azywaniu odpowiednich liniowych zada"n najmniejszych
kwadrat"ow.

\vspace{\bigskipamount}
\cprog{%
int mbs\_multiKnotRemoveClosedf ( int degree, int *lastknot, \\
\ind{33}float *knots, \\
\ind{33}int ncurves, int spdimen, \\
\ind{33}int inpitch, int outpitch, \\
\ind{33}float *ctlpoints, \\
\ind{33}int knotnum );}
\begin{sloppypar}
Procedura \texttt{mbs\_multiKnotRemoveClosedf} s"lu"ry do
usuwania w"ez"la z~reprezentacji zamkni"etych krzywych B-sklejanych.
\end{sloppypar}

Parametry \texttt{degree}, \texttt{ncurves} i~\texttt{spdimen} opisuj"a
odpowiednio stopie"n krzywych, ich liczb"e i~wymiar przestrzeni, w~kt"orej
krzywe le"r"a. Parametry \texttt{*lastknot} i~\texttt{knots} opisuj"a
pocz"atkowo pocz"atkowy ci"ag w"ez"l"ow. Po wykonaniu procedury parametry te
opisuj"a ko"ncowy ci"ag w"ez"l"ow.
Parametr \texttt{knotnum} okre"sla numer w"ez"la, kt"ory nale"ry usun"a"c.
Parametry \texttt{inpitch} i~\texttt{outpitch} okre"slaj"a pocz"atkow"a
i~ko"ncow"a podzia"lk"e tablicy punkt"ow kontrolnych
\texttt{ctlpoints}, tj.\ odleg"lo"sci pocz"atk"ow "lamanych
kontrolnych poszczeg"olnych krzywych. W~tablicy \texttt{ctlpoints} nale"ry
poda"c punkty kontrolne pocz"atkowej reprezentacji krzywych; procedura
umieszcza w~niej obliczon"a reprezentacj"e krzywych z~usuni"etym w"ez"lem.

\vspace{\bigskipamount}
\cprog{%
\#define mbs\_KnotRemoveClosedC1f(degree,lastknot,knots,coeff, \bsl \\
\ind{4}knotnum) \bsl \\
\ind{2}mbs\_multiKnotRemoveClosedf(degree,lastknot,knots,1,1,0,0,coeff,
\bsl \\
\ind{4}knotnum) \\
\#define mbs\_KnotRemoveClosedC2f(degree,lastknot,knots,ctlpoints, \bsl \\
\ind{4}knotnum) \bsl \\
\ind{2}mbs\_multiKnotRemoveClosedf(degree,lastknot,knots,1,2,0,0, \bsl \\
\ind{4}(float*)ctlpoints,knotnum) \\
\#define mbs\_KnotRemoveClosedC3f(degree,lastknot,knots,ctlpoints, \bsl \\
\ind{4}knotnum) ... \\
\#define mbs\_KnotRemoveClosedC4f(degree,lastknot,knots,ctlpoints, \bsl \\
\ind{4}knotnum) ...}
Cztery makra wywo"luj"ace procedur"e \texttt{mbs\_multiKnotRemoveClosedf}
w~celu usuni"ecia wskazanego w"ez"la z reprezentacji \emph{jednej}
zamkni"etej funkcji sklejanej lub krzywej B-sklejanej w~przestrzeni
dwu-, tr"oj- lub czterowymiarowej. Parametry musz"a spe"lnia"c warunki
podane w~opisie procedury \texttt{mbs\_multiKnotRemoveClosedf}.


\vspace{\bigskipamount}
\cprog{%
void mbs\_multiRemoveSuperfluousKnotsf ( int ncurves, \\
\ind{40}int spdimen, int degree, \\
\ind{40}int *lastknot, \\
\ind{40}float *knots, \\
\ind{40}int inpitch, int outpitch, \\
\ind{40}float *ctlpoints );}
\begin{sloppypar}
Procedura \texttt{mbs\_multiRemoveSuperfluousKnotsf} s"lu"ry do takiego
usuni"ecia z~reprezentacji krzywych w"ez"l"ow, aby krotno"s"c "radnego
z~pozosta"lych w"ez"l"ow nie przekracza"la stopnia (parametr \texttt{degree})
plus~$1$. Nie powoduje to zmiany krzywej, natomiast pozwala unikn"a"c
k"lopot"ow zwi"azanych z~wyst"epowaniem takich w"ez"l"ow (m.in.\
funkcja B-sklejana, kt"orej wszystkie w"ez"ly s"a t"a sam"a liczb"a,
jest to"r\-sa\-mo"s\-cio\-wo r"owna $0$, a~zatem uk"lad funkcji B-sklejanych
stopnia~$n$ opartych na ci"agu w"ez"l"ow, kt"ory zawiera w"eze"l
o~krotno"sci wi"ekszej ni"r $n+1$, nie jest baz"a).
\end{sloppypar}

Obliczenie jest wykonywane ,,w miejscu'', tj.\ obszar zajmowany
pocz"atkowo przez reprezentacj"e dan"a, po wykonaniu procedury zawiera now"a
reprezentacj"e krzywych. Usuwanie w"ez"l"ow polega na ,,przemieszczaniu''
danych (w"ez"l"ow i~punkt"ow kontrolnych) w~tablicach, bez
"radnych oblicze"n numerycznych.


\subsection{Algorytm Oslo}

Algorytm Oslo jest metod"a przej"scia od opartej na
ci"agu w"ez"l"ow $u_0,\ldots,u_N$ reprezentacji B-sklejanej krzywej
do reprezentacji z~dodatkowymi w"ez\-"lami (kt"ore razem z~poprzednimi
tworz"a ci"ag $\hat{u}_0,\ldots,\hat{u}_{\hat{N}}$).
Inaczej ni"r w~algorytmie Boehma (zobacz
p.~\ref{ssect:knot:ins}), kt"ory s"lu"ry do wstawiania jednego w"ez"la
(i~w~razie potrzeby nale"ry go stosowa"c wielokrotnie), tu wszystkie w"ez"ly
s"a wstawiane jednocze"snie.

Je"sli punkty kontrolne $\bm{d}_i$ krzywej B-sklejanej
stopnia~$n$ odpowiadaj"a ci"agowi w"ez"l"ow $u_0,\ldots,\allowbreak u_N$,
za"s punkty kontrolne $\hat{\bm{d}}_l$ odpowiadaj"a ci"agowi
$\hat{u}_0,\ldots,\hat{u}_{\hat{N}}$, to
\begin{align}
  \hat{\bm{d}}_l = \sum_{i=0}^{N-n-1}a_{il}^n\bm{d}_i,
\end{align}
gdzie wsp"o"lczynniki $a^{n}_{kl}$ s"a okre"slone za pomoc"a rekurencyjnych
wzor"ow
\begin{align}\label{eq:Oslo:0}
  a^0_{kl} &{}= \left\{\begin{array}{ll}1 & \mbox{dla $u_k\leq \hat{u}_l<u_{k+1}$,} \\
    0 & \mbox{w przeciwnym razie,} \end{array}\right. \\
  \label{eq:Oslo:j}
  a^n_{il} &{}= \frac{\hat{u}_{l+n}-u_i}{u_{i+n}-u_i} a^{n-1}_{il} +
    \frac{u_{i+n+1}-\hat{u}_{l+n}}{u_{i+n+1}-u_{i+1}} a^{n-1}_{i+1,l}.
\end{align}
Implementacja algorytmu Oslo w~bibliotece \texttt{libmultibs} polega na tym,
"re najpierw jest obliczana macierz~$A$ wsp"o"lczynnik"ow
$a^n_{il}$, a~nast"epnie jest ona mno"rona przez macierz punkt"ow
kontrolnych $\bm{d}_0,\ldots,\bm{d}_{N-n-1}$.
\begin{figure}[htb]
  \centerline{\epsfig{file=oslo.ps}}
  \caption{Wstawianie wielu w"ez"l"ow za pomoc"a algorytmu Oslo}
\end{figure}

Macierz~$A$ umo"rliwia r"ownie"r usuni"ecie
w~jednym kroku wielu w"ez"l"ow, przez rozwi"azanie nadokre"slonego uk"ladu
r"owna"n liniowych (w kt"orym jest wi"ecej r"owna"n ni"r niewiadomych).
Taki uk"lad r"owna"n, nawet je"sli jest niesprzeczny, najlepiej jest
rozwi"azywa"c jako liniowe zadanie najmniejszych kwadrat"ow.

Macierz zmiany reprezentacji jest reprezentowana jako macierz wst"egowa, za
pomoc"a tablicy opisuj"acej profil (tj.\ po"lo"renia niezerowych
wsp"o"lczynnik"ow w~kolejnych kolumnach) i~tablicy z niezerowymi
wsp"o"lczynnikami. Szczeg"o"lowy opis tej reprezentacji i~procedur jej
przetwarzania jest tre"sci"a p.~\ref{sect:band:matrix}.

\vspace{\bigskipamount}
\cprog{%
boolean mbs\_OsloKnotsCorrectf ( int lastuknot, const float *uknots, \\
\ind{30}int lastvknot, const float *vknots );}
\begin{sloppypar}
Procedura \texttt{mbs\_OsloKnotsCorrectf} sprawdza, czy dane dwa ci"agi
w"ez"l"ow umo"rliwiaj"a skonstruowanie macierzy przej"scia. Sprawdzane
warunki s"a takie: oba ci"agi s"a niemalej"ace, a ponadto pierwszy ci"ag
($\mathord{\mbox{\texttt{lastuknot}}}+1$ liczb podanych w~tablicy
\texttt{uknots}) jest podci"agiem drugiego ci"agu
($\mathord{\mbox{\texttt{lastvknot}}}+1$ liczb w~tablicy
\texttt{vknots}). Je"sli warunki te s"a spe"lnione, to warto"sci"a procedury
jest \texttt{true} (czyli \texttt{1}), a~w~przeciwnym razie \texttt{false}
(czyli \texttt{0}).
\end{sloppypar}

\vspace{\bigskipamount}
\cprog{%
int mbs\_BuildOsloMatrixProfilef ( int degree, \\
\ind{32}int lastuknot, const float *uknots, \\
\ind{32}int lastvknot, const float *vknots, \\
\ind{32}bandm\_profile *prof );}
Procedura \texttt{mbs\_BuildOsloMatrixProfilef} konstruuje na podstawie
parametr"ow (okre"slaj"acych stopie"n reprezentacji, \texttt{degree}) i~dwa
ci"agi w"ez"l"ow (zobacz wy"rej opis procedury
\texttt{mbs\_OsloKnotsCorrectf}) profil (tj.\ opis rozmieszczenia
niezerowych wsp"o"lczynnik"ow) macierzy zmiany reprezentacji. Profil ten
jest umieszczany w tablicy \texttt{prof}, kt"ora musi mie"c d"lugo"s"c co
najmniej
$\mathord{\mbox{\texttt{lastuknot}}}-\mathord{\mbox{\texttt{degree}}}+1$
(o~$1$ wi"eksz"a ni"r liczba kolumn macierzy).

Warto"sci"a procedury jest liczba niezerowych wsp"o"lczynnik"ow macierzy,
tj.\ d"lugo"s"c tablicy potrzebnej do ich przechowywania.

\vspace{\bigskipamount}
\cprog{%
void mbs\_BuildOsloMatrixf ( int degree, int lastuknot, \\
\ind{28}const float *uknots, \\
\ind{28}const float *vknots, \\
\ind{28}const bandm\_profile *prof, float *a );}
Procedura \texttt{mbs\_BuildOsloMatrixf} oblicza wsp"o"lczynniki macierzy
zmiany reprezentacji, za pomoc"a algorytmu Oslo. Parametry procedury to:
\texttt{degree} --- stopie"n reprezentacji, \texttt{uknots} --- tablica
w"ez"l"ow reprezentacji pocz"atkowej, o~d"lugo"sci
$\mathord{\mbox{\texttt{lastuknot}}}+1$, \texttt{vknots} --- tablica
w"ez"l"ow reprezentacji z dodatkowymi w"ez"lami. Liczba tych w"ez"l"ow
jest wyznaczana na podstawie zawarto"sci tablic, dlatego nie ma
okre"slaj"acego j"a parametru. Ci"agi w"ez"l"ow w~tych tablicach
musz"a spe"lnia"c warunki sprawdzane przez procedur"e
\texttt{mbs\_OsloKnotsCorrectf}.

Tablica \texttt{prof} zawiera opis struktury macierzy, kt"ory musi by"c
skonstruowany wcze"sniej, za pomoc"a procedury
\texttt{mbs\_BuildOsloMatrixProfilef}. Do tablicy \texttt{a} o~d"lugo"sci
obliczonej przez procedur"e
\texttt{mbs\_BuildOsloMatrixProfilef} procedura
\texttt{mbs\_BuildOsloMatrixf} wpisuje wsp"o"lczynniki macierzy przej"scia.

\vspace{\bigskipamount}
\cprog{%
void mbs\_multiOsloInsertKnotsf ( int ncurves, int spdimen, \\
\ind{28}int degree, \\
\ind{28}int inlastknot, const float *inknots, \\
\ind{28}int inpitch, float *inctlpoints, \\
\ind{28}int outlastknot, const float *outknots, \\
\ind{28}int outpitch, float *outctlpoints );}
\begin{sloppypar}
Procedura \texttt{mbs\_multiOsloInsertKnotsf} s"lu"ry do jednoczesnego
wstawienia \emph{wielu} w"ez"l"ow do reprezentacji \texttt{ncurves} krzywych
B-sklejanych po"lo"ronych w przestrzeni o wymiarze \texttt{spdimen}.
Stopie"n krzywych jest okre"slony przez warto"s"c parametru \texttt{degree}.
Reprezentacja pocz"atkowa sk"lada si"e z~w"ez"l"ow umieszczonych w~tablicy
\texttt{inknots} (jest ich $\mathord{\mbox{\texttt{inlastknot}}}+1$)
oraz "lamanych kontrolnych umieszczonych w~tablicy \texttt{inctlpoints}
o~podzia"lce \texttt{inpitch}.

Ko"ncowa reprezentacja krzywych ma by"c oparta na ci"agu w"ez"l"ow
o~d"lugo"sci $\mathord{\mbox{\texttt{outlastknot}}}+1$ podanym w~tablicy
\texttt{outknots}, przy czym ci"ag w"ez"l"ow pocz"atkowej reprezentacji musi
by"c podci"agiem tego ci"agu.
\end{sloppypar}

Dzia"lanie procedury polega na wyznaczeniu za pomoc"a algorytmu Oslo
odpowiedniej macierzy, a~nast"epnie pomno"reniu jej przez macierz utworzon"a
z~punkt"ow kontrolnych danych krzywych.

Je"sli warto"sci parametr"ow \texttt{inlastknot} i~\texttt{outlastknot}
s"a r"owne, to procedura, przy za"lo"reniu, "re oba ci"agi w"ez"l"ow s"a
identyczne (co \emph{nie jest} sprawdzane), kopiuje dane z~tablicy
\texttt{inctlpoints} do \texttt{outctlpoints} (z~uwzgl"ednieniem podzia"lek
tych tablic, okre"slonych przez parametry \texttt{inpitch}
i~\texttt{outpitch}).

\vspace{\bigskipamount}
\cprog{%
void mbs\_multiOsloRemoveKnotsLSQf ( int ncurves, int spdimen, \\
\ind{28}int degree, \\
\ind{28}int inlastknot, const float *inknots, \\ 
\ind{28}int inpitch, float *inctlpoints, \\ 
\ind{28}int outlastknot, const float *outknots, \\ 
\ind{28}int outpitch, float *outctlpoints );}
Procedura \texttt{mbs\_multiOsloRemoveKnotsLSQf} s"lu"ry do jednoczesnego
usuni"ecia \emph{wielu} w"ez"l"ow z~reprezentacji \texttt{ncurves} krzywych
B-sklejanych po"lo"ronych w przestrzeni o wymiarze \texttt{spdimen}.
Stopie"n krzywych jest okre"slony przez warto"s"c parametru \texttt{degree}.
Reprezentacja pocz"atkowa sk"lada si"e z~w"ez"l"ow umieszczonych w~tablicy
\texttt{inknots} (jest ich $\mathord{\mbox{\texttt{inlastknot}}}+1$)
oraz "lamanych kontrolnych umieszczonych w~tablicy \texttt{inctlpoints}
o~podzia"lce \texttt{inpitch}.

Ko"ncowa reprezentacja krzywych ma by"c oparta na ci"agu w"ez"l"ow
o~d"lugo"sci $\mathord{\mbox{\texttt{outlastknot}}}+1$ podanym w~tablicy
\texttt{outknots}, przy czym musi to by"c podci"ag ci"agu pocz"atkowego.
Ponadto krotno"s"c "radnego w"ez"la w ci"agu ko"ncowym nie mo"re by"c
wi"eksza ni"r $\mathord{\mbox{\texttt{degree}}}+1$ (poniewa"r w przeciwnym
razie macierz opisana wy"rej mia"laby kolumny liniowo zale"rne).

Dzia"lanie procedury polega na wyznaczeniu za pomoc"a algorytmu Oslo
odpowiedniej macierzy, a~nast"epnie rozwi"azaniu liniowego zadania
najmniejszych kwad\-ra\-t"ow z~t"a macierz"a.

Je"sli warto"sci parametr"ow \texttt{inlastknot} i~\texttt{outlastknot}
s"a r"owne, to procedura, przy za"lo"reniu, "re oba ci"agi w'ez"l"ow s"a
identyczne (co \emph{nie jest} sprawdzane), kopiuje dane z~tablicy
\texttt{inctlpoints} do \texttt{outctlpoints} (z~uwzgl"ednieniem podzia"lek
tych tablic, okre"slonych przez parametry \texttt{inpitch}
i~\texttt{outpitch}).


\subsection{\label{ssect:max:knot:ins}Maksymalne wstawianie w"ez"l"ow}

Procedury opisane w tym punkcie s"lu"r"a do wstawienia w"ez"l"ow do
reprezentacji krzywych i~p"lat"ow B-sklejanych w~taki spos"ob, aby
krotno"s"c ka"rdego w"ez"la wewn"etrznego by"la o~$1$ wi"eksza od stopnia.
W~ten spos"ob powstaje szczeg"olna reprezentacja B-sklejana, kt"ora sk"lada
si"e z~reprezentacji ka"rdego "luku wielomianowego w bazie Bernsteina lokalnej
zmiennej, tj.\ reprezentacja kawa"lkami B\'{e}ziera. Reprezentacja taka
umo"rliwia m.in.\ szybkie obliczanie punkt"ow krzywych (za pomoc"a schematu
Hornera) i~dzia"lania algebraiczne (mno"renie) funkcji sklejanych
i~krzywych. Procedury te nie usuwaj"a
zb"ednych w"ez"l"ow i~punkt"ow kontrolnych, kt"ore mog"a pozosta"c po
wstawieniu w"ez"l"ow. Procedury, kt"ore to robi"a (czyli daj"a wynik
,,czysty'') s"a opisane w nast"epnym punkcie.

\vspace{\bigskipamount}
\cprog{%
void mbs\_multiMaxKnotInsf ( int ncurves, int spdimen, int degree, \\
\ind{28}int inlastknot, const float *inknots, \\
\ind{28}int inpitch, const float *inctlpoints, \\
\ind{28}int *outlastknot, float *outknots, \\
\ind{28}int outpitch, float *outctlpoints, \\
\ind{28}int *skipl, int *skipr );}
\begin{sloppypar}
Procedura \texttt{mbs\_multiMaxKnotInsf} wstawia w"ez"ly do reprezentacji
\texttt{ncurves} krzywych B-sklejanych stopnia \texttt{degree}
w~przestrzeni o~wymiarze \texttt{spdimen}.

Pocz"atkowa reprezentacja
krzywych jest opisana przez parametry \texttt{inlastknot} (indeks ostatniego
w"ez"la), \texttt{inctlpoints} (tablica zawieraj"aca w"ez"l"ow),
\texttt{inpitch} (podzia"lka, tj.\ odleg"lo"s"c pocz"atk"ow danych "lamanych
kontrolnych) i~\texttt{inctlpoints} (tab\-li\-ca zawieraj"aca wsp"o"lrz"edne
punkt"ow kontrolnych kolejnych krzywych).

Procedura wyznacza reprezentacje krzywych odpowiadaj"ace ci"agowi w"ez"l"ow,
w~kt"orym ka"rdy w"eze"l wewn"etrzny (zobacz p.~\ref{ssect:BSC}) ma krotno"s"c
$\mathord{\makebox{\texttt{degree}}}+1$, a~w"ez"ly brzegowe maj"a krotno"sci
\texttt{degree} lub $\mathord{\makebox{\texttt{degree}}}+1$.

Krotno"sci w"ez"l"ow zewn"etrznych nie ulegaj"a zmianie, w~zwi"azku z~czym
mo"re powsta"c reprezentacja, kt"ora zawiera zb"edne w"ez"ly i~punkty
kontrolne. Parametry \texttt{*skipl} i~\texttt{*skipr} na wyj"sciu
z~procedury okre"slaj"a liczb"e zb"ednych w"ez"l"ow i punkt"ow
kontrolnych odpowiednio z~lewej i~prawej strony.

Nowa reprezentacja krzywej jest umieszczana w~tablicach
\texttt{outknots} (ci"ag w"ez\-"l"ow, indeks ostatniego w"ez"la jest zwracany
poprzez parametr \texttt{outlastknot}) i~\texttt{outctlpoints} ("lamane
kontrolne, pocz"atki "lamanych kolejnych krzywych s"a od siebie odleg"le
liczb"e, kt"ora jest warto"sci"a parametru \emph{wej"sciowego}
\texttt{outpitch}).

Je"sli w~pocz"atkowej reprezentacji krzywych wyst"epuj"a w"ez"ly
o~krotno"sci wi"ekszej ni"r docelowa, to w~pierwszym kroku s"a one
usuwane (z~roboczej kopii reprezentacji krzywych), za pomoc"a procedury
\texttt{mbs\_multiRemoveSuperfluousKnots}).
Nast"epnie jest wywo"lywana procedura \texttt{mbs\_multiOsloInsertKnotsf}),
kt"ora dokonuje wstawienia w"ez"l"ow za pomoc"a algorytmu Oslo.

D"lugo"sci tablic potrzebnych do pomieszczenia nowej reprezentacji krzywej
mo"rna obliczy"c za pomoc"a procedury \texttt{mbs\_LastknotMaxInsf}, kt"ora
oblicza indeks ostatniego w"ez"la nowej reprezentacji.
\end{sloppypar}


\vspace{\bigskipamount}
\cprog{%
\#define mbs\_MaxKnotInsC1f(degree,inlastknot,inknots,incoeff, \bsl \\
\ind{4}outlastknot,outknots,outcoeff,skipl,skipr) \bsl \\
\ind{2}mbs\_multiMaxKnotInsf(1,1,degree,inlastknot,inknots,0,incoeff, \bsl \\
\ind{4}outlastknot,outknots,0,outcoeff,skipl,skipr) \\
\#define mbs\_MaxKnotInsC2f(degree,inlastknot,inknots,inctlpoints, \bsl \\
\ind{4}outlastknot,outknots,outctlpoints,skipl,skipr) \bsl \\
\ind{2}mbs\_multiMaxKnotInsf(1,2,degree,inlastknot,inknots,0, \bsl \\
\ind{4}(float*)inctlpoints,outlastknot,outknots,0, \bsl \\
\ind{4}(float*)outctlpoints,skipl,skipr) \\
\#define mbs\_MaxKnotInsC3f(degree,inlastknot,inknots,inctlpoints, \bsl \\
\ind{4}outlastknot,outknots,outctlpoints,skipl,skipr) ... \\
\#define mbs\_MaxKnotInsC4f(degree,inlastknot,inknots,inctlpoints, \bsl \\
\ind{4}outlastknot,outknots,outctlpoints,skipl,skipr) ...}
Cztery makra wywo"luj"ace procedur"e \texttt{mbs\_multiMaxKnotInsf} w~celu
wyznaczenia reprezentacji B-sklejanej funkcji lub krzywej sklejanej
w~przestrzeni dwu-, tr"oj- i czterowymiarowej, w kt"orej wszystkie w"ez"ly
wewn"etrzne maj"a krotno"s"c o $1$ wi"eksz"a ni"r stopie"n, czyli
reprezentacji kawa"lkami B\'{e}ziera. Parametry musz"a
spe"lnia"c warunki podane w opisie procedury \texttt{mbs\_multiMaxKnotInsf}.


\subsection{Konwersja krzywych i p"lat"ow do postaci kawa"lkami B\'{e}ziera}

Opisane tu procedury dokonuj"a konwersji krzywych i~p"lat"ow B-sklejanych do
postaci kawa"lkami B\'{e}ziera, za pomoc"a procedury
\texttt{mbs\_multiMaxKnotInsf}. Mo"rna ich u"rywa"c do wygodnego
rysowania krzywych lub p"lat"ow.

\vspace{\bigskipamount}
\cprog{%
void mbs\_multiBSCurvesToBezf ( int spdimen, int ncurves, \\
\ind{31}int degree, int lastinknot, \\
\ind{31}const float *inknots, \\
\ind{31}int inpitch, const float *inctlp, \\
\ind{31}int *kpcs, int *lastoutknot, \\
\ind{31}float *outknots, \\
\ind{31}int outpitch, float *outctlp );}
Procedura \texttt{mbs\_multiBSCurvesToBezf} dokonuje konwersji
\texttt{ncurves} krzywych B-sklejanych stopnia \texttt{degree}, po"lo"ronych
w~przestrzeni o~wymiarze \texttt{spdimen} do postaci kawa"lkami B\'{e}ziera.

Parametry opisuj"ace dan"a reprezentacj"e krzywych to \texttt{lastinknot}
(indeks ostatniego w"ez"la), \texttt{inknots} (tablica z~ci"agiem w"ez"l"ow),
\texttt{inpitch} i~\texttt{inctlp} (podzia"lka i~tablica z~punktami
kontrolnymi krzywych).

Warto"s"c parametru \texttt{*kpcs} po wyj"sciu z~procedury jest r"owna
liczbie wielomianowych "luk"ow, z~kt"orych sk"lada si"e ka"rda krzywa.
Parametr \texttt{*lastoutknot} jest indeksem ostatniego w"ez"la w~ci"agu
nale"r"acym do wynikowej reprezentacji krzywych, tablica \texttt{outknots}
zawiera ten ci"ag w"ez"l"ow, parametr \emph{wej"sciowy} \texttt{outpitch}
okre"sla podzia"lk"e tablicy \texttt{outctlp}, w~kt"orej procedura umieszcza
punkty kontrolne B\'{e}ziera poszczeg"olnych "luk"ow wielomianowych.
Dok"ladniej, w~tablicy tej ka"rdej z~krzywych B-sklejanych odpowiada
\texttt{*kpcs*(degree+1)*spdimen} liczb zmiennopozycyjnych; ka"rdy "luk
wielomianowy jest reprezentowany przez kolejne \texttt{(degree+1)*spdimen}
liczb (opisuj"acych \texttt{degree+1} punkt"ow); warto"s"c parametru
\texttt{outpitch} okre"sla odleg"lo"s"c w~tablicy pocz"atk"ow reprezentacji
pierwszego "luku ka"rdej z~krzywych B-sklejanych.

Je"sli parametr \texttt{kpcs}, \texttt{lastoutknot} lub \texttt{outknots}
jest wska"znikiem pustym, to procedura pomija przykazywanie odpowiedniej
informacji.


\vspace{\bigskipamount}
\cprog{%
\#define mbs\_BSToBezC1f(degree,lastinknot,inknots,incoeff,kpcs, \bsl \\
\ind{4}lastoutknot,outknots,outcoeff) \bsl \\
\ind{2}mbs\_multiBSCurvesToBezf(1,1,degree,lastinknot,inknots,0,incoeff,\bsl \\
\ind{4}kpcs,lastoutknot,outknots,0,outcoeff) \\
\#define mbs\_BSToBezC2f(degree,lastinknot,inknots,inctlp,kpcs, \bsl \\
\ind{4}lastoutknot,outknots,outctlp) \bsl \\
\ind{2}mbs\_multiBSCurvesToBezf(2,1,degree,lastinknot,inknots,0, \bsl \\
\ind{4}(float*)inctlp,kpcs,lastoutknot,outknots,0,(float*)outctlp) \\
\#define mbs\_BSToBezC3f(degree,lastinknot,inknots,inctlp,kpcs, \bsl \\
\ind{4}lastoutknot,outknots,outctlp) ... \\
\#define mbs\_BSToBezC4f(degree,lastinknot,inknots,inctlp,kpcs, \bsl \\
\ind{4}lastoutknot,outknots,outctlp) ...}
\begin{sloppypar}
Podane wy"rej makra wywo"luj"a procedur"e \texttt{mbs\_multiBSCurvesToBezf}
w~celu otrzymania reprezentacji kawa"lkami B\'{e}ziera \emph{jednej} funkcji
sklejanej lub krzywej \mbox{B-sklejanej} w~przestrzeni dwu-, tr"oj- lub
czterowymiarowej. Parametr"ow tych makr maj"a takie samo znaczenie jak
parametry procedury \texttt{mbs\_multiBSCurvesToBezf} o~tych samych nazwach. 
\end{sloppypar}


\vspace{\bigskipamount}
\cprog{%
void mbs\_BSPatchToBezf ( int spdimen, \\
\ind{25}int degreeu, int lastuknot, \\
\ind{25}const float *uknots, \\
\ind{25}int degreev, int lastvknot, \\
\ind{25}const float *vknots, \\
\ind{25}int inpitch, const float *inctlp, \\
\ind{25}int *kupcs, int *lastoutuknot, \\
\ind{25}float *outuknots, \\
\ind{25}int *kvpcs, int *lastoutvknot, \\
\ind{25}float *outvknots, \\
\ind{25}int outpitch, float *outctlp );}
\begin{sloppypar}
Procedura \texttt{mbs\_BSPatchToBezf} znajduje reprezentacj"e
p"lata B-sklejanego z~w"ez\-"la\-mi o~krotno"sciach o~$1$ wi"ekszych od stopni
p"lata ze wzgl"edu na oba parametry, czyli reprezentacje B\'{e}ziera
wielomianowych kawa"lk"ow p"lata. Reprezentacja taka nadaje si"e m.in.\ do
wygodnego wy"swietlania p"lata. Procedura dopuszcza p"laty o~brzegach
swobodnych.
\end{sloppypar}

Parametr \texttt{spdimen} okre"sla wymiar przestrzeni, w kt"orej jest
po"lo"rony p"lat. Stopie"n p"lata jest okre"slony przez parametry
\texttt{degreeu} i~\texttt{degreev}, ci"agi w"ez"l"ow reprezentacji danej
s"a reprezentowane przez parametry \texttt{lastuknot}, \texttt{uknots},
\texttt{lastvknot} i~\texttt{vknots}, a~punkty kontrolne s"a podane
w~tablicy \texttt{inctlp} o~podzia"lce (odleg"lo"sci pocz"atk"ow kolejnych
kolumn siatki kontrolnej) b"ed"acej warto"sci"a parametru \texttt{inpitch}.

Parametry \texttt{*kupcs} i~\texttt{*kvpcs} s"lu"r"a do przekazania
informacji o~liczbie wielomianowych kawa"lk"ow p"lata; p"lat sk"lada si"e
z~\texttt{*kupcs} ,,pas"ow'', z~kt"orych ka"rdy sk"lada si"e
z~\texttt{*kvpcs} p"lat"ow wielomianowych. Parametry \texttt{lastoutuknot},
\texttt{outuknots}, \texttt{lastvknot} i~\texttt{outvknots} s"lu"r"a do
przekazania przez procedur"e ci"ag"ow w"ez"l"ow ko"ncowej reprezentacji
p"lata. Je"sli dowolny z~tych parametr"ow jest wska"znikiem pustym
(\texttt{NULL}), to odpowiednia informacja nie jest wyprowadzana przez
procedur"e (dla potrzeb rysowania p"lat"ow idane te s"a zb"edne).

\begin{sloppypar}
Punkty kontrolne wynikowej reprezentacji p"lata s"a umieszczane w~tablicy
\texttt{outctlp} o~podzia"lce \texttt{outpitch} (\textbf{Uwaga:} to jest
parametr wej"sciowy). Podzia"lka ta powinna by"c wi"eksza lub r"owna
$d(m+1)k_v$, gdzie liczba~$d$ jest wymiarem przestrzeni (warto"s"c parametru
\texttt{spdimen}), $m$~(warto"s"c parametru \texttt{degreev}) jest stopniem
p"lata ze wzgl"edu na parametr~$v$, za"s $k_v$ jest liczb"a odcink"ow, na
kt"ore w"ez"ly w ci"agu ,,$v$'' dziel"a przedzia"l $[v_m,v_{M-m}]$
(liczba $M$ jest warto"sci"a parametru \texttt{lastvknot}). Liczba $k_v$
jest przypisywana przez procedur"e parametrowi $\texttt{kvpcs}$, ale mo"rna
j"a otrzyma"c wcze"sniej, za pomoc"a procedury
\texttt{mbs\_NumKnotIntervalsf}.

Liczba kolumn ko"ncowej reprezentacji p"lata jest r"owna $(n+1)k_u$, gdzie
$n$~jest stopniem p"lata ze wzgl"edu na parametr~$u$, za"s $k_u$ jest
liczb"a ,,pas"ow'', z~kt"orych sk"lada si"e p"lat. Te"r mo"rna j"a obliczy"c
zawczasu, wywo"luj"ac procedur"e \texttt{mbs\_NumKnotIntervalsf}.
\end{sloppypar}

W"la"sciwe obliczenia (przede wszystkim wstawianie w"ez"l"ow) wykonuje
procedura \texttt{mbs\_multiMaxKnotInsf}.


\vspace{\medskipamount}\noindent
\textbf{Przyk"lad.} 
Za"lo"rymy, "re p"lat jest okre"slony wzorem~(\ref{eq:BSpatch:def}) za
pomoc"a dw"och niemalej"acych ci"ag"ow w"ez"l"ow, $u_0,\ldots,u_N$
i~$v_0,\ldots,v_M$, umieszczonych odpowiednio w~tablicach \texttt{u}
i~\texttt{v}. Stopie"n p"lata jest r"owny $n$ ze
wzgl"edu na parametr $u$ i $m$ ze wzgl"edu na parametr $v$.
Punkty kontrolne p"lata le"r"a w przestrzeni $d$-wymiarowej (tej samej co
p"lat) i~s"a ustawione w~kolumnach, w~tablicy~\texttt{cp}. Kolumna $i$-ta dla
$i\in\{0,\ldots,N-n-1\}$ sk"lada si"e z~$M-m$ punkt"ow, a~zatem jest ona
reprezentowana przez $(M-m)d$ liczb zmiennopozycyjnych.

\vspace{\medskipamount}
\noindent{\ttfamily
\ind{2}ku = mbs\_NumKnotIntervalsf ( $n$, $N$, u ); \\
\ind{2}kv = mbs\_NumKnotIntervalsf ( $m$, $M$, v ); \\
\ind{2}pitch = $(m+1)d$*kv; \\
\ind{2}b = pkv\_GetScratchMemf ( pitch*ku*(n+1) ); \\
\ind{2}mbs\_BSPatchToBezf ( $d$, $n$, $N$, u, $m$, $M$, v, $d$*($M-m$), cp, \\
\ind{22}\&ku, NULL, NULL, \&kv, NULL, NULL, pitch, b );}
\vspace{\medskipamount}

Po wykonaniu przedstawionego wy"rej fragmentu programu
w~tablicy \texttt{b} ma\-my punkty kontrolne B\'{e}ziera
p"lat"ow wielomianowych, z kt"orych sk"lada si"e dany p"lat B-sklejany.
Aby do tablicy \texttt{c} (o~d"lugo"sci co najmniej
$(n+1)(m+1)d$ liczb zmiennopozycyjnych) przenie"s"c ,,spakowan"a'' siatk"e
kontroln"a (tj.\ punkty kontrolne w~kolejnych kolumnach bez przerw)
$j$-tego p"lata z~$i$-tego pasa (licz"ac od zera), mo"rna wykona"c
instrukcje

\vspace{\medskipamount}
\noindent{\ttfamily
\ind{2}md = $(m+1)d$;\ind{16}\,/* d"lugo"s"c kolumny p"lata B\'{e}ziera */ \\
\ind{2}start = $(n+1)i$*pitch + md*$j$;\ind{4}/* po"lo"renie pierwszego punktu */\\
\ind{2}pkv\_Selectf ( $n+1$, md, pitch, md, \&b[start], c );
}\vspace{\medskipamount}

\begin{figure}[ht]
  \centerline{\epsfig{file=bspbez.ps}}
  \caption{P"lat B-sklejany i jego reprezentacja kawa"lkami B\'{e}ziera}
\end{figure}


\section{Algorytm Lane'a-Riesenfelda}

\cprog{%
boolean mbs\_multiLaneRiesenfeldf ( int spdimen, int ncurves, \\
\ind{20}int degree, \\
\ind{20}int inlastknot, int inpitch, const float *incp, \\
\ind{20}int *outlastknot, int outpitch, float *outcp );}

\vspace{\bigskipamount}
\cprog{%
\#define mbs\_LaneRiesenfeldC1f(degree,inlastknot,incp,outlastknot, \bsl \\
\ind{4}outcp) \bsl \\
\ind{2}mbs\_multiLaneRiesenfeldf ( 1, 1, degree, inlastknot, 0, incp, \bsl \\
\ind{4}outlastknot, 0, outcp ) \\
\#define mbs\_LaneRiesenfeldC2f(degree,inlastknot,incp,outlastknot, \bsl \\
\ind{4}outcp) \bsl \\
\ind{2}mbs\_multiLaneRiesenfeldf ( 2, 1, degree, inlastknot, 0, \bsl \\
\ind{4}(float*)incp, outlastknot, 0, (float*)outcp ) \\
\#define mbs\_LaneRiesenfeldC3f(degree,inlastknot,incp,outlastknot, \bsl \\
\ind{4}outcp) ... \\
\#define mbs\_LaneRiesenfeldC4f(degree,inlastknot,incp,outlastknot, \bsl \\
\ind{4}outcp) ...}

\begin{figure}[ht]
  \centerline{\epsfig{file=bsplane.ps}}
  \caption{Zastosowanie algorytmu Lane'a-Riesenfelda do p"lata B-sklejanego}
\end{figure}

\clearpage
