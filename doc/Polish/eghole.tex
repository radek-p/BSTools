
%/* //////////////////////////////////////////////////// */
%/* This file is a part of the BSTools procedure package */
%/* written by Przemyslaw Kiciak.                        */
%/* //////////////////////////////////////////////////// */

\chapter{Biblioteka \texttt{libeghole}}

Biblioteka \texttt{libeghole} zawiera procedury wype"lniania wielok"atnego
otworu w~powierzchni sklejanej z~p"lat"ow stopnia~$(3,3)$. Podstawy
teoretyczne i~opis konstrukcji s"a opisane w~pracy \emph{Konstrukcje
powierzchni g"ladko wype"lniaj"acych wielok"atne otwory}.

\section{\label{sect:G2:data}Przygotowanie danych}

Dane dla procedur wype"lniania otworu sk"ladaj"a si"e z~czterech cz"e"sci:
\begin{itemize}
  \item Liczby wierzcho"lk"ow otworu, $k$,
  \item $k$ jedenastoelementowych ci"ag"ow w"ez"l"ow,
  \item $12k+1$ punkt"ow kontrolnych dziedziny,
  \item $12k+1$ punkt"ow kontrolnych powierzchni.
\end{itemize}
Opr"ocz wymienionych wy"rej danych mo"rna okre"sli"c \textbf{wi"ezy}, czyli
r"ownania liniowe, kt"ore maj"a by"c spe"lnione przez powierzchni"e
wype"lniaj"ac"a otw"or. Spos"ob ich przygotowania jest opisany
w~p.~\ref{ssect:g2h:constraints}.

Liczba ca"lkowita $k$ musi by"c nie mniejsza ni"r~$3$ i~nie wi"eksza
ni"r~$16$. Powierzchnia z~otworem sk"lada si"e z~$3k$ p"lat"ow wielomianowych
stopnia~$(3,3)$, otaczaj"acych \mbox{$k$-k"atny} otw"or.

\textbf{Ci"agi w"ez"l"ow} $u^{(n)}_0,\ldots,u^{(n)}_{10}$, dla $n=0,\ldots,k-1$,
musz"a spe"lnia"c warunki
\begin{align*}
  u^{(n)}_0\leq u^{(n)}_1<\cdots<u^{(n)}_9\leq u^{(n)}_{10},
\end{align*}
oraz
\begin{align*}
  u^{(n)}_i-u^{(n)}_1 = u^{(m)}_0-u^{(m)}_{10-i},
\end{align*}
dla $m=(n+2)\bmod k$ i~$i=4,\ldots,9$.
Ci"agi te podaje si"e w~jednowymiarowej tablicy o~d"lugo"sci $11k$;
musi ona zawiera"c wyrazy tych ci"ag"ow kolejno po sobie, bez przerw.

\textbf{Punkty kontrolne dziedziny}, $\bm{c}_0,\ldots,\bm{c}_{12k}$ le"r"a
w~p"laszczy"znie i~stanowi"a wierzcho"lki
siatki kontrolnej. Schemat siatki i~spos"ob numeracji tych punkt"ow jest
pokazany na rysunku~\ref{fig:domain:cnet}. Siatka ta zawiera $k$~siatek
kontrolnych p"laskich bikubicznych p"lat"ow B-sklejanych, kt"ore maj"a
wsp"olne kawa"lki wielomianowe.%
\begin{figure}[ht]
  \centerline{\epsfig{file=g2hdomain.ps}}
  \caption{\label{fig:domain:cnet}Reprezentacja dziedziny powierzchni
    wype"lniaj"acej otw"or}
\end{figure}

Dla $n=0,\ldots,k-1$ $n$-ty p"lat B-sklejany jest reprezentowany przez
ci"agi w"ez"l"ow $u^{(n)}_0,\ldots,u^{(n)}_{10}$ i~$u^{(m)}_0,\ldots,u^{(m)}_7$,
gdzie $m=(n+1)\bmod k$, oraz punkty kontrolne $\bm{c}^{(n)}_{ij}$, $i=0,\ldots,6$,
$j=0,\ldots,3$, takie "re:
\begin{itemize}
  \item $\bm{c}^{(n)}_{ij}=\bm{c}_{12(n+1)-3j-i}$ dla $i=0,\ldots,2$, $j=0,\ldots,3$
  \item $\bm{c}^{(n)}_{ij}=\bm{c}_{12m-3i-j}$ dla $i=3,\ldots,6$, $j=0,\ldots,2$,
    gdzie $m=(n+1)\bmod k$,
  \item $\bm{c}^{(n)}_{3,3}=\bm{c}_0$,
  \item $\bm{c}^{(n)}_{ij}=\bm{c}_{12m+i-3}$ dla $i=4,\ldots,6$, $j=3$,
    gdzie $m=(n+2)\bmod k$. 
\end{itemize}
P"laty B-sklejane reprezentowane przez te ci"agi w"ez"l"ow i~punkty
kontrolne musz"a by"c regularne i~z~wyj"atkiem wsp"olnych fragment"ow,
kt"orych istnienie zapewnia reprezentacja, roz"l"aczne.

\begin{sloppypar}
Zbi"or punkt"ow p"lat"ow B-sklejanych reprezentowanych przez ci"agi w"ez"l"ow
i~punk\-ty kontrolne opisane wy"rej jest dziedzin"a pewnej parametryzacji
powierzchni z~otworem, za"s otoczony tymi p"latami obszar~$\varOmega$, kt"ory
jest krzywoliniowym $k$-k"atem, jest dziedzin"a pewnej parametryzacji
powierzchni wype"lniaj"acej, kt"ora ma by"c skonstruowana. Dla takiej
parametryzacji b"edzie okre"slony funkcjona"l~$F$, kt"orego warto"s"c
jest miar"a jako"sci powierzchni (ma on by"c minimalizowany).
Zmieniaj"ac punkty kontrolne~$\bm{c}_i$ zmienia si"e ten funkcjona"l,
co wp"lywa na wynik konstrukcji.%
\end{sloppypar}

\textbf{Punkty kontrolne powierzchni}, $\bm{b}_0,\ldots,\bm{b}_{12k}$, le"r"a
w~przestrzeni o~wymiarze~$d$ (w~praktyce zwykle b"edzie $d=3$ dla powierzchni
wielomianowej, albo $d=4$, je"sli ma by"c wype"lniony otw"or w~powierzchni
b"ed"acej jednorodn"a reprezentacj"a powierzchni kawa"lkami wymiernej).
Siatka kontrolna powierzchni jest zbudowana analogicznie jak siatka
kontrolna dziedziny, tj.\ mo"rna w~niej wyr"o"rni"c $k$~siatek kontrolnych
p"lat"ow B-sklejanych stopnia~$(3,3)$. Dla $n=0,\ldots,k-1$ p"lat $n$-ty jest
reprezentowany przez ci"agi w"ez"l"ow $u^{(n)}_0,\ldots,u^{(n)}_{10}$
i~$u^{(m)}_0,\ldots,u^{(m)}_7$, gdzie $m=(n+1)\bmod k$, oraz punkty kontrolne
$\bm{b}^{(n)}_{ij}$, $i=0,\ldots,6$, $j=0,\ldots,3$, kt"ore s"a punktami
$\bm{b}_l$ o~indeksach~$l$ okre"slonych tak samo jak indeksy punkt"ow
kontrolnych $\bm{c}^{(n)}_{ij}$ p"lat"ow B-sklejanych otaczaj"acych dziedzin"e.

Tablica punkt"ow kontrolnych powierzchni, kt"ora ma by"c parametrem procedur
konstruuj"acych wype"lnienie otworu, sk"lada si"e z~$(12k+1)d$ liczb
zmiennopozycyjnych --- ka"rde kolejne $d$~z~nich to wsp"o"lrz"edne kolejnego
punktu~$\bm{b}_l$. Punkty $\bm{c}^{(n)}_{ij}$ i~$\bm{b}^{(n)}_{ij}$
dla $i\in\{0,6\}$ oraz dla $j=0$ nie maj"a wp"lywu na wynik konstrukcji
(tj.\ na powierzchni"e wype"lniaj"ac"a otw"or), podobnie jak w"ez"ly
$u^{(n)}_0$ i~$u^{(n)}_{10}$, ale trzeba je poda"c. Na
rysunku~\ref{fig:domain:cnet} punkty, kt"ore maj"a wp"lyw na wynik konstrukcji,
s"a zaznaczone czarnymi kropkami.


\section{Minimum teorii}

Dok"ladny opis podstaw teoretycznych konstrukcji realizowanych przez procedury
z~biblioteki \texttt{libeghole} znajduje si"e w~pracy \emph{Konstrukcje
powierzchni g"ladko wype"lniaj"acych wielok"atne otwory}. Opis poni"rej
zawiera tylko wiadomo"sci teoretyczne niezb"edne do poprawnego przygotowania
danych dla procedur.

\subsection{Bazy u"rywane w~konstrukcjach}

Aby skonstruowa"c powierzchni"e wype"lniaj"ac"a otw"or, procedury
biblioteczne konstruuj"a baz"e $\phi_0,\ldots,\phi_{n+m}$ pewnej
przestrzeni liniowej~$V$, do kt"orej nale"r"a funkcje skalarne klasy~$C^1$
albo~$C^2$ opisuj"ace wsp"o"lrz"edne powierzchni wype"lniaj"acej.
Powierzchnia ta okre"slona jest wzorem
\begin{align}\label{eq:G2:surface}
  \bm{p} =
  \sum_{i=0}^{n-1}\bm{a}_i\phi_i+\sum_{i=0}^{m-1}\bm{b}_i\phi_{n+i}.
\end{align}
Wektory $\bm{b}_i\in\R^d$ s"a danymi punktami kontrolnymi powierzchni
z~otworem. Siatka kontrolna powierzchni, kt"orej to s"a wierzcho"lki,
jest grafem izomorficznym z~siatk"a kontroln"a dziedziny pokazan"a na
rysunku~\ref{fig:domain:cnet} i~jej wierzcho"lki s"a ponumerowane
analogicznie. Wierzcho"lki te podaje si"e (w~takiej kolejno"sci)
w~tablicy przekazywanej procedurom konstrukcji powierzchni jako parametr.
Tablica zawiera $12k+1$ punkt"ow kontrolnych, z~kt"orych $m=6k+1$
ma wp"lyw na powierzchni"e wype"lniaj"ac"a otw"or.

\begin{sloppypar}
Zadaniem procedur konstrukcji jest obliczenie wektor"ow
$\bm{a}_0,\ldots,\bm{a}_{n-1}\in\R^d$, kt"ore minimalizuj"a pewne
funkcjona"ly przyj"ete za miar"e ,,brzydoty'' powierzchni.
Funkcje bazowe $\phi_0,\ldots,\phi_{n+m-1}$ s"a okre"slone w~obszarze
$\varOmega\in\R^2$, kt"ory jest otworem w~p"las\-kiej powierzchni
reprezentowanej przez w"ez"ly i~siatk"e kontroln"a dziedziny, opisan"a
w~poprzednim punkcie. Obszar~$\varOmega$ jest podzielony na
$k$~czworok"at"ow krzywoliniowych $\varOmega_0,\ldots,\varOmega_{k-1}$,
kt"ore s"a obrazami kwadratu jednostkowego
w~przekszta"lceniach~$\bm{d}_0,\ldots,\bm{d}_{k-1}$, zwanych
\textbf{p"latami dziedziny}. Funkcja~$\phi_i$ jest zdefiniowana wzorem
\begin{align*}
  \phi_i(\bm{x}) = p_{li}(\bm{d}_l^{-1}(\bm{x}))\qquad
  \mbox{dla $\bm{x}\in\varOmega_i$,}
\end{align*}
za pomoc"a p"lat"ow dziedziny i~funkcji $p_{i0},\ldots,p_{i,k-1}$,
zwanych \textbf{p"latami funkcji bazowych}. Powierzchnia wype"lniaj"aca
otw"or sk"lada si"e z~$k$~p"lat"ow wielomianowych lub sklejanych
$\bm{p}_0,\ldots,\bm{p}_{k-1}$, okre"slonych wzorem
\begin{align*}
  \bm{p}_l =
  \sum_{i=0}^{n-1}\bm{a}_ip_{li}+\sum_{i=0}^{m-1}\bm{b}_ip_{l,i+n}.
\end{align*}
Reprezentacja B\'{e}ziera lub B-sklejana tych p"lat"ow jest ostatecznym
wynikiem konstrukcji.%
\end{sloppypar}

\begin{sloppypar}
Funkcje bazowe $\phi_0,\ldots,\phi_{n+m-1}$ mo"rna podzieli"c na dwa
podzbiory. Funkcje $\phi_0,\ldots,\phi_n$ spe"lniaj"a jednorodny warunek
brzegowy, tj.\ ich warto"sci i~pochodne cz"astkowe rz"edu $1$ (lub $1$ i~$2$)
na brzegu obszaru~$\varOmega$ s"a r"owne~$0$. Funkcje
$\phi_n,\ldots,\phi_{n+m-1}$ spe"lniaj"a warunki brzegowe dobrane tak, aby
dla dowolnych wektor"ow $\bm{a}_0,\ldots,\bm{a}_{n-1}$ powierzchnia opisana
wzorem~(\ref{eq:G2:surface}) "l"aczy"la si"e z~powierzchni"a dan"a
z~ci"ag"lo"sci"a p"laszczyzny stycznej lub krzywizny.
Funkcje $\phi_n,\ldots,\phi_{n+m-1}$ i~ich
pochodne rz"edu $1$ i~$2$ (albo $1,\ldots,4$) w~\textbf{punkcie "srodkowym},
tj.\ wsp"olnym punkcie wszystkich obszar"ow $\varOmega_l$ maj"a warto"s"c~$0$.%
\end{sloppypar}

P"o"lproste styczne do wsp"olnych krzywych obszar"ow
$\varOmega_0,\ldots,\varOmega_{k-1}$ s"a nachylone pod k"atami
$\alpha_0,\ldots,\alpha_{k-1}$, przy czym
$\alpha_0<\alpha_1<\cdots<\alpha_{k-1}<\alpha_0+2\pi$. Zbi"or
$\varDelta=\{\alpha_0,\ldots,\alpha_{k-1}\}$ nazywa si"e \textbf{podzia"lem
k"ata pe"lnego}. Niech $h$ oznacza liczb"e par
$\{\alpha_i,\alpha_i+\pi\}\subset\varDelta$. W~przypadku powierzchni wype"lniaj"acej
klasy~$G^1$ lub~$G^1Q^2$, oznaczmy
\begin{align*}
  n' = 3 + \max\{k,h+3\}.
\end{align*}
Liczba $n'$ jest liczb"a element"ow tzw.\ \textbf{bazy podstawowej}, w~kt"orej
wszystkie p"laty funkcji bazowych s"a bikubicznymi p"latami Coonsa,
okre"slonymi przez wielomiany stopnia~$5$ (s"a to zatem p"laty wielomianowe
stopnia~$(5,5)$).

Mo"rna przyj"a"c $n=n'$ lub $n=n'+4k$; w~tym drugim przypadku mamy \textbf{baz"e
rozszerzon"a} odpowiednio powi"ekszonej przestrzeni
$V_0=\mathord{\mathrm{lin}}\{\phi_0,\ldots,\phi_{n-1}\}\subset~V$. P"laty funkcji
bazowych do"l"aczonych $4k$ funkcji bazowych s"a iloczynami tensorowymi
wielomian"ow Bernsteina $B^5_2$ i~$B^5_3$. Podobnie jak w~przypadku
u"rycia bazy podstawowej, wynik konstrukcji sk"lada si"e z~$k$~p"lat"ow
B\'{e}ziera stopnia~$(5,5)$.

Jeszcze jedna mo"rliwo"s"c to wype"lnianie otworu p"latami B-sklejanymi
stopnia~$(5,5)$. Baza odpowiedniej przestrzeni~$V_0$ opr"ocz element"ow
bazy podstawowej zawiera dwie rodziny funkcji: w~pierwszej z~nich
p"laty funkcji bazowych s"a iloczynami tensorowymi funkcji B-sklejanych
$N^5_i$ i~$N^5_j$ dla $i,j\in\{2,\ldots,3+n_km_2\}$. Musi by"c $1\leq m_2\leq 4$
P"laty funkcji bazowych drugiej rodziny funkcji s"a bikubicznymi p"latami
Coonsa, okre"slonymi przez krzywe sklejane stopnia~$5$, maj"ace $n_km_1$
w"ez"l"ow, gdzie $1\leq m_1\leq 2$. Wymiar przestrzeni~$V_0$ jest wtedy r"owny
$n'+k\LP(2+n_km_2)^2+2n_km_1\RP$. Wynik konstrukcji ma posta"c $k$~p"lat"ow
B-sklejanych stopnia~$(5,5)$.

Dla konstrukcji powierzchni klasy~$G^2$ niech
\begin{align*}
  n' = 6+\max\{k,h+4\}+\max\{2k,2h+5\}.
\end{align*}
Je"sli $n=n'$, to mamy baz"e podstawow"a, w~kt"orej p"laty funkcji bazowych s"a
dwupi"etnymi p"latami Coonsa stopnia~$(9,9)$.

Baza rozszerzona zawiera dodatkowo $16k$ funkcji, kt"orych p"laty funkcji bazowych
s"a iloczynami tensorowymi wielomian"ow Bernsteina $B^9_3,\ldots,B^^9_6$.
W~obu przypadkach wynik konstrukcji sk"lada si"e z~$k$ wielomianowych p"lat"ow
stopnia~$(9,9)$, reprezentowanych w~postaci B\'{e}ziera.

Mo"rna te"r u"ry"c bazy sklejanej, okre"slonej za pomoc"a trzech parametr"ow,
$n_k$, $m_1$ i~$m_2$; w~tym przypadku przestrze"n~$V_0$ ma wymiar
\begin{align*}
  n' + k\LP(4+n_km_2)^2+3n_km_1\RP,
\end{align*}
a~wynik konstrukcji sk"lada si"e z~$k$~p"lat"ow B-sklejanych stopnia~$(9,9)$.
Musi by"c $1\leq m_1\leq 3$, $1\leq m_2\leq 7$.


\subsection{Kryteria optymalizacji powierzchni klasy~$G^1$}

Powierzchnie wype"lniaj"ace klasy~$G^1$ maj"a stopie"n~$(5,5)$.
Powierzchnie te s"a konstruowane przez minimalizacj"e
funkcjona"l"ow
\begin{align*}
  F_a(\bm{p}) &{}\stackrel{\mathrm{def}}{=}
  \int_{\varOmega}\|\Delta\bm{p}\|_2^2\,\mathrm{d}\varOmega, \\
  F_b(\bm{p}) &{}\stackrel{\mathrm{def}}{=}
  \int_{\varOmega}H^2\sqrt{\det G}\,\mathrm{d}\varOmega,
\end{align*}
gdzie $G$~oznacza macierz pierwszej formy podstawowej, a~$H$ oznacza
krzywizn"e "sredni"a powierzchni. Funkcjona"l~$F_a$ jest
form"a kwadratow"a, kt"ora ma jednoznacznie okre"slone minimum
(tak"re dla dowolnych niesprzecznych w"ez"l"ow).
Funkcjona"l~$F_b$, podobnie jak~$F_d$, jest istotnie nieliniowy,
a~jego warto"s"c zale"ry tylko od kszta"ltu powierzchni.
Minimalizacja funkcjona"lu~$F_b$ jest bardziej k"lopotliwa i~czasoch"lonna ni"r
minimalizacja~$F_a$ i~nie dla ka"rdej powierzchni danej jest wykonalna.


\subsection{Kryteria optymalizacji powierzchni klasy~$G^2$}

Wektory $\bm{a}_0,\ldots,\bm{a}_{n-1}$ s"a dobierane tak, aby zminimalizowa"c
warto"s"c jednego z~nast"epuj"acych funkcjona"l"ow:
\begin{align*}
  F_c(\bm{p}) \stackrel{\mathrm{def}}{=}{}&
  \int_{\varOmega}\|\nabla\!\Delta\bm{p}\|_F^2\,\mathrm{d}\varOmega, \\
  F_d(\bm{p}) \stackrel{\mathrm{def}}{=}{}&
  \int_{\varOmega}\|\nabla_{\!\cal M}H\|_2^2\sqrt{\det G}\,\mathrm{d}\varOmega.
\end{align*}
Kolejne wiersze macierzy $\nabla\!\Delta\bm{p}$ s"a gradientami laplasjan"ow
$d$~funkcji skalarnych opisuj"acych parametryzacj"e~$\bm{p}$; symbol
$\|\cdot\|_F$ oznacza norm"e Frobeniusa, tj.\ pierwiastek sumy kwadrat"ow
wszystkich wsp"o"lczynnik"ow macierzy.

Funkcjona"l~$F_c$ jest okre"slony dla powierzchni w~przestrzeni o~dowolnym
wymiarze~$d$, natomiast w~przypadku funkcjona"lu~$F_d$ musi by"c $d=3$.
Symbol~$H$ oznacza krzywizn"e "sredni"a powierzchni, $\nabla_{\!\cal M}H$
oznacza gradient krzywizny "sredniej na powierzchni, za"s~$G$ oznacza
macierz pierwszej formy podstawowej.

Funkcjona"l~$F_c$ jest form"a kwadratow"a, kt"orej minimalizacja polega na
rozwi"azaniu uk"ladu r"owna"n liniowych
\begin{align}\label{eq:g2h:Ritz:eq}
  A\bm{a}=-B\bm{b},
\end{align}
w~kt"orym wyst"epuj"a macierze $A=[a_{ij}]_{i,j}$ i~$B=[b_{ij}]_{i,j}$
o~wymiarach $n\times n$ i~$n\times m$, kt"orych wsp"o"lczynniki
\begin{align*}
  a_{ij} = a(\phi_i,\phi_j), \qquad
  b_{ij} = a(\phi_i,\phi_{j+n}),
\end{align*}
s"a warto"sciami formy dwuliniowej
\begin{align*}
  a(f,g) = \int_{\varOmega}\scp{\nabla\!\Delta f}%
  {\nabla\!\Delta g}\,\mathrm{d}\varOmega.
\end{align*}
Macierz $\bm{b}$ o~wymiarach $m\times d$ sk"lada si"e z~punkt"ow kontrolnych
powierzchni danej, macierz~$\bm{a}$, o~wymiarach $n\times d$ sk"lada si"e
z~niewiadomych wektor"ow $\bm{a}_0,\ldots,\bm{a}_{n-1}$. Liczba~$d$ jest
wymiarem przestrzeni, w~kt"orej jest powierzchnia, na przyk"lad~$3$
(ale mo"re te"r by"c~$4$ w~konstrukcji wielomianowej powierzchni
jednorodnej, reprezentuj"acej powierzchni"e wymiern"a).


Warto"s"c funkcjona"lu~$F_d$ nie zale"ry od parametryzacji powierzchni
(kt"ora musi le"re"c w~$\R^3$),
tylko od jej kszta"ltu. Znalezienie jego minimum jest trudniejsze, bardziej
czasoch"lonne i~nie zawsze wykonalne (wykonalno"s"c konstrukcji zale"ry od
danej powierzchni z~otworem). Polega ono na rozwi"azaniu uk"ladu r"owna"n
nieliniowych
\begin{align}
  \nabla F(a_0,\ldots,a_{n-1}) = \bm{0},
\end{align}
gdzie funkcja~$F$ jest okre"slona wzorem
\begin{align*}
  F(a_0,\ldots,a_{n-1}) = F_d(\bm{p}),
\end{align*}
dla parametryzacji $\bm{p}$ danej wzorem
\begin{align*}
  \bm{p}(u,v) &{}= \left[\begin{array}{c} u \\ v \\ p(u,v) \end{array}\right],
\qquad
  p(u,v) = \sum_{i=0}^{n-1}a_i\phi_i+\sum_{i=0}^{m-1}b_i\phi_{n+i}.
\end{align*}
Powierzchnia z~otworem jest przedstawiana w~takim uk"ladzie
wsp"o"lrz"ednych $uvw$, aby by"la w~nim wykresem funkcji skalarnej,
$w=q(u,v)$. Dziedzin"e $\varOmega$ konstruuje si"e przez zrzutowanie
powierzchni na p"laszczyzn"e~$uv$. Liczby $b_0,\ldots,b_{m-1}$ s"a
wsp"o"lrz"ednymi~$w$ punkt"ow kontrolnych powierzchni danej.


\subsection{Kryteria optymalizacji dla powierzchni klasy~$G^1Q^2$}

\subsection{R"ownania wi"ez"ow}

\begin{sloppypar}
Konstrukcje umo"rliwiaj"a nak"ladanie wi"ez"ow opisanych przez r"ownania
liniowe, np.\ wi"ez"ow interpolacyjnych. Minimum funkcjona"lu~$F_c$ lub~$F_d$
mo"rna poszukiwa"c w~zbiorze powierzchni, kt"orych wsp"o"lczynniki
spe"lniaj"a uk"lad r"owna"n
\begin{align}\label{eq:constraints}
  C\bm{a} = \bm{d}.
\end{align}
Macierz~$C$ o~wymiarach $w\times n$, musi by"c wierszowo regularna.
Macierz~$\bm{d}$, o~wymiarach $\bm{w}\times d$ opisuje praw"a stron"e
uk"ladu r"owna"n wi"ez"ow, przy czym~$w$ jest liczb"a wi"ez"ow, a~$d$ jest
wymiarem przestrzeni, w~kt"orej jest powierzchnia; dla konstrukcji
z~minimalizacj"a funkcjona"lu $F_d$ musi by"c $d=3$.
\end{sloppypar}

Kolejne wiersze niewiadomej macierzy~$\bm{a}$ s"a wektorami
$\bm{a}_0,\ldots,\bm{a}_{n-1}$, wyst"epuj"acymi we
wzorze~(\ref{eq:G2:surface}). Je"sli $i$-ty warunek na"lo"rony na powierzchni"e
ma posta"c $\bm{p}(\bm{x})=\bm{p}_0$ (to jest warunek interpolacyjny, kt"ory
narzuca punkt powierzchni odpowiadaj"acy punktowi~$\bm{x}\in\varOmega$), to
wsp"o"lczynniki w~$i$-tym wierszu macierzy~$C$ maj"a by"c r"owne
$\phi_0(\bm{x}),\ldots,\phi_{n-1}(\bm{x})$, a~$i$-ty wiersz
macierzy~$\bm{d}$ ma by"c r"owny
$\bm{p}_0-\sum_{i=0}^{m-1}\bm{b}_i\phi_{n+i}(\bm{x})$. Podobnie, narzucenie
warto"sci $\bm{v}$ pochodnej cz"astkowej na przyk"lad ze wzgl"edu na~$u$
w~punkcie~$\bm{x}$ nast"epuje za pomoc"a r"ownania wi"ezu, dla kt"orego
odpowiedni wiersz macierzy~$C$ sk"lada si"e z~liczb
$\frac{\partial}{\partial u}\phi_0(\bm{x}),\ldots,%
\frac{\partial}{\partial u}\phi_{n-1}(\bm{x})$, a~po prawej stronie
(tj.\ w~macierzy~$\bm{d}$) jest $\bm{v}-\sum_{i=0}^{m-1}\bm{b}_i%
\frac{\partial}{\partial u}\phi_{n+i}(\bm{x})$.

W~przypadku funkcjona"lu $F_d$, w~razie zastosowania bazy rozszerzonej,
dopuszczalne jest tylko narzucanie wi"ez"ow interpolacyjnych w~punkcie
"srodkowym dziedziny (tj.\ we wsp"olnym punkcie obszar"ow~$\varOmega_i$).
Biblioteka zawiera procedury obliczaj"ace warto"sci funkcji bazowych i~ich
pochodnych cz"astkowych w~tym punkcie. Ponadto s"a te"r procedury
udost"epniaj"ace pe"ln"a informacj"e o~dowolnej funkcji bazowej. Na
podstawie tej informacji program mo"re obliczy"c warto"sci dowolnego
funkcjona"lu liniowego na wszystkich funkcjach bazowych. Warto"sci tych
mo"rna nast"epnie u"ry"c jako wsp"o"lczynniki w~r"ownaniu wi"ezu.

\vspace{\medskipamount}
Opisany wy"rej spos"ob narzuca jednocze"snie i~niezale"rnie na wszystkie
wsp"o"lrz"edne po\-wierzch\-ni wype"lniaj"acej wi"ezy tej samej natury.
Alternatywna posta"c r"owna"n wi"ez"ow jest nast"epuj"aca:
\begin{align}\label{eq:alt:constraints}
  C_0\bm{a}_0+\cdots+C_{d-1}\bm{a}_{d-1} = \bm{d}.
\end{align}
Macierze $C_0,\ldots,C_{d-1}$ maj"a wymiary $w\times n$, przy czym
macierz~$C=[C_0,\ldots,C_{d-1}]$ (o~wymiarach $w\times nd$) musi by"c
wierszowo regularna. Ta posta"c wi"ez"ow jest og"olniejsza i~dopuszcza
okre"slenie warto"sci dowolnego funkcjona"lu liniowego dla
parametryzacji~$\bm{p}$. Mo"rna na przyk"lad narzuci"c tylko warto"s"c
pierwszej wsp"o"lrz"ednej punktu~$\bm{p}(\bm{x})$, przyjmuj"ac $i$-ty wiersz
macierzy~$C_0$ z"lo"rony ze wsp"o"lczynnik"ow
$\phi_0(\bm{x}),\ldots,\phi_{n-1}(\bm{x})$,
a~w~macierzach $C_1,\ldots,C_{d-1}$ umieszczaj"ac w~$i$-tym wierszu zera.


\subsection{Tabela procedur konstrukcji powierzchni}

Orientacj"e w"sr"od dost"epnych procedurach konstrukcji powierzchni
wype"lniaj"acych powinna u"latwi"c nast"epuj"aca tabela:

\vspace{\bigskipamount}
\centerline{%
\begin{picture}(2800,950)
\put(50,480){%
\begin{tabular}{@{}c@{\:}|@{\:}c@{\:}|@{\:}c@{\:}|@{\:}c@{\:}|@{\:}c@{\:}|@{\:}c@{\:}|@{\:}c@{\:}|@{\:}c@{\:}|@{\:}c@{\:}|@{\:}c@{\:}}
      & Coons & B\'{e}zier & B-spline & Coons & B\'{e}zier & B-spline & Coons & B\'{e}zier & B-spline \\ \hline
$G^1$ &   1.  &     2.     &    3.    &   4.  &     5.     &          &   7.  &     8.     &          \\ \hline
$G^2$ &  10.  &    11.     &   12.    &  13.  &    14.     &          &  16.  &    17.     &          \\ \hline
$G^1Q^2$ & 19.&    20.     &   21.    &  22.  &    23.     &          &  25.  &    26.     &          \\ \hline
$G^1$ &  28.  &    29.     &   30.    &  31.  &    32.     &          &  34.  &    35.     &          \\ \hline
$G^2$ &  37.  &    38.     &   39.    &  40.  &    41.     &          &  43.  &    44.     &          \\ \hline
$G^1Q^2$ & 46.&    47.     &   48.    &       &            &          &       &            &
\end{tabular}}%
\put(-70,594){L$\left\{\makebox[0pt][l]{\rule{0pt}{20pt}}\right.$}
\put(-70,258){\makebox[0pt][r]{N}L$\left\{\makebox[0pt][l]{\rule{0pt}{20pt}}\right.$}
\put(300,100){$\underbrace{\makebox[100pt][c]{}}_{\mbox{bez wi"ez"ow}}$}
\put(1163,100){$\underbrace{\makebox[100pt][c]{}}_{\mbox{wi"ezy~(\ref{eq:constraints})}}$}
\put(2027,100){$\underbrace{\makebox[100pt][c]{}}_{\mbox{wi"ezy~(\ref{eq:alt:constraints})}}$}
\end{picture}}

\vspace{\bigskipamount}
Procedury zaznaczone w~pierwszych trzech wierszach realizuj"a konstrukcje
z~minimalizacj"a form kwadratowych (przez rozwi"azanie uk"ladu r"owna"n liniowych).

Procedury w~kolejnych trzech wierszach rozwi"azuj"a uk"lady r"owna"n nieliniowych,
w~celu dokonania minimalizacji odpowiednich funkcjona"l"ow niezale"rnych od
parametryzacji.

Procedury w~pierwszych trzech kolumnach dokonuj"a konstrukcji bez nak"ladania
wi"ez"ow. W~kolejnych trzech kolumnach s"a wyliczone procedury konstrukcji
z~wi"ezami o~postaci~(\ref{eq:constraints}), a~w~nast"epnych trzech
z~wi"ezami~(\ref{eq:alt:constraints}).

Na g"orze ka"rdej kolumny jest podana posta"c bazy u"rywanej w~konstrukcji.
Nazwy procedur s"a nast"epuj"ace:

\begin{enumerate}
\item \texttt{g1h\_FillHolef}.
\item \texttt{g1h\_ExtFillHolef}.
\item \texttt{g1h\_SplFillHolef}.
\item \texttt{g1h\_FillHoleConstrf}.
\item \texttt{g1h\_ExtFillHoleConstrf}.
\addtocounter{enumi}{1}
\item \texttt{g1h\_FillHoleAltConstrf}.
\item \texttt{g1h\_ExtFillHoleAltConstrf}.
\addtocounter{enumi}{1}
\item \texttt{g2h\_FillHolef}.
\item \texttt{g2h\_ExtFillHolef}.
\item \texttt{g2h\_SplFillHolef}.
\item \texttt{g2h\_FillHoleConstrf}.
\item \texttt{g2h\_ExtFillHoleConstrf}.
\addtocounter{enumi}{1}
\item \texttt{g2h\_FillHoleAltConstrf}.
\item \texttt{g2h\_ExtFillHoleAltConstrf}.
\addtocounter{enumi}{1}
\item \texttt{g1h\_Q2FillHolef}.
\item \texttt{g1h\_Q2ExtFillHolef}.
\item \texttt{g1h\_Q2SplFillHolef}.
\item \texttt{g1h\_Q2FillHoleConstrf}.
\item \texttt{g1h\_Q2ExtFillHoleConstrf}.
\addtocounter{enumi}{1}
\item \texttt{g1h\_Q2FillHoleAltConstrf}.
\item \texttt{g1h\_Q2ExtFillHoleAltConstrf}.
\addtocounter{enumi}{1}
\item \texttt{g1h\_NLFillHolef}.
\item \texttt{g1h\_NLExtFillHolef}.
\item \texttt{g1h\_NLSplFillHolef}.
\item \texttt{g1h\_NLFillHoleConstrf}.
\item \texttt{g1h\_NLExtFillHoleConstrf}.
\addtocounter{enumi}{1}
\item \texttt{g1h\_NLFillHoleAltConstrf}.
\item \texttt{g1h\_NLExtFillHoleAltConstrf}.
\addtocounter{enumi}{1}
\item \texttt{g2h\_NLFillHolef}.
\item \texttt{g2h\_NLExtFillHolef}.
\item \texttt{g2h\_NLSplFillHolef}.
\item \texttt{g2h\_NLFillHoleConstrf}.
\item \texttt{g2h\_NLExtFillHoleConstrf}.
\addtocounter{enumi}{1}
\item \texttt{g2h\_NLFillHoleAltConstrf}.
\item \texttt{g2h\_NLExtFillHoleAltConstrf}.
\addtocounter{enumi}{1}
\item \texttt{g1h\_Q2NLFillHolef}.
\item \texttt{g1h\_Q2NLExtFillHolef}.
\item \texttt{g1h\_Q2NLSplFillHolef}.
\end{enumerate}

\newpage
\section{Spos"ob u"rycia procedur}

\subsection{Konstrukcja podstawowa}

\begin{sloppypar}
Konstrukcja powierzchni wype"lniaj"acej sk"lada si"e z~dw"och g"l"ownych
etap"ow. Pierwszy etap polega na skonstruowaniu bazy przestrzeni liniowej~$V$,
kt"orej elementami s"a funkcje opisuj"ace parametryzacj"e powierzchni
(tj.\ ka"rd"a ze wsp"o"lrz"ednych), oraz obliczenie zale"rnych od tej bazy
wsp"o"lczynnik"ow macierzy, kt"ore wyst"epuj"a w~rozwi"azywanych w~drugim
etapie uk"lad"ow r"owna"n. To obliczenie jest do"s"c czasoch"lonne,
ale przetwarzana jest w~nim tylko reprezentacja dziedziny.%
\end{sloppypar}

W~drugim etapie na podstawie punkt"ow kontrolnych powierzchni i~ewentualnie
wi"ez"ow (je"sli zosta"ly one okre"slone) jest obliczana prawa strona
uk"ladu r"owna"n, kt"ory jest nast"epnie rozwi"azywany. Rozwi"azanie uk"ladu
jest u"rywane do wyznaczenia $k$~p"lat"ow B\'{e}ziera stopnia~$(9,9)$,
z~kt"orych sk"lada si"e powierzchnia wype"lniaj"aca. Drugi etap
zabiera znacznie mniej czasu i~w~praktyce mo"re by"c powtarzany wielokrotnie,
gdy u"rytkownik programu interakcyjnego manipuluje punktami kontrolnymi
powierzchni lub wi"ezami (ale zmiana w"ez"l"ow wymaga powt"orzenia pierwszego
etapu konstrukcji).

Pierwszy etap konstrukcji zostanie zrealizowany przez wykonanie
nast"epuj"acych instrukcji:

\vspace{\medskipamount}
\noindent{\ttfamily%
GHoleDomainf *domain; \\
\ldots \\
if ( !(domain = gh\_CreateDomainf ( k, knots, domain\_cp )) ) \\
\mbox{} \ exit ( 1 ); \\
if ( !g2h\_ComputeBasisf ( domain ) ) \\
\mbox{} \ exit ( 1 ); \\
if ( !g2h\_DecomposeMatrixf ( domain ) ) \\
\mbox{} \ exit ( 1 );}

\vspace{\medskipamount}
\begin{sloppypar}
Parametr \texttt{k} okre"sla liczb"e wierzcho"lk"ow otworu, za"s
tablice \texttt{knots} i~\texttt{domain\_cp} zawieraj"a odpowiednio ci"agi
w"ez"l"ow i~punkty kontrolne dziedziny. Procedura \texttt{gh\_CreateDomainf}
tworzy struktur"e danych, kt"ora zawiera reprezentacj"e dziedziny~$\varOmega$
parametryzacji powierzchni wype"lniaj"acej, a~tak"re sposobu jej podzia"lu
na fragmenty ($k$~krzywoliniowych czworok"at"ow) i~bazy przestrzeni~$V$.
Struktura ta dalej b"edzie nazywana \textbf{rekordem dziedziny}.%
\end{sloppypar}

\begin{sloppypar}
Po utworzeniu rekordu dziedziny (przed wywo"laniem \texttt{g2h\_ComputeBasisf}),
mo"r\-na wywo"la"c procedur"e \texttt{g2h\_SetOptionProcf} w~celu u"rycia innych
ni"r domy"slne opcji w~konstrukcji. Procedura \texttt{g2h\_ComputeBasisf}
oblicza reprezentacj"e funkcji, z~kt"orych sk"lada si"e baza przestrzeni~$V$,
co zajmuje raczej ma"lo czasu.%
\end{sloppypar}

Procedura \texttt{g2h\_DecomposeMatrixf} oblicza wsp"o"lczynniki macierzy
$A$ i~$B$, kt"ore wyst"epuj"a w~uk"ladzie r"owna"n~(\ref{eq:g2h:Ritz:eq}).
Wsp"o"lczynniki te s"a warto"sciami formy dwuliniowej w~przestrzeni~$V$ dla
par funkcji bazowych. Wektor~$\bm{b}$ sk"lada si"e z~punkt"ow kontrolnych
powierzchni (b"ed"a one okre"slone w~drugim etapie konstrukcji),
a~niewiadomy wektor~$\bm{a}$ sk"lada si"e z~pozosta"lych wsp"o"lczynnik"ow
u"rywanej w~konstrukcji reprezentacji powierzchni wype"lniaj"acej. Macierz~$A$,
kt"ora jest symetryczna i~dodatnio okre"slona, jest nast"epnie rozk"ladana
metod"a Choleskiego na czynniki tr"ojk"atne: $A=LL^T$, kt"ore b"ed"a potrzebne
podczas rozwi"azywania uk"ladu. Obliczenie wsp"o"lczynnik"ow macierzy $A$ i~$B$
jest najbardziej czasoch"lonnym krokiem konstrukcji --- dla $k=8$ procesor
Pentium~IV z~zegarem $1.8$GHz mo"re na to zu"ry"c ok.~$0.15$s.

\vspace{\medskipamount}
Wykonanie drugiego etapu konstrukcji polega na wykonaniu instrukcji

\vspace{\medskipamount}
\noindent{\ttfamily%
if ( !g2h\_FillHolef ( domain, d, surf\_cp, acoeff, output ) ) \\
\mbox{} \ exit ( 1 );}

\vspace{\medskipamount}
Parametr \texttt{domain} jest wska"znikiem rekordu dziedziny, dla kt"orego
pierwszy etap konstrukcji zosta"l (z~sukcesem) zako"nczony. Parametr~\texttt{d}
okre"sla wymiar przestrzeni, w~kt"orej le"ry powierzchnia, tablica
\texttt{surf\_cp} zawiera punkty kontrolne powierzchni, a~parametr
\texttt{output} jest wska"znikiem procedury, kt"ora zostanie wywo"lana
$k$-krotnie. Za ka"rdym razem jej parametry b"ed"a opisywa"c reprezentacj"e
B\'{e}ziera kolejnego p"lata b"ed"acego cz"e"sci"a powierzchni wype"lniaj"acej
otw"or.

Parametr \texttt{acoeff} jest tablic"a, do kt"orej ma by"c wstawione
rozwi"azanie uk"ladu r"owna"n~(\ref{eq:g2h:Ritz:eq}). Mo"re ono by"c
potrzebne, ja"sli kogo"s interesuje warto"s"c funkcjona"lu~$F$
dla skonstruowanej powierzchni ("sci"slej, suma warto"sci dla funkcji
opisuj"acych wsp"o"lrz"edne powierzchni). Parametr teh mo"re mie"c
te"r warto"s"c \texttt{NULL} i~wtedy jest ignorowany.


\subsection{Konstrukcja nieliniowa}

Aby otrzyma"c powierzchni"e wype"lniaj"ac"a, dla kt"orej funkcjona"l~$F_d$
przyjmuje minimaln"a warto"s"c, nale"ry utworzy"c reprezentacj"e dziedziny,
skonstruowa"c baz"e i~obliczy"c wsp"o"lczynniki macierzy~$A$ i~$B$,
a~nast"epnie zamiast procedury \texttt{g2h\_FillHolef} wywo"la"c procedur"e
\texttt{g2h\_NLFillHolef}.

Obliczenia w~tej konstrukcji s"a znacznie bardziej pracoch"lonne (do paru
sekund), a~ponadto wykonalno"s"c tej konstrukcji zale"ry od powierzchni
z~otworem. Je"sli powierzchnia ta nie jest dostatecznie p"laska lub jest
zdegenerowana, to konstrukcja mo"re zako"nczy"c si"e niepowodzeniem.


\subsection{Konstrukcje z~przestrzeni"a rozszerzon"a}

P"laty wype"lniaj"ace otw"or skonstruowane w~spos"ob opisany wy"rej s"a
okre"slone jako dwupi"etne p"laty Coonsa, kt"orych krzywe opisuj"ace
brzegi i~pochodne w~kierunku poprzecznym do brzegu maj"a stopie"n~$9$
lub mniejszy. Reprezentacja Coonsa u"rywana w~konstrukcji jest poddawana
konwersji do postaci B\'{e}ziera. Poniewa"r przes\-trze"n wielomian"ow dw"och
zmiennych stopnia~$(9,9)$ ma wymiar~$100$, a~wielomiany stopnia~$(9,9)$
maj"ace dwupi"etn"a reprezentacj"e Coonsa tworz"a podprzestrze"n
o~wymiarze~$84$, wi"ec istnieje mo"rliwo"s"c rozszerzenia przestrzeni
funkcji, za pomoc"a kt"orych s"a reprezentowane powierzchnie wype"lniaj"ace,
do przestrzeni, kt"orej wymiar jest o~$16k$ wi"ekszy. Powierzchnie, wyznaczone
za pomoc"a minimalizacji funkcjona"lu~$F$ w~rozszerzonej przestrzeni,
mog"a mie"c lepszy kszta"lt (i~cz"esto maj"a).

Aby skorzysta"c z~tej mo"rliwo"sci, po utworzeniu rekordu dziedziny
za pomoc"a procedury \texttt{gh\_CreateDomainf} i~ewentualnym
zarejestrowaniu procedury wprowadzania opcji, nale"ry utworzy"c baz"e
przestrzeni podstawowej, wywo"luj"ac jak poprzednio \texttt{g2h\_ComputeBasisf}
(konstrukcja dodatkowych funkcji, kt"ore wchodz"a w~sk"lad bazy rozszerzonej
przestrzeni nie wymaga "radnych dodatkowych oblicze"n). Nast"epnie
\emph{zamiast} procedury \texttt{g2h\_DecomposeMatrixf} nale"ry wywo"la"c
procedur"e \texttt{g2h\_DecomposeExtMatrixf}, kt"ora obliczy macierze~$A$
i~$B$ odpowiednio powi"ekszonego uk"ladu r"owna"n~(\ref{eq:g2h:Ritz:eq})
i~roz"lo"ry macierz~$A$ na czynniki tr"ojk"atne.

Drugi etap konstrukcji przy u"ryciu przestrzeni rozszerzonej wykonuje
procedura \texttt{g2h\_ExtFillHolef}, kt"or"a nale"ry wywo"la"c \emph{zamiast}
procedury \texttt{g2h\_FillHolef}. Parametry tych procedur s"a identyczne.

Dane wykorzystywane w~obu konstrukcjach s"a tworzone i~przechowywane
w~rekordzie dziedziny niezale"rnie, w~zwi"azku z~czym mo"rna najpierw
utworzy"c i~roz"lo"ry"c macierze uk"lad"ow dla obu tych konstrukcji,
a~potem wykona"c je w~dowolnej kolejno"sci i~por"owna"c wyniki.
Czas oblicze"n dla przestrzeni rozszerzonej jest d"lu"rszy, ale r"o"rnica
jest praktycznie niezauwa"ralna. Obliczenia dla przestrzeni rozszerzonej
wymagaj"a za to wi"ecej pami"eci --- zar"owno na wyniki (wsp"o"lczynniki
macierzy, kt"ore przechowuje si"e w~obszarach rezerwowanych przez
\texttt{malloc}), jak i~na tablice pomocnicze (w~puli pami"eci ,,podr"ecznej'',
obs"lugiwanej przez procedury opisane w~p.~\ref{sect:scratch:mem}).
W~wersji podw"ojnej precyzji dla $k=8$ tablice pomocnicze mog"a zajmowa"c
oko"lo~$2$MB.

Aby skonstruowa"c powierzchni"e minimaln"a funkcjona"lu~$F_d$
przy u"ryciu przes\-trze\-ni rozszerzonej, nale"ry wywo"la"c procedur"e
\texttt{g2h\_NLExtFillHolef}. Pami"e"c ,,podr"eczna'' potrzebna w~tej
konstrukcji mo"re mie"c wielko"s"c do ok.\ $8$MB.


\subsection{\label{ssect:g2h:constraints}Konstrukcje z~wi"ezami}

W~celu skonstruowania powierzchni z~wi"ezami nale"ry skonstruowa"c
reprezentacj"e dziedziny (przy u"ryciu \texttt{gh\_CreateDomainf}),
skonstruowa"c baz"e, a~nast"epnie wprowadzi"c macierz~$C$ uk"ladu r"owna"n
wi"ez"ow i~wywo"la"c procedur"e konstrukcji powierzchni z~wi"ezami.

\begin{sloppypar}
Dla \textbf{przestrzeni podstawowej}, do wprowadzania macierzy uk"ladu r"owna"n
wi"ez"ow~(\ref{eq:constraints}) s"lu"ry
procedura~\texttt{g2h\_SetConstraintMatrixf}. Powierzchni"e minimaln"a
funkcjona"lu~$F_c$ z~wi"ezami konstruuje procedura
\texttt{g2h\_FillHoleConstrf}. Powierzchni"e minimaln"a funkcjona"lu~$F_d$
z~wi"ezami o~tej postaci konstruuje procedura \texttt{g2h\_NLFillHoleConstrf}.%
\end{sloppypar}

Do wprowadzenia macierzy uk"ladu r"owna"n wi"ez"ow
o~postaci~(\ref{eq:alt:constraints})
s"lu"ry procedura \texttt{g2h\_SetAltConstraintMatrixf}. Powierzchni"e
minimaln"a funkcjona"lu~$F_c$ z~wi"ezami o~tej postaci konstruuje procedura
\texttt{g2h\_FillHoleAltConstrf}, a~powierzchni"e minimaln"a
funkcjona"lu~$F_d$ procedura \texttt{g2h\_NLFillHoleAltConstrf}.

Dla \textbf{przestrzeni rozszerzonej} do wprowadzenia macierzy uk"ladu
r"owna"n~(\ref{eq:constraints}) s"lu"ry procedura
\texttt{g2h\_SetExtConstraintMatrixf}. Powierzchnia minimalna funkcjona"lu
$F_c$ jest konstruowana przez procedur"e
\texttt{g2h\_ExtFillHoleConstrf}, a~powierzchnia minimalna~$F_d$ przez
\texttt{g2h\_NLExtFillHoleConstrf}.

Macierz uk"ladu~(\ref{eq:alt:constraints}) dla przestrzeni rozszerzonej
wprowadza si"e za pomoc"a procedury
\texttt{g2h\_SetExtAltConstraintMatrixf}. Powierzchnie minimalne
funkcjona"l"ow $F_c$ i~$F_d$ otrzymuje si"e za pomoc"a procedur
\texttt{g2h\_ExtFillHoleAltConstrf} i~\texttt{g2h\_NLExtFillHoleAltConstrf}.

Macierz ka"rdego z~czterech rodzaj"a wi"ez"ow
(tj.\ o~postaci~(\ref{eq:constraints}) i~(\ref{eq:alt:constraints}) dla
przestrzeni podstawowej i~rozszerzonej) mo"re by"c okre"slona niezale"rnie
od~pozosta"lych. Aby zmieni"c macierz uk"ladu wi"ez"ow, wystarczy ponownie
wywo"la"c odpowiedni"a procedur"e spo"sr"od wymienionych wy"rej.


\section{Procedury podstawowe}

\cprog{%
\#define G2H\_FINALDEG  9 \\
\#define GH\_MAX\_K \ 16}
\begin{sloppypar}
\hspace*{\parindent}
Powy"rsze dwie sta"le symboliczne okre"slaj"a stopie"n p"lat"ow wype"lniaj"acych
otw"or i~najwi"eksz"a dopuszczaln"a liczb"e wierzcho"lk"ow wielok"atnego otworu.%
\end{sloppypar}

Sta"lych tych nie mo"rna po prostu zmieni"c --- stopie"n~$9$ jest rezultatem
zastosowania odpowiedniego schematu interpolacyjnego. Procedury biblioteczne
mog"a zrealizowa"c te"r schemat interpolacyjny, kt"orego wynikiem s"a p"laty
stopnia~$10$. W~tym celu trzeba usun"a"c definicj"e symbolu
\texttt{G2H\_FINALDEG9} w~pliku nag"l"owkowym i~skompilowa"c procedury.

Dziedzina parametryzacji powierzchni wype"lniaj"acej $k$-k"atny otw"or jest
dzielona na $k$~cz"e"sci. Do reprezentowania zbior"ow tych cz"e"sci s"a
u"rywane liczby ca"lkowite kr"otkie, czyli zmienne o~d"lugo"sci $16$~bit"ow.
Aby wype"lnia"c
otwory wi"ecej ni"r szesnastok"atne, trzeba przerobi"c odpowiedni fragment
procedur, tak, aby u"rywa"ly one np.\ s"l"ow $32$-bitowych, co umo"rliwi
wype"lnianie otwor"ow trzydziestodwuk"atnych.

\vspace{\bigskipamount}
\cprog{%
typedef struct GHoleDomainf \{ \\
\ind{4}int     hole\_k; \\
\ind{4}float   *hole\_knots; \\
\ind{4}point2f *domain\_cp; \\
\ind{4}boolean basisG1, basisG2; \\
\ind{4}void    *privateG; \\
\ind{4}void    *privateG1; \\
\ind{4}void    *SprivateG1; \\
\ind{4}void    *privateG2; \\
\ind{4}void    *SprivateG2; \\
\ind{4}int     error\_code; \\
\ind{2}\} GHoleDomainf;}
Typ struktury \texttt{GHoleDomainf} opisuje rekord dziedziny, tj.\ obiekt
reprezentuj"acy dane potrzebne do skonstruowania powierzchni
wype"lniaj"acej otw"or. W~programie deklaruje si"e zmienn"a wska"znikow"a
do rekordu tego typu, poniewa"r za ich tworzenie i~poprawno"s"c danych
odpowiedzialne s"a procedury biblioteczne.

Pole \texttt{hole\_k} opisuje liczb"e wierzcho"lk"ow wielok"atnego
otworu do wype"lnienia (od $3$ do $16$).

Pole \texttt{hole\_knots} jest wska"znikiem do tablicy $11k$ liczb,
kt"ore s"a w"ez"lami w~reprezentacji powierzchni z~otworem.

Pole \texttt{domain\_cp} jest wska"znikiem do tablicy $12k+1$ punkt"ow
statki kontrolnej reprezentuj"acej dziedzin"e parametryzacji
powierzchni wype"lniaj"acej.

Pola \texttt{privateG}, \texttt{privateG1}, \texttt{SprivateG1},
\texttt{privateG2} i~\texttt{SprivateG2} s"a wska"znikami rekord"ow
(ich budowa i~zawarto"s"c s"a niewidoczne dla aplikacji) z~wszystkimi
innymi danymi potrzebnymi w~poszczeg"olnych konstrukcjach powierzchni
wype"lniaj"acych.

Pole \texttt{error\_code} s"lu"ry do przechowywania informacji o~sukcesie,
albo o~przyczynie niepowodzenia oblicze"n.


\vspace{\bigskipamount}
\cprog{%
GHoleDomainf* gh\_CreateDomainf ( int     hole\_k, \\
\ind{33}float   *hole\_knots, \\
\ind{33}point2f *domain\_cp ); \\
void gh\_DestroyDomainf ( GHoleDomainf *domain );}
Procedura \texttt{gh\_CreateDomainf} tworzy obiekt typu~\texttt{GHoleDomainf},
kt"ory reprezentuje dziedzin"e powierzchni wype"lniaj"acej i~zwraca wska"znik
do tego obiektu. Bloki pami"eci dla tego obiektu i~wszystkich danych
wskazywanych przez zawarte w~nim wska"zniki s"a rezerwowane za pomoc"a procedury
\texttt{malloc}.

Parameter \texttt{hole\_k} okre"sla liczb"e~$k$ wierzcho"lk"ow dziedziny otworu
(musi by"c od $3$ do~$16$).

Parameter \texttt{hole\_knots} jest tablic"a z~$11k$ liczbami --- w"ez"lami
reprezentacji powierzchni z~otworem i~dziedziny.

Parameter \texttt{domain\_cp} jest tablic"a z~$12k+1$ punktami kontrolnymi
reprezentacji dziedziny. Zawarto"s"c tych dw"och tablic jest kopiowana
do pami"eci rezerwowanej przez procedur"e \texttt{gh\_CreateDomainf}.    

Je"sli nie mo"rna zarezerwowa"c pami"eci lub w~danych zosta"l wykryty b"l"ad,
to warto"sci"a procedury jest \texttt{NULL}.

Obiekt natychmiast po utworzeniu nie jest got"ow do wype"lniania powierzchni
--- ta procedura nie konstruuje bazy. Obliczenia przygotowawcze s"a
nieco czasoch"lonne, dlatego w~aplikacji mo"re by"c wygodne ich oddzielenie
od utworzenia obiektu.

\vspace{\medskipamount}
Procedura \texttt{gh\_DestroyDomainf} zwalnia (za pomoc"a procedury \texttt{free})
pami"e"c zajmowan"a przez reprezentacj"e dziedziny (w~szczeg"olno"sci wszystkie
bloki pami"eci zarezerwowane podczas oblicze"n wykonanych podczas przetwarzania
tej reprezentacji).

\vspace{\bigskipamount}
\cprog{%
void g2h\_SetOptionProcf ( GHoleDomainf *domain, \\
\ind{4}int (*OptionProc)( GHoleDomainf *domain, int query, int qn, \\
\ind{23}int *ndata, int **idata, float **fdata ) );}
Procedura \texttt{g2h\_SetOptionProcf} rejestruje dostarczon"a przez aplikacj"e
procedur"e, kt"orej zadaniem jest okre"slenie opcji dla konstrukcji
i~przekazanie odpowiednich danych. Je"sli po utworzeniu reprezentacji
dziedziny nie zarejestrujemy takiej procedury, to u"rywana jest procedura
domy"slna (kt"ora na ka"rde pytanie o~opcj"e daje odpowied"z domy"sln"a).

Ten spos"ob okre"slania opcji zosta"l zrealizowany w~celu ustalenia
listy parametr"ow procedur bibliotecznych podczas opracowywania konstrukcji.
Korzy"s"c z~niego jest taka, "re aplikacja, kt"ora nie u"rywa opcji
innych ni"r standardowe, nie musi wywo"lywa"c procedur bibliotecznych
z~parametrami pozbawionymi dla tej aplikacji znaczenia.

Zasady okre"slania opcji s"a opisane w~p.~\ref{ssect:g2h:options}.

\vspace{\bigskipamount}
\cprog{%
boolean g2h\_ComputeBasisf ( GHoleDomainf *domain );}
Procedura \texttt{g2h\_ComputeBasisf} konstruuje funkcje bazowe, przy u"ryciu
kt"orych mo"rna b"edzie wype"lnia"c otwory w~powierzchniach.
Parametr procedury jest wska"znikiem obiektu utworzonego przez
procedur"e \texttt{gh\_CreateDomainf}. Warto"s"c procedury \texttt{true}
oznacza sukces, a~\texttt{false} oznacza niepowodzenie obliczenia.

Je"sli dla dziedziny przekazanej jako parametr zosta"la zarejestrowana
procedura okre"slania opcji, to b"edzie ona wywo"lana pewn"a liczb"e razy.
Procedura ta ma wp"lyw na wynik oblicze"n (tj.\ posta"c funkcji bazowych),
czyli tak"re na powierzchnie wype"lniaj"ace otrzymane przy u"ryciu tej bazy.

Procedura \texttt{g2h\_ComputeBasisf} powinna by"c wywo"lana tylko raz
dla reprezentacji dziedziny utworzonej przez procedur"e
\texttt{gh\_CreateDomainf}. Je"sli trzeba dla tej samej dziedziny
skonstruowa"c wi"ecej ni"r jedn"a baz"e, np.\ przy u"ryciu r"o"rnych opcji,
to trzeba za ka"rdym razem zlikwidowa"c reprezentacj"e dziedziny (wywo"luj"ac
\texttt{gh\_DestroyDomainf}) i~utworzy"c j"a od nowa.

Obliczenie realizowane przez t"e procedur"e zajmuje umiarkowan"a ilo"s"c czasu,
dlatego mo"rna je wykona"c ,,na poczekaniu'' podcase obs"lugi komunikatu
(zwi"azane z~tym op"o"znienie nie powinno by"c zauwa"ralne dla u"rytkownika
programu interakcyjnego).

\vspace{\bigskipamount}
\cprog{%
boolean g2h\_ComputeFormMatrixf ( GHoleDomainf *domain ); \\
boolean g2h\_DecomposeMatrixf ( GHoleDomainf *domain );}
\begin{sloppypar}
Procedura \texttt{g2h\_ComputeFormMatrixf} oblicza wsp"o"lczynniki macierzy
uk"ladu r"owna"n rozwi"azywanego podczas konstruowania powierzchni
wype"lniaj"acej otw"or, \textbf{przy u"ryciu przestrzeni podstawowej}.
Parametrem procedury jest reprezentacja dziedziny utworzona przez procedur"e
\texttt{gh\_CreateDomainf}, dla kt"orej procedura \texttt{g2h\_ComputeBasisf}
skonstruowa"la z~sukcesem reprezentacj"e funkcji bazowych.%
\end{sloppypar}

\vspace{\medskipamount}
\begin{sloppypar}
Procedura \texttt{g2h\_DecomposeMatrixf} rozk"lada (metod"a Choleskiego)
macierz uk"la\-du, skonstruowan"a przez procedur"e
\texttt{g2h\_ComputeFormMatrixf}. Je"sli wsp"o"lczynniki nie zosta"ly obliczone,
to procedura \texttt{g2h\_DecomposeMatrixf} najpierw oblicza je, wywo"luj"ac
\texttt{g2h\_ComputeFormMatrixf}.%
\end{sloppypar}

\vspace{\medskipamount}
Warto"sci"a obu procedur jest \texttt{true} je"sli obliczenie przebieg"lo
poprawnie, albo \texttt{false} w~przeciwnym razie.

\vspace{\bigskipamount}
\cprog{%
boolean g2h\_FillHolef ( GHoleDomainf *domain, \\
\ind{14}int spdimen, const float *hole\_cp, float *acoeff, \\
\ind{14}void (*outpatch) ( int n, int m, const float *cp ) );}
\begin{sloppypar}
Procedura \texttt{g2h\_FillHolef} konstruuje powierzchni"e wype"lniaj"ac"a
wielok"atny otw"or, \textbf{w~oparciu o~przestrze"n podstawow"a}.%
\end{sloppypar}

\begin{sloppypar}
Parametr \texttt{domain} wskazuje reprezentacj"e dziedziny
utworzon"a przez procedur"e \texttt{gh\_CreateDomainf}, dla kt"orej procedura
\texttt{g2h\_ComputeBasisf} skonstruowa"la (z~sukcesem) reprezentacj"e
funkcji bazowych. Liczba~$k$ wierzcho"lk"ow otworu i~ci"agi w"ez\-"l"ow
nale"r"ace do reprezentacji powierzchni zosta"ly okre"slone przy wywo"laniu
procedury \texttt{gh\_CreateDomainf}.%
\end{sloppypar}

Parametr \texttt{spdimen} okre"sla wymiar~$d$ przestrzeni, w~kt"orej le"ry
powierzchnia. Dla powierzchni wielomianowej w~$\R^3$ parametr ten b"edzie
mia"l warto"s"c~$3$. Dla powierzchni wielomianowej w~$\R^4$, kt"ora jest
jednorodn"a reprezentacj"a powierzchni wymiernej w~$\R^3$ trzeba poda"c~$4$.

Parametr \texttt{hole\_cp} jest tablic"a, kt"ora zawiera $(12k+1)d$ liczb
zmiennopozycyjnych, b"ed"acych wsp"o"lrz"ednymi $12k+1$ punkt"ow kontrolnych
powierzchni.

Parametr \texttt{acoeff} mo"re mie"c warto"s"c \texttt{NULL}
(i~wtedy jest ignorowany), lub wskazywa"c tablic"e, do kt"orej procedura
wstawi rozwi"azanie uk"ladu r"owna"n~(\ref{eq:g2h:Ritz:eq}). Tab\-li\-ca ta musi
mie"c d"lugo"s"c $nd$, gdzie $d$~jest wymiarem przestrzeni, w~kt"orej le"ry
powierzchnia (tj.\ warto"sci"a parametru \texttt{spdimen}), a~$n$ jest
wymiarem przestrzeni podstawowej --- mo"rna go otrzyma"c za pomoc"a procedury
\texttt{g2h\_V0SpaceDimf}.

Parametr \texttt{outpatch} jest wska"znikiem procedury (dostarczonej przez
aplikacj"e), kt"ora b"edzie wywo"lana $k$~razy, w~celu wyprowadzenia
punkt"ow kontrolnych $k$ p"lat"ow B\'{e}ziera stopnia~$(9,9)$ wype"lniaj"acych
otw"or.


\vspace{\bigskipamount}
\cprog{%
boolean g2h\_ComputeExtFormMatrixf ( GHoleDomainf *domain ); \\
boolean g2h\_DecomposeExtMatrixf ( GHoleDomainf *domain );}
\begin{sloppypar}
Procedura \texttt{g2h\_ComputeExtFormMatrixf} oblicza wsp"o"lczynniki macierzy
uk"la\-du r"owna"n rozwi"azywanego podczas konstruowania powierzchni
wype"lniaj"acej otw"or, \textbf{przy u"ryciu przestrzeni rozszerzonej}.
Parametrem procedury jest reprezentacja dziedziny utworzona przez procedur"e
\texttt{gh\_CreateDomainf}, dla kt"orej procedura \texttt{g2h\_ComputeBasisf}
skonstruowa"la z~sukcesem reprezentacj"e funkcji bazowych.%
\end{sloppypar}

\vspace{\medskipamount}
\begin{sloppypar}
Procedura \texttt{g2h\_DecomposeExtMatrixf} rozk"lada (metod"a Choleskiego)
macierz uk"la\-du, skonstruowan"a przez procedur"e
\texttt{g2h\_ComputeExtFormMatrixf}. Je"sli wsp"o"lczynniki nie zosta"ly obliczone,
to procedura \texttt{g2h\_DecomposeExtMatrixf} najpierw oblicza je, wywo"luj"ac
\texttt{g2h\_ComputeExtFormMatrixf}.%
\end{sloppypar}

\vspace{\medskipamount}
Warto"sci"a obu procedur jest \texttt{true} je"sli obliczenie przebieg"lo
poprawnie, albo \texttt{false} w~przeciwnym razie.

\vspace{\bigskipamount}
\cprog{%
boolean g2h\_ExtFillHolef ( GHoleDomainf *domain, \\
\ind{14}int spdimen, const float *hole\_cp, float *acoeff, \\
\ind{14}void (*outpatch) ( int n, int m, const float *cp ) );}
\begin{sloppypar}
Procedura \texttt{g2h\_ExtFillHolef} konstruuje powierzchni"e wype"lniaj"ac"a
wielok"atny otw"or, \textbf{w~oparciu o~przestrze"n rozszerzon"a}.%
\end{sloppypar}

\begin{sloppypar}
Parametr \texttt{domain} wskazuje reprezentacj"e dziedziny
utworzon"a przez procedur"e \texttt{gh\_CreateDomainf}, dla kt"orej procedura
\texttt{g2h\_ComputeBasisf} skonstruowa"la (z~sukcesem) reprezentacj"e
funkcji bazowych. Liczba~$k$ wierzcho"lk"ow otworu i~ci"agi w"ez\-"l"ow
nale"r"ace do reprezentacji powierzchni zosta"ly okre"slone przy wywo"laniu
procedury \texttt{gh\_CreateDomainf}.%
\end{sloppypar}

Parametr \texttt{spdimen} okre"sla wymiar~$d$ przestrzeni, w~kt"orej le"ry
powierzchnia. Dla powierzchni wielomianowej w~$\R^3$ parametr ten b"edzie
mia"l warto"s"c~$3$. Dla powierzchni wielomianowej w~$\R^4$, kt"ora jest
jednorodn"a reprezentacj"a powierzchni wymiernej w~$\R^3$ trzeba poda"c~$4$.

Parametr \texttt{hole\_cp} jest tablic"a, kt"ora zawiera $(12k+1)d$ liczb
zmiennopozycyjnych, b"ed"acych wsp"o"lrz"ednymi $12k+1$ punkt"ow kontrolnych
powierzchni.

Parametr \texttt{acoeff} mo"re mie"c warto"s"c \texttt{NULL}
(i~wtedy jest ignorowany), lub wskazywa"c tablic"e, do kt"orej procedura
wstawi rozwi"azanie uk"ladu r"owna"n~(\ref{eq:g2h:Ritz:eq}). Tab\-li\-ca ta musi
mie"c d"lugo"s"c $nd$, gdzie $d$~jest wymiarem przestrzeni, w~kt"orej le"ry
powierzchnia (tj.\ warto"sci"a parametru \texttt{spdimen}), a~$n$ jest
wymiarem przestrzeni rozszerzonej --- mo"rna go otrzyma"c za pomoc"a procedury
\texttt{g2h\_ExtV0SpaceDimf}.

Parametr \texttt{outpatch} jest wska"znikiem procedury (dostarczonej przez
aplikacj"e), kt"ora b"edzie wywo"lana $k$~razy, w~celu wyprowadzenia
punkt"ow kontrolnych $k$ p"lat"ow B\'{e}ziera stopnia~$(9,9)$ wype"lniaj"acych
otw"or.


\vspace{\bigskipamount}
\cprog{%
int g2h\_GetErrorCodef ( GHoleDomainf *domain, \\
\ind{24}char **ErrorString );}
Procedura \texttt{g2h\_GetErrorCodef} mo"re by"c wywo"lana w~razie
niepowodzenia kt"orego"s etapu konstrukcji, w~celu ustalenia przyczyny.
Jej warto"sci"a jest numer (kod) b"l"edu. Je"sli parametr \texttt{ErrorString}
jest r"o"rny od \texttt{NULL}, to zmienna \texttt{*ErrorString} po powrocie
z~procedury wskazuje napis, kt"ory jest opisem b"l"edu (po angielsku).


\section{\label{ssect:g2h:options}Wprowadzanie opcji}

Procedura \texttt{g2h\_SetOptionProcf} opisana w~poprzednim punkcie s"lu"ry
do zarejestrowania procedury (nale"r"acej do aplikacji), kt"ora ma ,,odpowiada"c
na pytania'' na temat opcji. Procedura ta ma mie"c nast"epuj"acy nag"l"owek
(nazwy procedury i~parametr"ow mog"a oczywi"scie by"c inne):

\vspace{\medskipamount}
\noindent{\ttfamily%
int SetOptionf ( GHoleDomainf *domain, int query, int qn, \\
\ind{17}int *ndata, int **idata, float **fdata );}

\vspace{\medskipamount}
Podczas konstruowania bazy procedura ta b"edzie wywo"lana pewn"a liczb"e razy.
Pierwszy jej parametr jest wska"znikiem reprezentacji dziedziny wype"lnianego
otwo\-ru. Drugi parametr (\texttt{query}) jest numerem opcji, kt"or"a procedura
ma okre"sli"c. Parametr~\texttt{qn} jest dodatkowym numerem, kt"ory mo"re
by"c potrzebny w~opcjach dodanych do przysz"lych wersji biblioteki, a~na razie
mo"rna go zignorowa"c.

Warto"s"c procedury jest interpretowana jako odpowied"z na pytanie o~opcj"e,
kt"or"a nale"ry zastosowa"c. Mo"rliwe numery opcji (tj.\ warto"sci parametru
\texttt{query}) i~odpowiedzi s"a sta"lymi symbolicznymi (o~nazwach
zaczynaj"acych si"e odpowiednio od \texttt{G2HQUERY\_} i~\texttt{G2H\_})
wymienionymi ni"rej. Lista ta mo"re si"e zmieni"c w~przysz"lych wersjach
biblioteki \texttt{libeghole}.

\vspace{\bigskipamount}
\cprog{%
\#define G2H\_DEFAULT                    0 \\
\mbox{} \\
\#define G2HQUERY\_CENTRAL\_POINT         1 \\
\#define G2H\_CENTRAL\_POINT\_GIVEN        1 \\
\mbox{} \\
\#define G2HQUERY\_CENTRAL\_DERIVATIVES1  2 \\
\#define G2H\_CENRTAL\_DERIVATIVES1\_ALT   1 \\
\#define G2H\_CENTRAL\_DERIVATIVES1\_GIVEN 2 \\
\mbox{} \\
\#define G2HQUERY\_DOMAIN\_CURVES         3 \\
\#define G2H\_DOMAIN\_CURVES\_DEG4         1 \\
\mbox{} \\
\#define G2HQUERY\_BASIS                 4 \\
\#define G2H\_USE\_RESTRICTED\_BASIS       1}
\begin{sloppypar}
Po ka"rdym wywo"laniu procedura wprowadzania opcji mo"re zwr"oci"c warto"s"c
\texttt{G2H\_DEFAULT}. W~szczeg"olno"sci taka powinna by"c warto"s"c zwracana
dla ka"rdej opcji (okre"slonej przez parametr \texttt{query}), kt"orej procedura
,,nie rozumie''. Dzi"eki temu aplikacja ma szanse dzia"la"c poprawnie po
skompilowaniu z~nowsz"a wersj"a biblioteki.%
\end{sloppypar}

Je"sli parametr \texttt{query} ma warto"s"c \texttt{G2HQUERY\_CENTRAL\_POINT},
to odpowied"z (tj.\ warto"s"c procedury) \texttt{G2H\_DEFAULT} spowoduje
przyj"ecie za punkt "srodkowy dziedziny (tj.\ wsp"olny naro"rnik obszar"ow,
na kt"ore dziedzina zostanie podzielona) "srodka ci"e"rko"sci "srodk"ow
bok"ow otworu (to jest konstrukcja opisana w~artyku"lach).
Je"s\-li procedura zwr"oci warto"s"c \texttt{G2H\_CENTRAL\_POINT\_GIVEN}, to
zmienna \texttt{*ndata} musi otrzyma"c warto"s"c~$2$, a~zmiennej \texttt{*fdata}
nale"ry przypisa"c warto"s"c wskazuj"ac"a tablic"e z~dwiema liczbami
zmiennopozycyjnymi, kt"ore s"a wsp"o"lrz"ednymi punktu "srodkowego podanego
przez aplikacj"e.

Je"sli warto"sci"a parametru \texttt{query} jest
\texttt{G2HQUERY\_CENTRAL\_DERIVATIVES1} i~procedura zwr"oci warto"s"c
\texttt{G2H\_DEFAULT}, to wektory pochodnych pierwszego rz"edu krzywych
podzia"lu dziedziny w~punkcie "srodkowym i~wektory pochodnych poprzecznych
p"lat"ow pomocniczych dziedziny zostan"a przyj"ete zgodnie z~opisem
w~artyku"lach. Warto"s"c \texttt{G2H\_CENTRAL\_DERIVATIVES\_ALT} spowoduje
przyj"ecie wektor"ow pochodnych krzywych tak samo, za"s wektory pochodnych
poprzecznych b"ed"a do nich prostopad"le. Warto"s"c
\texttt{G2H\_CENTRAL\_DERIVATIVES\_GIVEN} oznacza, "re aplikacja podaje
wektory pochodnych krzywych. Zmienna \texttt{*ndata} ma mie"c warto"s"c
$2k$ (dla $k$-k"atnego otworu), a~tablica wskazywana przez zmienn"a
\texttt{*fdata} ma zawiera"c $2k$ liczb zmiennopozycyjnych. Ka"rde kolejne
dwie z~tych liczb s"a wsp"o"lrz"ednymi wektora pochodnej kolejnej krzywej.

Parametr \texttt{query} o~warto"sci \texttt{G2HQUERY\_DOMAIN\_CURVES} oznacza
pytanie o~pochodne krzywych podzia"lu dziedziny rz"edu wy"rszego ni"r~$1$
w~punkcie "srodkowym. W~odpowiedzi na to pytanie nale"ry zwr"oci"c
warto"s"c \texttt{G2H\_DEFAULT}, co spowoduje przyj"ecie zerowych pochodnych
rz"edu $2$, $3$ i~$4$. Inne opcje w~tym przypadku s"a na razie niedopracowane
i~mog"a da"c niepoprawne skutki.

Je"sli parametr \texttt{query} ma warto"s"c \texttt{G2HQUERY\_BASIS}, to
odpowied"z \texttt{G2H\_DEFAULT} oznacza u"rycie wszystkich stopni swobody
wyboru pochodnych cz"astkowych p"lat"ow wype"lniaj"acych otw"or w~ich punkcie
wsp"olnym (liczba ta jest wymiarem przestrzeni podstawowej). Zale"rnie od
liczby wierzcho"lk"ow otworu i~podzia"lu dziedziny, liczba tych stopni jest
od~$16$ do~$30$ (dla otwor"ow tr"oj- do o"smiok"atnych). Odpowied"z
\texttt{G2H\_USE\_RESTRICTED\_BASIS} powoduje ograniczenie liczby stopni
swobody do~$15$ (jest to wymiar przestrzeni wielomian"ow dw"och zmiennych
stopnia co najwy"rej~$4$).


\section{\label{sect:g2h:constraints}Nak"ladanie wi"ez"ow}

\cprog{%
int g2h\_V0SpaceDimf ( GHoleDomainf *domain ); \\
int g2h\_ExtV0SpaceDimf ( GHoleDomainf *domain );}
\begin{sloppypar}
\hspace*{\parindent}Procedury \texttt{g2h\_V0SpaceDimf}
i~\texttt{g1h\_ExtV0SpaceDimf} obliczaj"a
odpowiednio wymiary przestrzeni podstawowej i~rozszerzonej, u"rywanych
w~konstrukcjach po\-wierzch\-ni wype"lniaj"acych otw"or. Parametr \texttt{domain}
jest wska"znikiem rekordu dziedziny, dla kt"orej baza przestrzeni podstawowej
zosta"la pomy"slnie skonstruowana.%
\end{sloppypar}

\vspace{\bigskipamount}
\cprog{%
boolean g2h\_GetBPDerivativesf ( GHoleDomainf *domain, \\
\ind{32}int cno, float *val );}
\begin{sloppypar}
Procedura \texttt{g1h\_GetBPDerivativesf} oblicza warto"sci funkcji
bazowych (bazy przestrzeni podstawowej) w~punkcie "srodkowym
i~warto"sci pochodnych rz"edu $1,\ldots,4$
krzywych brzegowych p"lat"ow funkcji bazowych. Numer krzywej jest okre"slony
przez warto"s"c parametru~\texttt{cno} (musi by"c od~$0$ do~$k-1$).
Obliczone warto"sci s"a wpisywane do tablicy \texttt{val}, o~d"lugo"sci~$5n$,
gdzie $n$ jest wymiarem przestrzeni podstawowej. Kolejne pi"atki liczb
wpisywanych do tej tablicy odpowiadaj"a kolejnym funkcjom bazowym.%
\end{sloppypar}

\vspace{\bigskipamount}
\cprog{%
boolean g2h\_GetBFuncPatchf ( GHoleDomainf *domain, \\
\ind{29}int fn, int pn, float *bp );}
Procedura \texttt{g1h\_GetBFuncPatchf} oblicza i~umieszcza w~tablicy~\texttt{bp}
wsp"o"lczynniki $j$-tego p"lata $i$-tej funkcji bazowej. Warto"s"c
parametru~\texttt{fn} okre"sla numer~$i\in\{0,\ldots,n-1\}$ funkcji bazowej,
a~\texttt{pn} okre"sla numer~$j\in\{0,\ldots,k-1\}$ p"lata tej funkcji.
Procedura ta mo"re si"e przyda"c, je"sli wi"ezy nak"ladane na powierzchni"e
wype"lniaj"ac"a otw"or nie s"a warunkami interpolacyjnymi w~punkcie "srodkowym
powierzchni.

\begin{sloppypar}
P"laty funkcji bazowych s"a (skalarnymi) wielomianami dw"och zmiennych, stopnia
\texttt{G2H\_FINALDEG} ze wzgl"edu na ka"rd"a zmienn"a. Wsp"o"lczynniki
reprezentuj"a te p"laty w~tensorowej bazie Bernsteina.%
\end{sloppypar}

\vspace{\bigskipamount}
\cprog{%
boolean g2h\_SetConstraintMatrixf ( GHoleDomainf *domain, \\
\ind{34}int nconstr, const float *cmat );}
Procedura \texttt{g2h\_SetConstraintMatrixf} zwi"azuje z~dziedzin"a
powierzchni wype"lniaj"acej macierz uk"ladu r"owna"n liniowych opisuj"acych
wi"ezy, kt"ore maj"a by"c na"lo"rone na powierzchni"e.
Parametr \texttt{nconstr} jest liczb"a wi"ez"ow (tj.\ r"owna"n), czyli liczb"a
wierszy macierzy. Liczba kolumn jest r"owna wymiarowi przestrzeni podstawowej.
Kolejne wiersze s"a podane w~tablicy \texttt{cmat}. Wiersze te musz"a by"c
liniowo niezale"rne.

Warto"s"c \texttt{true} procedury oznacza powodzenie, za"s \texttt{false}
oznacza, "re podana macierz nie jest wierszowo-regularna.

\vspace{\bigskipamount}
\cprog{%
boolean g2h\_FillHoleConstrf ( GHoleDomainf *domain, \\
\ind{14}int spdimen, const float *hole\_cp, \\
\ind{14}int nconstr, const float *constr, \\
\ind{14}float *acoeff, \\
\ind{14}void (*outpatch) ( int n, int m, const float *cp ) );}
Procedura \texttt{g1h\_FillHoleConstrf} konstruuje powierzchni"e wype"lniaj"ac"a
otw"or, z~na"lo"ronymi wi"ezami, przy u"ryciu przestrzeni podstawowej.
Przed wywo"laniem tej procedury nale"ry z~dziedzin"a powierzchni zwi"aza"c
macierz uk"ladu r"owna"n opisuj"acego wi"ezy (to okre"sla m.in.\ liczb"e
na"lo"ronych wi"ez"ow). Parametry \texttt{domain}, \texttt{spdimen},
\texttt{hole\_cp}, \texttt{acoeff} i~\texttt{outpatch} maj"a takie samo znaczenie
jak w~procedurze \texttt{g2h\_FillHolef}. Parametr \texttt{nconstrf} okre"sla
liczb"e wi"ez"ow (musi ona zgadza"c si"e z~liczb"a podan"a w~wywo"laniu
\texttt{g2h\_SetConstraintMatrixf}. Tablica \texttt{constr} zawiera
macierz prawej strony r"owna"n wi"ez"ow --- \texttt{nconstr} wierszy po
\texttt{spdimen} liczb.

Zmiana wi"ez"ow (zar"owno macierzy jak i~prawej strony) nie wymaga tworzenia
rekordu dziedziny od pocz"atku. Aby zmieni"c wi"ezy, wystarczy ponownie
wywo"la"c \texttt{g2h\_SetConstraintMatrixf} i~\texttt{g1h\_FillHoleConstrf}.

\vspace{\bigskipamount}
\cprog{%
boolean g2h\_SetAltConstraintMatrixf ( GHoleDomainf *domain, \\
\ind{20}int spdimen, \\
\ind{20}int nconstr, const float *cmat );}
\begin{sloppypar}
Procedura \texttt{g2h\_SetAltConstraintMatrixf} s"lu"ry do wprowadzenia
macierzy~$C$ uk"ladu r"owna"n wi"ez"ow o~postaci~(\ref{eq:alt:constraints})
dla konstrukcji z~przestrzeni"a podstawow"a. Macierz ta ma wymiary $nd\times w$,
gdzie~$n$ jest wymiarem przestrzeni (mo"rna go otrzyma"c, wywo"luj"ac procedur"e
\texttt{g2h\_V0SpaceDimf}), $d$~jest wymiarem przestrzeni, w~kt"orej znajduje
si"e powierzchnia (czyli np.~$3$), $w$~jest liczb"a wi"ez"ow.%
\end{sloppypar}

Parametr \texttt{spdimen} okre"sla wymiar~$d$, parametr \texttt{nconstr}
okre"sla liczb"e wi"ez"ow. Wsp"o"lczynniki macierzy~$C$ nale"ry poda"c
w~tablicy~\texttt{cmat}. Podzia"lka tej tablicy jest r"owna d"lugo"sci
wiersza, tj.~$nd$. Macierz~$C$ musi by"c wierszowo regularna.

Warto"s"c \texttt{true} procedury \texttt{g2h\_SetAltConstraintMatrixf} oznacza,
"re macierz jest wierszowo regularna. Warto"s"c \texttt{false} oznacza, "re nie
(tj.\ procedura numeryczna uzna"la, "re wiersze s"a liniowo zale"rne)
i~z~tak"a macierz"a nie mo"rna wykona"c konstrukcji.

\vspace{\bigskipamount}
\cprog{%
boolean g2h\_FillHoleAltConstrf ( GHoleDomainf *domain, \\
\ind{10}int spdimen, const float *hole\_cp, \\
\ind{10}int naconstr, const float *constr, \\
\ind{10}float *acoeff, \\
\ind{10}void (*outpatch) ( int n, int m, const float *cp ) );}
\begin{sloppypar}
Procedura \texttt{g2h\_FillHoleAltConstrf} konstruuje powierzchni"e
wype"lniaj"ac"a, kt"ora jest punktem minimalnym funkcjona"lu~$F_c$ w~zbiorze
powierzchni reprezentowalnych przy u"ryciu przestrzeni podstawowej
i~spe"lniaj"acych r"ownania wi"ez"ow~(\ref{eq:alt:constraints}).
Macierz~$C$ tego uk"ladu musi by"c wprowadzona wcze"sniej, przez wywo"lanie
procedeury \texttt{g2h\_SetAltConstraintMatrixf}.%
\end{sloppypar}

Parametr \texttt{spdimen} okre"sla wymiar~$d$ przestrzeni, w~kt"orej znajduje
si"e powierzchnia. Tablica \texttt{hole\_cp} zawiera punkty kontrolne tej
powierzchni. Parametr \texttt{naconstr} okre"sla liczb"e wi"ez"ow~$w$.
W~tablicy \texttt{constr} nale"ry poda"c wsp"o"lczynniki wektora prawej
strony uk"ladu r"owna"n wi"ez"ow ($w$~liczb). Liczby~$d$ i~$w$ musz"a si"e
zgadza"c z~warto"sciami odpowiednich parametr"ow poprzedzaj"acego wywo"lania
procedury \texttt{g2h\_SetAltConstraintMatrixf}.

Je"sli parametr \texttt{acoeff} jest r"o"rny od~\texttt{NULL}, to ma wskazywa"c
tablic"e, do kt"orej procedura wpisze obliczone jako wynik optymalizacji
wektory $\bm{a}_0,\ldots,\bm{a}_{n-1}$. Procedura wskazywana przez
parametr \texttt{output} zostanie wywo"lana $k$~razy w~celu wyprowadzenia
wyniku konstrukcji, tj.\ p"lat"ow B\'{e}ziera wype"lniaj"acych otw"or.

\vspace{\bigskipamount}
\cprog{%
boolean g2h\_SetExtConstraintMatrixf ( GHoleDomainf *domain, \\
\ind{34}int nconstr, const float *cmat );}
Procedura \texttt{g2h\_SetExtConstraintMatrixf} zwi"azuje z~dziedzin"a
powierzchni wype"lniaj"acej macierz~$C$ uk"ladu r"owna"n
liniowych~(\ref{eq:constraints}) opisuj"acych wi"ezy, kt"ore maj"a by"c
na"lo"rone na powierzchni"e reprezentowan"a przy u"ryciu przestrzeni rozszerzonej..
Parametr \texttt{nconstr} jest liczb"a wi"ez"ow (tj.\ r"owna"n), czyli liczb"a
wierszy macierzy. Liczba kolumn jest r"owna wymiarowi przestrzeni rozszerzonej.
Kolejne wiersze s"a podane w~tablicy \texttt{cmat}. Wiersze te musz"a by"c
liniowo niezale"rne. D"lugo"s"c wiersza~$n$ jest r"owna wymiarowi przestrzeni~$V_0$,
kt"ory mo"rna pozna"c, wywo"luj"ac procedur"e \texttt{g2h\_ExtV0SpaceDimf}.

Warto"s"c \texttt{true} procedury oznacza powodzenie, za"s \texttt{false}
oznacza, "re podana macierz nie jest wierszowo-regularna.

Pierwsze~$16k$ wsp"o"lczynnik"ow odpowiada funkcjom bazowym, kt"ore w~punkcie
"srodkowym dziedziny maj"a warto"sci i~pochodne do czwartego rz"edu w"l"acznie
r"owne~$0$. Ostatnie $n'$~wsp"o"lczynnik"ow w~wierszu to warto"sci
odpowiedniego funkcjona"lu liniowego dla funkcji bazowych przestrzeni podstawowej.
Je"sli wi"ezy s"a okre"slane w~celu skonstruowania powierzchni minimalnej
funkcjona"lu~$F_d$ (za pomoc"a procedury \texttt{g2h\_NLExtFillHoleConstrf}),
to pierwsze~$16k$ wsp"o"lczynniki w~ka"rdym wierszu musz"a by"c zerami.
Ograniczenie to wynika z~zastosowanej w~tej konstrukcji metody numerycznej.

\vspace{\bigskipamount}
\cprog{%
boolean g2h\_ExtFillHoleConstrf ( GHoleDomainf *domain, \\
\ind{14}int spdimen, const float *hole\_cp, \\
\ind{14}int nconstr, const float *constr, \\
\ind{14}float *acoeff, \\
\ind{14}void (*outpatch) ( int n, int m, const float *cp ) );}
\begin{sloppypar}
Procedura \texttt{g1h\_ExtFillHoleConstrf} konstruuje powierzchni"e wype"lniaj"ac"a
otw"or, z~na"lo"ronymi wi"ezami, przy u"ryciu przestrzeni rozszerzonej.
Przed wywo"laniem tej procedury nale"ry z~dziedzin"a powierzchni zwi"aza"c
macierz uk"ladu r"owna"n opisuj"acego wi"ezy (to okre"sla m.in.\ liczb"e
na"lo"ronych wi"ez"ow). Parametry \texttt{domain}, \texttt{spdimen},
\texttt{hole\_cp}, \texttt{acoeff} i~\texttt{outpatch} maj"a takie samo znaczenie
jak w~procedurze \texttt{g2h\_ExtFillHolef}. Parametr \texttt{nconstrf} okre"sla
liczb"e wi"ez"ow (musi ona zgadza"c si"e z~liczb"a podan"a w~wywo"laniu
\texttt{g2h\_SetExtConstraintMatrixf}. Tablica \texttt{constr} zawiera
macierz prawej strony r"owna"n wi"ez"ow --- \texttt{nconstr} wierszy po
\texttt{spdimen} liczb.%
\end{sloppypar}

Zmiana wi"ez"ow (zar"owno macierzy jak i~prawej strony) nie wymaga tworzenia
rekordu dziedziny od pocz"atku. Aby zmieni"c wi"ezy, wystarczy ponownie
wywo"la"c \texttt{g2h\_SetExtConstraintMatrixf} i~\texttt{g1h\_ExtFillHoleConstrf}.

\vspace{\bigskipamount}
\cprog{% 
boolean g2h\_SetExtAltConstraintMatrixf ( GHoleDomainf *domain, \\
\ind{32}int spdimen, \\
\ind{32}int naconstr, const float *acmat );}
\begin{sloppypar}
Zadaniem procedury \texttt{g2h\_SetExtAltConstraintMatrixf} jest wprowadzenie
macierzy~$C$ uk"ladu r"owna"n wi"ez"ow~(\ref{eq:alt:constraints}) dla konstrukcji
powierzchni wype"lniaj"acej reprezentowanej za pomoc"a przestrzeni rozszerzonej.
Wymiary tej macierzy to~$nd\times w$, gdzie $n$ jest wymiarem przestrzeni
rozszerzonej, $d$~jest wymiarem przestrzeni, w~kt"orej jest powierzchnia
(np.~$3$), a~liczba wierszy~$w$ jest liczb"a wi"ez"ow.%
\end{sloppypar}

Parametry: \texttt{spdimen} --- wymiar~$d$, \texttt{naconstr} --- liczba
wi"ez"ow~$w$, \texttt{acmat} --- tablica wsp"o"lczynnik"ow macierzy~$C$.
Tablica ta ma podzia"lk"e r"own"a d"lugo"sci wiersza, tj.~$nd$.

Tablic"e~$C$ mo"rna podzieli"c na bloki $C_0,\ldots,C_{d-1}$, o~wymiarach
$w\times d$. Je"sli wi"ezy s"a okre"slane na potrzeby konstrukcji opartej
na minimalizacji funkcjona"lu~$F_d$, to musi by"c $d=3$. W~ka"rdym wierszu
ka"rdego bloku pierwsze $16k$~wspo"lczynniki musz"a by"c zerami, a~ponadto
macierz~$C$ musi by"c wierszowo-regularna.
Ograniczenie to wynika z~zastosowanej w~konstrukcji metody numerycznej.

Warto"s"c \texttt{true} procedury oznacza akceptacj"e macierzy,
za"s warto"s"c \texttt{false} oznacza, "re na podstawie rachunku numerycznego
macierz zosta"la przez procedur"e uznana za wierszowo-nieregularn"a.

\vspace{\bigskipamount}
\cprog{% 
boolean g2h\_ExtFillHoleAltConstrf ( GHoleDomainf *domain, \\                 
\ind{10}int spdimen, const float *hole\_cp, \\                             
\ind{10}int naconstr, const float *constr, \\          
\ind{10}float *acoeff, \\
\ind{10}void (*outpatch) ( int n, int m, const float *cp ) );}
Procedura \texttt{g2h\_ExtFillHoleAltConstrf} dokonuje konstrukcji powierzchni
wype"lniaj"acej, kt"ora minimalizuje funkcjona"l~$F_c$ w~zbiorze powierzchni
spe"lniaj"acych r"ownania wi"ez"ow o~postaci~(\ref{eq:alt:constraints}).
Macierz tego uk"ladu musi by"c wcze"sniej wprowadzona za pomoc"a procedury
\texttt{g2h\_SetExtAltConstraintsf}.

Parametry \texttt{spdimen} i~\texttt{naconstr} okre"slaj"a wymiar~$d$
przestrzeni, w~kt"orej jest powierzchnia i~licz"e wi"ez"ow~$w$. Liczby te musz"a
by"c takie same jak we wcze"sniejszym wywo"laniu
\texttt{g2h\_SetExtAltConstraintsf}. W~tablicy \texttt{hole\_cp} nale"ry poda"c
wsp"o"lrz"edne punkt"ow kontrolnych powierzchni. W~tablicy \texttt{constr}
nale"ry poda"c wsp"o"lczynniki wektora prawej strony uk"ladu
r"owna"n~(\ref{eq:alt:constraints}). Je"sli parametr \texttt{acoeff} ma
warto"s"c r"o"rn"a od~\texttt{NULL}, to do wskazywanej przeze"n tablicy
procedura wpisze wsp"o"lrz"edne wektor"ow $\bm{a}_0,\ldots,\bm{a}_{n-1}$,
otrzymanych w~wyniku optymalizacji. Parametr \texttt{outpatch} wskazuje
procedur"e, kt"ora s"lu"ry do wyprowadzenia wyniku konstrukcji w~postaci
p"lat"ow B\'{e}ziera.

Warto"s"c \texttt{true} oznacza zako"nczenie konstrukcji sukcesem,
a~\texttt{false} oznacza niepowodzenie.

\vspace{\bigskipamount}
\cprog{%
float g2h\_FunctionalValuef ( GHoleDomainf *domain, int spdimen, \\
\ind{23}const float *hole\_cp, const float *acoeff ); \\
float g2h\_ExtFunctionalValuef ( GHoleDomainf *domain, int spdimen, \\
\ind{23}const float *hole\_cp, const float *acoeff );}
Procedury \texttt{g2h\_FunctionalValuef} i~\texttt{g2h\_ExtFunctionalValuef}
obliczaj"a warto"sci funkcjona"lu~$F_c$ dla powierzchni wype"lniaj"acej
otw"or, okre"slonej przez punkty kontrolne dane w~tablicy \texttt{hole\_cp}
i~wektory $\bm{a}_0,\ldots,\bm{a}_{n-1}$ podane w~tablicy~\texttt{acoeff},
b"ed"ace odpowiednio wsp"o"lczynnikami reprezentacji powierzchni w~bazie
przestrzeni podstawowej i~rozszerzonej.

Po skonstruowaniu powierzchni za pomoc"a dowolnej procedury opisanej w~tym
lub w~nast"epnym punkcie, mo"rna wywo"la"c jedn"a z~powy"rszych procedur,
w~celu obliczenia warto"sci tego funkcjona"lu; w~tym celu nale"ry utworzy"c
odpowiedni"a tablic"e (o~d"lugo"sci $dn$ liczb typu \texttt{float}, gdzie
$d$~jest wymiarem przestrzeni, w~kt"orej le"ry powierzchnia (warto"sci"a
parametru~\texttt{spdimen}), a~$n$ jest wymiarem przestrzeni~$V_0$),
przekaza"c t"e tablic"e jako parametr \texttt{acoeff} procedury dokonuj"acej
konstrukcji, a~nast"epnie procedury obliczaj"acej warto"s"c funkcjona"lu.

\subsection{Wype"lnianie otwor"ow p"latami B-sklejanymi}

\cprog{%
\#define G2H\_S\_MAX\_NK  4 \\
\#define G2H\_S\_MAX\_M1  3 \\
\#define G2H\_S\_MAX\_M2  7}

\vspace{\bigskipamount}
\cprog{%
boolean g2h\_ComputeSplBasisf ( GHoleDomainf *domain, \\
\ind{31}int nk, int m1, int m2 );}

\vspace{\bigskipamount}
\cprog{%
boolean g2h\_ComputeSplFormMatrixf ( GHoleDomainf *domain ); \\
boolean g2h\_DecomposeSplMatrixf ( GHoleDomainf *domain );}

\vspace{\bigskipamount}
\cprog{%
boolean g2h\_SplFillHolef ( GHoleDomainf *domain, \\
\ind{11}int spdimen, const float *hole\_cp, \\
\ind{11}float *acoeff, \\
\ind{11}void (*outpatch) ( int n, int lknu, const float *knu, \\
\ind{30}int m, int lknv, const float *knv, \\
\ind{30}const float *cp ) );}

\vspace{\bigskipamount}

\section{Procedury konstrukcji nieliniowych}

Opisane w~tym punkcie procedury dokonuj"a konstrukcji powierzchni wype"lniaj"acych
przez minimalizacj"e funkcjona"lu~$F_d$. Wykonalno"s"c tych konstrukcji
zale"ry w~znacznie wi"ekszym stopniu od danej powierzchni z~otworem.
Ponadto s"a one bardziej czasosch"lonne. Powierzchnia z~otworem do wype"lnienia
przy u"ryciu tych procedur musi le"re"c w~przestrzeni~$\R^3$ (a~wi"ec nie
mo"re to by"c np.\ jednorodna reprezentacja powierzchni wymiernej).

\vspace{\bigskipamount}
\cprog{%
boolean g2h\_ComputeNLNormalf ( GHoleDomainf *domain, \\
\ind{31}const point3f *hole\_cp, \\
\ind{31}vector3f *anv );}
Procedura \texttt{g2h\_ComputeNLNormalf} konstruuje wersor jednej z~osi
uk"ladu wsp"o"lrz"ednych, w~kt"orym powierzchnia z~otworem i~konstruowana
powierzchnia wype"lniaj"aca b"ed"a przedstawione jako wykresy funkcji
skalarnych. Parametry wej"sciowe tej procedury to \texttt{domain} --- wska"znik
reprezentacji dziedziny i~\texttt{hole\_cp} --- tablica punkt"ow kontrolnych
powierzchni. Parametr \texttt{anv} wskazuje zmienn"a, do kt"orej procedura
wpisuje wynik.

Warto"s"c \texttt{true} procedury oznacza sukces, a~\texttt{false} --- jego brak
(je"sli powierzchnia okre"slona przez podane punkty kontrolne nie jest
dostatecznie p"laska). Procedura \texttt{g2h\_ComputeNLNormalf} w~zasadzie
nie jest przeznaczona do wywo"lywania przez aplikacj"e.

\vspace{\bigskipamount}
\cprog{%
boolean g2h\_NLFillHolef ( GHoleDomainf *domain, \\
\ind{12}const point3f *hole\_cp, float *acoeff, \\
\ind{12}void (*outpatch) ( int n, int m, const point3f *cp ) );}
Procedura~\texttt{g2h\_NLFillHolef} dokonuje konstrukcji powierzchni
wype"lniaj"acej, b"ed"acej punktem minimalnym funkcjona"lu~$F_d$
w~przestrzeni podstawowej, bez na"lo"ronych wi"ez"ow.
Procedura ta jest odpowiednikiem procedury \texttt{g2h\_FillHolef}
i~ma takie same parametry z~wyj"atkiem \texttt{spdimen}.

\vspace{\bigskipamount}
\cprog{%
boolean g2h\_NLFillHoleConstrf ( GHoleDomainf *domain, \\
\ind{12}const point3f *hole\_cp, \\
\ind{12}int nconstr, const vector3f *constr, \\
\ind{12}float *acoeff, \\
\ind{12}void (*outpatch) ( int n, int m, const point3f *cp ) );}
\begin{sloppypar}
Procedura \texttt{ g2h\_NLFillHoleConstrf } konstruuje powierzchni"e
wype"lniaj"ac"a, kt"o\-ra minimalizuje funkcjona"l~$F_d$ w~przestrzeni
podstawowej, z~na"lo"ronymi wi"ezami opisanymi przez uk"lad
r"owna"n~(\ref{eq:constraints}). Procedura ta jest odpowiednikiem
procedury \texttt{g2h\_FillHoleConstrf}.%
\end{sloppypar}

\vspace{\bigskipamount}
\cprog{%
boolean g2h\_NLFillHoleAltConstrf ( GHoleDomainf *domain, \\
\ind{12}const point3f *hole\_cp, \\
\ind{12}int nconstr, const float *constr, \\
\ind{12}float *acoeff, \\
\ind{12}void (*outpatch) ( int n, int m, const point3f *cp ) );}
Procedura \texttt{g2h\_NLFillHoleAltConstrf} konstruuje powierzchni"e
wype"lniaj"ac"a, kt"ora minimalizuje funkcjona"l~$F_d$ w~przestrzeni
podstawowej, z~na"lo"ronymi wi"ezami opisanymi przez uk"lad
r"owna"n~(\ref{eq:alt:constraints}). Jest ona odpowiednikiem
procedury \texttt{g2h\_FillHoleAltConstrf}.

\vspace{\bigskipamount}
\cprog{%
boolean g2h\_NLExtFillHolef ( GHoleDomainf *domain, \\
\ind{12}const point3f *hole\_cp, \\
\ind{12}float *acoeff, \\
\ind{12}void (*outpatch) ( int n, int m, const point3f *cp ) );}
Procedura \texttt{g2h\_NLExtFillHolef} dokonuje konstrukcji powierzchni
minimalnej funkcjona"lu~$F_d$ w~przestrzeni rozszerzonej, bez na"lo"ronych
wi"ez"ow. Procedura ta jest odpowiednikiem procedury \texttt{g2h\_ExtFillHolef}
i~ma takie same parametry (z~wyj"atkiem \texttt{spdimen}).

\vspace{\bigskipamount}
\cprog{%
boolean g2h\_NLExtFillHoleConstrf ( GHoleDomainf *domain, \\
\ind{12}const point3f *hole\_cp, \\
\ind{12}int nconstr, const vector3f *constr, \\
\ind{12}float *acoeff, \\
\ind{12}void (*outpatch) ( int n, int m, const point3f *cp ) );}
Procedura \texttt{g2h\_NLExtFillHoleConstrf} konstruuje powierzchni"e
wype"lniaj"ac"a, kt"ora minimalizuje funkcjona"l~$F_d$ w~przestrzeni
rozszerzonej, z~na"lo"ronymi wi"ezami opisanymi przez uk"lad
r"owna"n~(\ref{eq:constraints}). Procedura ta jest odpowiednikiem
procedury \texttt{g2h\_ExtFillHoleConstrf}.

Dopuszczalna w~konstrukcji wykonywanej przez t"e procedur"e
macierz~$C$ uk"ladu r"owna"n opisuj"acych wi"ezy, wprowadzona przez
wcze"sniejsze wywo"lanie procedury \texttt{g2h\_SetExtConstraintMatrixf},
musi mie"c w~ka"rdym wierszu pierwsze $16k$ wsp"o"lczynnik"ow zerowych.

\vspace{\bigskipamount}
\cprog{%
boolean g2h\_NLExtFillHoleAltConstrf ( GHoleDomainf *domain, \\
\ind{12}const point3f *hole\_cp, \\
\ind{12}int naconstr, const float *constr, \\
\ind{12}float *acoeff, \\
\ind{12}void (*outpatch) ( int n, int m, const point3f *cp ) );}
Procedura \texttt{g2h\_NLExtFillHoleAltConstrf} konstruuje powierzchni"e
wype"lniaj"ac"a, kt"ora minimalizuje funkcjona"l~$F_d$ w~przestrzeni
rozszerzonej, z~na"lo"ronymi wi"ezami opisanymi przez uk"lad
r"owna"n~(\ref{eq:alt:constraints}). Jest ona odpowiednikiem
procedury \texttt{g2h\_ExtFillHoleAltConstrf}.

Bloki~$C_0,C_1,C_2$ dopuszczalnej w~konstrukcji wykonywanej przez t"e
procedur"e macierzy~$C=[C_0,C_1,C_2]$ uk"ladu r"owna"n opisuj"acych wi"ezy,
wprowadzonej przez wcze"sniejsze wywo"lanie procedury
\texttt{g2h\_SetExtAltConstraintMatrixf}, musi mie"c w~ka"rdym wierszu
pierwsze $16k$ wsp"o"lczynnik"ow zerowych.

\vspace{\bigskipamount}
\cprog{%
boolean g2h\_NLFunctionalValuef ( GHoleDomainf *domain, \\
\ind{33}const point3f *hole\_cp, \\
\ind{33}const vector3f *acoeff, \\
\ind{33}float *funcval );}
Procedura \texttt{g2h\_NLFunctionalValuef} dla powierzchni wype"lniaj"acej
okre"slonej przez punkty kontrolne w~tablicy~\texttt{hole\_cp} i~wektory
$\bm{a}_0,\ldots,\bm{a}_{n-1}$ (wsp"o"lczynnik"ow reprezentacji powierzchni
w~bazie przestrzeni podstawowej) oblicza warto"s"c funkcjona"lu~$F_d$.

\vspace{\bigskipamount}
\cprog{%
boolean g2h\_NLExtFunctionalValuef ( GHoleDomainf *domain, \\
\ind{30}const point3f *hole\_cp, \\
\ind{30}const vector3f *acoeff, \\
\ind{30}float *funcval );}
Procedura \texttt{g2h\_NLExtFunctionalValuef} dla powierzchni wype"lniaj"acej
okre"slonej przez punkty kontrolne w~tablicy~\texttt{hole\_cp} i~wektory
$\bm{a}_0,\ldots,\bm{a}_{n-1}$ (wsp"o"lczynnik"ow reprezentacji powierzchni
w~bazie przestrzeni rozszerzonej) oblicza warto"s"c funkcjona"lu~$F_d$.

\vspace{\bigskipamount}
\cprog{%
boolean g2h\_NLSplFillHolef ( GHoleDomainf *domain, \\
\ind{10}const point3f *hole\_cp, \\
\ind{10}float *acoeff, \\
\ind{10}void (*outpatch) ( int n, int lknu, const float *knu, \\
\ind{29}int m, int lknv, const float *knv, \\
\ind{29}const point3f *cp ) );}


\newpage
\section{Procedury wizualizacji}

Nazwa ,,procedury wizualizacji'' dotyczy procedur, kt"ore wyci"agaj"a
z~rekordu dziedziny rozmaite dane, na podstawie kt"orych mo"rna uzyska"c
wgl"ad w~dzia"lanie konstrukcji, na przyk"lad rysuj"ac rozmaite obrazki.

\vspace{\bigskipamount}
\cprog{%
void g2h\_DrawDomSurrndPatchesf ( GHoleDomainf *domain, \\
\ind{10}void (*drawpatch) ( int n, int m, const point2f *cp ) );}
Procedura \texttt{g2h\_DrawDomSurrndPatchesf} wyci"aga reprezentacje
B\'{e}ziera stopnia~$(3,3)$ p"lat"ow otaczaj"acych dziedzin"e,
tj.\ wielomianowych kawa"lk"ow p"lat"ow B-sklejanych reprezentowanych
przez w"ez"ly i~punkty kontrolne dziedziny podane podczas tworzenia
rekordu dziedziny.

Parametr \texttt{domain} wskazuje rekord dziedziny, parametr
\texttt{drawpatch} jest wska"znikiem procedury, kt"ora dla otworu
$k$-k"atnego b"edzie wywo"lana $3k$ razy, z~parametrami opisuj"acymi kolejne
p"laty otaczaj"ace dziedzin"e.


\vspace{\bigskipamount}
\cprog{%
void g2h\_DrawDomAuxPatchesf ( GHoleDomainf *domain, \\
\ind{10}void (*drawpatch) ( int n, int m, const point2f *cp ) );}
Procedura \texttt{g2h\_DrawDomAuxPatchesf} wyci"aga reprezentacje B\'{e}ziera
p"lat"ow pomocniczych dziedziny. Parametr \texttt{domain} wskazuje rekord
dziedziny, a~parametr \texttt{drawpatch} jest wska"znikiem procedury,
kt"ora b"edzie wywo"lana $k$~razy, w~celu przekazania aplikacji
kolejnych p"lat"ow.

\vspace{\bigskipamount}
\cprog{%
void g2h\_DrawBasAuxPatchesf ( GHoleDomainf *domain, int fn, \\
\ind{10}void (*drawpatch) ( int n, int m, const float *cp ) );}
Procedura \texttt{g2h\_DrawBasAuxPatchesf} wyci"aga reprezentacje B\'{e}ziera
p"lat"ow pomocniczych funkcji bazowych. S"a to wielomiany dw"och zmiennych;
jest ich~$k$ dla ka"rdej funkcji b"ed"acej elementem bazy przestrzeni
podstawowej. Wymiar~$n$ tej przestrzeni mo"rna pozna"c wywo"luj"ac
procedur"e \texttt{g2h\_V0SpaceDimf}.

Parametr \texttt{domain} wskazuje rekord dziedziny, parametr \texttt{fn}
jest numerem funkcji bazowej (musi mie"c warto"s"c od~$0$ do~$n-1$),
parametr \texttt{drawpatch} jest wska"znikiem procedury, kt"ora zostanie
wywo"lana $k$~razy w~celu przekazania aplikacji kolejnych p"lat"ow
pomocniczych funkcji bazowej o~numerze~\texttt{fn}.

\vspace{\bigskipamount}
\cprog{%
void g2h\_DrawJFunctionf ( GHoleDomainf *domain, int i, int l, \\
\ind{10}void (*drawpoly) ( int deg, const float *f ) );}
Procedura \texttt{g2h\_DrawJFunctionf} wyci"aga wsp"o"lczynniki wielomianu
(w~bazie Bernsteina odpowiedniego stopnia), kt"ory jest funkcj"a po"l"aczenia
u"rywan"a w~konstrukcji funkcji bazowych. Na ka"rdy podobszar~$\varOmega_i$
dziedziny takich funkcji jest~$16$.

Parametr \texttt{domain} wskazuje rekord dziedziny. Parametr~\texttt{i}
jest numerem obszaru (musi by"c od~$0$ do~$k-1$), parametr \texttt{l} wybiera
funkcj"e po"l"aczenia, kt"ora ma by"c wyprowadzona za pomoc"a procedury
wskazywanej przez parametr~\texttt{drawpoly}.

Warto"s"c parametru~\texttt{l} od~$0$ do~$15$ okre"sla jedn"a z~szesnastu
funkcji po"l"aczenia, a~od~$16$ do~$27$ iloczyn odpowiednich funkcji, u"rywany
w~konstrukcji. Po informacj"e, kt"ora funkcja ma kt"ory numer, odsy"lam
do tekstu "zr"od"lowego procedury.

\vspace{\bigskipamount}
\cprog{%
void g2h\_DrawDiPatchesf ( GHoleDomainf *domain, \\
\ind{10}void (*drawpatch) ( int n, int m, const point2f *cp ) );}
\begin{sloppypar}
Procedura \texttt{g2h\_DrawDiPatchesf} wyci"aga reprezentacje B\'{e}ziera
stopnia~$(9,9)$ p"lat"ow dziedziny. Parametr \texttt{domain} wskazuje
rekord dziedziny, a~parametr \texttt{drawpatch} wskazuje procedur"e, kt"ora
ma by"c wywo"lana $k$~razy w~celu wyprowadzenia kolejnych p"lat"ow.%
\end{sloppypar}

\vspace{\bigskipamount}
\cprog{%
void g2h\_ExtractPartitionf ( GHoleDomainf *domain, \\
\ind{10}int *hole\_k, int *hole\_m, \\
\ind{10}float *partition, float *part\_delta, float *spart\_alpha, \\
\ind{10}float *spart\_malpha, float *spart\_salpha, \\
\ind{10}float *spart\_knot, float *alpha0, \\
\ind{10}boolean *spart\_sgn, boolean *spart\_both );}
Procedura \texttt{g2h\_ExtractPartitionf} wyci"aga informacj"e na temat
podzia"lu k"ata pe"lnego w~punkcie "srodkowym dziedziny~$\varOmega$
podzielonej na podobszary~$\varOmega_i$.

\vspace{\bigskipamount}
\cprog{%
void g2h\_ExtractCentralPointf ( GHoleDomainf *domain, \\
\ind{26}point2f *centp, vector2f *centder );}
\begin{sloppypar}
Procedura \texttt{g2h\_ExtractCentralPointf} wyci"aga punkt "srodkowy dziedziny
i~pochodne pierwszego rz"edu krzywych podzia"lu dziedziny w~punkcie
"srodkowym.%
\end{sloppypar}

Parametr \texttt{domain} wskazuje rekord dziedziny, parametr
\texttt{centp} wskazuje zmienn"a, kt"orej ma by"c przypisany punkt "srodkowy,
a~parametr \texttt{centder} wskazuje tablic"e o~d"lugo"sci~$k$,
do kt"orej zostan"a wpisane wektory pochodnych kolejnych krzywych.

\vspace{\bigskipamount}
\cprog{%
void g2h\_DrawBasAFunctionf ( GHoleDomainf *domain, int fn, \\
\ind{10}void (*drawpatch) ( int n, int m, const point3f *cp ) );}
Procedura \texttt{g2h\_DrawBasAFunctionf} s"lu"ry do uzyskania informacji na
temat funkcji bazowej o~numerze \texttt{fn}$\in\{0,\ldots,n'-1\}$.
Procedura \texttt{*drawpatch} jest wywo"lywana $k$~razy, za ka"rdym
wywo"laniem jej parametry opisuj"a jeden p"lat dziedziny (wsp"o"lrz"edne
$x$, $y$ punkt"ow w~tablicy~\texttt{cp}) i~odpowiedni
p"lat funkcji bazowej (wsp"o"lrz"edne~$z$). Parametry tej procedury
opisuj"a stopie"n, a~w~tablicy~\texttt{cp} s"a punkty kontrolne B\'{e}ziera.

\vspace{\bigskipamount}
\cprog{%
void g2h\_DrawBasBFunctionf ( GHoleDomainf *domain, int fn, \\
\ind{10}void (*drawpatch) ( int n, int m, const point3f *cp ) );}
Procedura \texttt{g2h\_DrawBasBFunctionf} s"lu"ry fo uzyskania informacji na
temat funkcji bazowej o~numerze \texttt{fn}$\in\{n,\ldots,n+m-1\}$.
Procedura \texttt{*drawpatch}, kt"ora s"lu"ry do przekazania tej informacji,
jest wywo"lywana tak samo jak dla procedury \texttt{g2h\_DrawBasAFunctionf}
(i~mo"re to by"c ta sama procedura w~programie).

\vspace{\bigskipamount}
\cprog{%
void g2h\_DrawBasCNetf ( GHoleDomainf *domain, int fn, \\
\ind{10}void (*drawnet) ( int n, int m, const point3f *cp ) );}
Procedura \texttt{g2h\_DrawBasCNetf} s"lu"ry do uzyskania B-sklejanych siatek
kontrolnych reprezentuj"acych funkcj"e $\varphi_i$ dla $i=$\texttt{fn}.

\vspace{\bigskipamount}
\cprog{%
void g2h\_DrawMatricesf ( GHoleDomainf *domain, \\
\ind{10}void (*drawmatrix)(int nfa, int nfb, \\
\ind{29}float *amat, float *bmat) );}
Procedura \texttt{g2h\_DrawMatricesf} s"lu"ry do uzyskania macierzy $A$
i~$B$, wyst"epuj"acych w~uk"ladzie r"owna"n~(\ref{eq:g2h:Ritz:eq}),
napisanym dla bazy podstawowej. Poniewa"r macierz~$A$ jest symetryczna,
jej reprezentacja jest ,,spakowana'', zgodnie z~opisem
w~p.~\ref{sect:packed:sym:array}.

Parametr~\texttt{drawmatrix} jest wska"znikiem procedury, kt"ora zostanie
wywo"lana z~parametrami opisuj"acymi macierze; \texttt{nfa} jest liczb"a
wierszy obu macierzy i~liczb"a kolumn macierzy~$A$. Parametr~\texttt{nfb}
jest liczb"a kolumn macierzy~$B$. Parametry \texttt{amat} i~\texttt{bmat}
s"a tablicami wsp"o"lczynnik"ow.

\vspace{\bigskipamount}
\cprog{%
void g2h\_DrawExtMatricesf ( GHoleDomainf *domain, \\
\ind{11}void (*drawmatrix)(int k, int r, int s, \\
\ind{30}float *Aii, float *Aki, float *Akk, \\
\ind{30}float *Bi, float *Bk) );}
Procedura \texttt{g2h\_DrawMatricesf} s"lu"ry do uzyskania macierzy $A$
i~$B$, wyst"epuj"acych w~uk"ladzie r"owna"n~(\ref{eq:g2h:Ritz:eq}),
napisanym dla bazy podstawowej. Macierz~$A$ jest symetryczna,
o~strukturze blokowej i~jest reprezentowana w~spos"ob opisany
w~p.~\ref{sect:block:sym:array}. Macierz~$B$ jest pe"lna, jest ona
reprezentowana w~postaci blokowej.

Parametr~\texttt{drawmatrix} jest wska"znikiem procedury, kt"ora zostanie
wywo"lana z~parametrami opisuj"acymi macierze; jej parametry \texttt{k},
\texttt{r} i~\texttt{s} opisuj"a liczb"e i~wielko"s"c blok"ow macierzy~$A$.
W~tablicach \texttt{Aii}, \texttt{Aki} i~\texttt{Akk} s"a podane
wsp"o"lczynniki macierzy~$A$, a~w~tablicach \texttt{Bi} i~\texttt{Bk}
wsp"o"lczynniki macierzy~$B$.

\vspace{\bigskipamount}
\cprog{%
int g2h\_DrawBFcpnf ( int hole\_k, unsigned char *bfcpn );}
Procedura \texttt{g2h\_DrawBFcpnf} umieszcza w~tablicy~\texttt{bfcpn}
indeksy tych punkt"ow kontrolnych powierzchni i~dziedziny, kt"ore s"a
istotne dla kszta"ltu brzegu otworu w~powierzchni i~dziedziny,
a~tak"re p"laszczyzny stycznej i~krzywizny powierzchni na brzegu.
Wszystkich punkt"ow kontrolnych dla powierzchni z~otworem $k$-k"atnym jest
$12k+1$, tych istotnych jest~$6k+1$; na rysunku~\ref{fig:domain:cnet}
s"a one zaznaczone czarnymi kropkami. Kolejno"s"c indeks"ow punkt"ow
w~tablicy odpowiada kolejno"sci funkcji bazowych
$\phi_n,\ldots,\phi_{n+m-1}$.

Warto"sci"a procedury jest $6k+1$.

\vspace{\bigskipamount}
\cprog{%
boolean g2h\_GetFinalPatchCurvesf ( GHoleDomainf *domain, \\
\ind{10}int spdimen, const float *hole\_cp, float *acoeff, \\
\ind{10}void (*outcurve) ( int n, const float *cp ) ); \\
boolean g2h\_GetExtFinalPatchCurvesf ( GHoleDomainf *domain, \\
\ind{10}int spdimen, const float *hole\_cp, float *acoeff, \\
\ind{10}void (*outcurve) ( int n, const float *cp ) ); \\
boolean g2h\_GetSplFinalPatchCurvesf ( GHoleDomainf *domain, \\
\ind{10}int spdimen, const float *hole\_cp, float *acoeff, \\
\ind{10}void (*outcurve) ( int n, int lkn, \\
\ind{20}const float *kn, const float *cp ) );}

