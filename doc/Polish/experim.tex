
%/* //////////////////////////////////////////////////// */
%/* This file is a part of the BSTools procedure package */
%/* written by Przemyslaw Kiciak.                        */
%/* //////////////////////////////////////////////////// */

\chapter{Eksperymenty}

Tu s"a opisy r"o"rnych eksperyment"ow, kt"ore jeszcze nie dojrza"ly do
w"l"aczenia do bibliotek. Gdzie"s to musi by"c udokumentowane.

\section{Konwersja p"lata tr"ojk"atnego do tensorowego}

P"lat tr"ojk"atny stopnia $n$ jest reprezentowany przez $(n+1)(n+2)/2$
punkt"ow kontrolnych. P"lat tensorowy stopnia $(n,n)$ jest reprezentowany za
pomoc"a $(n+1)^2$ punkt"ow kontrolnych. Znalezienie reprezentacji p"lata
polega na pomno"reniu odpowiedniej macierzy przej"scia przez wektor
punkt"ow kontrolnych reprezentacji tr"ojk"atnej --- powstaje wektor punkt"ow
kontrolnych reprezentacji tensorowej.

Macierz przej"scia ma wymiary $(n+1)^2\times(n+1)(n+2)/2$ i jest rzadka,
tj.\ wi"ekszo"s"c jej wsp"o"lczynnik"ow r"o"rni si"e od zera. Rozmieszczenie
niezerowych wsp"o"lczynnik"ow dla $n\in\{1,\ldots,9\}$ jest pokazane na
rys.~\ref{fig:trconv:mat}.

\begin{figure}[h]
  \[
    \begin{array}{c}\mbox{\epsfig{file=tbez1.ps}}\end{array}\quad
    \begin{array}{c}\mbox{\epsfig{file=tbez2.ps}}\end{array}\quad
    \begin{array}{c}\mbox{\epsfig{file=tbez3.ps}}\end{array}\quad
    \begin{array}{c}\mbox{\epsfig{file=tbez4.ps}}\end{array}\quad
    \begin{array}{c}\mbox{\epsfig{file=tbez5.ps}}\end{array}\quad
    \begin{array}{c}\mbox{\epsfig{file=tbez6.ps}}\end{array}
  \]
  \[
    \begin{array}{c}\mbox{\epsfig{file=tbez7.ps}}\end{array}\quad
    \begin{array}{c}\mbox{\epsfig{file=tbez8.ps}}\end{array}\quad
    \begin{array}{c}\mbox{\epsfig{file=tbez9.ps}}\end{array}
  \]
  \caption{\label{fig:trconv:mat}Struktura macierzy przej"scia od postaci
    tr"ojk"atnej do tensorowej}
\end{figure}
Dzi"eki temu, "re macierze przej"scia s"a rzadkie, mo"rna je reprezentowa"c
w~ma"lej ilo"sci pami"eci. Reprezentacja u"rywana przez procedur"e opisan"a
ni"rej sk"lada si"e z~tablicy, kt"ora zawiera warto"sci bezwzgl"edne
wsp"o"lczynnik"ow r"o"rne od $0$ i~$1$ (ka"rda liczba wyst"epuje w~niej
tylko raz) i~kodu, kt"ory opisuje mno"renie (a~w~szcze\-g"ol\-no"s\-ci
umo"rliwia odtworzenie macierzy). Procedura konwersji jest interpreterem
tego kodu.

\vspace{\bigskipamount}
\cprog{%
void mbs\_ConvertTBezToTensorf ( int spdimen, int degree, \\
\ind{20}const float *trctlp, float *rpctlp );}

...................................

