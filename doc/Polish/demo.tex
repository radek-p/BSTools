
%/* //////////////////////////////////////////////////// */
%/* This file is a part of the BSTools procedure package */
%/* written by Przemyslaw Kiciak.                        */
%/* //////////////////////////////////////////////////// */

\chapter{Programy demonstracyjne}

Programy demonstracyjne do"l"aczone do bibliotek (w~poddrzewie katalog"ow
\texttt{demo}) maj"a trzy cele. Po pierwsze wy"swietlaj"a ruchome obrazki,
kt"ore mog"a pom"oc w~zaznajamianiu si"e z~krzywymi i~powierzchniami
sklejanymi. Po drugie, umo"rliwiaj"a testowanie procedur z~bibliotek (wiele
b"l"ed"ow zosta"lo wykryte dzi"eki tym programom).
Po trzecie, mog"a stanowi"c "zr"od"lo informacji o~sposobie wykorzystania
bibliotek w~innych zastosowaniach.

Programy te powstawa"ly w~spos"ob niezbyt systematyczny, na zasadzie
dorabiania kolejnych funkcji w~miar"e potrzeb zg"laszanych przez moje
widzimisi"e. Dlatego \textbf{nie s"a one przyk"ladem} szczeg"olnie
eleganckiego programowania. S"a one oparte o~za"lo"renie, "re poza
dzia"laj"acym "srodowiskiem XWindow nie ma w~systemie "radnych
specyficznych bibliotek graficznych ani innych (np.\ Gnome, KDE,
Athena, Motif, OpenGL). Programy te korzystaj"a wi"ec z~interfejsu
z~systemem realizowanego tylko przez bibliotek"e \texttt{Xlib}.
Dzi"eki temu powinno da"c si"e je skompilowa"c i~uruchomi"c na~dowolnym
komputerze wyposa"ronym w~system XWindow. Osoby, kt"orych wygl"ad,
mo"rliwo"sci lub spos"ob obs"lugi tych program"ow nie zadowala,
zach"ecam do zaprojektowania i~napisania w"lasnych program"ow; nie
w"atpi"e, "re b"ed"a znacznie lepsze.


\section{Program \texttt{pokrzyw}}

Program \texttt{pokrzyw} (katalog \texttt{demo/pokrzyw})
wy"swietla i~umo"rliwia kszta"ltowanie (pokrzywienie) p"laskiej
(,,zwyk"lej'' lub wymiernej) krzywej B-sklejanej stopnia od~$1$ do~$8$.
Dwa okna z~prawej strony ekranu (okna programu utworzonego przez system
XWindow) zawieraj"a obraz krzywej (wraz z~laman"a kontroln"a i~innymi
obiektami) oraz ci"agu w"ez"l"ow.

Wi"ekszo"s"c polece"n wydaje si"e za pomoc"a lewego guzika myszy; s"lu"ry on
do ,,chwytania i~trzymania'' punkt"ow kontrolnych i~w"ez"l"ow, a~tak"re do
pstrykania%
\footnote{to jest jedyny uznawany przeze mnie polski odpowiednik
angielskiego czasownika \emph{``to click''}.}
wihajstr"ow%
\footnote{ang.~\emph{``widgets''}.}
(guzik"ow itp.).

Prawy guzik s"lu"ry tylko do wstawiania w"ez"l"ow. Aby przesun"a"c w"eze"l
lub punkt kontrolny nale"ry ustawi"c na nim kursor i~nacisn"a"c \emph{lewy}
guzik, a~nast"epnie przesun"a"c kursor dok"adkolwiek. Aby usun"a"c w"eze"l
nale"ry wskaza"c go kursorem, nacisn"a"c \emph{lewy} guzik i~trzymaj"ac go
nacisn"a"c klawisz \ovalbox{\texttt{R}}.

Lewa strona ekranu zawiera menu, tj.\ kolekcj"e wihajstr"ow, kt"ore s"lu"r"a
do wydawania polece"n. Aby zako"nczy"c dzia"lanie programu nale"ry
pstrykn"a"c (lewym guzikiem myszy) guzik \framebox{\texttt{Quit}}, albo
nacisn"a"c klawisz \ovalbox{\texttt{Q}}.

Naci"sni"ecie klawisza \ovalbox{\texttt{M}} nadaje oknu programu wielko"s"c
maksymaln"a, za"s \ovalbox{\texttt{m}} minimaln"a. Pozosta"le polecenia
interpretowane przez program (poprzez wihajstry) mo"rna odgadn"a"c po jego
uruchomieniu.


\section{Program \texttt{pognij}}

Program \texttt{pognij} (katalog \texttt{demo/pognij}) wy"swietla
i~umo"rliwia kszta"ltowanie (pogi"ecie) krzywej (,,zwyk"lej'' lub wymiernej)
B-sklejanej w~przestrzeni tr"ojwymiarowej. Od programu \texttt{pokrzyw}
r"o"rni si"e on g"l"ownie tym, "re ma cztery okna z~obrazem krzywej
zamiast jednego.

Trzy z~tych okien przedstawiaj"a obraz krzywej w~rzutach prostopad"lych na
p"laszczyzny $xy$, $yz$, $zx$ uk"ladu wsp"o"lrz"ednych, a~czwarte (na~dole
z~prawej strony) obraz krzywej w~rzucie perspektywicznym. Pierwsze trzy okna
umo"rliwiaj"a zmienianie punkt"ow kontrolnych krzywej; mo"rna w~nich
,,przesuwa"c'' punkty kontrolne (trzymaj"ac naci"sni"ety lewy guzik myszy).
W~oknie czwartym lewy guzik s"lu"ry do zmieniania po"lo"renia obserwatora,
a~prawy guzik umo"rliwia ,,zmienianie ogniskowej'' kamery u"rytej do
otrzymania obrazu krzywej w~tym oknie (nale"ry go nacisn"a"c i~przesun"a"c
kursor w~g"or"e lub w~d"o"l).

Naci"sni"ecie klawisza \ovalbox{\texttt{R}} w~chwili, gdy kursor jest
w~oknie rzutu perspektywicznego powoduje przywr"ocenie pocz"atkowego
po"lo"renia obserwatora.

Wielko"s"c czterech okien z~obrazami krzywej mo"rna zmienia"c. W~tym celu
nale"ry ustawi"c kursor mi"edzy tymi oknami (wy"swietlany kszta"lt kursora
zmieni si"e), a~nast"epnie nacisn"a"c lewy guzik i~przesun"a"c kursor.
Naci"sni"ecie klawisza \ovalbox{\texttt{R}} w~chwili gdy kursor znajduje
si"e mi"edzy tymi oknami powoduje nadanie im jednakowych wielko"sci.

Pozosta"le elementy obs"lugi tego programu s"a takie same jak programu
\texttt{pokrzyw}.


\section{Program \texttt{pomnij}}

\section{Program \texttt{polep}}

\begin{sloppypar}
Program \texttt{polep} s"lu"ry do demonstrowania procedury wype"lniania
wielok"atnego otwo\-ru w~powierzchni sklejanej (z"lo"ronej z~p"lat"ow
wielomianowych stopnia $(3,3)$). Procedura ta jest opisana szczeg"o"lowo
w~ksi"a"rce \emph{Podstawy modelowania krzywych i~powierzchni}. Cztery okna,
w~kt"orych program wy"swietla obrazy powierzchni, s"a obs"lugiwane w~spos"ob
identyczny jak cztery okna z~obrazami krzywej w~programie
\texttt{pognij}.
\end{sloppypar}

Punkty kontrolne powierzchni mo"rna zmienia"c za pomoc"a myszy, w~oknach
z~rzutami r"ownoleg"lymi. Opr"ocz tego program zawiera generator
,,gotowych'' danych, obs"lugiwany przez trzy suwaki w~g"ornej cz"e"sci menu.


