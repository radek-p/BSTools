
%/* //////////////////////////////////////////////////// */
%/* This file is a part of the BSTools procedure package */
%/* written by Przemyslaw Kiciak.                        */
%/* //////////////////////////////////////////////////// */

\chapter{Projekty}

Ten rozdzia"l dokumentacji jest po"swi"econy procedurom ,,wysokopoziomowym''
rozwi"azuj"acym r"o"rne, raczej skomplikowane zadania, zrealizowanym przy
u"ryciu bibliotek opisanych wcze"sniej. Obecnie w~pakiecie jest jedna
taka procedura, konstruuj"aca powierzchnie g"ladko wype"lniaj"ace
wielok"atne otwory w~powierzchniach z"lo"ronych z~p"lat"ow prostok"atnych.


\section{Wype"lnianie wielok"atnych otwor"ow}

W~programie demonstracyjnym \texttt{polep} jest procedura wype"lniania
wielok"atnego otworu w~uog"olnionej powierzchni B-sklejanej trzeciego
stopnia, p"latami B\'{e}ziera stopnia $(5,5)$, z~zachowaniem
ci"ag"lo"sci~$G^1$ po"l"acze"n tych p"lat"ow mi"edzy sob"a i~z~p"latami
otaczaj"acymi otw"or. Dok"ladny opis teoretyczny tej procedury znajduje
si"e w~ksi"a"rce \emph{Podstawy modelowania krzywych i~powierzchni}.
Konstrukcja realizowana przez t"e procedur"e jest znacznie prostsza
i~mniej og"olna ni"r konstrukcje realizowane za pomoc"a procedur
znajduj"acych si"e w~bibliotekach \texttt{libg1hole} i~\texttt{libg2hole};
by"la ona opracowana znacznie wcze"sniej i~zebrane przy tej okazji
do"swiadczenia pomog"ly przy opracowaniu konstrukcji zaimplementowanych
w~tych bibliotekach.

W~tym miejscu jest opis sposobu reprezentowania danych dla tej procedury
i~jej parametr"ow.

Kod "zr"od"lowy procedury w~wersji pojedynczej precyzji jest w~pliku
\texttt{g1holef.c}, a~odpowiedni plik nag"l"owkowy to \texttt{g1holef.h}.
Odpowiednie pliki dla podw"ojnej precyzji maj"a nazwy odpowiednio
\texttt{g1holed.c} i~\texttt{g1holed.h}.

\vspace{\bigskipamount}
\cprog{%
boolean FillG1Holef ( int hole\_k, point3f*(*GetBezp)(int i, int j), \\
\ind{22}float beta1, float beta2, \\
\ind{22}point3f *hpcp );}
Procedura \texttt{FillG1Holef} konstruuje powierzchni"e wype"lniaj"ac"a
$k$-k"atny otw"or w~powierzchni. Parametry tej procedury s"a nast"epuj"ace:

Parametr \texttt{hole\_k} okre"sla liczb"e $k$ wierzcho"lk"ow otworu.
Powinna to by"c liczba~$3$, $5$, $6$, $7$ lub~$8$.

Parametr \texttt{GetBezp} jest wska"znikiem procedury kt"or"a
\texttt{FillG1Holef} wywo"luje w~celu otrzymania punkt"ow kontrolnych
p"lat"ow B\'{e}ziera stopnia $(3,3)$ otaczaj"acych otw"or. Procedura ta ma
zwr"oci"c (jako warto"s"c) wska"znik tablicy, w~kt"orej znajduj"a si"e
punkty kontrolne tych p"lat"ow.

P"laty otaczaj"ace otw"or s"a okre"slane przez pary liczb $(i,j)$ zgodnie ze
schematem na rysunku~\ref{fig:g1:patch:num:1}; zmienna~$i$
(parametr~\texttt{i}) przyjmuje warto"sci od $0$ do $k-1$, zmienna~$j$
(parametr~\texttt{j}) warto"sci $1$ lub $2$.
Na rysunku tym jest pokazana r"ownie"r kolejno"s"c, w~jakiej nale"ry
umie"sci"c punkty kontrolne tych p"lat"ow w~tablicy.
Dla ka"rdego p"lata wystarczy poda"c w~tablicy tylko~$8$ punkt"ow,
kt"orych numery s"a widoczne na rysunku.
\begin{figure}[ht]
  \centerline{\epsfig{file=g1patches1.ps}}
  \caption{\label{fig:g1:patch:num:1}Schemat numeracji p"lat"ow
    otaczaj"acych otw"or}
  \centerline{i kolejno"s"c punkt"ow kontrolnych w tablicach}
\end{figure}

P"laty otaczaj"ace otw"or musz"a spe"lnia"c podane ni"rej warunki
zgodno"sci naro"rnik"ow, pochodnych cz"astkowych i~pochodnych mieszanych.
$m$-ty punkt kontrolny p"lata $(i,j)$ jest oznaczony symbolem
$\bm{p}^{(i,j)}_m$. Dodatkowo $l=i+1\bmod k$.
\begin{itemize}
  \item Warunki zgodno"sci naro"rnik"ow:
    \begin{align*}
      &{}\bm{p}^{(i,1)}_0=\bm{p}^{(i,2)}_3 \quad\mbox{oraz}\quad
      \bm{p}^{(i,2)}_0=\bm{p}^{(l,1)}_3.
    \end{align*}
  \item Warunki zgodno"sci pochodnych cz"astkowych:
    \begin{align*}
      &{}\bm{p}^{(i,1)}_0-\bm{p}^{(i,1)}_1=\bm{p}^{(i,2)}_2-\bm{p}^{(i,2)}_3, \\
      &{}\bm{p}^{(i,2)}_0-\bm{p}^{(i,2)}_1=\bm{p}^{(l,1)}_7-\bm{p}^{(l,1)}_3, \\
      &{}\bm{p}^{(i,2)}_0-\bm{p}^{(i,2)}_4=\bm{p}^{(l,1)}_2-\bm{p}^{(l,1)}_3.
    \end{align*}
  \item Warunki zgodno"sci pochodnych mieszanych:
    \begin{align*}
      &{}\bm{p}^{(i,1)}_4-\bm{p}^{(i,1)}_5=\bm{p}^{(i,2)}_6-\bm{p}^{(i,2)}_7, \\
      &{}\bm{p}^{(i,2)}_0-\bm{p}^{(i,2)}_1-\bm{p}^{(i,2)}_4+\bm{p}^{(i,2)}_5=
      \bm{p}^{(l,1)}_6-\bm{p}^{(l,1)}_7-\bm{p}^{(l,1)}_2+\bm{p}^{(l,1)}_3.
    \end{align*}
\end{itemize}

Parametry \texttt{beta1} i~\texttt{beta2} s"a czynnikami, przez kt"ore
procedura \texttt{FillG1Holef} mno"ry pewne wektory konstruowane w~trakcie
oblicze"n. W~zasadzie ich warto"s"c powinna by"c r"owna~$1$, ale mo"rna
poda"c inn"a w~celu skorygowania kszta"ltu (je"sli warto"s"c $1$ daje
nieodpowiedni efekt).

Parametr \texttt{hpcp} jest wska"znikiem tablicy, do kt"orej procedura
wstawia punkty kontrolne p"lat"ow B\'{e}ziera stopnia $(5,5)$, z~kt"orych
sk"lada si"e powierzchnia wype"lniaj"aca otw"or.

Warto"sci"a procedury jest \texttt{true} je"sli konstrukcja powierzchni
wype"lniaj"acej zako"nczy"la si"e sukcesem, albo \texttt{false} w~przeciwnym
razie.

\vspace{\bigskipamount}
\cprog{%
extern void (*G1OutCentralPointf)( point3f *p ); \\
extern void (*G1OutAuxCurvesf)( int ncurves, int degree, \\
\ind{32}const point3f *accp, float t ); \\
extern void (*G1OutStarCurvesf)( int ncurves, int degree, \\
\ind{33}const point3f *sccp ); \\
extern void (*G1OutAuxPatchesf)( int npatches, int degu, int degv, \\
\ind{33}const point3f *apcp );}
\begin{sloppypar}
Powy"rsze zmienne umo"rliwiaj"a ,,podczepienie'' procedur wyprowadzaj"acych
cz"e"s\-cio\-we wyniki konstrukcji powierzchni wype"lniaj"acej. Ich warto"s"c
pocz"atkowa jest r"owna \texttt{NULL}. Przypisanie dowolnej z~tych zmiennych
odpowiedniej procedury przed wywo"laniem \texttt{FillG1Hole} powoduje
wywo"lanie tej procedury; mo"re ona wyprowadzi"c dane lub narysowa"c na ich
podstawie obrazek.
\end{sloppypar}

Procedura wskazywana przez zmienn"a \texttt{G1OutCentralPointf} otrzymuje
jako parametr wska"znik punktu ,,"srodkowego'' (tj.\ wsp"olnego naro"rnika
$k$~p"lat"ow wype"lniaj"acych otw"or). Procedura ta ma prawo zmieni"c ten
punkt (tj.\ przypisa"c mu nowe wsp"o"lrz"edne), poniewa"r taka ingerencja
w~konstrukcj"e jest dopuszczalna i~w~pewnych sytuacjach potrzebna.

Pozosta"le procedury nie mog"a zmienia"c warto"sci zmiennych wskazywanych
przez parametry, z~kt"orymi zosta"ly wywo"lane. Procedura wskazywana przez
zmienn"a \texttt{G1OutAuxCurvesf} jest wywo"lywana z~parametrami opisuj"acymi
tzw.\ krzywe pomocnicze w~konstrukcji --- krzywe B\'{e}ziera stopnia
\texttt{degree} (w~tej implementacji krzywe te s"a stopnia~$3$). W~ka"rdym
wywo"laniu jest podawana jedna krzywa.

\begin{sloppypar}
Procedura wskazywana przez zmienn"a \texttt{G1OutStarCurvesf} jest
wywo"lywana z~parametrami opisuj"acymi krzywe brzegowe p"lat"ow
wype"lniaj"acych otw"or (konstrukcja tych krzywych jest jednym z~pierwszych
etap"ow konstrukcji). Parametry procedury opisuj"a reprezentacj"e
B\'{e}ziera stopnia~$3$ tych krzywych. W~ka"rdym wywo"laniu jest podawana
jedna krzywa.
\end{sloppypar}

\begin{sloppypar}
Procedura wskazywana przez zmienn"a \texttt{G1OutAuxPatchesf} jest
wywo"lywana z~parametrami opisuj"acymi tzw.\ p"laty pomocnicze, kt"ore
okre"slaj"a p"laszczyzny styczne do konstruowanej powierzchni we wszystkich
punktach krzywej brzegowej. P"laty te maj"a stopie"n $(3,1)$
(parametry \texttt{ndegu} i~\texttt{ndegv} maj"a takie warto"sci).
Parametr \texttt{apcp} wskazuje tablic"e punkt"ow kontrolnych. W~ka"rdym
wywo"laniu przekazywana jest reprezentacja jednego p"lata pomocniczego.
\end{sloppypar}%
\begin{figure}[ht]
  \centerline{\epsfig{file=g1patches2.ps}}
  \caption{\label{fig:g1:patch:num:2}Powierzchnia z~otworem wype"lnionym
    przez procedur"e \texttt{FillG1Holef}}
\end{figure}

\vspace{\medskipamount}
\noindent\textbf{Uwaga:}
Obecna wersja procedury nie zawiera obs"lugi sytuacji wyj"atkowych, kt"ore
umo"rliwiaj"a ,,bezpieczny powr"ot'' w~razie niepowodzenia konstrukcji.
Opracowanie takiej obs"lugi jest konieczne przed wbudowaniem procedury do
systemu ,,produkcyjnego'', tj.\ nadaj"acego si"e do u"rycia w~celu
projektowania przemys"lowego. Co wi"ecej, wi"ekszo"s"c procedur
w~bibliotekach opisanych wcze"sniej nie ma obs"lugi b"l"ed"ow (w~razie
wyst"apienia b"l"edu jest wywo"lywana procedura \texttt{exit}), a~zatem
obecnie ca"ly pakiet nadaje si"e tylko do cel"ow eksperymentalnych
(i~zreszt"a po to go zacz"a"lem pisa"c).

