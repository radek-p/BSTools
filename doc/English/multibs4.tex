
%/* //////////////////////////////////////////////////// */
%/* This file is a part of the BSTools procedure package */
%/* written by Przemyslaw Kiciak.                        */
%/* //////////////////////////////////////////////////// */

\section{B\'{e}zier curves and patches subdivision}

B\'{e}zier curves and patches may be divided into pieces with the de~Casteljau
algorithm, and there are two cases implemented separately.
The first case is bisection of the domain (e.g.\ the division of the
interval $[0.25,0.5]$ into the parts $[0.25,0.375]$
and~$[0.375,0.5]$, of equal length), and the algorithm computes only
averages of pairs of numbers. The second case is general; the domain
may be divided into intervals or rectangles at any place, which includes
also the possibility of extrapolation.

\vspace{\bigskipamount}
\cprog{%
void mbs\_multiBisectBezCurvesf ( int degree, int ncurves, \\
\ind{33}int spdimen, int pitch, \\
\ind{33}float *ctlp, float *ctlq );}
The procedure \texttt{mbs\_multiBisectBezCurvesf} divides \texttt{ncurves}
B\'{e}zier curves of degree \texttt{degree} in the space of dimension
\texttt{spdimen}. The domain (an interval $[a,b]$) is divided into two
parts of equal length.

\begin{sloppypar}
The control points are given in the array \texttt{ctlp}, whose pitch is
\texttt{pitch}. Upon return this array contains the control points of the second
arc of each curve (the one corresponding to the interval $[\frac{a+b}{2},b]$).
In the array $\texttt{ctlq}$ the procedure stores the control points of the
arcs corresponding to the interval $[a,\frac{a+b}{2}]$. The pitch of
this array is the same as the pitch of \texttt{ctlp} (i.e.\ equal to
the value of the parameter \texttt{pitch}).
\end{sloppypar}

\vspace{\bigskipamount}
\cprog{%
\#define mbs\_BisectBC1f(degree,ctlp,ctlq) \bsl \\
\ind{2}mbs\_multiBisectBezCurvesf(degree,1,1,0,ctlp,ctlq) \\
\#define mbs\_BisectBC2f(degree,ctlp,ctlq) \bsl \\
\ind{2}mbs\_multiBisectBezCurvesf(degree,1,2,0,(float*)ctlp,(float*)ctlq) \\
\#define mbs\_BisectBC3f(degree,ctlp,ctlq) ... \\
\#define mbs\_BisectBC4f(degree,ctlp,ctlq) ...}
The above macros call \texttt{mbs\_multiBisectBezCurvesf} in order to
bisect one polynomial or B\'{e}zier curve in the space of dimension $2$, $3$
or $4$, i.e.\ to compute its local representations corresponding to the
halves of the initial interval.

\vspace{\bigskipamount}
\ucprog{%
\#define mbs\_BisectBP1uf(degreeu,degreev,ctlp,ctlq) \bsl \\
\ind{2}mbs\_multiBisectBezCurvesf(degreeu,1,(degreev+1),0,ctlp,ctlq) \\
\#define mbs\_BisectBP1vf(degreeu,degreev,ctlp,ctlq) \bsl \\
\ind{2}mbs\_multiBisectBezCurvesf(degreev,degreeu+1,1,degreev+1, \bsl \\
\ind{2}ctlp,ctlq) \\
\#define mbs\_BisectBP2uf(degreeu,degreev,ctlp,ctlq) \bsl \\
\ind{2}mbs\_multiBisectBezCurvesf(degreeu,1,2*(degreev+1),0, \bsl \\
\ind{2}(float*)ctlp,(float*)ctlq)}

\dcprog{%
\#define mbs\_BisectBP2vf(degreeu,degreev,ctlp,ctlq) \bsl \\
\ind{2}mbs\_multiBisectBezCurvesf(degreev,degreeu+1,2,2*(degreev+1), \bsl \\
\ind{2}(float*)ctlp,(float*)ctlq) \\
\#define mbs\_BisectBP3uf(degreeu,degreev,ctlp,ctlq) ... \\
\#define mbs\_BisectBP3vf(degreeu,degreev,ctlp,ctlq) ... \\
\#define mbs\_BisectBP4uf(degreeu,degreev,ctlp,ctlq) ... \\
\#define mbs\_BisectBP4vf(degreeu,degreev,ctlp,ctlq) ...}
The above macros call \texttt{mbs\_multiBisectBezCurvesf} in order to
divide the rectangular domain of one bivariate polynomial (given in
a~tensor product Bernstein basis) or a~B\'{e}zier patch (of dimension
$2$, $3$ or $4$) into two equal parts and to compute the local representations
of the polynomial or the patch related with the two parts.
The macros with the letter \texttt{u} in the identifier bisect the
interval of the first parameter, and the macros with the letter
\texttt{v} bisect the interval of the second parameter of the polynomial
or the patch.

\vspace{\bigskipamount}
\cprog{%
void mbs\_multiDivideBezCurvesf ( int degree, int ncurves, \\
\ind{33}int spdimen, int pitch, float t, \\
\ind{33}float *ctlp, float *ctlq );}
The procedure \texttt{mbs\_multiDivideBezCurvesf} divides
\texttt{ncurves} B\'{e}zier curves of degree \texttt{degree} in the space
of dimension \texttt{spdimen}. The domain (the interval $[0,1]$) is divided at
the point $t$, specified by the parameter \texttt{t}, and if
$t\notin[0,1]$, then the division is in fact an extrapolation.

The control points are given in the array \texttt{ctlp}, whose pitch is
\texttt{pitch}. Upon return this array contains the control points of the
second arc of each curve (related with the interval $[t,1]$).
The array $\texttt{ctlq}$ is filled with the control points of the first
arcs, related with the interval $[0,t]$. The pitch of this array is the
same as the pitch of \texttt{ctlp} (it is equal to the value of
the parameter \texttt{pitch}).


\vspace{\bigskipamount}
\cprog{%
\#define mbs\_DivideBC1f(degree,t,ctlp,ctlq) \bsl \\
\ind{2}mbs\_multiDivideBezCurvesf(degree,1,1,0,t,ctlp,ctlq) \\
\#define mbs\_DivideBC2f(degree,t,ctlp,ctlq) \bsl \\
\ind{2}mbs\_multiDivideBezCurvesf(degree,1,2,0,t, \bsl \\
\ind{2}(float*)ctlp,(float*)ctlq) \\
\#define mbs\_DivideBC3f(degree,t,ctlp,ctlq) ... \\
\#define mbs\_DivideBC4f(degree,t,ctlp,ctlq) ...}
\begin{sloppypar}
The above macros call \texttt{mbs\_multiDivideBezCurvesf} in order to
divide the domain of one polynomial or one B\'{e}zier curve
(two-, three- or four-dimensional) at the proportion $t:1-t$, where $t$~is
the value of the parameter \texttt{t}, and to find the local representations
of the polynomial or the curve.
\end{sloppypar}

\vspace{\bigskipamount}
\cprog{%
\#define mbs\_DivideBP1uf(degreeu,degreev,u,ctlp,ctlq) \bsl \\
\ind{2}mbs\_multiDivideBezCurvesf(degreeu,1,(degreev)+1,0,u,ctlp,ctlq) \\
\#define mbs\_DivideBP1vf(degreeu,degreev,v,ctlp,ctlq) \bsl \\
\ind{2}mbs\_multiDivideBezCurvesf(degreev,(degreeu)+1,1,degreev+1,v, \bsl \\
\ind{4}ctlp,ctlq) \\
\#define mbs\_DivideBP2uf(degreeu,degreev,u,ctlp,ctlq) \bsl \\
\ind{2}mbs\_multiDivideBezCurvesf(degreeu,1,2*(degreev)+1,0,u, \bsl \\
\ind{4}(float*)ctlp,(float*)ctlq) \\
\#define mbs\_DivideBP2vf(degreeu,degreev,v,ctlp,ctlq) \bsl \\
\ind{1}mbs\_multiDivideBezCurvesf(degreev,(degreeu)+1,2,2*(degreev)+1,v, \bsl \\
\ind{4}(float*)ctlp,(float*)ctlq) \\
\#define mbs\_DivideBP3uf(degreeu,degreev,u,ctlp,ctlq) ... \\
\#define mbs\_DivideBP3vf(degreeu,degreev,v,ctlp,ctlq) ... \\
\#define mbs\_DivideBP4uf(degreeu,degreev,u,ctlp,ctlq) ... \\
\#define mbs\_DivideBP4vf(degreeu,degreev,v,ctlp,ctlq) ...}
The macros calling \texttt{mbs\_multiDivideBezCurvesf} in order to divide
a~bivariate polynomial or a~B\'{e}zier patch in the space of dimension
$2$, $3$ or $4$ --- in the ``$u$'' direction (i.e.\ the interval
of the first patch parameter is to be divided) or in the ``$v$''
direction (the interval of the second parameter is to be divided).
The number $u$ or $v$, which is the value of the parameter
\texttt{u} or \texttt{v} is the point of division of the interval
$[0,1]$ (the interval of the parameter is divided in the proportion
e.g.\ $u:1-u$).


\newpage
\section{Degree elevation}

Degree elevation is a~computation of a~new representation of
curves in the Bernstein or B-spline basis of degree greater by
a~specified amount.

\subsection{Degree elevation of B\'{e}zier curves and patches}

%\vspace{\bigskipamount}
\cprog{%
void mbs\_multiBCDegElevf ( int ncurves, int spdimen, \\
\ind{27}int inpitch, int indegree, \\
\ind{27}const float *inctlpoints, \\
\ind{27}int deltadeg, \\
\ind{27}int outpitch, int *outdegree, \\
\ind{27}float *outctlpoints );}
\begin{sloppypar}
\hspace*{\parindent}The procedure \texttt{mbs\_multiBCDegElevf} performs
the degree elevation of \texttt{ncurves} B\'{e}zier curves of degree
\texttt{indegree} in the space of dimension \texttt{spdimen}, to the degree
$\mathord{\mbox{\texttt{indegree}}}+\mathord{\mbox{\texttt{deltadeg}}}$
(the resulting degree is assigned to the parameter \texttt{*outdegree}).

The control polygons of the curves are given in the array \texttt{inctlpoints},
whose pitch is \texttt{inpitch}. The computed control polygons are stored
by the procedure in the array \texttt{outctlpoints}, with the pitch
\texttt{outpitch}.
\end{sloppypar}

\vspace{\bigskipamount}
\cprog{%
\#define mbs\_BCDegElevC1f(indegree,incoeff,deltadeg, \bsl \\
\ind{4}outdegree,outcoeff) \bsl \\
\ind{2}mbs\_multiBCDegElevf ( 1, 1, 0, indegree, incoeff, deltadeg, \bsl \\
\ind{4}0, outdegree, outcoeff ) \\
\#define mbs\_BCDegElevC2f(indegree,inctlpoints,deltadeg, \bsl \\
\ind{4}outdegree,outctlpoints) \bsl \\
\ind{2}mbs\_multiBCDegElevf ( 1, 2, 0, indegree, (float*)inctlpoints, \bsl \\
\ind{4}deltadeg, 0, outdegree, (float*)outctlpoints ) \\
\#define mbs\_BCDegElevC3f(indegree,inctlpoints,deltadeg, \bsl \\
\ind{4}outdegree,outctlpoints) ... \\
\#define mbs\_BCDegElevC4f(indegree,inctlpoints,deltadeg, \bsl \\
\ind{4}outdegree,outctlpoints) ...}
\begin{sloppypar}
The four macros above may be used for degree elevation of one polynomial
(given by the coefficients in the Bernstein basis) or a~B\'{e}zier curve
in the space of dimension $2$, $3$ or~$4$. The parameters of the macros
are described with the procedure \texttt{mbs\_multiBCDegElevf}.
\end{sloppypar}

\vspace{\bigskipamount}
\cprog{%
void mbs\_BCDegElevPf ( int spdimen, \\
\ind{23}int indegreeu, int indegreev, \\
\ind{23}const float *inctlp, \\
\ind{23}int deltadegu, int deltadegv, \\
\ind{23}int *outdegreeu, int *outdegreev, \\
\ind{23}float *outctlp );}
The procedure \texttt{mbs\_BCDegElevPf} performs the degree elevation of
a~B\'{e}zier patch in the space of dimension \texttt{spdimen}, with respect to
one or both parameters.

The parameters \texttt{indegu} and \texttt{indegv} specify the degree of the
initial patch representation, with respect to its two parameters.
The array \texttt{inctlp} contains the control points of
the patch, organized in subsequent columns. The array is packed, i.e.\ without
unused areas between consecutive columns, hence the pitch is equal to the
length of the column representation:
$(\mathord{\mbox{\texttt{indegu}}}+1)*\mathord{\mbox{\texttt{spdimen}}}$.
The array \texttt{outctlp}, in which the procedure stores the control points of
the resulting representation, is packed in a~similar way.

The parameters \texttt{deltadegu} and \texttt{deltadegv} must be nonnegative.
They specify the numbers, by which the degrees of the patch are to increase.
The final degrees (sums of the initial degrees and the increments)
are assigned to the parameters \texttt{*outdegu} and~\texttt{outdegv}.


\vspace{\bigskipamount}
\cprog{%
\#define mbs\_BCDegElevP1f(indegreeu,indegreev,incoeff, \bsl \\
\ind{4}deltadegu,deltadegv,outdegreeu,outdegreev,outcoeff) \bsl \\
\ind{2}mbs\_BCDegElevPf ( 1, indegreeu, indegreev, incoeff, \bsl \\
\ind{4}deltadegu, deltadegv, outdegreeu, outdegreev, outcoeff ) \\
\#define mbs\_BCDegElevP2f(indegreeu,indegreev,inctlp, \bsl \\
\ind{4}deltadegu,deltadegv,outdegreeu,outdegreev,outctlp) \bsl \\
\ind{2}mbs\_BCDegElevPf ( 2, indegreeu, indegreev, (float*)inctlp, \bsl \\
\ind{4}deltadegu, deltadegv, outdegreeu, outdegreev, (float*)outctlp ) \\
\#define mbs\_BCDegElevP3f(indegreeu,indegreev,inctlp, \bsl \\
\ind{4}deltadegu,deltadegv,outdegreeu,outdegreev,outctlp) \ldots \\
\#define mbs\_BCDegElevP4f(indegreeu,indegreev,inctlp, \bsl \\
\ind{4}deltadegu,deltadegv,outdegreeu,outdegreev,outctlp) \ldots}
The above macros may be used for degree elevation of a~bivariate polynomial
or a~B\'{e}zier patch in the space of dimension $2$, $3$ or $4$.
They call the procedure \texttt{mbs\_BCDegElevPf}.


\subsection{Degree elevation of B-spline curves and patches}

%\vspace{\bigskipamount}
\cprog{%
void mbs\_multiBSDegElevf ( int ncurves, int spdimen, \\
\ind{27}int indegree, int inlastknot, \\
\ind{27}const float *inknots, \\
\ind{27}int inpitch, const float *inctlpoints, \\
\ind{27}int deltadeg, \\
\ind{27}int *outdegree, int *outlastknot, \\
\ind{27}float *outknots, \\
\ind{27}int outpitch, float *outctlpoints, \\
\ind{27}boolean freeend );}
\begin{sloppypar}
\hspace*{\parindent}
The procedure \texttt{mbs\_multiBSDegElevf} performs the degree elevation
of \texttt{ncurves} B-spline curves of degree \texttt{indegree}
in the space of dimension \texttt{spdimen}, up to the degree
$\mathord{\mbox{\texttt{indegree}}}+\mathord{\mbox{\texttt{deltadeg}}}$,
which is assigned to the parameter \texttt{*outdegree}.
\end{sloppypar}

The procedure is able to process curves with clamped or free ends.
If the value of the parameter \texttt{freeend} is \texttt{false}, then
the resulting representation of the curves has clamped ends,
with the only external knots being the extremal knots
(see Section~\ref{ssect:BSC}). If the value of
\texttt{freeend} is \texttt{true}, then the resulting representation
has free ends. The knot sequence of this representation is obtained by
increasing the multiplicities of all knots by the value of the parameter
\texttt{deltadeg}, and then by rejecting knots from the beginning and end
of the sequence so as to obtain $\hat{u}_n<\hat{u}_{n+1}$
and~$\hat{u}_{N-n}>\hat{u}_{N-n-1}$ ($n$ here denotes the degree of
the result representation, and $N$~is the number of its last knot).

\begin{sloppypar}
The method of degree elevation does not depend on the value of
\texttt{freeend}. The representation obtained after the degree elevation
if one with clamped ends. For \texttt{freeend=true} the procedure calls
\texttt{mbs\_multiBSChangeLeftKnotsf} and
\texttt{mbs\_multiBSChangeRightKnotsf}. This involves additional rounding errors.
The procedure \texttt{mbs\_multiBSDegElevf} may be used for degree elevation of
a~closed curve; the parameter \texttt{freeend} should then be \texttt{true},
to obtain a~closed representation of higher degree. In such a~representation
the appropriate number of initial control points coincide with the final
control points \emph{up to the rounding errors}.%
\end{sloppypar}

The initial representation is given in the arrays \texttt{inknots}
(knots, their number is $\mathord{\mbox{\texttt{inlastknot}}}+1$) and
\texttt{inctlpoints} (control points the pitch of this array is
\texttt{inpitch}).

The resulting representation is stored in the arrays \texttt{outknots} (knots,
their number is assigned to~\texttt{*outlastknot})
and \texttt{outctlpoints} (control poitns, the pitch of this array is
specified by \texttt{outpitch}).

It is necessary to provide the arrays long enough to accomodate the result.
The rule is as follows: if the last knot of the representation of degree~$n$
has the number~$N$, the number of polynomial arcs of the curve is~$l$
(it may be found using the procedure \texttt{mbs\_NumKnotIntervalsf}),
and the resulting representation has the degree~$n'$, then the last knot
of this representation has the number
\begin{align*}
  N' = N+(l+1-d_0-d_1)(n'-n),
\end{align*}
where $d_0$ and $d_1$ are numbers such that
\begin{align*}
  u_n=\cdots=u_{n+d_0}<u_{n+d_0+1}\quad\mbox{and}\quad
  u_{N-n-d_1-1}<u_{N-n-d_1}=\cdots=u_{N-n}.
\end{align*}
The number of control points of the new curve representation is is $N'-n'$.
For $m$~curves in the space of dimension~$d$ it is necessary to allocate
an array of length $N'+1$ floating point numbers for the knots and an array
of length $(N'-n)md$ floating point numbers for the control points.


\vspace{\bigskipamount}
\cprog{%
\#define mbs\_BSDegElevC1f(indegree,inlastknot,inknots,incoeff, \bsl \\
\ind{4}deltadeg,outdegree,outlastknot,outknots,outcoeff,freeend) \bsl \\
\ind{2}mbs\_multiBSDegElevf(1,1,indegree,inlastknot,inknots,0,incoeff, \bsl \\
\ind{4}deltadeg,outdegree,outlastknot,outknots,0,outcoeff,freeend) \\
\#define mbs\_BSDegElevC2f(indegree,inlastknot,inknots,inctlpoints, \bsl \\
\ind{4}deltadeg,outdegree,outlastknot,outknots,outctlpoints,freeend) \bsl \\
\ind{2}mbs\_multiBSDegElevf(1,2,indegree,inlastknot,inknots, \bsl \\
\ind{4}0,(float*)inctlpoints,deltadeg, \bsl \\
\ind{4}outdegree,outlastknot,outknots,0,(float*)outctlpoints,freeend) \\
\#define mbs\_BSDegElevC3f(indegree,inlastknot,inknots,inctlpoints, \bsl \\
\ind{2}deltadeg,outdegree,outlastknot,outknots,outctlpoints,freeend) ... \\
\#define mbs\_BSDegElevC4f(indegree,inlastknot,inknots,inctlpoints, \bsl \\
\ind{2}deltadeg,outdegree,outlastknot,outknots,outctlpoints,freeend) ...}
Four macros, which call \texttt{mbs\_multiBSDegElevf} for degree elevation
of one scalar spline function of B-spline curve in the space of dimension
$2$, $3$ or $4$.%
\begin{figure}[ht]
  \centerline{\epsfig{file=degel.ps}}
  \caption{Degree elevation of a~planar B-spline curve from $3$ to $4$.}
\end{figure}

\newpage
%\vspace{\bigskipamount}
\cprog{%
void mbs\_multiBSDegElevClosedf ( int ncurves, int spdimen, \\
\ind{14}int indegree, int inlastknot, const float *inknots, \\
\ind{14}int inpitch, const float *inctlpoints, \\
\ind{14}int deltadeg, \\
\ind{14}int *outdegree, int *outlastknot, \\
\ind{14}float *outknots, int outpitch, float *outctlpoints );}
\begin{sloppypar}
The procedure \texttt{mbs\_multiBSDegElevClosedf} performs degree elevation
of closed B-spline curves.%
\end{sloppypar}

The parameter \texttt{ncurves} specifies the number of curves,
the value of the parameter \texttt{spdimen} is the dimension of the space
in which they are located.

\begin{sloppypar}
The parameters \texttt{indegree},
\texttt{inlastknot}, \texttt{inknots}, \texttt{inpitch}, \texttt{inctlpoints}
describe the input data --- degree~$n$, number~$N$ of the last knot,
the knot sequence~$u_0,\ldots,u_N$ the pitch of the array with the control
points and that array respectively. The parameter \texttt{deltadeg} (whose value
must be nonnegative) specifies the degree increment.%
\end{sloppypar}

The parameters \texttt{*outdegree} and~\texttt{*outlastknot} are variables,
to which the assigns the final representation degree and the number
of the last knot of this representation.
The final knot sequence is stored in the array \texttt{outknots}.
The parameter~\texttt{outpitch} specifies the pitch of the
array~\texttt{outctlpoints}, used to store the control points of the
final representation.

The number of the last knot of the resulting representation of the closed
curve is
\begin{align*}
  N' = N+(l+1+r-d_0-d_1)(n'-n),
\end{align*}
where~$r$ is the multiplicity of the knot~$u_n$ in the given representation
of degree~$n$ (without counting~$u_0$), and~$d_0$ and~$d_1$ are nmbers such that
\begin{align*}
  u_n=\cdots=u_{n+d_0}<u_{n+d_0+1}\quad\mbox{and}\quad
  u_{N-n-d_1-1}<u_{N-n-d_1}=\cdots=u_{N-n}.
\end{align*}

\vspace{\bigskipamount}
\cprog{%
\#define mbs\_BSDegElevClosedC1f(indegree,inlastknot,inknots, \bsl \\
\ind{4}incoeff,deltadeg,outdegree,outlastknot,outknots,outcoeff) \bsl \\
\ind{2}mbs\_multiBSDegElevClosedf(1,1,indegree,inlastknot,inknots,0, \bsl \\
\ind{4}incoeff,deltadeg,outdegree,outlastknot,outknots,0,outcoeff) \\
\#define mbs\_BSDegElevClosedC2f(indegree,inlastknot,inknots, \bsl \\
\ind{4}inctlpoints,deltadeg,outdegree,outlastknot,outknots, \bsl \\
\ind{4}outctlpoints) \bsl \\
\ind{2}mbs\_multiBSDegElevClosedf(1,2,indegree,inlastknot,inknots,0, \bsl \\
\ind{4}(float*)inctlpoints,deltadeg,outdegree,outlastknot,outknots, \bsl \\
\ind{4}0,(float*)outctlpoints) \\
\#define mbs\_BSDegElevClosedC3f(indegree,inlastknot,inknots, \bsl \\
\ind{4}inctlpoints,deltadeg,outdegree,outlastknot,outknots, \bsl \\
\ind{4}outctlpoints) ... \\
\#define mbs\_BSDegElevClosedC4f(indegree,inlastknot,inknots, \bsl \\
\ind{4}inctlpoints,deltadeg,outdegree,outlastknot,outknots, \bsl\\
\ind{4}outctlpoints) ...}
Four macros calling the procedure \texttt{mbs\_multiBSDegElevClosedf}
in order to perform degree elevation of one closed B-spline curve
located in the space of dimension $1,2,3,4$ respectively. For the description
of the parameters see the description of that procedure.


\subsection*{Example --- degree elevation of a~B-spline patch}

The degree of a~patch may be elevated with respect to the first (,,$u$'')
or the second (,,$v$'') parameter. The methods of calling the appropriate
procedures for both cases, shown in the example below, are based on
assumption that all control nets of the patch (the initial and the final ones)
are ``packed'', i.e.\ the pitch of each array with the control points
is equal to the length of representation of one column.

We have the numbers $n$ and $m$, which specify the degree of the initial
representation of the patch, the numbers $N$ and $M$, which specify the
lengths of knot sequences, the arrays \texttt{uknots} and \texttt{vknots}
(of length $N+1$ and $M+1$ respectively) with the knots, and the array
\texttt{ctlp} with $(N-n)(M-m)d$ floating point numbers, the coordinates of
the control points of the patch. The pitch of the last array is
$(M-m)d$.

To raise the degree with respect to the ``$u$'' parameter, we can see this
patch as a~B-spline curve in the space of dimension $(M-m)d$. Then we
compute the lengths of the necessary arrays, we allocate the memory and
we call the procedure of degree elevation (here the degree increment is~$1$):

\vspace{\medskipamount}
\noindent{\ttfamily
\ind{2}ku = mbs\_NumKnotIntervalsf ( $n$, $N$, uknots ); \\
\ind{2}for ( d0 = 0; uknots[$n+\mathord{\texttt{d0}}+1$] == uknots[$n$]; d0++ ) \\
\ind{4}; \\
\ind{2}for ( d1 = 0; uknots[$N-n-\mathord{\texttt{d1}}-1$] == unkots[$N-n$]; d1++ ) \\
\ind{4}; \\
\ind{2}ua = pkv\_GetScratchMemf ( $N+2+\mathord{\texttt{ku}}-\mathord{\texttt{d0}}-\mathord{\texttt{d1}}$ ); \\
\ind{2}cpa = pkv\_GetScratchMemf ( $(N-n+\mathord{\texttt{ku}}-\mathord{\texttt{d0}}-\mathord{\texttt{d1}})(M-m)d$ ); \\
\ind{2}mbs\_multiBSDegElevf ( 1, $(M-m)d$, $n$, $N$, uknots, 0, ctlp, 1, \\
\ind{24}\&na, \&Na, ua, 0, cpa, false );
}\vspace{\medskipamount}

The pitches of the arrays \texttt{cp} and \texttt{cpa} are irrelevant
(the parameters which specify them are~$0$), because here only one curve
is subject to the degree elevation. The variables \texttt{na} and \texttt{Na}
are assigned the degree (equal to $n+1$) and the number of the last
knot (equal to
$N+\mathord{\texttt{ku}}-\mathord{\texttt{d0}}-\mathord{\texttt{d1}}+1$)
of the resulting representation of the patch. The degree with respect to the
parameter ``$v$'' and the knot sequence related with this parameter are identical
as in the initial representation of the patch.

Degree elevation with respect to the parameter ``$v$'' is equivalent to
the degree elevation of B-spline curves represented by the columns of the
control net. The appropriate code, which raises the degree by~$1$,
looks like this:

\vspace{\medskipamount}
\noindent{\ttfamily
\ind{2}kv = mbs\_NumKnotIntervalsf ( $m$, $M$, vknots ); \\
\ind{2}for ( d0 = 0; vknots[$m+\mathord{\texttt{d0}}+1$] == vknots[$m$]; d0++ ) \\
\ind{4}; \\
\ind{2}for ( d1 = 0; vknots[$M-m-\mathord{\texttt{d1}}-1$] == vnkots[$M-m$]; d1++ ) \\
\ind{4}; \\
\ind{2}va = pkv\_GetScratchMemf ( $M+2+\mathord{\texttt{kv}}-\mathord{\texttt{d0}}-\mathord{\texttt{d1}}$ ); \\
\ind{2}cpa = pkv\_GetScratchMemf ( $(N-n)(M-m+\mathord{\texttt{kv}}-\mathord{\texttt{d0}}-\mathord{\texttt{d1}})d$ ); \\
\ind{2}pitch1 = $(M-m)d$; \\
\ind{2}pitch2 = $(M-m+\mathord{\texttt{kv}}-\mathord{\texttt{d0}}-\mathord{\texttt{d1}})d$; \\
\ind{2}mbs\_multiBSDegElevf ( $N-n$, $d$, $m$, $M$, vknots, pitch1, ctlp, 1, \\
\ind{24}\&ma, \&Ma, va, pitch2, cpa, false );
}\vspace{\medskipamount}

If degree elevation by an increment greater than~$1$ is needed, one can
execute the code above a~number of times, but it is much faster and
more accurate to specify the appropriate parameter \texttt{deltadeg}.
It is necessary then to compute correctly the lengths and pitches
of the arrays to accomodate the resulting representation of the patch.
The sufficient information may be found in the description of the
procedure \texttt{mbs\_multiBSDegElevf}%
\begin{figure}[ht]
  \centerline{\epsfig{file=bspdegel.ps}}
  \caption{Degree elevation of a~B-spline patch.}
\end{figure}


\newpage
\section{Degree reduction}

Degree reduction of a~B-spline curve is a~problem of approximation
(somewhat like knot removing). The aim is to obtain a~B-spline
curve~$\tilde{\bm{s}}$ of degree~$\tilde{n}=n-d$ (for
$d\in\{1,\ldots,n\}$), which is close to a~given curve~$\bm{s}$ of degree~$n$.
A~delicate point of this construction is the arbitrary choice of
the knot sequence for the resulting curve, because of its influence on the
curve shape. The following assumptions seem obvious:
\begin{itemize}
  \item The resulting curve must have the same domain.
  \item If the given curve~$\bm{s}$ was obtained by degree elevation by~$d$
    of a~curve~$\tilde{\bm{s}}$ of degree~$n'$, then the result
    of degree reduction must be the curve~$\tilde{\bm{s}}$.
\end{itemize}
In the constructions implemented in the procedures described in this section
the set of knots of the resulting curve is a~subset of the set of knots
of the given curve. The rule of choosing the multiplicities of the knots
is as follows: let a~knot $u_i$ of the given curve has multiplicity~$r$.
\pagebreak[2]
If $r<=d$, then the multiplicity~$\tilde{r}$ of this knot in the result
representation is~$1$. If $d<r<=n+1$, then $\tilde{r}=r-d$,
and if $r>n+1$, then $\tilde{r}=n-d+1$.

For a~\textbf{non-closed curve} the knot sequence obtained based on the rule
above is modified so as to obtain a~sequence
$\tilde{u}_0,\ldots,\tilde{u}_{\tilde{N}}$ such that
$\tilde{u}_{n'}<\tilde{u}_{n'+1}$ and
$\tilde{u}_{\tilde{N}-n'-1}<\tilde{u}_{\tilde{N}-n'-1}$. To do this,
at both ends of the sequence some knots may be rejected or appended
(the first and the last knot may be appended).

\begin{sloppypar}
The next step is to find an~auxiliary knot sequence
$\hat{u}_0,\ldots,\hat{u}_{\hat{N}}$, which contains all knots of the resulting
sequence, with multiplicities greater by~$d$. By knot insertion
(with the Oslo algorithm, the procedure \texttt{mbs\_multiOsloInsertKnotsf})
and removing knots of multiplicities exceeding~$n+1$,
(with the procedure \texttt{mbs\_multiRemoveSuperfluousKnotsf}) the
auxiliary representation of the given curve~$\bm{s}$, based on the auxiliary
knot sequence, is obtained:
\begin{align*}
  \bm{s}(t) = \sum_{i=0}^{N-n-1}\bm{d}_iN^n_i(t) =
  \sum_{i=0}^{\hat{N}-n-1}\hat{\bm{d}}_i\hat{N}^n_i(t).
\end{align*}
Then the matrix~$A$, which describes degree elevation by~$d$
of the B-spline curve of degree~$n'$ based on the resulting knot sequence,
is constructed. The control points
$\tilde{\bm{d}}_0,\ldots,\tilde{\bm{d}}_{\tilde{N}-n'-1}$ of the
resulting curve~$\tilde{\bm{s}}$ are computed by solving the linear least
squares problem for the system of equations
\begin{align*}
  A\bm{x}=\bm{b},
\end{align*}
where $\bm{x}=[\tilde{\bm{d}}_0,\ldots,\tilde{\bm{d}}_{\tilde{N}-n'-1}]^T$
and $\bm{b}=[\hat{\bm{d}}_0,\ldots,\hat{\bm{d}}_{\hat{N}-n-1}]^T$.%
\end{sloppypar}


\vspace{\bigskipamount}
\cprog{%
boolean mbs\_multiBSDegRedf ( int ncurves, int spdimen, \\
\ind{16}int indegree, int inlastknot, const float *inknots, \\
\ind{16}int inpitch, const float *inctlpoints, \\
\ind{16}int deltadeg, \\
\ind{16}int *outdegree, int *outlastknot, float *outknots, \\
\ind{16}int outpitch, float *outctlpoints );}
The procedure \texttt{mbs\_multiBSDegRedf} reduces the degree of
non-closed B-spline curves, as described above. The input parameters
specify:
\texttt{ncurves} --- the number of curves, \texttt{spdimen} --- dimension
of the space with the curves, \texttt{indegree} --- degree~$n$,
\texttt{inlastknot} --- the index~$N$ of the last knot of the given curves,
\texttt{inknots} --- the knot sequence of the given curves (in an
array of length~$N+1$), \texttt{inpitch} --- pitch of the array with
the given control points, \texttt{deltadeg} --- the number~$d$, by which the
degree is to be reduced.

Output parameters: \texttt{*outdegree} --- the variable, to which the degree~$n'$
of the result curves will be assigned, \texttt{*outlastknot} ---
the variable, to which the index~$\tilde{N}$ of the last knot
of the result knot sequence will be assigned, \texttt{outknots} --- an array
for storing these knots $\tilde{u}_0,\ldots,\tilde{u}_{\tilde{N}}$,
\texttt{outpitch} --- pitch of the array \texttt{outctlpoints},
in which the control points of the resulting curves will be stored.

\vspace{\medskipamount}
\textbf{Caution:} Currently there is no procedure to compute the
length of the resulting knot sequence, which might be called before the
allocation of the arrays for the result knots and control points.
Before such a~procedure is implemented, one has to guess the sufficient
sizes for these arrays and guess the sufficiently large pitch
for the array \texttt{outctlpoints}.

The return value is \texttt{true} if the construction succeeded
and~\texttt{false} otherwise. However, in case of error
the procedure \texttt{pkv\_SignalError} is called, and its default behaviour
causes the program termination.


\begin{figure}[ht]
  \centerline{\epsfig{file=degred1.ps}}
  \caption{Degree reduction of a~B-spline curve from $5$ to~$4$}
\end{figure}

\cprog{%
\#define mbs\_BSDegRedC1f(indegree,inlastknot,inknots,incoeff, \bsl \\
\ind{4}deltadeg,outdegree,outlastknot,outknots,outcoeff) \bsl \\
\ind{2}mbs\_multiBSDegRedf(1,1,indegree,inlastknot,inknots,0,incoeff, \bsl \\
\ind{4}deltadeg,outdegree,outlastknot,outknots,0,outcoeff) \\
\#define mbs\_BSDegRedC2f(indegree,inlastknot,inknots,incpoints, \bsl \\
\ind{4}deltadeg,outdegree,outlastknot,outknots,outcpoints) \bsl \\
\ind{2}mbs\_multiBSDegRedf(1,2,indegree,inlastknot,inknots,0, \bsl \\
\ind{4}(float*)incpoints, \bsl \\
\ind{4}deltadeg,outdegree,outlastknot,outknots,0,(float*)outcpoints) \\
\#define mbs\_BSDegRedC3f(indegree,inlastknot,inknots,incpoints, \bsl \\
\ind{4}deltadeg,outdegree,outlastknot,outknots,outcpoints) ... \\
\#define mbs\_BSDegRedC4f(indegree,inlastknot,inknots,incpoints, \bsl \\
\ind{4}deltadeg,outdegree,outlastknot,outknots,outcpoints) ...}
The macros above call the procedure \texttt{mbs\_multiBSDegRedf} in order
to reduce the degree of one curve in the space of dimension $1,\ldots,4$.


\vspace{\bigskipamount}
\cprog{%
boolean mbs\_multiBSDegRedClosedf ( int ncurves, int spdimen, \\
\ind{16}int indegree, int inlastknot, const float *inknots, \\
\ind{16}int inpitch, const float *inctlpoints, \\
\ind{16}int deltadeg, \\
\ind{16}int *outdegree, int *outlastknot, float *outknots, \\
\ind{16}int outpitch, float *outctlpoints );}
The procedure \texttt{mbs\_multiBSDegRedClosedf} reduces degree of
closed B-spline curves. Its parameters have identical descriptions as the
parameters of the procedure \texttt{mbs\_multiBSDegRedf}.

\vspace{\bigskipamount}
\cprog{%
\#define mbs\_BSDegRedClosedC1f(indegree,inlastknot,inknots, \bsl \\
\ind{4}incoeff,deltadeg,outdegree,outlastknot,outknots,outcoeff) \bsl \\
\ind{2}mbs\_multiBSDegRedClosedf(1,1,indegree,inlastknot,inknots,0, \bsl \\
\ind{4}incoeff,deltadeg,outdegree,outlastknot,outknots,0,outcoeff) \\
\#define mbs\_BSDegRedClosedC2f(indegree,inlastknot,inknots, \bsl \\
\ind{4}incpoints,deltadeg,outdegree,outlastknot,outknots,outcpoints) \bsl \\
\ind{2}mbs\_multiBSDegRedClosedf(1,2,indegree,inlastknot,inknots,0, \bsl \\
\ind{4}(float*)incpoints, \bsl \\
\ind{4}deltadeg,outdegree,outlastknot,outknots,0,(float*)outcpoints) \\
\#define mbs\_BSDegRedClosedC3f(indegree,inlastknot,inknots, \bsl \\
\ind{2}incpoints,deltadeg,outdegree,outlastknot,outknots,outcpoints) ... \\
\#define mbs\_BSDegRedClosedC4f(indegree,inlastknot,inknots, \bsl \\
\ind{2}incpoints,deltadeg,outdegree,outlastknot,outknots,outcpoints) ...}
\begin{sloppypar}
The macros above call \texttt{mbs\_multiBSDegRedClosedf} in order to reduce
degree of one closed curve located in the space of dimension $1,\ldots,4$.%
\end{sloppypar}

\begin{figure}[ht]
  \centerline{\epsfig{file=degred2.ps}}
  \caption{Degree reduction of a~closed B-spline curve from $5$ to~$4$}
\end{figure}

