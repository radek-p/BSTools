
%/* //////////////////////////////////////////////////// */
%/* This file is a part of the BSTools procedure package */
%/* written by Przemyslaw Kiciak.                        */
%/* //////////////////////////////////////////////////// */

\chapter{The \texttt{libmengerc} library}

This library consists of procedures, whose purpose is to find minimal curves
of the integral Menger curvature, a~functional defined with the formula
\begin{align*}
  K_p({\cal C}) = \int\int\limits_{{\cal C}^3}\int
 K(\bm{p}_1,\bm{p}_2,\bm{p}_3)^p\,\mathrm{d}\mu({\cal C})\,\mathrm{d}\mu({\cal
 C})\,\mathrm{d}\mu({\cal C})
\end{align*}
where $K(\bm{p}_1,\bm{p}_2,\bm{p}_3)$ is the Menger curvature of the triple
of points of the curve~$\cal C$ (in $\R^3$), $p$ is an exponent, which should be
greater than~$3$ (in practice: from $4$ to $20$) and the integral is taken
over all triples of points of the curve (with respect to the arc length
measure).

The curve $\cal C$ is a closed B-spline curve of degree at least~$3$ with
uniorm knots. Such a~curve is a~knot in $\R^3$. Given an initial curve, the
procedures search local minima of the integral Menger curvature in the set
of curves, whose length is that of the initial curve. The local minimum is
a~knot topologocally equivalent to the initial curve.

The optimization method is described in the paper \emph{Shape optimization of
closed B-spline curves by minimization of the integral Menger curvature},
in preparation.

A~batch-mode program reading the curve and searching for a~minimum may be
found in the directory \texttt{bstools/test/mengerc}.

The procedures are also built in the demonstration program \texttt{pozwalaj}.
To experiment, create or read in a~closed cubic B-spline curve with uniform
knots, click the \fbox{\texttt{Options}} button and the \fbox{\texttt{Menger
curv.}} switch, set the parameters and click the \fbox{\texttt{optimize}}
button.

\cprog{}

