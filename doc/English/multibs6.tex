
%/* //////////////////////////////////////////////////// */
%/* This file is a part of the BSTools procedure package */
%/* written by Przemyslaw Kiciak.                        */
%/* //////////////////////////////////////////////////// */

\newpage
\section{\label{sect:Coons:patch:procedures}Processing Coons patches}

\subsection{Polynomial patches}

Polynomial Coons patches are represented with B\'{e}zier curves, whose degrees
need not be the same. The domain of the patch is the square $[0,1]^2$,
thus the numbers $a,b,c,d$ discussed in Section~\ref{ssect:Coons:patch:def}
are $0,1,0,1$ respectively.

\vspace{\bigskipamount}
\cprog{%
void mbs\_BezC1CoonsFindCornersf ( int spdimen, \\
\ind{34}int degc00, const float *c00, \\
\ind{34}int degc01, const float *c01, \\
\ind{34}int degc10, const float *c10, \\
\ind{34}int degc11, const float *c11, \\
\ind{34}float *pcorners );}
\begin{sloppypar}
The procedure \texttt{mbs\_BezC1CoonsFindCornersf} computes the matrix~$\bm{P}$
of dimensions $4\times\nobreak 4$, whose elements are the points and derivatives
of the curves $\bm{c}_{00},\bm{c}_{10},\bm{c}_{01},\bm{c}_{11}$.%
\end{sloppypar}

Parameters: \texttt{spdimen} --- dimension~$d$ of the space with the
curves and the bicubic polynomial Coons patch (of class $C^1$) represented
by these curves. Each pair of the parameters \texttt{degc??} and \texttt{c??}    
describes one of the curves, its degree and the control points.

The parameter \texttt{pcorners} points to the array, in which the result
is to be stored; the array length must be at least~$16d$.

\vspace{\bigskipamount}
\cprog{%
boolean mbs\_BezC1CoonsToBezf ( int spdimen, \\
\ind{31}int degc00, const float *c00, \\
\ind{31}int degc01, const float *c01, \\
\ind{31}int degc10, const float *c10, \\
\ind{31}int degc11, const float *c11, \\
\ind{31}int degd00, const float *d00, \\
\ind{31}int degd01, const float *d01, \\
\ind{31}int degd10, const float *d10, \\
\ind{31}int degd11, const float *d11, \\
\ind{31}int *n, int *m, float *p );}
The procedure \texttt{mbs\_BezC1CoonsToBezf} finds the B\'{e}zier
representation of the bicubic Coons patch (of class~$C^1$), defined by
given polynomial curves. The procedure value is \texttt{true}, if the
computation was successful and~\texttt{false} otherwise; the reason
of failure may be insufficient space on the scratch memory stack.

The value of the parametr \texttt{spdimen} is the dimension~$d$ of the space,
in which the curves and the patch are located. Each pair of parameters
\texttt{degc??} and~\texttt{c??} describes one of the curves
$\bm{c}_{00},\bm{c}_{01},\bm{c}_{10},\bm{c}_{11}$, by specifying its
degree and B\'{e}zier control points. Each pair of parameters
\texttt{degd??} and~\texttt{d??} describe in the same way one of the curves
$\bm{d}_{00},\bm{d}_{01},\bm{d}_{10},\bm{d}_{11}$. The curves must
satisfy (up to rounding errors) the compatibility
conditions~(\ref{eq:Coons:compat:cond}), which \emph{is not} verified.

The variables \texttt{*n} and~\texttt{*m} obtain values, which describe
the degree of the B\'{e}zier patch representation.
The value~$n$ assigned to \texttt{*n} is the greatest value of
the parameters \texttt{degc??} or~$3$ (if the number~$3$ is greater).
Similarly the value~$m$ assigned to \texttt{*m} is the greatest value
of the parameters \texttt{degd??}, or~$3$. The B\'{e}zier control points
of the patch are stored in the array pointed by the parameter~\texttt{p};
it must be long enough (at least $(n+1)(m+1)d$).

\vspace{\bigskipamount}
\cprog{%
void mbs\_TabCubicHFuncDer2f ( float a, float b, \\
\ind{20}int nkn, const float *kn, \\
\ind{20}float *hfunc, float *dhfunc, float *ddhfunc );}
The procedure \texttt{mbs\_TabCubicHFuncDer2f} evaluates the polynomials
$\tilde{H}_{00},\tilde{H}_{10}$, $\tilde{H}_{01}$ and $\tilde{H}_{11}$,
being the basis of definition of bicubic Coons patches, and their
derivatives of order~$1$ and~$2$. The result of this computation may be used
to compute a~number of points of a~Coons patch corresponding to
a~rectangular net in the domain, using the procedure
\texttt{mbs\_TabBezC1CoonsDer2f} (a~polynomial patch) or
\texttt{mbs\_TabBSC1CoonsDer2f} (a~spline patch).

The parameters \texttt{a} and~\texttt{b} describe the end points of the
interval taken as the domain of the curves $\bm{c}_{ij}$ or $\bm{d}_{ij}$;
for polynomial Coons patches these parameters must have values
$0$ and~$1$ respectively.

The parameter \texttt{nkn} specifies the number~$k$ of points $u_m\in[a,b]$,
at which the polynomials are to be evaluated; these points
(floating point numbers) are given in the array \texttt{kn}.

The values of the polynomials and their derivatives of order~$1$ and~$2$
are stored in the arrays \texttt{hfunc}, \texttt{dhfunc} and~\texttt{ddhfunc}
respectively.
The arrays must have length at least~$4k$; to each subsequent four positions
in the array the values of the four polynomials or their derivatives at
the subsequent point $u_m$ are assigned.

\vspace{\bigskipamount}
\cprog{%
void mbs\_TabCubicHFuncDer3f ( float a, float b, int nkn, \\
\ind{20}const float *kn, \\
\ind{20}float *hfunc, float *dhfunc, float *ddhfunc, \\
\ind{20}float *dddhfunc );}


\vspace{\bigskipamount}
\cprog{%
boolean mbs\_TabBezC1CoonsDer2f ( int spdimen, \\
\ind{8}int nknu, const float *knu, const float *hfuncu, \\
\ind{8}const float *dhfuncu, const float *ddhfuncu, \\
\ind{8}int nknv, const float *knv, const float *hfuncv, \\
\ind{8}const float *dhfuncv, const float *ddhfuncv, \\
\ind{8}int degc00, const float *c00, \\
\ind{8}int degc01, const float *c01, \\
\ind{8}int degc10, const float *c10, \\
\ind{8}int degc11, const float *c11, \\
\ind{8}int degd00, const float *d00, \\
\ind{8}int degd01, const float *d01, \\
\ind{8}int degd10, const float *d10, \\
\ind{8}int degd11, const float *d11, \\
\ind{8}float *p, float *pu, float *pv, \\
\ind{8}float *puu, float *puv, float *pvv );}
The procedure \texttt{mbs\_TabBezC1CoonsDer2f} performs a~fast computation
of points and derivatives of order $1$ and~$2$ of a~bicubic polynomial Coons
patch, at the points $(u_i,v_j)$, where $i\in\{0,\ldots,k-1\}$,
$j\in\{0,\ldots,l-1\}$.

The parameter~\texttt{spdimen} specifies the dimension~$d$ of the space with
the patch. The parameter~\texttt{nknu} specifies the number~$k$, the
array~\texttt{knu} contains the numbers $u_0,\ldots,u_{k-1}$. The contents
of the arrays \texttt{hfuncu}, \texttt{dhfuncu}, \texttt{ddhfuncu} must be
respectively the values of the polynomials $H_{00},H_{10},H_{01},H_{11}$
and their derivatives of order~$1$ and~$2$ at the points
$u_0,\ldots,u_{k-1}$; these values are simplest to obtain by calling
the procedure \texttt{mbs\_TabCubicHFuncDer2f}
(with the parameters \texttt{a}${}=0$, \texttt{b}${}=1$).

The numbers $v_0,\ldots,v_{l-1}$ and the values of the functions $H_{ij}$
and their derivatives at $v_j$ are analoguously specified by the parameters
\texttt{nknv}, \texttt{knv}, \texttt{hfuncv}, \texttt{dhfuncv},
\texttt{ddhfuncv}.

The pairs of parameters \texttt{degc??},~\texttt{c??} and \texttt{degd??},
\texttt{d??} describe the B\'{e}zier curves, which define the patch.
These curves must satisfy (up to the rounding errors)
the compatibility conditions~(\ref{eq:Coons:compat:cond}).

In the arrays pointed by the parameters \texttt{p}, \texttt{pu}, \texttt{pv},
\texttt{puu}, \texttt{puv}, \texttt{pvv} the procedure stores the computed
points and derivatives of order $1$ and~$2$; if any of the parameters
is~\texttt{NULL}, then the corresponding points or vectors are not computed.
Otherwise the pointed array must have length at least~$k^2d$.

The value returned is \texttt{true} in case of success and \texttt{false}
after failure (caused by insufficient space on the scratch memory stack).


\vspace{\bigskipamount}
\cprog{%
boolean mbs\_TabBezC1CoonsDer3f ( int spdimen, \\
\ind{8}int nknu, const float *knu, const float *hfuncu, \\
\ind{8}const float *dhfuncu, const float *ddhfuncu, \\
\ind{8}const float *dddhfuncu, \\
\ind{8}int nknv, const float *knv, const float *hfuncv, \\
\ind{8}const float *dhfuncv, const float *ddhfuncv, \\
\ind{8}const float *dddhfuncv, \\
\ind{8}int degc00, const float *c00, \\
\ind{8}int degc01, const float *c01, \\
\ind{8}int degc10, const float *c10, \\
\ind{8}int degc11, const float *c11, \\
\ind{8}int degd00, const float *d00, \\
\ind{8}int degd01, const float *d01, \\
\ind{8}int degd10, const float *d10, \\
\ind{8}int degd11, const float *d11, \\
\ind{8}float *p, float *pu, float *pv, \\
\ind{8}float *puu, float *puv, float *pvv, \\
\ind{8}float *puuu, float *puuv, float *puvv, float *pvvv );}

\vspace{\bigskipamount}
\cprog{%
boolean mbs\_TabBezC1Coons0Der2f ( int spdimen, \\
\ind{8}int nknu, const float *knu, const float *hfuncu, \\
\ind{8}const float *dhfuncu, const float *ddhfuncu, \\
\ind{8}int nknv, const float *knv, const float *hfuncv, \\
\ind{8}const float *dhfuncv, const float *ddhfuncv, \\
\ind{8}int degc00, const float *c00, \\
\ind{8}int degc01, const float *c01, \\
\ind{8}int degd00, const float *d00, \\
\ind{8}int degd01, const float *d01, \\
\ind{8}float *p, float *pu, float *pv, \\
\ind{8}float *puu, float *puv, float *pvv );}
\begin{sloppypar}
The procedure \texttt{mbs\_TabBezC1Coons0Der2f} is a~simplified version
of the procedure \texttt{mbs\_TabBezC1CoonsDer2f} for the case,
when the curves $\bm{c}_{10},\bm{c}_{11},\bm{d}_{10}$ and~$\bm{d}_{11}$ are
null (i.e.\ when all their control points have all coordinates zero).
Computing points of such a~patch may be done in a~shorter time;
this procedure is used by the library \texttt{libg1hole}.%
\end{sloppypar}

The parameters of \texttt{mbs\_TabBezC1Coons0Der2f} are the same as
the parameters of the procedure \texttt{mbs\_TabBezC1CoonsDer2f}
with the same names.

\vspace{\bigskipamount}
\cprog{%
boolean mbs\_TabBezC1Coons0Der3f ( int spdimen, \\
\ind{8}int nknu, const float *knu, const float *hfuncu, \\
\ind{8}const float *dhfuncu, const float *ddhfuncu, \\
\ind{8}const float *dddhfuncu, \\
\ind{8}int nknv, const float *knv, const float *hfuncv, \\
\ind{8}const float *dhfuncv, const float *ddhfuncv, \\
\ind{8}const float *dddhfuncv, \\
\ind{8}int degc00, const float *c00, \\
\ind{8}int degc01, const float *c01, \\
\ind{8}int degd00, const float *d00, \\
\ind{8}int degd01, const float *d01, \\
\ind{8}float *p, float *pu, float *pv, \\
\ind{8}float *puu, float *puv, float *pvv, \\
\ind{8}float *puuu, float *puuv, float *puvv, float *pvvv );}


\vspace{\bigskipamount}
\cprog{%
void mbs\_BezC2CoonsFindCornersf ( int spdimen, \\
\ind{34}int degc00, const float *c00, \\
\ind{34}int degc01, const float *c01, \\
\ind{34}int degc02, const float *c02, \\
\ind{34}int degc10, const float *c10, \\
\ind{34}int degc11, const float *c11, \\
\ind{34}int degc12, const float *c12, \\
\ind{34}float *pcorners );}
\begin{sloppypar}
The procedure \texttt{mbs\_BezC2CoonsFindCornersf} computes the matrix~$\bm{P}$
of dimensions $6\times\nobreak 6$, whose elements are the points and derivatives
of the curves
$\bm{c}_{00},\bm{c}_{10},\bm{c}_{01},\bm{c}_{11},\bm{c}_{02},\bm{c}_{12}$.%
\end{sloppypar}

Parameters: \texttt{spdimen} --- dimension~$d$ of the space with the
curves and the biquintic polynomial Coons patch (of class $C^2$) represented
by these curves. Each pair of the parameters \texttt{degc??} and \texttt{c??}    
describes one of the curves, its degree and the control points.

The parameter \texttt{pcorners} points to the array, in which the result
is to be stored; the array length must be at least~$36d$.


\vspace{\bigskipamount}
\cprog{%
boolean mbs\_BezC2CoonsToBezf ( int spdimen, \\
\ind{31}int degc00, const float *c00, \\
\ind{31}int degc01, const float *c01, \\
\ind{31}int degc02, const float *c02, \\
\ind{31}int degc10, const float *c10, \\
\ind{31}int degc11, const float *c11, \\
\ind{31}int degc12, const float *c12, \\
\ind{31}int degd00, const float *d00, \\
\ind{31}int degd01, const float *d01, \\
\ind{31}int degd02, const float *d02, \\
\ind{31}int degd10, const float *d10, \\
\ind{31}int degd11, const float *d11, \\
\ind{31}int degd12, const float *d12, \\
\ind{31}int *n, int *m, float *p );}
The procedure \texttt{mbs\_BezC2CoonsToBezf} converts a biquintic
Coons patch to the B\'{e}zier form. The patch is represented by
$12$~polynomial curves, which describe its boundary (the curves $c_{00}$,
$c_{10}$, $d_{00}$, $d_{10}$) and the cross derivatives of the first
($c_{01}$, $c_{11}$, $d_{01}$, $d_{11}$) and second ($c_{02}$, $c_{12}$,
$d_{02}$, $d_{12}$)) order. All these curves are given in B\'{e}zier form,
their degrees are specified respectively by the parameters
\texttt{degc00}, \ldots, \texttt{degd12}, their control points (in the
space of dimension~\texttt{spdimen}) are given in the arrays
\texttt{c00}, \ldots, \texttt{d12}.

The output parameters are \texttt{*n} and~\texttt{*m}, which obtain the values
indicating the degree and the array \texttt{p}, in which the B\'{e}zier control
points of the patch are stored.

\vspace{\bigskipamount}
\cprog{%
void mbs\_TabQuinticHFuncDer3f (  float a, float b, \\
\ind{32}int nkn, const float *kn, \\
\ind{32}float *hfunc, float *dhfunc, \\
\ind{32}float *ddhfunc, float *dddhfunc );}
\begin{sloppypar}
The procedure \texttt{mbs\_TabQuinticHFuncDer3f} evaluates the polynomials
$\tilde{H}_{00}$, $\tilde{H}_{10}$, $\tilde{H}_{01}$, $\tilde{H}_{11}$,
$\tilde{H}_{02}$ and $\tilde{H}_{12}$, 
being the basis of definition of biquintic Coons patches, and their
derivatives of order~$1$, $2$ and~$3$. The result of this computation may be
used to compute a~number of points of a~Coons patch corresponding to
a~rectangular net in the domain, using the procedure
\texttt{mbs\_TabBezC2CoonsDer3f} (a~polynomial patch) or
\texttt{mbs\_TabBSC2CoonsDer3f} (a~spline patch).%
\end{sloppypar}

The parameters \texttt{a} and~\texttt{b} describe the end points of the
interval taken as the domain of the curves $\bm{c}_{ij}$ or $\bm{d}_{ij}$;
for polynomial Coons patches these parameters must have values
$0$ and~$1$ respectively.

The parameter \texttt{nkn} specifies the number~$k$ of points $u_m\in[a,b]$,
at which the polynomials are to be evaluated; these points
(floating point numbers) are given in the array \texttt{kn}.

The values of the polynomials and their derivatives of order~$1$, $2$
and~$3$ are stored in the arrays \texttt{hfunc}, \texttt{dhfunc},
\texttt{ddhfunc} and~\texttt{dddhfunc} respectively.
The arrays must have length at least~$4k$; to each subsequent four positions
in the array the values of the four polynomials or their derivatives at
the subsequent point $u_m$ are assigned.

\vspace{\bigskipamount}
\cprog{%
boolean mbs\_TabBezC2CoonsDer3f ( int spdimen, \\
\ind{8}int nknu, const float *knu, const float *hfuncu, \\
\ind{8}const float *dhfuncu, const float *ddhfuncu, \\
\ind{8}const float *dddhfuncu, \\
\ind{8}int nknv, const float *knv, const float *hfuncv, \\
\ind{8}const float *dhfuncv, const float *ddhfuncv, \\
\ind{8}const float *dddhfuncv, \\
\ind{8}int degc00, const float *c00, \\
\ind{8}int degc01, const float *c01, \\
\ind{8}int degc02, const float *c02, \\
\ind{8}int degc10, const float *c10, \\
\ind{8}int degc11, const float *c11, \\
\ind{8}int degc12, const float *c12, \\
\ind{8}int degd00, const float *d00, \\
\ind{8}int degd01, const float *d01, \\
\ind{8}int degd02, const float *d02, \\
\ind{8}int degd10, const float *d10, \\
\ind{8}int degd11, const float *d11, \\
\ind{8}int degd12, const float *d12, \\
\ind{8}float *p, float *pu, float *pv, float *puu, \\
\ind{8}float *puv, float *pvv, \\
\ind{8}float *puuu, float *puuv, float *puvv, float *pvvv );}
The procedure \texttt{mbs\_TabBezC2CoonsDer3f} performs a~fast computation
of points and derivatives of order $1$, $2$ and~$3$ of a~biquintic polynomial
Coons patch, at the points $(u_i,v_j)$, where $i\in\{0,\ldots,k-1\}$,
$j\in\{0,\ldots,l-1\}$.

The parameter~\texttt{spdimen} specifies the dimension~$d$ of the space with
the patch. The parameter~\texttt{nknu} specifies the number~$k$, the
array~\texttt{knu} contains the numbers $u_0,\ldots,u_{k-1}$. The contents
of the arrays \texttt{hfuncu}, \texttt{dhfuncu}, \texttt{ddhfuncu},
\texttt{dddhfuncu} must be respectively the values of the polynomials
$H_{00},H_{10},H_{01},H_{11},H_{02},H_{12}$
and their derivatives of order~$1$, $2$ and~$3$ at the points
$u_0,\ldots,u_{k-1}$; these values are simplest to obtain by calling
the procedure \texttt{mbs\_TabQuinticHFuncDer3f}
(with the parameters \texttt{a}${}=0$, \texttt{b}${}=1$).

The sequence $v_0,\ldots.v_{l-1}$ and the values of $H_{ij}$ at the points
of this sequence are analoguously represented by the parameters
\texttt{nknv}, \texttt{knv}, \texttt{hfuncv}, \texttt{dhfuncv},
\texttt{ddhfuncv}, \texttt{dddhfuncv}.

The pairs of parameters \texttt{degc??},~\texttt{c??} and \texttt{degd??},
\texttt{d??} describe the B\'{e}zier curves, which define the patch.
These curves must satisfy (up to the rounding errors)
the compatibility conditions~(\ref{eq:Coons:compat:cond}).

In the arrays pointed by the parameters \texttt{p}, \texttt{pu}, \texttt{pv},
\texttt{puu}, \texttt{puv}, \texttt{pvv}, \texttt{puuu}, \texttt{puuv},
\texttt{puvv}, \texttt{pvvv} the procedure stores the computed
points and derivatives of order $1$, $2$ and~$3$; if any of the parameters
is~\texttt{NULL}, then the corresponding points or vectors are not computed.
Otherwise the pointed array must have length at least~$k^2d$.

The value returned is \texttt{true} in case of success and \texttt{false}
after failure (caused by insufficient space on the scratch memory stack).


\vspace{\bigskipamount}
\cprog{%
boolean mbs\_TabBezC2Coons0Der3f ( int spdimen, \\      
\ind{8}int nknu, const float *knu, const float *hfuncu, \\
\ind{8}const float *dhfuncu, const float *ddhfuncu, \\
\ind{8}const float *dddhfuncu, \\
\ind{8}int nknv, const float *knv, const float *hfuncv, \\
\ind{8}const float *dhfuncv, const float *ddhfuncv, \\
\ind{8}const float *dddhfuncv, \\
\ind{8}int degc00, const float *c00, \\
\ind{8}int degc01, const float *c01, \\
\ind{8}int degc02, const float *c02, \\
\ind{8}int degd00, const float *d00, \\
\ind{8}int degd01, const float *d01, \\
\ind{8}int degd02, const float *d02, \\
\ind{8}float *p, float *pu, float *pv, float *puu, float *puv, \\
\ind{8}float *pvv, \\
\ind{8}float *puuu, float *puuv, float *puvv, float *pvvv );}
\begin{sloppypar}
The procedure \texttt{mbs\_TabBezC2Coons0Der3f} is a~simplified version
of the procedure \texttt{mbs\_TabBezC2CoonsDer3f} for the case,
when the curves $\bm{c}_{10},\bm{c}_{11},\bm{c}_{12},\bm{d}_{10},\bm{d}_{11}$
and~$\bm{d}_{12}$ are null (i.e.\ when all their control points have all
coordinates zero). Computing points of such a~patch may be done in a~shorter time;
this procedure is used by the library \texttt{libg1hole}.%
\end{sloppypar}

The parameters of \texttt{mbs\_TabBezC2Coons0Der3f} are the same as
the parameters of the procedure \texttt{mbs\_TabBezC2CoonsDer3f}   
with the same names.


\newpage
\subsection{Spline patches}

Spline Coons patches are defined with B-spline curves; they may have
different degrees and different knot sequences; the only restriction is that
all curves $\bm{c}_{ij}$ must have the same domain (determined by their
boundary knots) and the same concerns the curves $\bm{d}_{ij}$.

\vspace{\bigskipamount}
\cprog{%
void mbs\_BSC1CoonsFindCornersf ( int spdimen, \\
\ind{8}int degc00, int lastknotc00, const float *knotsc00, \\
\ind{8}const float *c00, \\
\ind{8}int degc01, int lastknotc01, const float *knotsc01, \\
\ind{8}const float *c01, \\
\ind{8}int degc10, int lastknotc10, const float *knotsc10, \\
\ind{8}const float *c10, \\
\ind{8}int degc11, int lastknotc11, const float *knotsc11, \\
\ind{8}const float *c11, \\
\ind{8}float *pcorners );}
\begin{sloppypar}
The procedure \texttt{mbs\_BSC1CoonsFindCornersf} computes the matrix~$\bm{P}$
of dimensions $4\times\nobreak 4$, whose elements are the points and derivatives
of the curves $\bm{c}_{00},\bm{c}_{10},\bm{c}_{01},\bm{c}_{11}$.%
\end{sloppypar}

Parameters: \texttt{spdimen} --- dimension~$d$ of the space with the
curves and the bicubic spline Coons patch (of class $C^1$) represented
by these curves. Each quadruple of the parameters \texttt{degc??},
\texttt{lastknotc??}, \texttt{knotsc??} and \texttt{c??}    
describes one of the curves, its degree, number of the last knot, knots
and the control points respectively.

The parameter \texttt{pcorners} points to the array, in which the result
is to be stored; the array length must be at least~$16d$.


\vspace{\bigskipamount}
\cprog{%
boolean mbs\_BSC1CoonsToBSf ( int spdimen, \\
\ind{8}int degc00, int lastknotc00, const float *knotsc00, \\
\ind{8}const float *c00, \\
\ind{8}int degc01, int lastknotc01, const float *knotsc01, \\
\ind{8}const float *c01, \\
\ind{8}int degc10, int lastknotc10, const float *knotsc10, \\
\ind{8}const float *c10, \\
\ind{8}int degc11, int lastknotc11, const float *knotsc11, \\
\ind{8}const float *c11, \\
\ind{8}int degd00, int lastknotd00, const float *knotsd00, \\
\ind{8}const float *d00, \\
\ind{8}int degd01, int lastknotd01, const float *knotsd01, \\
\ind{8}const float *d01, \\
\ind{8}int degd10, int lastknotd10, const float *knotsd10, \\
\ind{8}const float *d10, \\
\ind{8}int degd11, int lastknotd11, const float *knotsd11, \\
\ind{8}const float *d11, \\
\ind{8}int *degreeu, int *lastuknot, float *uknots, \\
\ind{8}int *degreev, int *lastvknot, float *vknots, float *p );}
The procedure \texttt{mbs\_BSC1CoonsToBSf} finds a~B-spline representation
of a~bicubic Coons patch (of class~$C^1$), defined by given spline curves.
The value returned is \texttt{true} if the computation has been successful
and \texttt{false} otherwise (the reason of failure may be insufficient space
on the scratch memory stack or incorrect knot sequences of the given curves).

The value of the parametes \texttt{spdimen} is the dimension~$d$ of the space
with the curves and the patch. Subsequent quadruples of parameters
\texttt{degc??}, \texttt{lastknotc??}, \texttt{knotsc??} and \texttt{c??}
describe the appropriate curve of the family
$\bm{c}_{00},\bm{c}_{01},\bm{c}_{10},\bm{c}_{11}$, by specifying the degree,
number of the last knot, knot sequence and the array of control points.
The quadruples of parameters \texttt{degd??}, \texttt{lastknotd??},
\texttt{knotsd??} and \texttt{d??} in the same way describe the curves
of the family $\bm{d}_{00},\bm{d}_{01},\bm{d}_{10},\bm{d}_{11}$.
The curves must satisfy (up to rounding errors) the compatibility
conditions~(\ref{eq:Coons:compat:cond}), \emph{which is not verified}.

The variables \texttt{*n} and \texttt{*m} are assigned the values, which describe
the degree of the B-spline representation of the patch. The value~$n$ of the
variable \texttt{*n} is the greatest of the values of the parameters
\texttt{degc??} or~$3$ (if the number~$3$ is greater).
Similarly, the value~$m$ of the variable \texttt{*m} is the greatest
of the values of the parameters \texttt{degd??} or~$3$. The parameters
\texttt{lastuknot}, \texttt{uknots}, \texttt{lastvknot} i~\texttt{vknots}
are used to output the knot sequences of the B-spline representation.
In the array pointed by the parameter~\texttt{p} the procedure stores the
control points.

The arrays \texttt{unkots}, \texttt{vknots} and~\texttt{p} must be long enough;
their lengths may be computed before the allocation by calling
\texttt{mbs\_FindBSCommonKnotSequencef} for the families of curves
$\bm{c}_{ij}$ and $\bm{d}_{ij}$; the variables pointed by the parameter
\texttt{lastknot} of this procedure must have the initial value~$3$.


\vspace{\bigskipamount}
\cprog{%
boolean mbs\_TabBSC1CoonsDer2f ( int spdimen, \\
\ind{8}int nknu, const float *knu, const float *hfuncu, \\
\ind{8}const float *dhfuncu, const float *ddhfuncu, \\
\ind{8}int nknv, const float *knv, const float *hfuncv, \\
\ind{8}const float *dhfuncv, const float *ddhfuncv, \\
\ind{8}int degc00, int lastknotc00, const float *knotsc00, \\
\ind{8}const float *c00, \\
\ind{8}int degc01, int lastknotc01, const float *knotsc01, \\
\ind{8}const float *c01, \\
\ind{8}int degc10, int lastknotc10, const float *knotsc10, \\
\ind{8}const float *c10, \\
\ind{8}int degc11, int lastknotc11, const float *knotsc11, \\
\ind{8}const float *c11, \\
\ind{8}int degd00, int lastknotd00, const float *knotsd00, \\
\ind{8}const float *d00, \\
\ind{8}int degd01, int lastknotd01, const float *knotsd01, \\
\ind{8}const float *d01, \\
\ind{8}int degd10, int lastknotd10, const float *knotsd10, \\
\ind{8}const float *d10, \\
\ind{8}int degd11, int lastknotd11, const float *knotsd11, \\
\ind{8}const float *d11, \\
\ind{8}float *p, float *pu, float *pv, \\
\ind{8}float *puu, float *puv, float *pvv );}
The procedure \texttt{mbs\_TabBSC1CoonsDer2f} performs a~fast computation
of points and derivatives of order $1$ and~$2$ of a~bicubic spline Coons
patch, at the points $(u_i,v_j)$, where $i\in\{0,\ldots,k_u-1\}$,
$j\in\{0,\ldots,k_v-1\}$.

The parameter~\texttt{spdimen} specifies the dimension~$d$ of the space with
the patch. The parameters~\texttt{nknu} and~\texttt{nknv} specify the
numbers~$k_u$ and~$k_v$, the arrays~\texttt{knu} and~\texttt{knv} contain
respectively the numbers $u_0,\ldots,u_{k_u-1}$ and $v_0,\ldots,v_{k_v-1}$.
The contents of the arrays \texttt{hfuncu}, \texttt{dhfuncu}, \texttt{ddhfuncu}
must be respectively the values of the polynomials
$\tilde{H}_{00},\tilde{H}_{10},\tilde{H}_{01},\tilde{H}_{11}$
and their derivatives of order~$1$ and~$2$ at the points
$u_0,\ldots,u_{k_u-1}$. The arrays \texttt{hfuncv}, \texttt{dhfuncv}
and~\texttt{ddhfuncv} must contain the values of the polynomials
$\hat{H}_{00},\hat{H}_{10},\hat{H}_{01},\hat{H}_{11}$ at $v_0,\ldots,v_{k_v-1}$;
these values are simplest to obtain by calling
the procedure \texttt{mbs\_TabCubicHFuncDer2f}.

\begin{sloppypar}
The quadruples of parameters \texttt{degc??}, \texttt{lastknotc??}, \texttt{knotsc??},
\texttt{c??} and \texttt{degd??}, \texttt{lastknotd??}, \texttt{knotsd??},
\texttt{d??} describe the B-spline curves, which define the patch.
These curves must satisfy (up to the rounding errors)
the compatibility conditions~(\ref{eq:Coons:compat:cond}).%
\end{sloppypar}

In the arrays pointed by the parameters \texttt{p}, \texttt{pu}, \texttt{pv},
\texttt{puu}, \texttt{puv}, \texttt{pvv} the procedure stores the computed
points and derivatives of order $1$ and~$2$; if any of the parameters
is~\texttt{NULL}, then the corresponding points or vectors are not computed.
Otherwise the pointed array must have length at least~$k_uk_vd$.

The value returned is \texttt{true} in case of success and \texttt{false}
after failure (caused by insufficient space on the scratch memory stack
or incorrect knot sequences of the curves).


\vspace{\bigskipamount}
\cprog{%
boolean mbs\_TabBSC1Coons0Der2f ( int spdimen, \\
\ind{8}int nknu, const float *knu, const float *hfuncu, \\
\ind{8}const float *dhfuncu, const float *ddhfuncu, \\
\ind{8}int nknv, const float *knv, const float *hfuncv, \\
\ind{8}const float *dhfuncv, const float *ddhfuncv, \\
\ind{8}int degc00, int lastknotc00, const float *knotsc00, \\
\ind{8}const float *c00, \\
\ind{8}int degc01, int lastknotc01, const float *knotsc01, \\
\ind{8}const float *c01, \\
\ind{8}int degd00, int lastknotd00, const float *knotsd00, \\
\ind{8}const float *d00, \\
\ind{8}int degd01, int lastknotd01, const float *knotsd01, \\
\ind{8}const float *d01, \\
\ind{8}float *p, float *pu, float *pv, \\
\ind{8}float *puu, float *puv, float *pvv );}
\begin{sloppypar}
The procedure \texttt{mbs\_TabBSC1Coons0Der2f} is a~simplified version
of the procedure \texttt{mbs\_TabBSC1CoonsDer2f} for the case,
when the curves $\bm{c}_{10},\bm{c}_{11},\bm{d}_{10}$ and~$\bm{d}_{11}$ are
null (i.e.\ when all their control points have all coordinates zero).
Computing points of such a~patch may be done in a~shorter time.%
\end{sloppypar}

The parameters of \texttt{mbs\_TabBSC1Coons0Der2f} are the same as
the parameters of the procedure \texttt{mbs\_TabBSC1CoonsDer2f}
with the same names.


\vspace{\bigskipamount}
\cprog{%
void mbs\_BSC2CoonsFindCornersf ( int spdimen, \\
\ind{8}int degc00, int lastknotc00, const float *knotsc00, \\
\ind{8}const float *c00, \\
\ind{8}int degc01, int lastknotc01, const float *knotsc01, \\
\ind{8}const float *c01, \\
\ind{8}int degc02, int lastknotc02, const float *knotsc02, \\
\ind{8}const float *c02, \\
\ind{8}int degc10, int lastknotc10, const float *knotsc10, \\
\ind{8}const float *c10, \\
\ind{8}int degc11, int lastknotc11, const float *knotsc11, \\
\ind{8}const float *c11, \\
\ind{8}int degc12, int lastknotc12, const float *knotsc12, \\
\ind{8}const float *c12, \\
\ind{8}float *pcorners );}
\begin{sloppypar}
The procedure \texttt{mbs\_BSC2CoonsFindCornersf} computes the matrix~$\bm{P}$
of dimensions $6\times\nobreak 6$, whose elements are the points and derivatives
of the curves
$\bm{c}_{00},\bm{c}_{10},\bm{c}_{01},\bm{c}_{11},\bm{c}_{02},\bm{c}_{12}$.%
\end{sloppypar}

Parameters: \texttt{spdimen} --- dimension~$d$ of the space with the
curves and the biquintic spline Coons patch (of class $C^2$) represented
by these curves. Each quadruple of the parameters \texttt{degc??},
\texttt{lastknotc??}, \texttt{knotsc??} and \texttt{c??}    
describes one of the curves, its degree, number of the last knot, knots
and the control points respectively.

The parameter \texttt{pcorners} points to the array, in which the result
is to be stored; the array length must be at least~$36d$.


\vspace{\bigskipamount}
\cprog{%
boolean mbs\_BSC2CoonsToBSf ( int spdimen, \\
\ind{8}int degc00, int lastknotc00, const float *knotsc00, \\
\ind{8}const float *c00, \\
\ind{8}int degc01, int lastknotc01, const float *knotsc01, \\
\ind{8}const float *c01, \\
\ind{8}Aint degc02, int lastknotc02, const float *knotsc02, \\
\ind{8}const float *c02, \\
\ind{8}int degc10, int lastknotc10, const float *knotsc10, \\
\ind{8}const float *c10, \\
\ind{8}int degc11, int lastknotc11, const float *knotsc11, \\
\ind{8}const float *c11, \\
\ind{8}int degc12, int lastknotc12, const float *knotsc12, \\
\ind{8}const float *c12, \\
\ind{8}int degd00, int lastknotd00, const float *knotsd00, \\
\ind{8}const float *d00, \\
\ind{8}int degd01, int lastknotd01, const float *knotsd01, \\
\ind{8}const float *d01, \\
\ind{8}int degd02, int lastknotd02, const float *knotsd02, \\
\ind{8}const float *d02, \\
\ind{8}int degd10, int lastknotd10, const float *knotsd10, \\
\ind{8}const float *d10, \\
\ind{8}int degd11, int lastknotd11, const float *knotsd11, \\
\ind{8}const float *d11, \\
\ind{8}int degd12, int lastknotd12, const float *knotsd12, \\
\ind{8}const float *d12, \\
\ind{8}int *degreeu, int *lastuknot, float *uknots, \\
\ind{8}int *degreev, int *lastvknot, float *vknots, float *p );}
The procedure \texttt{mbs\_BSC2CoonsToBSf} finds a~B-spline representation
of a~biquintic Coons patch (of class~$C^2$), defined by given spline curves.
The value returned is \texttt{true} if the computation has been successful
and \texttt{false} otherwise (the reason of failure may be insufficient space
on the scratch memory stack or incorrect knot sequences of the given curves).

The value of the parametes \texttt{spdimen} is the dimension~$d$ of the space
with the curves and the patch. Subsequent quadruples of parameters
\texttt{degc??}, \texttt{lastknotc??}, \texttt{knotsc??} and \texttt{c??}
describe the appropriate curve of the family
$\bm{c}_{00},\bm{c}_{01},\bm{c}_{10},\bm{c}_{11}$, by specifying the degree,
number of the last knot, knot sequence and the array of control points.
The quadruples of parameters \texttt{degd??}, \texttt{lastknotd??},
\texttt{knotsd??} and \texttt{d??} in the same way describe the curves
of the family $\bm{d}_{00},\bm{d}_{01},\bm{d}_{10},\bm{d}_{11}$.
The curves must satisfy (up to rounding errors) the compatibility
conditions~(\ref{eq:Coons:compat:cond}), \emph{which is not verified}.

The variables \texttt{*n} and \texttt{*m} are assigned the values, which describe
the degree of the B-spline representation of the patch. The value~$n$ of the
variable \texttt{*n} is the greatest of the values of the parameters
\texttt{degc??} or~$5$ (if the number~$5$ is greater).
Similarly, the value~$m$ of the variable \texttt{*m} is the greatest
of the values of the parameters \texttt{degd??} or~$5$. The parameters
\texttt{lastuknot}, \texttt{uknots}, \texttt{lastvknot} i~\texttt{vknots}
are used to output the knot sequences of the B-spline representation.
In the array pointed by the parameter~\texttt{p} the procedure stores the
control points.

The arrays \texttt{unkots}, \texttt{vknots} and~\texttt{p} must be long enough;
their lengths may be computed before the allocation by calling
\texttt{mbs\_FindBSCommonKnotSequencef} for the families of curves
$\bm{c}_{ij}$ and $\bm{d}_{ij}$; the variables pointed by the parameter
\texttt{lastknot} of this procedure must have the initial value~$5$.


\vspace{\bigskipamount}
\cprog{%
boolean mbs\_TabBSC2CoonsDer3f ( int spdimen, \\
\ind{8}int nknu, const float *knu, const float *hfuncu, \\
\ind{8}const float *dhfuncu, const float *ddhfuncu, \\
\ind{8}const float *dddhfuncu, \\
\ind{8}int nknv, const float *knv, const float *hfuncv, \\
\ind{8}const float *dhfuncv, const float *ddhfuncv, \\
\ind{8}const float *dddhfuncv, \\
\ind{8}int degc00, int lastknotc00, const float *knotsc00, \\
\ind{8}const float *c00, \\
\ind{8}int degc01, int lastknotc01, const float *knotsc01, \\
\ind{8}const float *c01, \\
\ind{8}int degc02, int lastknotc02, const float *knotsc02, \\
\ind{8}const float *c02, \\
\ind{8}int degc10, int lastknotc10, const float *knotsc10, \\
\ind{8}const float *c10, \\
\ind{8}int degc11, int lastknotc11, const float *knotsc11, \\
\ind{8}const float *c11, \\
\ind{8}int degc12, int lastknotc12, const float *knotsc12, \\
\ind{8}const float *c12, \\
\ind{8}int degd00, int lastknotd00, const float *knotsd00, \\
\ind{8}const float *d00, \\
\ind{8}int degd01, int lastknotd01, const float *knotsd01, \\
\ind{8}const float *d01, \\
\ind{8}int degd02, int lastknotd02, const float *knotsd02, \\
\ind{8}const float *d02, \\
\ind{8}int degd10, int lastknotd10, const float *knotsd10, \\
\ind{8}const float *d10, \\
\ind{8}int degd11, int lastknotd11, const float *knotsd11, \\
\ind{8}const float *d11, \\
\ind{8}int degd12, int lastknotd12, const float *knotsd12, \\
\ind{8}const float *d12, \\
\ind{8}float *p, float *pu, float *pv, \\
\ind{8}float *puu, float *puv, float *pvv, \\
\ind{8}float *puuu, float *puuv, float *puvv, float *pvvv );}
The procedure \texttt{mbs\_TabBSC2CoonsDer3f} performs a~fast computation
of points and derivatives of order $1$, $2$ and~$3$ of a~biquintic spline
Coons patch, at the points $(u_i,v_j)$, where $i\in\{0,\ldots,k_u-1\}$,
$j\in\{0,\ldots,k_v-1\}$.

The parameter~\texttt{spdimen} specifies the dimension~$d$ of the space with
the patch. The parameters~\texttt{nknu} and~\texttt{nknv} specify the
numbers~$k_u$ and~$k_v$, the arrays~\texttt{knu} and~\texttt{knv} contain
respectively the numbers $u_0,\ldots,u_{k_u-1}$ and $v_0,\ldots,v_{k_v-1}$.
The contents of the arrays \texttt{hfuncu}, \texttt{dhfuncu}, \texttt{ddhfuncu}
and \texttt{dddhfuncu} must be respectively the values of the polynomials
$\tilde{H}_{00},\tilde{H}_{10},\tilde{H}_{01},\tilde{H}_{11},\tilde{H}_{02},\tilde{H}_{12}$
and their derivatives of order~$1$, $2$ and~$3$ at the points
$u_0,\ldots,u_{k_u-1}$. The arrays \texttt{hfuncv}, \texttt{dhfuncv},
\texttt{ddhfuncv} and~\texttt{dddhfuncv} must contain the values of the polynomials
$\hat{H}_{00},\hat{H}_{10},\hat{H}_{01},\hat{H}_{11},\hat{H}_{02},\hat{H}_{12}$
at $v_0,\ldots,v_{k_v-1}$; these values are simplest to obtain by calling
the procedure \texttt{mbs\_TabQuinticHFuncDer3f}.

\begin{sloppypar}
The quadruples of parameters \texttt{degc??}, \texttt{lastknotc??}, \texttt{knotsc??},
\texttt{c??} and \texttt{degd??}, \texttt{lastknotd??}, \texttt{knotsd??}
\texttt{d??} describe the B-spline curves, which define the patch.
These curves must satisfy (up to the rounding errors)
the compatibility conditions~(\ref{eq:Coons:compat:cond}).%
\end{sloppypar}

In the arrays pointed by the parameters \texttt{p}, \texttt{pu}, \texttt{pv},
\texttt{puu}, \texttt{puv}, \texttt{pvv}, \texttt{puuu}, \texttt{puuv},
\texttt{puvv} and \texttt{pvvv} the procedure stores the computed
points and derivatives of order $1$, $2$ and~$3$; if any of the parameters
is~\texttt{NULL}, then the corresponding points or vectors are not computed.
Otherwise the pointed array must have length at least~$k_uk_vd$.

The value returned is \texttt{true} in case of success and \texttt{false}
after failure (caused by insufficient space on the scratch memory stack
or incorrect knot sequences of the curves).


\vspace{\bigskipamount}
\cprog{%
boolean mbs\_TabBSC2Coons0Der3f ( int spdimen, \\
\ind{8}int nknu, const float *knu, const float *hfuncu, \\
\ind{8}const float *dhfuncu, const float *ddhfuncu, \\
\ind{8}const float *dddhfuncu, \\
\ind{8}int nknv, const float *knv, const float *hfuncv, \\
\ind{8}const float *dhfuncv, const float *ddhfuncv, \\
\ind{8}const float *dddhfuncv, \\
\ind{8}int degc00, int lastknotc00, const float *knotsc00, \\
\ind{8}const float *c00, \\
\ind{8}int degc01, int lastknotc01, const float *knotsc01, \\
\ind{8}const float *c01, \\
\ind{8}int degc02, int lastknotc02, const float *knotsc02, \\
\ind{8}const float *c02, \\
\ind{8}int degd00, int lastknotd00, const float *knotsd00, \\
\ind{8}const float *d00, \\
\ind{8}int degd01, int lastknotd01, const float *knotsd01, \\
\ind{8}const float *d01, \\
\ind{8}int degd02, int lastknotd02, const float *knotsd02, \\
\ind{8}const float *d02, \\
\ind{8}float *p, float *pu, float *pv, \\
\ind{8}float *puu, float *puv, float *pvv, \\
\ind{8}float *puuu, float *puuv, float *puvv, float *pvvv );}
\begin{sloppypar}
The procedure \texttt{mbs\_TabBSC2Coons0Der3f} is a~simplified version
of the procedure \texttt{mbs\_TabBSC2CoonsDer3f} for the case,
when the curves $\bm{c}_{10},\bm{c}_{11},\bm{c}_{12},\bm{d}_{10},\bm{d}_{11}$
and~$\bm{d}_{12}$ are
null (i.e.\ when all their control points have all coordinates zero).
Computing points of such a~patch may be done in a~shorter time.%
\end{sloppypar}

The parameters of \texttt{mbs\_TabBSC2Coons0Der3f} are the same as
the parameters of the procedure \texttt{mbs\_TabBSC2CoonsDer3f}
with the same names.


\newpage
\section{Spherical product}

\begin{sloppypar}
The spherical product of two planar parametric curves,
$\bm{p}(t)=[x_{\bm{p}}(t),y_{\bm{p}}(t)]^T$
and~$\bm{q}(t)=[x_{\bm{q}}(t),y_{\bm{q}}(t)]^T$,
is the parametric surface in~$\R^3$, given by
\begin{align*}
  \bm{s}(u,v) = \left[\begin{array}{c}
    x_{\bm{p}}(u)x_{\bm{q}}(v) \\
    y_{\bm{p}}(u)x_{\bm{q}}(v) \\
    y_{\bm{q}}(v)
  \end{array}\right].
\end{align*}
The curves~$\bm{p}$ and~$\bm{q}$ are called respectively equator
and meridian. The procedures described below compute the
control points of the B-spline representation of the spherical product of
planar B-spline curves, piecewise polynomial and piecewise rational
respectively.%
\end{sloppypar}

The knot sequence of the equator is the ,,$u$'' knot sequence
and the knot sequence of the meridian is the ,,$v$'' knot
sequence of the spherical product.

\vspace{\bigskipamount}
\cprog{%
void mbs\_SphericalProductf ( \\
\ind{6}int degree\_eq, int lastknot\_eq, const point2f *cpoints\_eq, \\
\ind{6}int degree\_mer, int lastknot\_mer, const point2f *cpoints\_mer, \\
\ind{6}int pitch, point3f *spr\_cp );}

\vspace{\bigskipamount}
\cprog{%
void mbs\_SphericalProductRf ( \\
\ind{6}int degree\_eq, int lastknot\_eq, const point3f *cpoints\_eq, \\
\ind{6}int degree\_mer, int lastknot\_mer, const point3f *cpoints\_mer, \\
\ind{6}int pitch, point4f *spr\_cp );}

