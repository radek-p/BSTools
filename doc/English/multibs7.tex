
%/* //////////////////////////////////////////////////// */
%/* This file is a part of the BSTools procedure package */
%/* written by Przemyslaw Kiciak.                        */
%/* //////////////////////////////////////////////////// */

\newpage
\section{Drawing trimmed patches}

\subsection{\label{ssect:trimpatch:bound}Domain representation}

The domain of a~trimmed B-spline patch of degree $(n,m)$, with the knots
$u_0,\ldots,u_N$ and $v_0,\ldots,v_M$ is a~subset of the rectangle
$[u_n,u_{N-n}]\times[v_m,v_{M-m}]$. In particular, it is always a~bounded
area. The boundary of the domain is the sum of planar curvilinear closed
polylines. Every such a~polyline may consist of
\begin{itemize}
  \item polylines (sequences of line segments),
  \item B\'{e}zier curves,
  \item B-spline curves,
\end{itemize}
called hereafter the boundary elements.
The points (vertices of the polylines and control points of the curves)
may be given with the cartesian coordinates (then they are of type
\texttt{point2f}) or homogeneous coordinates (in this case
they are of type \texttt{vector3f}).

The data describing every such a~polyline must satisfy the following condition:
the B-spline curves must be continuous and the end point of each boundary
element (polyline or curve) is the first point of the of the element that
follows (the last element is followed by the first one). If this condition
is not satisfied then the procedures of drawing trimmed patches will insert
the appropriate line segments.

Another condition is the absence of points at infinity. It is sufficient
that all weight coordinates are positive, though it is not required.
However, specifying an unbounded patch boundary may cause a~program
execution error. Therefore all polylines and curves must lie in the rectangle
specified as the domain of the untrimmed patch.

\vspace{\medskipamount}
The boundary of the domain is represented with an array containing
structures of type \texttt{polycurvef}.

\vspace{\bigskipamount}
\cprog{%
typedef struct\{ \\
\ind{4}boolean closing; \\
\ind{4}byte\ind{4}spdimen; \\
\ind{4}short\ind{3}degree; \\
\ind{4}int\ind{5}lastknot; \\
\ind{4}float\ind{3}*knots; \\
\ind{4}float\ind{3}*points; \\
\ind{2}\} polycurvef;
}
The attribute \texttt{closing} specifies the boundary element following
the current one. If its value is \texttt{false}, then there is a~new element
after the current one (and it is described by the next structure in the array).
If \texttt{closing} is \texttt{true}, then the end of the current element
should be connected with the beginning of the first element in the array,
or with the last of the preceding elements, preceded by an element
with the \texttt{closing} attribute \texttt{true}
(in this way the curvilinear polygon is closed).

\begin{sloppypar}
The attribute \texttt{spdimen} may be equal to \texttt{2} or \texttt{3}.
In the former case the attribute \texttt{points} points to an array
of structures \texttt{point2f} with the cartesian coordinates of points
in the plane. In the latter case it points to an array of structures
\texttt{vector3f}, with the homogeneous coordinates of points
(the curve, whose control points are represented in this way
is rational, and its representation is homogeneous).
\end{sloppypar}

The attribute \texttt{degree} specifies the degree~$n$ of the curve,
it must not be less than~$1$.

The attribute \texttt{lastknot} specifies the number~$N$ of the last
vertex of the polyline or the last knot of the B-spline curve.

The attribute \texttt{knots} points to an array with the knots of
the B-spline curve, of length $N+1$.

The attribute \texttt{points} points to an array with the vertices of the
polyline or the control polygon. Depending on the value of the attribute
\texttt{spdimen} this array contains pairs of triples of floating point
numbers.

To specify a~\textbf{polyline} consisting of~$N$ line segments,
one should set the attributes
\texttt{degree=1}, \texttt{lastknot=$N$}, \texttt{knots=NULL}. The array
pointed by \texttt{points} must contain \texttt{spdimen*$(N+1)$}
floating point numbers (or $N+1$ structures of type \texttt{point2f} or
\texttt{vector3f}).

To specify a~\textbf{B\'{e}zier curve} of degree $n>0$, the attributes
should be \texttt{degree=$n$}, \texttt{lastknot=-1}, \texttt{knots=NULL}.
The array pointed by \texttt{points} must contain \\
\texttt{spdimen*$(n+1)$}
floating point numbers (or $n+1$ structur of type \texttt{point2f} or
\texttt{vector3f}).

To specify a~\textbf{B-spline curve} of degree $n>0$, one should set up the
attributes \texttt{degree=$n$}, \texttt{lastknot=$N$}. The attribute
\texttt{knots} must point an array with $N+1$ floataing point numbers,
being the knots, and \texttt{points} must point an array with
\texttt{spdimen*$(N-n)$} floating point numbers, the coordinates
of the control points.

It is not required that B-spline curves be represented with clamped knots,
but the curve should be connected with the neighbouring boundary elements.
In particular one can specify a~connected part of the boundary domain
as a~single closed B-spline curve.

\vspace{\medskipamount}
The boundary of the domain may (but it does not have to) be oriented.
One can use the convention that moving along the boundary according to their
natural parameterization one has the inside of the domain on the
left hand (or right hand) side. The drawing procedures must be implemented
in such a~way that the appropriate information is available.
The boundary elements may intersect, and it must not cause execution
errors.

\vspace{\medskipamount}\noindent
\textbf{Example.} The boundary of the domain in Figure~\ref{fig:trimpatch}
is described as follows:%
\begin{figure}[ht]
  \centerline{\epsfig{file=trimpatch.ps}}
  \caption{\label{fig:trimpatch}A~trimmed domain and a~trimmed B-spline patch.}
\end{figure}

\vspace{\medskipamount}
\noindent{\ttfamily
\#define n  3 \\
\#define NNt1a 10 \\
float ut1a[NNt1a+1] = \\
\ind{2}\{-0.5, 0.0, 0.0, 0.0, 1.4, 2.8, 4.2, 5.6, 5.6, 5.6, 6.1\}; \\
point2f cpt1a[NNt1a-n] = \\
\ind{2}\{\{4.0,0.4\},\{3.5,0.4\},\{2.8,0.8\},\{2.6,2.0\},\{2.8,3.2\},\{3.5,3.6\}, \\
\ind{3}\{4.0,3.6\}\}; \\
point2f cpd1a[3] = \{\{4.0,3.6\},\{4.0,4.0\},\{1.2,4.0\}\}; \\
point3f cpt1b[4] = \{\{1.2,4.0,1.0\},\{1.5,3.0,1.0\},\{0.5,2.5,0.75\},\\
\ind{20}\{0.0,2.5,1.0\}\}; \\
point2f cpd1b[4] = \{\{0.0,2.5\},\{0.0,0.0\},\{4.0,0.0\},\{4.0,0.4\}\}; \\
point2f cpd1c[5] = \{\{0.3,1.1\},\{1.6,1.6\},\{2.1,1.1\},\{1.6,0.6\}, \\
\ind{20}\{0.3,1.1\}\}; \\
point3f cpt1c[4] = \{\{2.0,3.3,1.0\},\{0.6,1.65,0.5\},\{0.6,1.25,0.5\}, \\
\ind{20}\{2.0,2.5,1.0\}\}; \\
point3f cpt1d[4] = \{\{2.0,2.5,1.0\},\{1.25,1.25,0.5\},\{1.25,1.5,0.5\}, \\
\ind{20}\{2.0,3.0,1.0\}\}; \\
point2f cpd1e[2] = \{\{2.0,3.0\},\{2.0,3.3\}\}; \\
polycurvef boundary1[8] = \\
\ind{2}\{\{false,2,n,NNt1a,\&ut1a[0],(float*)\&cpt1a[0]\}, /* B-spline */ \\
\ind{3}\{false,2, 1, \ 2, NULL,(float*)\&cpd1a[0]\}, \ /* polyline */ \\
\ind{3}\{false,3, 3, -1, NULL,(float*)\&cpt1b[0]\}, \ /* B\'{e}zier curve */ \\
\ind{3}\{true, 2, 1, \ 3, NULL,(float*)\&cpd1b[0]\}, \ /* polyline */ \\
\ind{3}\{true, 2, 1, \ 4, NULL,(float*)\&cpd1c[0]\}, \ /* polyline */ \\
\ind{3}\{false,3, 3, -1, NULL,(float*)\&cpt1c[0]\}, \ /* B\'{e}zier curve */ \\
\ind{3}\{false,3, 3, -1, NULL,(float*)\&cpt1d[0]\}, \ /* B\'{e}zier curve */ \\
\ind{3}\{true, 2, 1, \ 1, NULL,(float*)\&cpd1e[0]\}\}; /* line segment */
}\vspace{\medskipamount}

The boundary in this example consists of four closed curves.
The first is the ``outer border'' and it consists of four elements:
a~B-spline curve, a~polyline of two line sements, a~rational B\'{e}zier curve
and a~polyline of three line segments. The second curve is one
closed polyline and the third curve consists of two halfcircles
(represented as rational cubic B\'{e}zier curves) and a~polyline
consisting of one line segment.

The \textbf{index} of a~point of a~plane is the number of circulations
(in the counterclockwise direction) of this point along the boundary
according to the orientation of this boundary. The first two of the three
closed curves above are oriented so that moving along these curves we have the
inside of the domain on the left hand side. The third curve has the opposite
orientation. Therefore the index of all points outside the ``outer border''
is~$0$ and the index of the points inside the polygon bounded by the second
curve is~$2$. The domain is defined as the set of points, whose index~is~$1$.


\subsection{Domain boundary compilation}

To draw a~trimmed patch it is necessary to compute many times the common points
of straight lines with the boundary of the domain. To save time the
representation described in the previous section is translated into
a~code, which describes the polylines and B\'{e}zier curves, which form
the boundary.

\vspace{\bigskipamount}
\cprog{%
int  mbs\_TrimCVBoundSizef ( int nelem, const polycurvef *bound );}
The procedure \texttt{mbs\_TrimCVBoundSizef} computes the length
(in bytes) of the code, which describes the boundary of the trimmed
patch domain. The given boundary representation is as described
in the previous section. Its parts are polylines, B\'{e}zier curves
and B-spline curves, described with the elements of the array
\texttt{bound} of length \texttt{nelem}.

This procedure may be used for allocation of the sufficient memory block.

\vspace{\bigskipamount}
\cprog{%
void *mbs\_CompileTrimPatchBoundf ( int nelem, \\
\ind{35}const polycurvef *bound, \\
\ind{35}void *buffer );}
The procedure \texttt{mbs\_CompileTrimPatchBoundf} ``compiles''
the boundary representation, i.e.\ it produces the code representing
the polylines and B\'{e}zier curves (the B-spline curves are replaced
with the appropriate sequences of B\'{e}zier arcs).

The parameter \texttt{nelem} specifies the length of the array \texttt{bound},
whose elements describe the domain boundary. The parameter \texttt{buffer}
points to the array, in which the code is to be stored. The array must be
long enough (one can compute the necessary length using
\texttt{mbs\_TrimCVBoundSizef}). If the parameter
\texttt{buffer} is (\texttt{NULL}),
then the procedure allocates a~long enough memory block from the scratch
memory pool (using \texttt{pkv\_GetScratchMem}).

The procedure returns the pointer to the array with the code
(i.e.\ the given value of the parameter \texttt{buffer} or the address
of the allocated memory block, if \texttt{buffer} was \texttt{NULL}).
It may also return \texttt{NULL}, which indicates an error
(e.g.\ insufficient scratch memory).


\subsection{Line pictures}

\begin{sloppypar}\hyphenpenalty=400
A line picture of a~patch or its domain consists of the curves being the
images of the parts of domain boundary and of lines of constant parameters.
To draw such a~picture one should find these lines, i.e.\ find the
intersections of appropriate straight lines with the domain boundary.
This is done by the procedure \texttt{mbs\_FindBoundLineIntersectionsf},
described below. The ``higher level'' procedure
\texttt{mbs\_DrawTrimBSPatchDomf} generates a~set of straight lines
and it computes their intersections with the domain.
Each such an intersection is output by calling the \textbf{output procedure}
given as a~parameter. That procedure should draw or display in some way
the line segment in the domain or its image (a~piece of a~curve of a~constant
parameter) on the patch.
\end{sloppypar}

\vspace{\bigskipamount}
\cprog{%
typedef struct \{ \\
\ind{4}float t; \\
\ind{4}char \ sign1, sign2; \\
\ind{2}\} signpoint1f;}
The structure of type \texttt{signpoint1f} is used to describe an
intersection point of a~straight line with the domain boundary.
The line is given in the parametric form, and it divides the plane
into two halfplanes. The attribute~\texttt{t} of the structure
is set to the value of the parameter of the line corresponding to
the intersection point, and the attributes \texttt{sign1} and~\texttt{sign2}
describe the orientation of the intersection. Their possible values are
$0$, $-1$ and~$+1$, which correspond to the cases when the initial point
(\texttt{sign1}) or the end point (\texttt{sign2}) of a~small part of the
boundary is located on the line or in one of the two halfplanes.

\vspace{\bigskipamount}
\cprog{%
void mbs\_FindBoundLineIntersectionsf ( const void *bound, \\
\ind{39}const point2f *p0, float t0, \\
\ind{39}const point2f* p1, float t1, \\
\ind{39}signpoint1f *inters, \\
\ind{39}int *ninters );}
\begin{sloppypar}
The procedure \texttt{mbs\_FindBoundLineIntersectionsf} computes the
intersection points of the straight line given by its two
points, \texttt{p0} and~\texttt{p1}, with the boundary of the domain of
the trimmed patch, represented by the code in the array
\texttt{bound} (generated by \texttt{mbs\_CompileTrimPatchBoundf}).
The intersection points are stored in the array \texttt{inters}.
If the boundary has a~common line segment with the straignt line,
then it is represented in the array \texttt{inters} by two elements,
corresponding to the end points of the segment. In that case the
attributes \texttt{sign1} and~\texttt{sign2} of the two elements
are set to~$0$.
\end{sloppypar}

The numbers \texttt{t0} and~\texttt{t1} are the parameters of the straight
line corresponding to the points \texttt{p0} and \texttt{p1}. Both these
points and their corresponding parameters must be different.

The initial value of the parameter \texttt{*ninters} specifies the length
(capacity) of the array \texttt{inters}, i.e.\ the maximal number of
the intersection points that the program expects to find. Upon exit
this parameter has the value of the intersection points found.
In case of error (e.g.\ in the code, or the overflow of the
array~\texttt{inters}), the parameter \texttt{inters} is assigned a~negative
value.

The array \texttt{inters} after finding all the intersection points
is sorted in the order of increasing values of the attributes~\texttt{t}.

\vspace{\bigskipamount}
\cprog{%
void mbs\_DrawTrimBSPatchDomf ( int degu, int lastuknot, \\
\ind{20}const float *uknots, \\
\ind{20}int degv, int lastvknot, \\
\ind{20}const float *vknots, \\
\ind{20}int nelem, const polycurvef *bound, \\
\ind{20}int nu, float au, float bu, \\
\ind{20}int nv, float av, float bv, \\
\ind{20}int maxinters, \\
\ind{20}void (*NotifyLine)(char,int,point2f*,point2f*), \\
\ind{20}void (*DrawLine)(point2f*,point2f*,int), \\
\ind{20}void (*DrawCurve)(int,int,const float*) );}
The procedure \texttt{mbs\_DrawTrimBSPatchDomf} may be used to draw a~line
image of the domain of a~trimmed B-spline patch, or the patch itself.
Its purpose is to generate a~set of line segments in the domain
and to pass each line segment to and output procedure.
The details of further processing of the line segments
(e.g.\ displaying on the screen or writing in a~PostScript file)
is thus kept away from the procedure
\texttt{mbs\_DrawTrimBSPatchDomf}.

The first $8$ parameters of the procedure describe the boundary of the
trimmed B-spline patch. They are: degree~$n$ of the patch with respect to
the parameetr~$u$ (\texttt{degu}), the number~$N$ of the last knot
in the knot sequence $u_0,\ldots,u_N$ (\texttt{lastuknot}), the array with
these knots (\texttt{uknots}), the degree~$m$ of the patch with respect to~$v$
(\texttt{degv}), the number~$M$ of the last knot in the sequence
$v_0,\ldots,v_M$, the array with these knots (\texttt{vknots}), the number of
boundary elements (\texttt{nelem}) and the array \texttt{bound}, whose
elements describe the domain boundary, as described in
Section~\ref{ssect:trimpatch:bound}.

The six parameters that follow specify the set of lines, whose intersections
with the patch boundary are to be found. The set consists of
``vertical'' and ``horizontal'' lines (lines of constant parameters
$u$ and $v$ respectively).

The ``vertical'' lines correspond to the knots $u_n,\ldots,u_{N-n}$
(and thus they have nonempty intersections with the domain)
and in addition to the numbers, which divide each interval $[u_i,u_{i+1}]$,
$i=n,\ldots,N-n-1$, into equal subintervals.
The default number of the subintervals is the value of~\texttt{nu},
but it may be modified so that the length of each subinterval be
not less than the value of the parameter \texttt{au} and not greater than
the value of \texttt{bu}.

In a~similar way the parameters \texttt{nv}, \texttt{av} and~\texttt{bv}
specify the set of ``horizontal'' lines (of constant parameter~$v$),
generated by the procedure.

The parameter \texttt{maxinters} is the maximal expected number of intersection
points od a~single line with the domain boundary. According to its value
the procedure allocates a~buffer for these points. In case of overflow in
this buffer the procedure may fail.

The last three parameters point to the output procedures.
Each of them may be \texttt{NULL}, which causes ignoring the appropriate
result of the computations.

The first output procedure, \texttt{NotifyLine}, is called for each
subsequent ``vertical'' or ``horizontal'' line from the set generated
by the procedure. Its first parameter (of type \texttt{char}) is~$1$
if the line is vertical, and~$2$ if horizontal.
The second parameter specifies the number of the appropriate interval
between the knots, and the next two parameters are the end points of the
intersection of the line with the domain of the untrimmed patch.
For example, if the first parameter is~$1$, and the second is~$k$,
then the line is vertical, i.e.\ it is a~line of constant parameter~$u$,
which is the number from the interval $[u_k,u_{k+1})$.
This number is the value of the $x$ coordinate of the points passed
as the third and the fourth parameter.

The output procedure \texttt{DrawLine} is called after finding the
intersections of the domain boundary with the line, for \emph{each} pair
of consecutive intersection points. The points are passed as the first two
parameters. The third parameter is the index of the points inside the line
segment (see Section~\ref{ssect:trimpatch:bound}). If the boundary
is oriented in such a~way that moving along it according to the orientation
we have the inside of the domain on the left hand side, then this index
will always be~$1$ (for the line segments inside the domain) or~$0$
(for the line segments outside the domain). In general the orientation
of the particular closed curves, which describe the domain boundary,
may be arbitrary. The decision, which line segments are inside
and which are outside the domain is left to the procedure
\texttt{DrawLine} (it may implement e.g.\ the parity rule:
the domain contains the line segments with the index odd).

The procedure \texttt{DrawCurve} is called in order to draw the
elements of the domain boundary. The first parameter is always~$d=2$
or~$3$, to distinguish between planar polynomial and rational (represented
in the homoheneous form) B\'{e}zier curves. The second parameter specifies
the degree~$n$ of the curve (if it is~$1$ then the curve is a~line segment,
the procedure may take advantage of that). The third parameter
is an array with the control points. It contains $(n+1)d$ floating point
numbers, the coordinates of the control points.


\subsection*{Example --- output proceures for line pictures}

Example procedures below were used to draw the domain and the patch
shown in Figure~\ref{fig:trimpatch}, using PostScript.

The picture on the left hand side shows the domain of a~B-spline patch,
i.e.\ the intersections of the constant parameter lines corresponding to
the knots of the patch, and the domain boundary.
The line segments are drawn with the procedure shown below.
It calls some procedure \texttt{MapPoint} in order to map
(scale and translate) the line segments.

\vspace{\medskipamount}
\noindent{\ttfamily
void DrawLine1 ( point2f *p0, point2f *p1, int index ) \\
\{ \\
\ind{2}point2f q0, q1; \\
\ind{2}if ( index == 1 ) \{ \\
\ind{4}ps\_Set\_Line\_Width ( 2.0 ); \\
\ind{4}MapPoint ( frame, p0, \&q0 ); \\
\ind{4}MapPoint ( frame, p1, \&q1 ); \\
\ind{4}ps\_Draw\_Line ( q0.x, q0.y, q1.x, q1.y ); \\
\ind{2}\} \\
\} /*DrawLine1*/
}\vspace{\medskipamount}

The domain boundary has been drawn by the procedure
\texttt{DrawCurve1}, whose shortened version follows
(the full version is in the file \texttt{trimpatch.c}):

\vspace{\medskipamount}
\noindent{\ttfamily
void DrawCurve1 ( int dim, int degree, const float *cp ) \\
\{ \\
\#define DENS 50 \\
\ind{2}int i, size; \\
\ind{2}float t; \\
\ind{2}point2f *c, p; \\
\ind{2}ps\_Set\_Line\_Width ( 6.0 ); \\
\ind{2}if ( degree == 1 ) \{ \\
\ind{4}/* A~B\'{e}zier curve of degree $1$ is a~line segment, so this case */ \\
\ind{4}/* is treated in a~special way.\ The array cp contains 4 */ \\
\ind{4}/* or 6 numbers, the cartesian or homogeneous coordinates */ \\
\ind{4}/* (depending on the value of dim) of the end points.\ */ \\
\ind{4}... \\
\ind{2}\} \\
\ind{2}else /* degree > 1, we draw a~polyline */ \{ \\
\ind{4}if ( c = pkv\_GetScratchMem ( size=(DENS+1)*sizeof(point2f) ) ) \{ \\
\ind{6}if ( dim == 2 ) \{ \\
\ind{8}for ( i = 0; i <= DENS; i++ ) \{ \\
\ind{10}t = (float)i/(float)DENS; \\
\ind{10}mbs\_BCHornerC2f ( degree, cp, t, \&p ); \\
\ind{10}MapPoint ( frame, \&p, \&c[i] ); \\
\ind{8}\} \\
\ind{6}\} \\
\ind{6}else if ( dim == 3 ) \{ \\
\ind{8}for ( i = 0; i <= DENS; i++ ) \{ \\
\ind{10}t = (float)i/(float)DENS; \\
\ind{10}mbs\_BCHornerC2Rf ( degree, (point3f*)cp, t, \&p ); \\
\ind{10}MapPoint ( frame, \&p, \&c[i] ); \\
\ind{8}\} \\
\ind{6}\} \\
\ind{6}else goto out; \\
\ind{6}ps\_Draw\_Polyline ( c, DENS ); \\
out: \\
\ind{6}pkv\_FreeScratchMem ( size ); \\
\ind{4}\} \\
\ind{2}\} \\
\#undef DENS \\
\} /*DrawCurve1*/
}\vspace{\medskipamount}

The call of \texttt{mbs\_DrawTrimBSPatchDomf}, which produced this picture,
looks like this:

\vspace{\medskipamount}
\noindent{\ttfamily
\ind{2}mbs\_DrawTrimBSPatchDomf ( n1, NN1, u1, m1, MM1, v1, 8, boundary1, \\
\ind{28}1, 2.0, 2.0, 1, 2.0, 2.0, \\
\ind{28}20, NULL, DrawLine1, DrawCurve1 );
}\vspace{\medskipamount}

The first $6$ parameters specify the degree and knots, and also the untrimmed
domain, as described before.
The domain is the rectangle $[0,4]\times[0,4]$, and the lengths of
the intervals between the knots are between $1$ and~$1.5$.
Therefore the values of the parameters \texttt{nu}, \texttt{au}, \texttt{bu},
\texttt{nv}, \texttt{av}, \texttt{bv} ensure drawing only the lines of constant
parameters corresponding to the knots.

To draw the picture of the trimmed patch as on the right hand side
of Figure~\ref{fig:trimpatch} one must map the lines in the domain
onto the patch, and then to use the appropriate perspective projection.
The procedure \texttt{DrawLine2} used in this case is the following:

\vspace{\medskipamount}
\noindent{\ttfamily
void DrawLine2 ( point2f *p0, point2f *p1, int index ) \\
\{ \\
\#define LGT 0.05 \\
\ind{2}void     *sp; \\
\ind{2}int      i, k; \\
\ind{2}float    t, d; \\
\ind{2}vector2f v; \\
\ind{2}point2f  q, *c; \\
\ind{2}point3f  p, r; \\
\ind{2}if ( index == 1 ) \{ \\
\ind{4}ps\_Set\_Line\_Width ( 2.0 ); \\
\ind{4}SubtractPoints2f ( p1, p0, \&v ); \\
\ind{4}d = sqrt ( DotProduct2f(\&v,\&v) ); \\
\ind{4}k = (int)(d/LGT+0.5); \\
\ind{4}sp = pkv\_GetScratchMemTop (); \\
\ind{4}c = (point2f*)pkv\_GetScratchMem ( (k+1)*sizeof(point2f) ); \\
\ind{4}for ( i = 0; i <= k; i++ ) \{ \\
\ind{6}t = (float)i/(float)k; \\
\ind{6}InterPoint2f ( p0, p1, t, \&q ); \\
\ind{6}mbs\_deBoorP3f ( n1, NN1, u1, m1, MM1, v1, 3*(MM1-m1), \\
\ind{22}\&cp1[0][0], q.x, q.y, \&p ); \\
\ind{6}PhotoPointUDf ( \&CPos, \&p, \&r ); \\
\ind{6}c[i].x = r.x;  c[i].y = r.y; \\
\ind{4}\} \\
\ind{4}ps\_Draw\_Polyline ( c, k ); \\
\ind{4}pkv\_SetScratchMemTop ( sp ); \\
\ind{2}\} \\
\#undef LGT \\
\} /*DrawLine2*/
}\vspace{\medskipamount}
 
An explanation: instead of a~curve the procedure draws a~polyline.
The number of its line segments depends on the length of the line segment
in the domain of the patch. One can take into account also the shape
of the patch, which is a~bit more complicated. The procedure
\texttt{DrawLine2} has an access to the representation of the patch
(its knots and control points) via global variables. The points of the patch
are computed using the de~Boor algorithm (by the \texttt{mbs\_deBoorP3f}
procedure). One can decrease the computational cost, by specifying with
the parameter \texttt{NotifyLine} a~procedure, whose task would be to
find the B-spline (or even a~piecewise B\'{e}zier) representation of the
curve of constant parameters $u$ or $v$. The procedure specified as
the parameter \texttt{DrawLine} after calling \texttt{NotifyLine}
would then just draw the arcs of this curve of the constant parameter.

