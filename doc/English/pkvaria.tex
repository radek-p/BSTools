
%/* //////////////////////////////////////////////////// */
%/* This file is a part of the BSTools procedure package */
%/* written by Przemyslaw Kiciak.                        */
%/* //////////////////////////////////////////////////// */

\chapter{The \texttt{libpkvaria} library}

The header file with the prototypes of procedures from the \texttt{libpkvaria}
library is \texttt{pkvaria.h}.

\section{Various small gadgets}

\cprog{%
\#define false\ind{2}0 \\
\#define true\ind{3}1 \\
\#define EXP1\ind{3}2.7182818284590452353 \\
\#define PI\ind{5}3.1415926535897932384 \\  
\#define SQRT2\ind{2}1.4142135623730950488 \\
\#define SQRT3\ind{2}1.7320508075688772935 \\  
\mbox{} \\
typedef unsigned char boolean; \\
typedef unsigned char byte;}
\hspace*{\parindent}
For boolean data it is better to use the name \texttt{boolean} than
e.g.\ \texttt{unsigned char} for its type, and to write
\texttt{true} and \texttt{false} instead of \texttt{0} and \texttt{1}.
Preserving this rule in the library procedures is however not quite perfect.

It is good to establish some conventions used in programs. It is known, that
if one thing may be done in a~number of ways, each person will do it
in a~different way. A~team may thus commit a~program with procedures,
whose parameters are specified in inches and meters.
For humans, degrees are more convenient as the angle measure unit.
In the program code --- radians. I am using the convention, that all
angles are processed by the program in radians, and they are input and output
in degrees, though e.g.\ PostScript and OpenGL use a~different convention.

\vspace{\bigskipamount}
\cprog{%
\#define min(a,b) ((a)<(b) ?\ (a) :\ (b)) \\
\#define max(a,b) ((a)>(b) ?\ (a) :\ (b))}
Two extremely useful macros.

\vspace{\bigskipamount}
\cprog{%
double pkv\_rpower ( double x, int e );}
The procedure \texttt{pkv\_rpower} computes $x^e$.

\vspace{\bigskipamount}
\cprog{%
void pkv\_HexByte ( byte b, char *s );}
The procedure \texttt{pkv\_HexByte} finds the hexadecimal representation of
the value of the parameter~\texttt{b}. The hexadecimal digits are
stored in the array~\texttt{s}, whose length must be at least~$3$.


\section{Boxes}

\cprog{%
typedef struct Box2i \{ \\
\mbox{} \ \ \ int  x0, x1, y0, y1; \\
\mbox{} \ \} Box2i; \\
\mbox{} \\
typedef struct Box2s \{ \\
\mbox{} \ \ \ short x0, x1, y0, y1; \\
\mbox{} \ \} Box2s;}

\mbox{} 


\section{\label{sect:scratch:mem}Scratch memory management}

Many procedures of this package use the scratch memory for storing
intermediate results of computations, and the memory blocks are deallocated
in the order reverse to that of allocating. The memory management is
implemented with use of a~stack, which is a~very fast and flexible
method.

The scratch memory pool serviced by the procedures described below
may also be used by other procedures --- the only condition is
creating a~pool large enough at the beginning of the program execution
and using this memory in a~strictly ``stack'' manner.

\vspace{\bigskipamount}
\cprog{%
char pkv\_InitScratchMem ( int size );}
The procedure \texttt{pkv\_InitScratchMem} allocates (with
\texttt{malloc}) a~memory block of size \texttt{size}
bytes and initializes the scratch memory management in this block.
The procedure returns~\texttt{0} if \texttt{malloc} failed and
\texttt{1} if the memory pool has been successfully initialized.

This procedure must be called before calling any procedures using
the scratch memory, i.e.\ which call
\texttt{pkv\_GetScratchMem}, \texttt{pkv\_GetScratchMemTop}, \\
\texttt{pkv\_FreeScratchMem} or \texttt{pkv\_SetScratchMemTop}).

Currently there is no extending of the scratch memory pool if it
turns out to be too small during the program execution.
The programmer therefore has to calculate the size of this pool
large enough for doing the computations.
Some help in doing this is experimenting and calling the procedure
\texttt{pkv\_MaxScratchTaken}.

\vspace{\bigskipamount}
\cprog{%
void pkv\_DestroyScratchMem ( void );}
The procedure \texttt{pkv\_DestroyScratchMem} deallocates (with
\texttt{free}) the scratch memory pool created by a~call to
\texttt{pkv\_InitScratchMem}.

\vspace{\bigskipamount}
\cprog{%
void *pkv\_GetScratchMem ( int size );}
\begin{sloppypar}
The procedure \texttt{mbs\_GetScratchMem} allocates a~memory block of size
\texttt{size} bytes in the scratch memory pool and it returns the
pointer to this block. If the allocation is impossible (because the scratch
memory pool is too small), the procedure returns the empty pointer
(\texttt{NULL}).
\end{sloppypar}

\vspace{\bigskipamount}
\cprog{%
void pkv\_FreeScratchMem ( int size );}
The procedure \texttt{pkv\_FreeScratchMem} deallocates the last \texttt{size}
bytes allocated earlier by calls to \texttt{pkv\_GetScratchMem}.

The memory blocks deallocation is always done in the order reverse to
that of their allocation. It is possible to deallocate a~number of
memory blocks with a~single call to this procedure, by specifying the parameter,
whose value is the sum of sizes of the blocks to be deallocated.

\vspace{\bigskipamount}
\cprog{%
\#define pkv\_GetScratchMemi(size) \bsl \\
\ind{2}(int*)pkv\_GetScratchMem ( (size)*sizeof(int) ) \\
\#define pkv\_FreeScratchMemi(size) \bsl \\
\ind{2}pkv\_FreeScratchMem ( (size)*sizeof(int) ) \\
\#define pkv\_ScratchMemAvaili() \bsl \\
\ind{2}(pkv\_ScratchMemAvail()/sizeof(int)) \\
\mbox{} \\
\#define pkv\_GetScratchMemf(size) \bsl \\
\ind{2}(float*)pkv\_GetScratchMem ( (size)*sizeof(float) ) \\
\#define pkv\_FreeScratchMemf(size) \bsl \\
\ind{2}pkv\_FreeScratchMem ( (size)*sizeof(float) ) \\
\#define pkv\_ScratchMemAvailf() \bsl \\
\ind{2}(pkv\_ScratchMemAvail()/sizeof(float)) \\
\mbox{} \\
\#define pkv\_GetScratchMemd(size) \bsl \\
\ind{2}(double*)pkv\_GetScratchMem ( (size)*sizeof(double) ) \\
\#define pkv\_FreeScratchMemd(size) \bsl \\
\ind{2}pkv\_FreeScratchMem ( (size)*sizeof(double) ) \\
\#define pkv\_ScratchMemAvaild() \bsl \\
\ind{2}(pkv\_ScratchMemAvail()/sizeof(double))}
The macros above may be used to allocate memory blocks for arrays of
floating point numbers. Using them makes the source code shorter and
better readable.

\vspace{\bigskipamount}
\cprog{%
void *pkv\_GetScratchMemTop ( void ); \\
void pkv\_SetScratchMemTop ( void *p );}
An alternative for remembering the number of allocated bytes
(to be deallocated with \texttt{pkv\_FreeScratchMem}) is to remember
the pointer to the end of the allocated area in the pool.
To do this, one may call \texttt{pkv\_GetScratchMemTop} and store
the value returned in a~local variable. Then one or more memory blocks may be
allocated with \texttt{pkv\_GetScratchMem}). Their deallocation is done
by calling \texttt{pkv\_SetScratchMemTop} with the pointer given by
\texttt{pkv\_GetScratchMemTop}.

\vspace{\bigskipamount}
\cprog{%
int pkv\_ScratchMemAvail ( void );}
The value of \texttt{pkv\_ScratchMemAvail} is the current number of available
bytes in the scratch memory pool. An attempt of allocating a~greater block
will fail --- the procedure \texttt{pkv\_GetScratchMem} will return
\texttt{NULL}.

\vspace{\bigskipamount}
\cprog{%
int pkv\_MaxScratchTaken ( void );}
The value of the procedure \texttt{pkv\_MaxScratchTaken} is the greatest number
of bytes allocated at a~certain moment in the scratch memory pool,
since the pool was created (with \texttt{pkv\_InitScratchMem}),
beforre calling this procedure.


\section{Square angle measure}

\cprog{%
double pkv\_SqAngle ( double x, double y );}\mbox{}
\indent
The procedure \texttt{pkv\_SqAngle} computes a~measure of the angle between
the vector $[x,y]$ and the $Ox$ axis. This measure is computed with a couple
of arithmetic operations, which is faster than using the cyclometric
functions. The function values are in the interval $[0,4)$.

The properties of this measure: if two vectors make the right angle,
then the differences of the values of this measure is~$1$. Similarly,
for the straight angle the difference is~$2$. The measure of the full angle is~$4$.


\section{Data exchanging}

\cprog{%
void pkv\_Exchange ( void *x, void *y, int size );}
\hspace*{\parindent}
The procedure \texttt{pkv\_Exchange} swaps the contents of two memory
blocks of size \texttt{size} pointed by the parameters \texttt{x}
and \texttt{y}. The blocks must be disjoint.

\begin{sloppypar}
The procedure uses a~buffer, whose length is not greater than~$1$KB, allocated
with the \texttt{pkv\_GetScratchMem} procedure, therefore to use this
procedure it is necessary to create the large enough scratch memory pool
(by calling \texttt{pkv\_InitScratchMem} at the beginning of the program
execution).
\end{sloppypar}

\vspace{\bigskipamount}
\cprog{%
void pkv\_Sort2f ( float *a, float *b ); \\
void pkv\_Sort2d ( double *a, double *b );}
The procedures \texttt{pkv\_Sort2f} and \texttt{pkv\_Sort2d} swap the
values of the variables \texttt{*a} and \texttt{*b}, if the first of them is
greater than the second.


\newpage
\section{Sorting}

\subsection{CountSort}

The procedures described below sort arrays of structures with numerical
data (keys), integer or floating point. The sorting method depends on the
length of the array. For a~small number, InsertionSort is used.
For longer arrays the CountSort algorithm is used.

The sorting method is stable, i.e. it does not swap the array elements
whose keys have equal values, except that it puts the floating point
$+0.0$s after the $-0.0$s.

\vspace{\bigskipamount}
\cprog{%
\#define ID\_SHORT\ind{2}0 \\
\#define ID\_USHORT 1 \\
\#define ID\_INT\ind{4}2 \\
\#define ID\_UINT\ind{3}3 \\
\#define ID\_FLOAT\ind{2}4 \\
\#define ID\_DOUBLE 5}
The identifiers above denote the possible types of the keys.
The types \texttt{short} and \texttt{unsigned short} are $16$-bit integers.
The types \texttt{int} and \texttt{unsigned int} are $32$-bit integers.
The types \texttt{float} and \texttt{double} are $32$-bit and $64$-bit IEEE-754
floating point numbers. The sorting procedures assume that the byte
ordering is \emph{little-endian} (Intel processors use this byte ordering).

\vspace{\bigskipamount}
\cprog{%
\#define SORT\_OK\ind{8}1 \\
\#define SORT\_NO\_MEMORY 0 \\
\#define SORT\_BAD\_DATA\ind{2}2}
The identifiers above denote possible values of the sorting procedures
described below. If there is no error, the value returned is
\texttt{SORT\_OK}. The other possibilities indicate not enough
scratch memory or invalid data.

\vspace{\bigskipamount}
\cprog{%
char pkv\_SortKernel ( void *ndata, int item\_length, int num\_offset, \\
\ind{22}int num\_type, int num\_data, int *permut );}
The procedure \texttt{pkv\_SortKernel} finds the proper sequence of the
elements in the \texttt{ndata} array, i.e.\ the permutation, which
puts the elements in the sorted (nondecreasing) order.
The \texttt{ndata} array consists of structures of size \texttt{item\_length}
bytes. The key, i.e.\ the integer or floating point number with respect to
which the data are to be sorted, is located in each structure
\texttt{num\_offset} bytes from the structure beginning. The number $n$
of the structures (i.e.\ the length of the \texttt{ndata} array)
is the value of the parametr \texttt{num\_data}.

The array \texttt{permut} contains numbers from $0$ to $n-1$. At the return
(when there is no error), the array \texttt{permut} contains the same numbers,
in the order corresponding to the proper permutation.
The initial ordering of the numbers in this array is important if the data have
to be sorted with respect to a~number of keys. For example, if the structures
consist of two numbers, $x$ and $y$, and the array has to be sorted so that
the $x$-s form a~nondecreasing sequence, and for $x$-s equal the
$y$-s have to form a~nondecreasing sequence, one should initialize the
permut array by filling it with the numbers $0,\ldots,n-1$ (in an arbitrary
order), then call \texttt{pkv\_SortKernel} twice: first to sort the array
with respect to $y$ and then with respect to $x$. Then one can call
\texttt{pkv\_SortPermute} to set the data in the array in the right order.

\vspace{\bigskipamount}
\cprog{%
void pkv\_SortPermute ( void *ndata, int item\_length, int num\_data, \\
\ind{23}int *permut );}
The procedure \texttt{pkv\_SortPermute} permutes the structures in the array
\texttt{ndata} according to the contents of the \texttt{permut} array,
which has to contain the integer numbers from $0$ to $n-1$. The number of
structures $n$ is the value of the parameter \texttt{num\_data}, the length
of the array element (in bytes) is specified by the parameter
\texttt{item\_length}.

\vspace{\bigskipamount}
\cprog{%
char pkv\_SortFast ( void *ndata, int item\_length, int num\_offset, \\
\ind{20}int num\_type, int num\_data );}
The procedure \texttt{pkv\_SortFast} sorts the array \texttt{ndata}, which contains
\texttt{num\_data} structures of size \texttt{item\_length} bytes,
which contain numeric keys of type specified with the parameter
\texttt{num\_type}, located \texttt{num\_offset} bytes from the structure beginning.


\subsection{QuickSort}

The procedure described below sorts elements of a given sequence by
comparisons. It is an implementation of the QuickSort algorithm,
and two basic operations on the sequence---comparing and swapping two
elements---are done by application-supplied procedures given
by parameters. The QuickSort algorithm is not stable, ie.\ the order of
two equal elements after sorting may be changed.

\vspace{\bigskipamount}
\cprog{%
void pkv\_QuickSort ( int n, boolean (*less)(int,int), \\
\ind{21}void (*swap)(int,int) );}
Thew parameter \texttt{n} specifies the length of the sequence to be sorted;
the elements are numbered from~$0$ to $n-1$.
The parameter \texttt{less} points to a procedure, whose value
\texttt{true} indicates that the $i$-th element of the sequence is less that
the $j$-th element (where the numbers $i$ and~$j$ are the values of parameters
of the procedure). The parameter \texttt{swap} points to a procedure swapping the
elements of the sequence indicated by its parameters.


\newpage
\subsection{Heap priority queue}

The priority queue implemented with the procedures described in this section
is an array of pointers to arbitrary objects; both inserting and removing an
object is addind or deleting a~pointer in the array.The priorities are defined
by the application, which ought to supply a~procedure, \texttt{cmp}, with
two pointer parameters; the procedure must return true if the priority
of the object pointed by the first parameter is higher.

\vspace{\bigskipamount}
\cprog{%
int pkv\_UpHeap ( void *a[], int l, boolean (*cmp)(void*,void*) ); \\
int pkv\_DownHeap ( void *a[], int l, int f, \\
\ind{19}boolean (*cmp)(void*,void*) ); \\
int pkv\_HeapInsert ( void *a[], int *l, void *newelem, \\
\ind{21}boolean (*cmp)(void*,void*) ); \\
void pkv\_HeapRemove ( void *a[], int *l, int el, \\
\ind{22}boolean (*cmp)(void*,void*) ); \\
void pkv\_HeapOrder ( void *a[], int n, \\
\ind{21}boolean (*cmp)(void*,void*) ); \\ 
void pkv\_HeapSort ( void *a[], int n, \\ 
\ind{20}boolean (*cmp)(void*,void*) );}   


\newpage
\section{Multidimensional array management}

The procedures for processing curves and surfaces in the \texttt{libmultibs}
library process the control points stored in one-dimensional arrays of
floating point numbers. Such an array may be declared for example as $n$
points in the three-dimensional space, but the memory area occupied by these
data contains $3n$ numbers stored one-by-one.

Two-dimensional arrays of points have a~similar contents and usually
they are used for storing rectangular nets of control points. Such an array
contains the coordinates of the points of the first column of the control net,
then the second etc. The basic parameter, which makes it possible to access
the right places in such an array is the \emph{pitch}, which is the
distance between the beginnings of two consecutive columns. Obviously,
the pitch is irrelevant for one-dimensional arrays (or arrays with only
one column).

From the point of view of the data processing arrays it is better to
interpret them as two-dimensional arrays, without taking care of
the actual number of points and the dimension of the space, whose elements
are these points. The array cosists of \emph{rows} of some fixed length
(usually not greater than the pitch).
After each row there may be an unused area, whose length is the difference
of the pitch and the row length. The procedures and macros described below
may be used to change the pitch, by moving closer or farther the rows,
which changes the length of the unused areas, to copy the data between two
arrays of various pitches, or to move the rows in an array without changing
its pitch.

For special purposes it may be necessary to process arrays of bytes in this
way (i.e.\ the smallest directly addressable memory cells). Therefore the
C procedures are implemented for the arrays of \texttt{char}. The arrays
of \texttt{float} or \texttt{double} are processed with macros, which
multiply the row lengths and pitches by the size of \texttt{float} or
\texttt{double}.

\vspace{\bigskipamount}
\cprog{%
void pkv\_Rearrangec ( int nrows, int rowlen, \\
\ind{22}int inpitch, int outpitch, \\
\ind{22}char *data );}
The procedure \texttt{pkv\_Rearrangec} moves the data in the array in order to
change the pitch. The array consists of \texttt{nrows} rows. Each of them
consists of \texttt{rowlen} bytes. The parameter \texttt{inpitch}
specifies the initial pitch (the distance between the beginnings of consecutive
rows).
The parameter \texttt{outpitch} is the target pitch. Both pitches cannot be
shorter than the row length.
\begin{figure}[ht]
  \centerline{\epsfig{file=memory.ps}}
  \caption{\label{fig:memory:1}The effect of the procedures \texttt{pkv\_Rearrangec}
    and \texttt{pkv\_Selectc}}
\end{figure}

\vspace{\bigskipamount}
\cprog{%
void pkv\_Selectc ( int nrows, int rowlen, \\
\ind{19}int inpitch, int outpitch, \\
\ind{19}const char *indata, char *outdata );}
The procedure \texttt{pkv\_Selectc} copies the data from the array \texttt{indata}
to the array \texttt{outdata}. The data are stored in \texttt{nrows} rows
of length \texttt{rowlen}. The pitch of the \texttt{indata} array is
specified by the parameter \texttt{inpitch}, and the pitch of the
\texttt{outdata} by \texttt{outpitch}. The arrays must occupy disjoint
memory areas.

The effect of the procedure \texttt{pkv\_Selectc} may be illustrated with the same
picture (fig.~\ref{fig:memory:1}) as the \texttt{pkv\_Rearrangec} procedure,
keeping in mind that the data are copied to a~\emph{different} array.

\begin{figure}[ht]
  \centerline{\epsfig{file=memshift.ps}}
  \caption{\label{fig:memory:2}Moving data by the procedure \texttt{pkv\_Movec}}
\end{figure}
\vspace{\bigskipamount}
\cprog{%
void pkv\_Movec ( int nrows, int rowlen, \\
\ind{17}int pitch, int shift, char *data );}
The procedure \texttt{pkv\_Movec} ``moves'' data in the array
by~\texttt{shift} bytes. The parameter \texttt{nrows} specifies the number of
rows, \texttt{rowlen} is the row length, \texttt{pitch} is the array
pitch (which remains unchanged). The parameter
\texttt{data} is a~pointer to the beginning of the firsy row before
moving it. The value of the \texttt{shift} parameter may be positive or
negative.

The contents of the array between the rows, if it is not overwritten by
the row contents moved by \texttt{pkv\_Movec} onto that place, are left
unchanged. This makes it possible to ``extend'' all rows (with shortening
the unused area), in order to make space in the rows for additional elements,
or to remove some elements from all rows (which extends the unused areas).

\newpage
%\vspace{\bigskipamount}
\cprog{%
void pkv\_ZeroMatc ( int nrows, int rowlen, int pitch, char *data );}
The procedure \texttt{pkv\_ZeroMatc} initializes an array of bytes, by
assigning the value~$0$ to all its elements. The contents of the unused areas
is left unchanged.

\vspace{\bigskipamount}
\cprog{%
void pkv\_ReverseMatc ( int nrows, int rowlen, \\
\ind{23}int pitch, char *data );}
The procedure \texttt{pkv\_ReverseMatc} puts the rows of an array of bytes
in the reverse order. The parameters \texttt{nrows} and \texttt{rowlen}
specify the dimensions of this array. The parameter \texttt{pitch} is the pitch
of the array \texttt{data}.

\vspace{\bigskipamount}
\cprog{%
\#define pkv\_Rearrangef(nrows,rowlen,inpitch,outpitch,data) \bsl \\
\ind{2}pkv\_Rearrangec(nrows,(rowlen)*sizeof(float), \bsl \\
\ind{4}(inpitch)*sizeof(float),(outpitch)*sizeof(float),(char*)data) \\
\#define pkv\_Selectf(nrows,rowlen,inpitch,outpitch,indata,outdata) \bsl \\
\ind{2}pkv\_Selectc(nrows,(rowlen)*sizeof(float), \bsl \\
\ind{4}(inpitch)*sizeof(float),(outpitch)*sizeof(float), \bsl \\
\ind{4}(char*)indata,(char*)outdata) \\
\#define pkv\_Movef(nrows,rowlen,pitch,shift,data) \bsl \\
\ind{2}pkv\_Movec(nrows,(rowlen)*sizeof(float),(pitch)*sizeof(float), \bsl \\
\ind{4}(shift)*sizeof(float),(char*)data) \\
\#define pkv\_ZeroMatf(nrows,rowlen,pitch,data) \bsl \\
\ind{2}pkv\_ZeroMatc(nrows,(rowlen)*sizeof(float), \bsl \\
\ind{4}(pitch)*sizeof(float),(char*)data) \\
\#define pkv\_ReverseMatf(nrows,rowlen,pitch,data) \bsl \\
\ind{2}pkv\_ReverseMatc ( nrows, (rowlen)*sizeof(float), \\
\ind{4}(pitch)*sizeof(float), (char*)data ) \\
\mbox{} \\
\#define pkv\_Rearranged(nrows,rowlen,inpitch,outpitch,data) \bsl \\
\ind{2}pkv\_Rearrangec(nrows,(rowlen)*sizeof(double), \bsl \\
\ind{4}(inpitch)*sizeof(double),(outpitch)*sizeof(double),(char*)data) \\
\#define pkv\_Selectd(nrows,rowlen,inpitch,outpitch,indata,outdata) \bsl \\
\ind{2}pkv\_Selectc(nrows,(rowlen)*sizeof(double), \bsl \\
\ind{4}(inpitch)*sizeof(double),(outpitch)*sizeof(double), \bsl \\
\ind{4}(char*)indata,(char*)outdata) \\
\#define pkv\_ZeroMatd(nrows,rowlen,pitch,data) \bsl \\
\ind{2}pkv\_ZeroMatc(nrows,(rowlen)*sizeof(double), \bsl \\
\ind{4}(pitch)*sizeof(double),(char*)data) \\
\#define pkv\_Moved(nrows,rowlen,pitch,shift,data) \bsl \\
\ind{2}pkv\_Movec(nrows,(rowlen)*sizeof(double),(pitch)*sizeof(double), \bsl \\
\ind{4}(shift)*sizeof(double),(char*)data) \\
\#define pkv\_ReverseMatd(nrows,rowlen,pitch,data) \bsl \\
\ind{2}pkv\_ReverseMatc ( nrows, (rowlen)*sizeof(double), \\
\ind{4}(pitch)*sizeof(double), (char*)data )}
The macros above may be used for processing arrays of \texttt{float}
or \texttt{double} in the way described earlier.

The macros \texttt{pkv\_Rearrangef} and \texttt{pkv\_Rearranged} change
the array pitch.

The macros \texttt{pkv\_Selectf} and \texttt{pkv\_Selectd} copy data
between arrays of different pitches.

The macros \texttt{pkv\_Movef} and \texttt{pkv\_Moved} move the rows
in the array.

The macros \texttt{pkv\_ZeroMatf} and \texttt{pkv\_ZeroMatd} initiaalize
the contents of arrays by setting to~$0.0$ all the elements
(floating point numbers --- this is a~trick based on the fact that
all bits of a~floating point $0$ are $0$).

The macros \texttt{pkv\_ReverseMatf} and \texttt{pkv\_ReverseMatd} reverse
the order of rows of floating point arrays.


\vspace{\bigskipamount}
\cprog{%
void pkv\_Selectfd ( int nrows, int rowlen, \\
\ind{20}int inpitch, int outpitch, \\
\ind{20}const float *indata, double *outdata ); \\
void pkv\_Selectdf ( int nrows, int rowlen, \\
\ind{20}int inpitch, int outpitch, \\
\ind{20}const double *indata, float *outdata );}
The above procedures copy data (floating point numbers) between arrays
of elements of different precisions. This is similar to the effect
of using \texttt{pkv\_Selectf}, except that a~conversion between \texttt{float}
and \texttt{double} is done.

The array pitches (i.e.\ distances between the beginnings of consecutive rows)
are expressed in the units being the lengths of \texttt{float} and \texttt{double}
as appropriate.

The procedure \texttt{pkv\_Selectfd} should work correctly for all possible
data (representing numbers). The other procedure may cause the floating
point overflow or underflow. Moreover, there may be rounding errors, which
result from the fact that the set of floats is a~subset of the set of doubles.

\vspace{\bigskipamount}
\cprog{%
void pkv\_TransposeMatrixc ( int nrows, int ncols, int elemsize, \\
\ind{28}int inpitch, const char *indata, \\
\ind{28}int outpitch, char *outdata );}
The procedure \texttt{pkv\_TransposeMatrixc} makes the transposition of
a~matrix $m\times n$. The parameters \texttt{nrows} and \texttt{ncols}
specify the numbers $m$ and $n$ respectively. The size (in bytes)
of the matrix element is the value of the \texttt{elemsize} parameter.
The consecutive rows of the input matrix (whose elements are packed one-by-one)
are given in the \texttt{indata} array, whose pitch (in bytes) is
\texttt{inpitch}. The procedure writes the consecutive rows of the matrix
transposition to the array \texttt{outdata}, whose pitch is \texttt{outpitch}.

\vspace{\bigskipamount}
\cprog{%
\#define pkv\_TransposeMatrixf(nrows,ncols,inpitch,indata, \bsl \\
\ind{4}outpitch,outdata) \bsl \\
\ind{2}pkv\_TransposeMatrixc ( nrows, ncols, sizeof(float), \bsl \\
\ind{4}(inpitch)*sizeof(float), (char*)indata, \bsl \\
\ind{4}(outpitch)*sizeof(float), (char*)outdata ) \\
\#define pkv\_TransposeMatrixd(nrows,ncols,inpitch,indata, \bsl \\
\ind{4}outpitch,outdata) \ldots}
The two macros above may be used to transpose conveniently numeric matrices,
consisting of \texttt{float} or \texttt{double} floating point numbers.
The parameters of those macros correspond to the parameters of the procedure
\texttt{pkv\_TransposeMatrixc} (except of \texttt{elemsize}).
The pitch unit is the size of \texttt{float} or \texttt{double}. 


\newpage
\section{Line segment rasterization}

The procedure of line rasterization is placed in this library, because
so far there is no better place. In future it is desirable to write
a~procedure of polygon rasterization, and if this is further developed,
then making a~separate library with raster graphics routines will be
worth doing.

\vspace{\bigskipamount}
\cprog{%
typedef struct \{ \\
\ind{2}short x, y; \\
\} xpoint;}
The structure \texttt{xpoint} is intended to represent
pixels; it is identical to the \texttt{Xpoint} structure defined in the file
\texttt{Xlib.h}. Due to this, the pixels computed by the line rasterization
procedures (and by the curve rasterization procedures of the \texttt{libmultibs}
library) may be displayed by an XWindow application without further
conversion. On the other hand, due to presence of this definition in the
\texttt{pkvaria.h} file it is not necessary to include \texttt{Xlib.h}
and it is possible to use this procedure in non-XWindow applications.

\vspace{\bigskipamount}
\cprog{%
extern void \ \ (*\_pkv\_OutputPixels)(const xpoint *buf, int n); \\
extern xpoint *\_pkv\_pixbuf; \\
extern int \ \ \ \_pkv\_npix;}
The variables above are used during the rasterization; these are:
pointer to the pixel outputting procedure (which must be supplied by the
application), pixel buffer pointer and pixel buffer counter. The application
should not refer to these variables directly.

\vspace{\bigskipamount}
\cprog{%
\#define PKV\_BUFSIZE 256 \\
\#define PKV\_FLUSH ... \\
\#define PKV\_PIXEL(p,px) ... \\
\#define PKV\_SETPIXEL(xx,yy) ...}
The macros above define the pixel buffer capacity ($256$ causes reserving
$1$KB for that buffer) and implement the buffer servicing. They are made
available in the header file for the needs of the curve rasterization
procedures from the \texttt{libmultibs} library.

\vspace{\bigskipamount}                              
\cprog{%
void \_pkv\_InitPixelBuffer ( void ); \\
void \_pkv\_DestroyPixelBuffer ( void );}
Auxiliary procedures, of which the first allocates the pixel buffer, and
the second deallocates it. The buffer is allocated in the scratch memory pool
(by a~call to \texttt{pkv\_GetScratchMem}), therefore all the scratch memory
allocated after its allocation must be deallocated up to the last byte
before the pixel buffer deallocation.

\vspace{\bigskipamount}                              
\cprog{%
void \_pkv\_DrawLine ( int x1, int y1, int x2, int y2 ); \\
void pkv\_DrawLine ( int x1, int y1, int x2, int y2, \\
\ind{20}void (*output)(const xpoint *buf, int n) );}
The procedure \texttt{\_pkv\_DrawLine} implements the Bresenham algorithm of
line segment rasterization. The procedure assumes that the pixel buffer
has been allocated prior to the call to it
(by \texttt{\_pkv\_InitPixelBuffer}),
and the variable \texttt{\_pkv\_OutputPixels} points to the proper pixel
output procedure (e.g.\ drawing the pixels on the screen).

The procedure intendeded to be called by applications is
\texttt{pkv\_DrawLine}, whose parameters: \texttt{x1}, \texttt{y1},
\texttt{x2}, \texttt{y2} specify the line segment end points,
and the parameter \texttt{output} points to the proper pixel output
routine. The procedure \texttt{pkv\_DrawLine} allocates and initializes the
pixel buffer and assigns the value of the parameter \texttt{output}
to the variable \texttt{\_pkv\_OutputPixels},
then it calls \texttt{\_pkv\_DrawLine}, flushes the buffer and deallocates it.

The procedure pointed by \texttt{output} must have two parameters;
the first is the pointer to the first pixel of the sequence to output,
and the second indicates the number of those pixels. The \texttt{output}
procedure may allocate scratch memory, but it must deallocate it before
the return.


\newpage
\section{Exception handling}

During the program execution there may appear exceptional situations,
and the program must be able to deal with them. A~typical problem
is the memory shortage; if no large enough memory block is available,
the program has to do at least one of the following actions:
\begin{itemize}
\item Halt (by a~call to \texttt{exit}); not doing that would cause
  the program abortion by the operating system because of its improper behaviour,
  i.e.\ an attempt to access memory at a~random address.
\item Inform the user (before halting) about the appearance,
  place and nature of the exceptional situation. Without that the user
  will have no idea of the reason of getting a~(guten) abend.
\item Terminate the computation impossible of the error appearance
  without halting the program. In that case the user should also be
  informed, why the program refused to do something, but it is still at the
  user command with some other services.
\end{itemize}

In case of exceptional situations the procedures of the bstools package
call the \texttt{pkv\_SignalError} procedure. Its default action is
writing (on \texttt{stderr}) a~message and halting the program.
Applications may (by calling the \texttt{pkv\_SetErrorHandler} procedure)
install their own exception handling procedures, which may display
messages in a~dialog box, and which may (with use of the procedures
\texttt{setjmp} and \texttt{longjmp}, see their description in man pages)
abort the unsuccessful computation by terminating a~number of unfinished
procedures and reset the program to some default state.

\vspace{\bigskipamount}
\cprog{%
\#define LIB\_PKVARIA 0 \\
\#define LIB\_PKNUM \ \ 1 \\
\#define LIB\_GEOM \ \ \ 2 \\
\#define LIB\_CAMERA \ 3 \\
\#define LIB\_PSOUT \ \ 4 \\
\#define LIB\_MULTIBS 5 \\
\#define LIB\_RAYBEZ \ 6}
The above symbolic names denote the library with the procedure signalling
the exception. Applications may define its own identifiers, which should
better be different from the above.

\vspace{\bigskipamount}
\cprog{%
void pkv\_SignalError ( \\
\ind{8}int module, int errno, const char *errstr );}
The procedure \texttt{pkv\_SignalError} by default prints a~message
to \texttt{stderr} and halts the program (with a~call to
\texttt{exit~(~1~);}). The message consists of the error number
(internal for the module, i.e.\ the library), being the value of
the \texttt{errno} parameter, the module number (the parameter
\texttt{module}) and the message text (pointed by the parameter
\texttt{errstr}).

If an exception handling procedure is installed (with
\texttt{pkv\_SetErrorHandler}), then the procedure
\texttt{pkv\_SignalError} will call it, passing it its parameters.

\vspace{\bigskipamount}
\cprog{%
void pkv\_SetErrorHandler ( \\
\ind{3}void (*ehandler)( int module, int errno, const char *errstr ) );}
\begin{sloppypar}
The procedure \texttt{pkv\_SetErrorHandler} installs an exception handling
procedure, which henceforth will be called by \texttt{pkv\_SignalError}.
Setting the parameter \texttt{ehandler} the \texttt{NULL} value
causes ``uninstalling'' any previously installed exception handler,
i.e.\ restoring the default action of the \texttt{pkv\_SignalError}
procedure.%
\end{sloppypar}


%\newpage
\section{\label{sect:MALLOC:FREE}Wrappings of \texttt{malloc} and~\texttt{free}}

Some demonstration programs (\texttt{pomnij} and \texttt{pozwalaj})
launch a~child process (using \texttt{fork} and~\texttt{exec}), whose
purpose is to perform the time-consuming numerical computations without
locking the user interaction. In particular it is possible to terminate the
computations before they are complete. If the child process gets the signal
\texttt{SIGUSR1}, it must break the computations
(this is done with \texttt{setjmp} and~\texttt{longjmp}) and free all memory
allocated dynamically in order to prepare for the next job.

The dynamic allocation/deallocation is a~critical phase of the computation;
\texttt{longjmp} is prohibited when \texttt{malloc} or \texttt{free} is
working, and in particular after memory block allocation, but before
assignment of its address to a~variable. Also an application must
register somehow all blocks currently allocated in order to clean up.

To make it possible, \texttt{malloc} and \texttt{free} may be called via the
macros described below. They provide hooks for the application, i.e.\
pointers to procedures to be called when necessary.

\textbf{Remark:} so far not all procedures in the libraries use these
macros.

\vspace{\bigskipamount}
\cprog{%
extern boolean pkv\_critical, pkv\_signal; \\
extern void (*pkv\_signal\_handler)( void ); \\
extern void (*pkv\_register\_memblock)( void *ptr, boolean alloc );}
The variable \texttt{pkv\_signal\_handler} is \texttt{NULL} by default;
an application may assign it the address of a~signal handler, which will
be called at once, except the signall arrived in the critical phase;
then this procedure will be called after return from \texttt{malloc} or
\texttt{free}.

A~signal handler (registered with the \texttt{signal} procedure)
should test the value of the variable \texttt{pkv\_critical}.
If it is \texttt{true}, then only the assignment
\texttt{pkv\_signal = true;} must be done. If the value of \texttt{pkv\_critical}
is \texttt{false}, then the procedure pointed by
\texttt{pkv\_signal\_handler} may be called, and this procedure is allowed
to call \texttt{longjmp}.
If after leaving the critical phase \texttt{pkv\_signal}
is \texttt{true}, the macro calls this procedure (so the signal is processed
later, but it is done).

The variable \texttt{pkv\_register\_memblock}, if not \texttt{NULL},
must point to a~procedure, which is called with the address of each block
allocated or deallocated via the macro \texttt{PKV\_MALLOC}
or~\texttt{PKV\_FREE} (only if \texttt{pkv\_signal\_handler} is not
\texttt{NULL}).

\vspace{\bigskipamount}
\cprog{%
\#define PKV\_MALLOC(ptr,size) \bsl \\
\ind{2}\{ \bsl \\
\ind{4}if ( pkv\_signal\_handler ) \{ \bsl \\
\ind{6}pkv\_signal = false; \bsl \\
\ind{6}pkv\_critical = true; \bsl \\
\ind{6}(ptr) = malloc ( size ); \bsl \\
\ind{6}if ( pkv\_register\_memblock ) \bsl \\
\ind{8}pkv\_register\_memblock ( (void*)(ptr), true ); \bsl \\
\ind{6}pkv\_critical = false; \bsl \\
\ind{6}if ( pkv\_signal ) \bsl \\
\ind{8}pkv\_signal\_handler (); \bsl \\
\ind{4}\} \bsl \\
\ind{4}else \bsl \\
\ind{6}(ptr) = malloc ( size ); \bsl \\
\ind{2}\} \\
\mbox{} \\
\#define PKV\_FREE(ptr) \bsl \\
\ind{2}\{ \bsl \\
\ind{4}if ( pkv\_signal\_handler ) \{ \bsl \\
\ind{6}pkv\_signal = false; \bsl \\
\ind{6}pkv\_critical = true; \bsl \\
\ind{6}free ( (void*)(ptr) ); \bsl \\
\ind{6}if ( pkv\_register\_memblock ) \bsl \\
\ind{8}pkv\_register\_memblock ( (void*)(ptr), false ); \bsl \\
\ind{6}(ptr) = NULL; \bsl \\
\ind{6}pkv\_critical = false; \bsl \\
\ind{6}if ( pkv\_signal ) \bsl \\
\ind{8}pkv\_signal\_handler (); \bsl \\
\ind{4}\} \bsl \\
\ind{4}else \{ \bsl \\
\ind{6}free ( (void*)(ptr) ); \bsl \\
\ind{6}(ptr) = NULL; \bsl \\
\ind{4}\} \bsl \\
\ind{2}\}}
The macro \texttt{PKV\_FREE}, apart from deallocation of a~memory block
pointed by the macro parameter (using \texttt{free}), assings \texttt{NULL}
to this parameter.


\section{Debugging}

\cprog{%
void WriteArrayf ( const char *name, int lgt, const float *tab ); \\
void WriteArrayd ( const char *name, int lgt, const double *tab );}
\hspace*{\parindent}
The two above procedures may be used to produce control printouts
during the program debugging. They print out (to \texttt{stdout})
a~text \texttt{name} and \texttt{lgt} floating point numbers from
the array \texttt{tab}.

\vspace{\bigskipamount}
\cprog{%
void *DMalloc ( size\_t size ); \\
void DFree ( void *ptr );}
These procedures may be called instead of \texttt{malloc} and \texttt{free},
if there is a~suspicion, that the program writes outside of the allocated
memory blocks. The procedure \texttt{DMalloc} allocates (with \texttt{malloc})
a~memory block with additional $16$~bytes, fills it with zeroes,
stores (in the first four bytes) the size and returns the address of the
eighth byte of the allocated block.

The procedure \texttt{DFree} verifies, whether the bytes $4,\ldots,7$ and
the last $8$~bytes of the block to deallocate are~$0$ and it writes out
a~warning.

