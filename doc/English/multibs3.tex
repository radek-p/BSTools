
%/* //////////////////////////////////////////////////// */
%/* This file is a part of the BSTools procedure package */
%/* written by Przemyslaw Kiciak.                        */
%/* //////////////////////////////////////////////////// */

\newpage
\section{Knot insertion and removal}

\subsection{\label{ssect:knot:ins}The Boehm algorithm}

The purpose of the procedure and the macros described below is
to insert a~single knot into the representation of B-spline curves,
using the Boehm algorithm. The representation of the curves is
\emph{modified}, i.e.\ the memory area occupied by the initial representation
(knot sequence and control points) after return contains the new
representation, with the additional knot. If both representations are
necessary, then the original representation should be copied by the
application, and then the copy may be modified.

\vspace{\bigskipamount}
\cprog{%
int mbs\_multiKnotInsf ( int degree, int *lastknot, \\
\ind{24}float *knots, \\
\ind{24}int ncurves, int spdimen, \\
\ind{24}int inpitch, int outpitch, \\
\ind{24}float *ctlpoints, float t );}
The procedure \texttt{mbs\_multiKnotInsf} inserts the knot~$t$ to
the representation of B-spline curves of degree $n=$\texttt{degree}.
In this way a~new representation of those curves is constructed, and it replaces
the original representation. The number~$t$ must be from the interval
$[\mbox{\texttt{knots[degree]}},\mbox{\texttt{knots[lastknot-degree]}}]$.

Initially the parameter \texttt{*lastknot} specifies the index~$N$ of the
last knot of the initial knot sequence; on return it is increased by~$1$,
which indicates the growth of the knot sequence by one number --- the
value of the parameter~\texttt{t}, inserted into the array
\texttt{knots}. Therefore this array must have the capacity at least
\texttt{*lastknot+2}, to accomodate the longer knot sequence.

\begin{sloppypar}
The parameter \texttt{ncurves} specifies the number of curves, and the
parameter \texttt{spdimen} is the dimension~$d$ of the space with the curves.
Each curve is initially represented by $N-n$ points
in the $d$-dimensional space.
The coordinates of those points ($(N-n)d$ numbers)
are stored in the array \texttt{ctlpoints}. The first coordinate of the first
control point of the first curve is at the begining of the array.
As on return the representation of each curve has one control point more,
there are two parameters to describe the pitch, i.e.\ the distance between
the beginnings of representations of two consecutive curves:
\texttt{inpitch} specifies the initial pitch, at least $(N-n)d$,
the parameter \texttt{outpitch} specifies the final pitch, which must not
be less than~$(N-n+1)d$.
\end{sloppypar}

The value returned by the procedure is the number~$k$ of the interval $[u_k,u_{k+1})$
for the initial knot sequence, whose element is the new knot~$t$.
Upon return it is inserted into the array \texttt{knots} at the position
$k+1$ and \texttt{*lastknot} is increased by one.

\vspace{\medskipamount}
\noindent
\textbf{Remark:} To insert a~knot into the representation of a~closed curve,
instead of \texttt{mbs\_multiKnotInsf} one should use the procedure
\texttt{mbs\_multiKnotInsClosedf}.


\vspace{\bigskipamount}
\cprog{%
\#define mbs\_KnotInsC1f(degree,lastknot,knots,coeff,t) \bsl \\
\ind{2}mbs\_multiKnotInsf(degree,lastknot,knots,1,1,0,0,coeff,t) \\
\#define mbs\_KnotInsC2f(degree,lastknot,knots,coeff,t) \bsl \\
\ind{2}mbs\_multiKnotInsf(degree,lastknot,knots,1,2,0,0,coeff,t) \\
\#define mbs\_KnotInsC3f(degree,lastknot,knots,coeff,t) ... \\
\#define mbs\_KnotInsC4f(degree,lastknot,knots,coeff,t) ...}
The four macros above call \texttt{mbs\_multiKnotInsf} in order to insert
a~knot to the representation of \emph{one} scalar spline function or
B-spline curve in the space of dimension $2$, $3$ and $4$.
The parameters must satisfy the conditions given in the description of
the procedure \texttt{mbs\_multiKnotInsf}.

\vspace{\bigskipamount}
\cprog{%
int mbs\_multiKnotInsClosedf ( int degree, int *lastknot, \\
\ind{30}float *knots, \\
\ind{30}int ncurves, int spdimen, \\
\ind{30}int inpitch, int outpitch, \\
\ind{30}float *ctlpoints, float t );}
The procedure \texttt{mbs\_multiKnotInsClosedf} inserts a~knot~$t$ to
the representation of \emph{closed} B-spline curves of degree
\texttt{degree}. It may also be used to insert a~knot to a~closed
B-spline patch (being a~tube or a~torus). The main conputation is done
by the procedure \texttt{mbs\_multiKnotInsf}. After it returns, the result
is further processed in order to restore the periodicity of the curves
representation.

The parameters: \texttt{degree} --- degree of the curves, \texttt{*lastknot}
--- on entry its value is the number of the last knot in the initial sequence,
on return its value is increased by~$1$. The array \texttt{knots} contains
the knot sequences, the initial and final one respectively. The parameter
\texttt{ncurves} specifies the number of curves. The parameter \texttt{spdimen}
specifies the space dimension. The parameters \texttt{inpitch} and \texttt{outpitch}
specify the pitch of the array \texttt{ctlpoints} with the control points,
before and after the knot insertion, see the description of the procedure
\texttt{mbs\_multiKnotInsf}.
The parameter \texttt{t} specifies the new knot, to be inserted.

\vspace{\bigskipamount}
\cprog{%
\#define mbs\_KnotInsClosedC1f(degree,lastknot,knots,coeff,t) \bsl \\
\ind{2}mbs\_multiKnotInsClosedf(degree,lastknot,knots,1,1,0,0,coeff,t) \\
\#define mbs\_KnotInsClosedC2f(degree,lastknot,knots,coeff,t) \bsl \\
\ind{2}mbs\_multiKnotInsClosedf(degree,lastknot,knots,1,2,0,0,coeff,t) \\
\#define mbs\_KnotInsClosedC3f(degree,lastknot,knots,coeff,t) ... \\
\#define mbs\_KnotInsClosedC4f(degree,lastknot,knots,coeff,t) ...}
The four macros above call \texttt{mbs\_multiKnotInsClosedf} in order to
insert a~knot to the representation of \emph{one} periodic spline function
or a~closed B-spline curve in the spaces of dimension $2$, $3$ and $4$.
The parameters must satisfy the condition given in the description
of the procedure \texttt{mbs\_multiKnotInsClosedf}.



\subsection{Removing knots}

This section describes a~procedure of removing a~single knot from the
representation of B-spline curves, and the macros, which make it easier to
use this procedure for a~single curve in the spaces of dimensions
$1$--$4$. The procedure sets up a~system of equations related with two
representations of the curves, with the matrix corresponding to the
change of representation by the Boehm algorithm, and then it solves this
system as a~linear least-squares problem. The curves obtained by
removing a~knot may differ from the original curves.

The knot removal takes place ``at the spot'', i.e.\ the memory area initially
occupied by the given representation, upon return contains a~new, shorter knot
sequence and new control points. If both representations are necessary,
then the application should copy the original representation of the curves
and remove a~knot from the copy.

\vspace{\bigskipamount}
\cprog{%
int mbs\_multiKnotRemovef ( int degree, int *lastknot, \\
\ind{27}float *knots, \\
\ind{27}int ncurves, int spdimen, \\
\ind{27}int inpitch, int outpitch, \\
\ind{27}float *ctlpoints, \\
\ind{27}int knotnum );}
\begin{sloppypar}
The procedure \texttt{mbs\_multiKnotRemovef} removes a~knot from the
representation of B-spline curves of degree \texttt{degree}, located in
the space of dimension \texttt{spdimen}.
The representation is defined for a~knot sequence of length
\texttt{*lastknot+1}, given in the array \texttt{knots}.
The control points of the curves are given in the array \texttt{ctlpoints}.
The parameter \texttt{inpitch} specifies the pitch, i.e.\ the initial distance
between the beginnings of the areas in the array \texttt{ctlpoints} with the
control points of the consecutive curves. The parameter \texttt{outpitch}
specifies the final pitch of this array (rearranged after the knot removal).
\end{sloppypar}

The knot to be removed is indicated by the parameter \texttt{knotnum},
whose value must be from \texttt{degree+1} to
\texttt{lastknot-degree-1}.

The new representation of the curves replaces the initial one
in the arrays \texttt{knots} and \texttt{ctlpoints}. The parameter
\texttt{*lastknot} is decreased by~$1$.

\begin{figure}[b]
  \centerline{\epsfig{file=knotrem.ps}}
  \caption{\label{fig:knotrem}Example of knot removal.}
\end{figure}
If the multiplicity of the knot being removed is equal to $r$ and the
derivative of the curve of order $\mbox{\texttt{degree}}-r+1$ is not
continuous at this knot, then the knot removal will change the curve.
The new control points are computed by solving a~linear least squares
problem, which is a~method of solving an approximation problem
(see example given below).

The value of the procedure is the number $k$, such that the removed
knot is the element of the interval $[u_k,u_{k+1})$ determined by the
\emph{final} knot sequence. If the knot, whose number is
\texttt{knotnum} is less than the next knot in the sequence, then
$k=\texttt{knotnum}-1$, but in general it may not be the case.

\vspace{\medskipamount}
\noindent
\textbf{Remark:} To remove a~knot from the representation of closed curves
one should call \texttt{mbs\_multiKnotRemoveClosedf}.
Using the procedure \texttt{mbs\_multiKnotRemovef} may result in getting
non-closed curves.

\vspace{\bigskipamount}
\cprog{%
\#define mbs\_KnotRemoveC1f(degree,lastknot,knots,coeff,knotnum) \bsl \\
\ind{2}mbs\_multiKnotRemovef(degree,lastknot,knots,1,1,0,0,coeff,knotnum) \\
\#define mbs\_KnotRemoveC2f(degree,lastknot,knots,ctlpoints, \bsl \\
\ind{4}knotnum) \bsl \\
\ind{2}mbs\_multiKnotRemovef(degree,lastknot,knots,1,2,0,0, \bsl \\
\ind{4}(float*)ctlpoints,knotnum) \\
\#define mbs\_KnotRemoveC3f(degree,lastknot,knots,ctlpoints, \bsl \\
\ind{4}knotnum) ... \\
\#define mbs\_KnotRemoveC4f(degree,lastknot,knots,ctlpoints, \bsl \\
\ind{4}knotnum) ...}
The four macros above call \texttt{mbs\_multiKnotRemovef} in order to
remove a~knot from the representation of \emph{one} scalar spline function
or one B-spline curve in the space of dimension $2$, $3$ and $4$.
The parameters must be as described in the description of the procedure
\texttt{mbs\_multiKnotRemovef}.

Example of knot removal for a~planar cubic B-spline curve is shown in
Figure~\ref{fig:knotrem} (see the program \texttt{test/knotrem.c}).
The initial multiplicity of the knot being removed is the degree of the curve
plus two; the curve consists of two disjoint pieces and one of its
control points does not influence its shape. Removing the knot causes
rejecting this point, without changing the curve.

Removing the knot of multiplicity degree plus one causes connecting the
curve pieces --- two control points are replaced by one, their midpoint.
Subsequent knot removal is done by solving the appropriate linear least
squares problems.

\vspace{\bigskipamount}
\cprog{%
int mbs\_multiKnotRemoveClosedf ( int degree, int *lastknot, \\
\ind{33}float *knots, \\
\ind{33}int ncurves, int spdimen, \\
\ind{33}int inpitch, int outpitch, \\
\ind{33}float *ctlpoints, \\
\ind{33}int knotnum );}
\begin{sloppypar}
The procedure \texttt{mbs\_multiKnotRemoveClosedf} removes knots from
representations of closed B-spline curves.
\end{sloppypar}

The parameters \texttt{degree}, \texttt{ncurves} and~\texttt{spdimen} specify
the degree and number of curves and space dimension respectively.
The parameters \texttt{*lastknot} and~\texttt{knots} initially describe
the initial knot sequence. After return these parameters describe the final
knot sequence.
The parameter \texttt{knotnum} specifies the number of knot to be removed.
The parameters \texttt{inpitch} and \texttt{outpitch} specify the initial and
final pitch of the array of control points,
\texttt{ctlpoints}, i.e.\ the distances between the beginnings
of control polygons of consecutive curves. The array \texttt{ctlpoints}
initially contains the control points of the initial representation of the
curves; the procedure replaces them by the control points of the final
representation, obtained by removing the knot.

\vspace{\bigskipamount}
\cprog{%
\#define mbs\_KnotRemoveClosedC1f(degree,lastknot,knots,coeff, \bsl \\
\ind{4}knotnum) \bsl \\
\ind{2}mbs\_multiKnotRemoveClosedf(degree,lastknot,knots,1,1,0,0,coeff,
\bsl \\
\ind{4}knotnum) \\
\#define mbs\_KnotRemoveClosedC2f(degree,lastknot,knots,ctlpoints, \bsl \\
\ind{4}knotnum) \bsl \\
\ind{2}mbs\_multiKnotRemoveClosedf(degree,lastknot,knots,1,2,0,0, \bsl \\
\ind{4}(float*)ctlpoints,knotnum) \\
\#define mbs\_KnotRemoveClosedC3f(degree,lastknot,knots,ctlpoints, \bsl \\
\ind{4}knotnum) ... \\
\#define mbs\_KnotRemoveClosedC4f(degree,lastknot,knots,ctlpoints, \bsl \\
\ind{4}knotnum) ...}
The four macros above call \texttt{mbs\_multiKnotRemoveClosedf} in order to
remove the indicated knot from the representation of \emph{one} periodic
spline function or closed B-spline curve in the space of dimension
two, three and four. The parameters are described with the procedure
\texttt{mbs\_multiKnotRemoveClosedf}.


\vspace{\bigskipamount}
\cprog{%
void mbs\_multiRemoveSuperfluousKnotsf ( int ncurves, \\
\ind{40}int spdimen, int degree, \\
\ind{40}int *lastknot, \\
\ind{40}float *knots, \\
\ind{40}int inpitch, int outpitch, \\
\ind{40}float *ctlpoints );}
\begin{sloppypar}
The procedure \texttt{mbs\_multiRemoveSuperfluousKnotsf} removes knots
from the representation of B-spline curves in such a~way, that all
the remaining knots have multiplicity of the degree (the parameter
\texttt{degree}) plus one. The curves are not changed, but problems caused
by the presence of such knots, may be avioded (e.g.\
a~B-spline function, whose all knots are the same, is the zero function,
hence any set of B-spline functions with a~knot of multiplicity
greater than $n+1$ is not a~basis).
\end{sloppypar}

The computation is done ,,at the spot'', i.e.\ the area initially occupied
by the initial representation, after return contains the new representation
of the curves. The knots are removed by moving the data (knots and control
points) in the arrays, without any
numerical computations.


\subsection{The Oslo algorithm}

The Oslo algorithm is a~method of finding a~representation of B-spline curves
corresponding to a~knot sequence $\hat{u}_0,\ldots,\hat{u}_{\hat{N}}$
given a~representation based on a~subsequence $u_0,\ldots,u_N$.
As opposed to the Boehm algorithm (see Section~\ref{ssect:knot:ins}),
which inserts one knot at a~time (and which may be used a~number of times
if necessary), here all knots are inserted at the same time.

If the control points $\bm{d}_i$ of a~B-spline curve of degree~$n$
correspond to the knots $u_0,\ldots,\allowbreak u_N$, and the control points
$\hat{\bm{d}}_l$ correspond to the knots
$\hat{u}_0,\ldots,\hat{u}_{\hat{N}}$, then
\begin{align}
  \hat{\bm{d}}_l = \sum_{i=0}^{N-n-1}a_{il}^n\bm{d}_i,
\end{align}
where the coefficients $a^{n}_{kl}$ are given by the recursive
formulae
\begin{align}\label{eq:Oslo:0}
  a^0_{kl} &{}= \left\{\begin{array}{ll}1 & \mbox{for $u_k\leq \hat{u}_l<u_{k+1}$,} \\
    0 & \mbox{else,} \end{array}\right. \\
  \label{eq:Oslo:j}
  a^n_{il} &{}= \frac{\hat{u}_{l+n}-u_i}{u_{i+n}-u_i} a^{n-1}_{il} +
    \frac{u_{i+n+1}-\hat{u}_{l+n}}{u_{i+n+1}-u_{i+1}} a^{n-1}_{i+1,l}.
\end{align}
The implementation of the Oslo algorithm in the library \texttt{libmultibs}
is such that initially the matrix~$A$, whose coefficients are
$a^n_{il}$, is computed and then it is multiplied by the matrix of the
control points $\bm{d}_0,\ldots,\bm{d}_{N-n-1}$.
\begin{figure}[htb]
  \centerline{\epsfig{file=oslo.ps}}
  \caption{Inserting knots with the Oslo algorithm.}
\end{figure}

The matrix~$A$ makes it possible also to remove a~number of knots at
a~time, by solving an over-definite system of linear equations
(with more equations than unknowns).
Such a~system, even when it is consistent, is solved best as a~linear least
squares problem.

The matrix~$A$ is represented as a~band matrix, by an array with its profile
(i.e.\ table of positions of nonzero coefficients in consecutive
columns) and an array with nonzero coefficients. A~detailed description
of this representation and the related procedures is in
Section~\ref{sect:band:matrix}.

\vspace{\bigskipamount}
\cprog{%
boolean mbs\_OsloKnotsCorrectf ( int lastuknot, const float *uknots, \\
\ind{30}int lastvknot, const float *vknots );}
\begin{sloppypar}
The procedure \texttt{mbs\_OsloKnotsCorrectf} verifies, whether two given
sequences of knots make it possible to construct the matrix~$A$. The
conditions verified are as follows: both sequences are nondecreasing,
and the first sequence of ($\mathord{\mbox{\texttt{lastuknot}}}+1$
numbers, given in the array \texttt{uknots}), is a~subsequence of the
second sequence of length ($\mathord{\mbox{\texttt{lastvknot}}}+1$,
given in the array \texttt{vknots}). If these conditions are satisfied,
the procedure returns \texttt{true}, otherwise it returns \texttt{false}.
\end{sloppypar}

\vspace{\bigskipamount}
\cprog{%
int mbs\_BuildOsloMatrixProfilef ( int degree, \\
\ind{32}int lastuknot, const float *uknots, \\
\ind{32}int lastvknot, const float *vknots, \\
\ind{32}bandm\_profile *prof );}
The procedure \texttt{mbs\_BuildOsloMatrixProfilef}, given the degree
of the representation (\texttt{degree}) and two knot sequences
(see description of the procedure \texttt{mbs\_OsloKnotsCorrectf} above),
constructs the profile of the matrix of the representation transformation.
The profile is stored in the array \texttt{prof}, whose length must be
at least
$\mathord{\mbox{\texttt{lastuknot}}}-\mathord{\mbox{\texttt{degree}}}+1$
(the number of columns plus one).

The procedure returns the number of nonzero coefficients of the matrix,
which is the length of the array to be allocated for storing the coefficients.

\vspace{\bigskipamount}
\cprog{%
void mbs\_BuildOsloMatrixf ( int degree, int lastuknot, \\
\ind{28}const float *uknots, \\
\ind{28}const float *vknots, \\
\ind{28}const bandm\_profile *prof, float *a );}
The procedure \texttt{mbs\_BuildOsloMatrixf} computes the coefficients of the
matrix of curve representation transformation, using the Oslo algorithm.
The parameters are:
\texttt{degree} --- degree of the curve, \texttt{uknots} --- array
of knots of the initial representation, of length
$\mathord{\mbox{\texttt{lastuknot}}}+1$, \texttt{vknots} --- array
of knots of the final representation, whose length is determined
based on the contents of the arrays, therefore there is no parameter
to specify it. The knot sequences have to satisfy the conditions
verified by the procedure \texttt{mbs\_OsloKnotsCorrectf}.

The array \texttt{prof} contains the description of the matrix structure (the
profile), which has to be found earlier, with the procedure
\texttt{mbs\_BuildOsloMatrixProfilef}. The coefficients
computed by the \texttt{mbs\_BuildOsloMatrixf} are stored in the
array \texttt{a}, whose length has been computed by
\texttt{mbs\_BuildOsloMatrixProfilef}.

\vspace{\bigskipamount}
\cprog{%
void mbs\_multiOsloInsertKnotsf ( int ncurves, int spdimen, \\
\ind{28}int degree, \\
\ind{28}int inlastknot, const float *inknots, \\
\ind{28}int inpitch, float *inctlpoints, \\
\ind{28}int outlastknot, const float *outknots, \\
\ind{28}int outpitch, float *outctlpoints );}
\begin{sloppypar}
The procedure \texttt{mbs\_multiOsloInsertKnotsf} inserts a~number of knots
to the representation of \texttt{ncurves} B-spline curves located in the space
of dimension \texttt{spdimen}.
The degree of the curves is specified by the parameter \texttt{degree}.
The initial representation consists of the knots given in the array
\texttt{inknots} (of length $\mathord{\mbox{\texttt{inlastknot}}}+1$)
and the control polygons stored in the array \texttt{inctlpoints},
whose pitch is given by \texttt{inpitch}.

The final representation is based on the knot sequence of length
$\mathord{\mbox{\texttt{outlastknot}}}+1$, given in the array
\texttt{outknots}, and the initial knot sequence must be a~subsequence of
this sequence.
\end{sloppypar}

The procedure sets up the appropriate matrix with the Oslo algorithm,
and then it multiplies it by the matrix of the given control points
of the curves.

If the values of the parameters \texttt{inlastknot} and~\texttt{outlastknot}
are the same, then the procedure assumes that the knot sequences are
identical (which is \emph{not} verified) only copies data from the array
\texttt{inctlpoints} to \texttt{outctlpoints} (according to the pitches
of the arrays, specified by the parameters \texttt{inpitch}
and~\texttt{outpitch} respectively).

\vspace{\bigskipamount}
\cprog{%
void mbs\_multiOsloRemoveKnotsLSQf ( int ncurves, int spdimen, \\
\ind{28}int degree, \\
\ind{28}int inlastknot, const float *inknots, \\ 
\ind{28}int inpitch, float *inctlpoints, \\ 
\ind{28}int outlastknot, const float *outknots, \\ 
\ind{28}int outpitch, float *outctlpoints );}
The procedure \texttt{mbs\_multiOsloRemoveKnotsLSQf} removes a~number
of knots from the representation of given \texttt{ncurves} B-spline curves
located in the space of dimension \texttt{spdimen}.
The degree of the curves is the value of the parameter \texttt{degree}.
The initial representation constsis of the knot sequence stored in the
array \texttt{inknots} (of length $\mathord{\mbox{\texttt{inlastknot}}}+1$)
and the control polygons stored in the array \texttt{inctlpoints},
whose pitch is \texttt{inpitch}.

The final representation is based on the knot sequence of length
$\mathord{\mbox{\texttt{outlastknot}}}+1$, given in the array
\texttt{outknots}, and this sequence must be a~subsequence of
the initial sequence. Moreover, no knot of the final knot sequence may have
multiplicity greater than $\mathord{\mbox{\texttt{degree}}}+1$ (otherwise
the matrix described above would have columns linearly dependent).

The procedure constructs the appropriate matrix, and then it solves a~linear
least squares problem with this matrix.

If the parameters \texttt{inlastknot} and~\texttt{outlastknot}
have the same value, then the procedure assumes that the knot
sequences are identical (which is not verified) and it only copies
data from the array \texttt{inctlpoints} to \texttt{outctlpoints}
according to the pitches of the arrays, specified by the parameters
\texttt{inpitch} and~\texttt{outpitch} respectively).


\subsection{\label{ssect:max:knot:ins}Maximal knot insertion}

The procedures described in this section may be used to insert knots
into the representation of B-spline curves and patches in such a~way,
that the multiplicity of each knot be equal to the degree plus one.
In this way a~particular B-spline representation is obtained; it consists
of the representations of polynomial arcs in local Bernstein bases,
i.e. a~piecewise B\'{e}zier representation. Such a~representation
makes it possible e.g.\ to quickly compute points of the curves
(with the Horner scheme) and algebraic operations (like multiplication)
on spline functions and curves. The procedures described here do not
remove unnecessary knots (of multiplicity greater than degree$+1$).
The procedures which do remove the unnecessary knots (thus producing
a~``clean'' result) are described in the next section.

\vspace{\bigskipamount}
\cprog{%
void mbs\_multiMaxKnotInsf ( int ncurves, int spdimen, int degree, \\
\ind{28}int inlastknot, const float *inknots, \\
\ind{28}int inpitch, const float *inctlpoints, \\
\ind{28}int *outlastknot, float *outknots, \\
\ind{28}int outpitch, float *outctlpoints, \\
\ind{28}int *skipl, int *skipr );}
\begin{sloppypar}
The procedure \texttt{mbs\_multiMaxKnotInsf} inserts knots to the
representation of \texttt{ncurves} B-spline curves of degree \texttt{degree}
in the space of dimension \texttt{spdimen}.

The initial representation of the curves is given by the parameters
\texttt{inlastknot} (number of the last knot),
\texttt{inctlpoints} (array with the knots),
\texttt{inpitch} (pitch, i.e.\ the distance between the beginnings of
the given control polylines),
and \texttt{inctlpoints} (array with the control points).

The procedure constructs the representation of the curves corresponding
to the knot sequence with all internal knots (see Section~\ref{ssect:BSC})
of multiplicity $\mathord{\makebox{\texttt{degree}}}+1$, and the
boundary knots have multiplicity \texttt{degree} or
$\mathord{\makebox{\texttt{degree}}}+1$.

The multiplicities of the extremal knots remain unchanged, therefore
the resulting representation may contain unnecessary knots and control
points. The parameters \texttt{*skipl} and \texttt{*skipr} upon return
indicate the number of unnecessary knots and control points
from the left side and the right side respectively.

The new representation is stored in the arrays
\texttt{outknots} (knot sequence, the index of the last knot is assigned
to the parameter \texttt{*outlastknot}) and \texttt{outctlpoints}
(control polygons, the pitch of this array is specified by the
\emph{input} parameter \texttt{outpitch}).

If the initial knot sequence contains knots of multiplicity greater than
desired, the procedure begins the computations with removing them
(from a~copy of the data), with use of the procedure
\texttt{mbs\_multiRemoveSuperfluousKnots}).
Then the procedure \texttt{mbs\_multiOsloInsertKnotsf}) is called
in order to insert the knots using the Oslo algorithm.

The lengths of the arrays necessaty to accomodate the new representation
of the curves may be found with use of the procedure
\texttt{mbs\_LastknotMaxInsf}, which computes the index of the last knot
of the new representation.
\end{sloppypar}


\vspace{\bigskipamount}
\cprog{%
\#define mbs\_MaxKnotInsC1f(degree,inlastknot,inknots,incoeff, \bsl \\
\ind{4}outlastknot,outknots,outcoeff,skipl,skipr) \bsl \\
\ind{2}mbs\_multiMaxKnotInsf(1,1,degree,inlastknot,inknots,0,incoeff, \bsl \\
\ind{4}outlastknot,outknots,0,outcoeff,skipl,skipr) \\
\#define mbs\_MaxKnotInsC2f(degree,inlastknot,inknots,inctlpoints, \bsl \\
\ind{4}outlastknot,outknots,outctlpoints,skipl,skipr) \bsl \\
\ind{2}mbs\_multiMaxKnotInsf(1,2,degree,inlastknot,inknots,0, \bsl \\
\ind{4}(float*)inctlpoints,outlastknot,outknots,0, \bsl \\
\ind{4}(float*)outctlpoints,skipl,skipr) \\
\#define mbs\_MaxKnotInsC3f(degree,inlastknot,inknots,inctlpoints, \bsl \\
\ind{4}outlastknot,outknots,outctlpoints,skipl,skipr) ... \\
\#define mbs\_MaxKnotInsC4f(degree,inlastknot,inknots,inctlpoints, \bsl \\
\ind{4}outlastknot,outknots,outctlpoints,skipl,skipr) ...}
The four macros above call \texttt{mbs\_multiMaxKnotInsf} in order to
construct the representation of one scalar spline function or a~B-spline
curve in the space of dimension $2$, $3$, $4$, with all internal knots
of multiplicity equal to the degree plus one. The parameters are
described with the procedure \texttt{mbs\_multiMaxKnotInsf}.


\subsection{Conversion of curves and patches to the piecewise B\'{e}zier form}

The procedures described below convert B-spline curves and patches to the
piecewise B\'{e}zier form, using the procedure
\texttt{mbs\_multiMaxKnotInsf}. They may help to draw curves and patches.

\vspace{\bigskipamount}
\cprog{%
void mbs\_multiBSCurvesToBezf ( int spdimen, int ncurves, \\
\ind{31}int degree, int lastinknot, \\
\ind{31}const float *inknots, \\
\ind{31}int inpitch, const float *inctlp, \\
\ind{31}int *kpcs, int *lastoutknot, \\
\ind{31}float *outknots, \\
\ind{31}int outpitch, float *outctlp );}
The procedure \texttt{mbs\_multiBSCurvesToBezf} converts
\texttt{ncurves} B-spline curves of degree \texttt{degree}, located in the
space of dimension \texttt{spdimen} to the piecewise B\'{e}zier form.

The parameters, which describe the given representation are
\texttt{lastinknot} (index of the last knot), \texttt{inknots}
(array with the knot sequence),
\texttt{inpitch} and~\texttt{inctlp} (pitch and the array with the control
points).

The value of \texttt{*kpcs} upon return from the proceure is equal to
the number of polynomial arcs of each curve.
The parameter \texttt{*lastoutknot} is the index of the last knot
of the sequence of the resulting representation, the array
\texttt{outknots} contains these knots, the \emph{input} parameter
\texttt{outpitch} specifies the pitch of the array \texttt{outctlp},
in which the procedure stores the control points of the B\'{e}zier
representations of the polynomial arcs.
More precisely, to each of the B-spline curves there correspond
\texttt{*kpcs*(degree+1)*spdimen} floating point numbers; each polynomial
arc is represented by \texttt{(degree+1)*spdimen} consecutive numbers
(coordinates of \texttt{degree+1} points); the value of the parameter
\texttt{outpitch} is the distance between the beginnings of
the representations of the first arcs of the consecutive B-spline curves.

If the parameter \texttt{kpcs}, \texttt{lastoutknot} or \texttt{outknots}
is \texttt{NULL}, then the procedure does not output the corresponding
information.


\vspace{\bigskipamount}
\cprog{%
\#define mbs\_BSToBezC1f(degree,lastinknot,inknots,incoeff,kpcs, \bsl \\
\ind{4}lastoutknot,outknots,outcoeff) \bsl \\
\ind{2}mbs\_multiBSCurvesToBezf(1,1,degree,lastinknot,inknots,0,incoeff,\bsl \\
\ind{4}kpcs,lastoutknot,outknots,0,outcoeff) \\
\#define mbs\_BSToBezC2f(degree,lastinknot,inknots,inctlp,kpcs, \bsl \\
\ind{4}lastoutknot,outknots,outctlp) \bsl \\
\ind{2}mbs\_multiBSCurvesToBezf(2,1,degree,lastinknot,inknots,0, \bsl \\
\ind{4}(float*)inctlp,kpcs,lastoutknot,outknots,0,(float*)outctlp) \\
\#define mbs\_BSToBezC3f(degree,lastinknot,inknots,inctlp,kpcs, \bsl \\
\ind{4}lastoutknot,outknots,outctlp) ... \\
\#define mbs\_BSToBezC4f(degree,lastinknot,inknots,inctlp,kpcs, \bsl \\
\ind{4}lastoutknot,outknots,outctlp) ...}
\begin{sloppypar}
The above macros call \texttt{mbs\_multiBSCurvesToBezf} in order to
obtain the piecewise B\'{e}zier representation of \emph{one} spline function or
a~B-spline curve in the space of dimension $2$, $3$ and $4$. The
parameters are described with the procedure
\texttt{mbs\_multiBSCurvesToBezf}.
\end{sloppypar}


\vspace{\bigskipamount}
\cprog{%
void mbs\_BSPatchToBezf ( int spdimen, \\
\ind{25}int degreeu, int lastuknot, \\
\ind{25}const float *uknots, \\
\ind{25}int degreev, int lastvknot, \\
\ind{25}const float *vknots, \\
\ind{25}int inpitch, const float *inctlp, \\
\ind{25}int *kupcs, int *lastoutuknot, \\
\ind{25}float *outuknots, \\
\ind{25}int *kvpcs, int *lastoutvknot, \\
\ind{25}float *outvknots, \\
\ind{25}int outpitch, float *outctlp );}
\begin{sloppypar}
The procedure \texttt{mbs\_BSPatchToBezf} finds the representation of
a~B-spline patch with knots of multiplicities equal to the degree plus one,
for both parameters. In other words, it is a~piecewise B\'{e}zier representation.
Such a~represnetation may be convenient when the patch is to be drawn.
The procedure may process patches with clamped boundary as well as with
free boundary.
\end{sloppypar}

The parameter \texttt{spdimen} specifies the space dimension.
The degree of the patch is given by the parameters
\texttt{degreeu} and \texttt{degreev}, the knot sequences of the given
representation are given by the parameters
\texttt{lastuknot}, \texttt{uknots}, \texttt{lastvknot} and \texttt{vknots},
and the control points are given in the array \texttt{inctlp}, whose pitch
(distance between the beginnings of consecutive columns of the control net)
specified with the parameter \texttt{inpitch}.

\begin{sloppypar}
The parameters \texttt{*kupcs} and \texttt{*kvpcs} may be used to pass
the information about the number of polynomial pieces of the patch;
the patch consists of \texttt{*kupcs} ``strips'', each of which consisting
of \texttt{*kvpcs} polynomial patches. The parameters \texttt{lastoutuknot},
\texttt{outuknots}, \texttt{lastvknot} and \texttt{outvknots} are used to
output the knot sequences of the final representation. If any of the four
parameters is \texttt{NULL}, then the corresponding information is not
output by the procedure (it is unnecessary for drawing the patch).%
\end{sloppypar}

\begin{sloppypar}
The control points of the final patch representation are stored in the array
\texttt{outctlp}, whose pitch is \texttt{outpitch} (it is an~\emph{input}
parameter). The pitch should be greater than 
$d(m+1)k_v$, where~$d$ is the dimension of the space (the value of the
parameter \texttt{spdimen}), $m$~(the value of \texttt{degreev}) is the degree
of the patch with respect to~$v$, and $k_v$ is the number of intervals between
knots in the interval $[v_m,v_{M-m}]$
(the number~$M$ is the value of the parameter \texttt{lastvknot}).
The number $k_v$ is assigned by the procedure \texttt{mbs\_BSPatchToBezf}
to the parameter $\texttt{kvpcs}$, but it may be obtained before calling it,
with the \texttt{mbs\_NumKnotIntervalsf} procedure.

The number of columns of the final representation of the patch is $(n+1)k_u$,
where $n$~is the degree of the patch with respect to~$u$, and $k_u$ is
the number of ``strips'', of which the patch consists.
It may also be computed earlier, by calling \texttt{mbs\_NumKnotIntervalsf}.
\end{sloppypar}

\begin{sloppypar}
The main computation (mainly knot insertion) is done by the procedure
\texttt{mbs\_multiMaxKnotInsf}.%
\end{sloppypar}


\vspace{\medskipamount}\noindent
\textbf{Example.}
Suppose that the patch is defined by Formula~(\ref{eq:BSpatch:def}) with two
nondecreasing knot sequences, $u_0,\ldots,u_N$ and~$v_0,\ldots,v_M$,
stored respectively in the arrays \texttt{u} and~\texttt{v}.
The degree is $n$ with respect to $u$ and $m$ with respect to $v$.
The control points in the $d$-dimensional space are orgaanized in the columns
and stored in the array \texttt{cp}. The $i$-th column, for
$i\in\{0,\ldots,N-n-1\}$, consists of $M-m$ points, therefore it is
represented by $(M-m)d$ floating point answers.

\vspace{\medskipamount}
\noindent{\ttfamily
\ind{2}ku = mbs\_NumKnotIntervalsf ( $n$, $N$, u ); \\
\ind{2}kv = mbs\_NumKnotIntervalsf ( $m$, $M$, v ); \\
\ind{2}pitch = $(m+1)d$*kv; \\
\ind{2}b = pkv\_GetScratchMemf ( pitch*ku*(n+1) ); \\
\ind{2}mbs\_BSPatchToBezf ( $d$, $n$, $N$, u, $m$, $M$, v, $d$*($M-m$), cp, \\
\ind{22}\&ku, NULL, NULL, \&kv, NULL, NULL, pitch, b );}
\vspace{\medskipamount}

After executing the above code the array~\texttt{b} contains the
B\'{e}zier control points of the polynomial pieces of the B-spline patch.
To move the control points of the $j$-th B\'{e}zier patch from the $i$-th
strip (counting from~$0$) to the array~\texttt{c} (of length at least
$(n+1)(m+1)d$), and obtain a~``packed'' control net (without unused areas
between he columns), one may use the code

\vspace{\medskipamount}
\noindent{\ttfamily
\ind{2}md = $(m+1)d$;\ind{6}\,/* length of one column of each B\'{e}zier patch */ \\
\ind{2}start = $(n+1)i$*pitch + md*$j$;\ind{4}/* position of the first point */\\
\ind{2}pkv\_Selectf ( $n+1$, md, pitch, md, \&b[start], c );
}\vspace{\medskipamount}

\begin{figure}[ht]
  \centerline{\epsfig{file=bspbez.ps}}
  \caption{A~B-spline patch and its piecewise B\'{e}zier representation.}
\end{figure}


\section{Lane-Riesenfeld algorithm}

\cprog{%
boolean mbs\_multiLaneRiesenfeldf ( int spdimen, int ncurves, \\
\ind{20}int degree, \\
\ind{20}int inlastknot, int inpitch, const float *incp, \\
\ind{20}int *outlastknot, int outpitch, float *outcp );}

\vspace{\bigskipamount}
\cprog{%
\#define mbs\_LaneRiesenfeldC1f(degree,inlastknot,incp,outlastknot, \bsl \\
\ind{4}outcp) \bsl \\
\ind{2}mbs\_multiLaneRiesenfeldf ( 1, 1, degree, inlastknot, 0, incp, \bsl \\
\ind{4}outlastknot, 0, outcp ) \\
\#define mbs\_LaneRiesenfeldC2f(degree,inlastknot,incp,outlastknot, \bsl \\
\ind{4}outcp) \bsl \\
\ind{2}mbs\_multiLaneRiesenfeldf ( 2, 1, degree, inlastknot, 0, \bsl \\
\ind{4}(float*)incp, outlastknot, 0, (float*)outcp ) \\
\#define mbs\_LaneRiesenfeldC3f(degree,inlastknot,incp,outlastknot, \bsl \\
\ind{4}outcp) ... \\
\#define mbs\_LaneRiesenfeldC4f(degree,inlastknot,incp,outlastknot, \bsl \\
\ind{4}outcp) ...}

\begin{figure}[ht]
  \centerline{\epsfig{file=bsplane.ps}}
  \caption{Application of the Lane-Riesenfeld algorithm to a~B-spline patch}
\end{figure}



\clearpage
