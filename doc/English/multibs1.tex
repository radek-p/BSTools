
%/* //////////////////////////////////////////////////// */
%/* This file is a part of the BSTools procedure package */
%/* written by Przemyslaw Kiciak.                        */
%/* //////////////////////////////////////////////////// */

\chapter{The \texttt{libmultibs} library}

\section[Basic definitions and representations of curves and patches]%
{Basic definitions and representations \\ of curves and patches}

\subsection{B\'{e}zier curves}

A~\unl{B\'{e}zier curve} is defined by the formula
\begin{align}\label{eq:Bezc:def}
  \bm{p}(t) = \sum_{i=0}^n \bm{p}_iB^n_i(t),
\end{align}
with \unl{control points} $\bm{p}_0,\ldots,\bm{p}_n$
and \unl{Bernstein polynomials}
\begin{align}
  B^n_i(t) \stackrel{\mathrm{def}}{=} \binom{n}{i}t^i(1-t)^{n-i},\quad
  i=0,\ldots,n.
\end{align}

The polyline, whose consecutive vertices are the points
$\bm{p}_0,\ldots,\bm{p}_n$, is called the \unl{control polygon} of the curve.
Each control point has $d$ coordinates and then the curve is located in the
$d$-dimensional space. In particular, for $d=1$ the formula~(\ref{eq:Bezc:def})
describes a~polynomial of the variable~$t$ of degree at most~$n$.

The representation of a~B\'{e}zier curve consists of the number~$n$
and of the sequence of $n+1$ control points, whose coordinates
($(n+1)d$ floating-point numbers) are packed in an array
(i.e.\ first come $d$~coordinates of $\bm{p}_0$, then $\bm{p}_1$ etc.).

\vspace{\medskipamount}
\unl{A~rational B\'{e}zier curve} is given by
\begin{align}
  \bm{p}(t) =
  \frac{\sum_{i=0}^n w_i\bm{p}_i B^n_i(t)}{\sum_{i=0}^n w_iB^n_i(t)},
\end{align}
with the Bernstein polynomials, control points
$\bm{p}_0,\ldots,\bm{p}_n$ and \unl{weights} $w_0,\ldots,w_n$. Such a~curve is
located in the same space as the control points.

If $w_i=0$ for some~$i$, then the expression $w_i\bm{p}_i$ may be replaced
by an arbitrary vector $\bm{v}_i$, thus extending the definition of the curve,
but at least one weight must be nonzero.

The control points $\bm{p}_i$ of the rational curve are convenient for the
program user, which may interactively modify them, but the procedures
of this library process the \unl{homogeneous representation}.
For a~curve in a~$d$-dimensional space it is a~polynomial curve in
the space of dimension $d+1$:
\begin{align}
  \bm{P}(t) = \sum_{i=0}^n\bm{P}_iB^n_i(t),
\end{align}
whose control points $\bm{P}_i$ are given by
\begin{align}
  \bm{P}_i = \left[\begin{array}{c}w_i\bm{p}_i \\ w_i \end{array}\right].
\end{align}
The last (i.e.\ $d+{}$first) homogeneous coordinate is thus the weight.
If $w_i=0$ then
\begin{align}
  \bm{P}_i = \left[\begin{array}{c}\bm{v}_i \\ 0 \end{array}\right].
\end{align}

The cartesian coordinates of the point $\bm{p}(t)$ of a~rational curve
are obtained by dividing the first $d$ coordinates of the point $\bm{P}(t)$
by its last coordinate.

The representation of a~rational curve e.g.\ in a~three-dimensional space
consists of the number~$n$ (which determines the degree of the representation)
and of an array of $4(n+1)$ floating point numbers, being the
coordinates of the consecutive points~$\bm{P}_i$.
As the homogeneous curves are ordinary polynomial curves, in most cases
they may be processed with the procedures appropriate for the polynomial
B\'{e}zier curves.


\subsection{Tensor product B\'{e}zier patches}

\begin{sloppypar}
A~\unl{tensor product (rectangular) B\'{e}zier patch} is defined by
the formula
\begin{align}
  \bm{p}(u,v) = \sum_{i=0}^n\sum_{j=0}^m \bm{p}_{ij}B^n_i(u)B^m_j(v),
\end{align}
with the Bernstein polynomials $B^n_i$ and $B^m_j$ of degrees
$n$ and $m$ respectively and with the control points $\bm{p}_{ij}$.
By convention, the \unl{row of the control net} is the polyline
with the vertices
$\bm{p}_{0j},\ldots,\bm{p}_{nj}$ (for $j\in\{0,\ldots,m\}$),
and the \unl{column} is the polyline with the vertices
$\bm{p}_{i0},\ldots,\bm{p}_{im}$ (for all $i\in\{0,\ldots,n\}$).
\end{sloppypar}

The representation of a~B\'{e}zier patch consists of
two positive numbers $n$ and $m$ and of $(n+1)(m+1)$ control points,
i.e.\ $(n+1)(m+1)d$ floating point numbers,
stored in the array in the following sequence: first
$d$~coordinates of $\bm{p}_{00}$, then $d$ coordinates of
$\bm{p}_{01}$ etc. After the coordinates of the point $\bm{p}_{0m}$
there ought to be the coordinates of $\bm{p}_{10}$ etc., up to the point
$\bm{p}_{nm}$. In other words, the control net is stored in the array
columnwise.

The array described above may be seen in many different ways.
For instance, to apply an affine transformation to the patch, it is necessary
to transform its control points. In that case the array may be seen as
a~one-dimensional array of points.

We can also divide this patch using the de~Casteljau algorithm,
by halving the interval of the parameter $u$ or $v$. In the latter case
we apply the algorithm to all columns, as if they were control polygons of
B\'{e}zier curves. The array contains thus $n+1$ curves and its pitch
is equal to $(m+1)d$ (where $d$ is the space dimension), i.e.\ the
second curve representation begins $(m+1)d$ places after the first, etc.

To divide the interval of $u$, it is necessary to
apply the de~Casteljau algorithm to all rows of the control net.
It turns out that the patch representation may be interpreted as
a~representation of a~B\'{e}zier curve in the space of dimension
$(m+1)d$ (each column of the control net is a~point of this space).
In this case we process only one B\'{e}zier curve of degree~$n$
in the $(m+1)d$-dimensional space, and the pitch is irrelevant, as there
is only one curve.

\vspace{\medskipamount}
A~\unl{rational B\'{e}zier patch} is given by the formula
\begin{align*}
  \bm{p}(u,v) =
    \frac{\sum_{i=0}^n\sum_{j=0}^m w_{ij}\bm{p}_{ij}B^n_i(u)B^m_j(v)}%
         {\sum_{i=0}^n\sum_{j=0}^m w_{ij}B^n_i(u)B^m_j(v)},
\end{align*}
where apart from the Bernstein polynomials and the control points there are
weights $w_{ij}$. The procedures processing rational B\'{e}zier patches
in the \texttt{libmultibs} library, process their homogeneous representations
i.e.\ the arrays of control points $\bm{P}_{ij}$
of polynomial B\'{e}zier patches
\begin{align}
  \bm{P}(u,v) = \sum_{i=0}^n\sum_{j=0}^m\bm{P}_{ij}B^n_i(u)B^m_j(v),
\end{align}
in the space of dimension $d+1$. The relation between the control points
$\bm{p}_{ij}$ and weights $w_{ij}$ of the rational patch with the
points $\bm{P}_{ij}$ here is the same as in the case of the rational
B\'{e}zier curves. The method of storing the control points $\bm{P}_{ij}$
of the homogeneous patch is the same as in the case of the nonrational patch
(except that the space dimension, where the homogeneous patch is located is
greater by~$1$).


\subsection{\label{ssect:BSC}B-spline curves}

Let $n\geq 0$ and let the nondecreasing sequence of \unl{knots} (real numbers)
$u_0,\ldots,u_N$, such that $N>2n$ and $u_n<u_{N-n}$ be fixed.
A~\unl{B-spline curve} of degree~$n$ based on this knot sequence is defined
by the formula
\begin{align}\label{eq:BScurve:def}
  \bm{s}(t) = \sum_{i=0}^{N-n-1} \bm{d}_iN^n_i(t),
\end{align}
with the control points $\bm{d}_0,\ldots,\bm{d}_{N-n-1}$ and
\unl{B-spline functions} $N^n_0,\ldots,N^n_{N-n-1}$. These functions have
a~number of equivalent definitions, e.g.\ they may be defined by the
recursive Mansfield-de~Boor-Cox formula:
\begin{align}\label{eq:BS:basis0}
  N^0_i(t) &{}= \left\{\begin{array}{ll}1 & \mbox{dla $u_i\leq t<u_{i+1}$,} \\
    0 & \mbox{w przeciwnym razie,} \end{array}\right. \\
  \label{eq:BS:basisn}
  N^j_i(t) &{}= \frac{t-u_i}{u_{i+j}-u_i} N^{j-1}_i(t) +
    \frac{u_{i+j+1}-t}{u_{i+j+1}-u_{i+1}} N^{j-1}_{i+1}(t)\quad
    \mbox{dla $j=1,\ldots,n$}.
\end{align}

The domain of the curve is the interval $[u_n,u_{N-n-1})$. In each interval
$[u_k,u_{k+1})$ (for $n\leq k<N-n$) the B-spline curve is a~polynomial arc
of degree at most~$n$.

The representation of a~B-spline curve consists of the integer numbers
$n$ and $N$, which determine respectively the degree and the
number of the last knot, the sequence of knots (array of floating point numbers)
$u_0,\ldots,u_N$ and the control points $\bm{d}_0,\ldots,\bm{d}_{N-n-1}$,
located in the same space that the curve --- if the dimension of this space is~$d$,
then the array of control points must contain $(N-n)d$ floating point numbers.

\vspace{\medskipamount}
\begin{sloppypar}\hyphenpenalty=200
To describe the details of procedures the following naming
convention is used: \unl{boundary knots} are the knots, which bound
the curve domain, i.e.\ $u_n$, $u_{N-n}$ and all knots equal to one of the two.
The boundary knots are \unl{left} and \unl{right}. The \unl{internal knots}
are all knots in the open interval $(u_n,u_{N-n})$; these knots have
the corresponding junction points of polynomial arcs. Apart from the above
there are also the \unl{external knots}, which are not elements of
the closed interval $[u_n,u_{N-n}]$.
Apart from the above, the knots $u_0$ and $u_N$ are called the \unl{extremal knots}.
For example, if $n=3$, $N=15$ and
\begin{align*}
  u_0<u_1=u_2<u_3=u_4=u_5<u_6\leq\cdots\leq u_{11}<u_{12}=u_{13}=u_{14}=u_{15},
\end{align*}
then the knots $u_0$, $u_1$ i~$u_2$ are external, the knots $u_3$, $u_4$
and $u_{12},\ldots,u_{15}$ are boundary, and the other knots
are internal. the extremal knots are $u_0$ and $u_{15}$.
\end{sloppypar}

The extremal knots are necessary to define the functions
$N^n_0$ and $N^n_{N-n-1}$, but they do not have any influence on the
values of those functions in $[u_n,u_{N-n})$, and thus they do not affect
the shape of the curve. Various software packages either require supplying these
knots or not. The \texttt{libmultibs} library requires specifying them
(it suffices that the conditions $u_0\leq u_1$ and $u_{N-1}\leq u_N$ are
satisfied).

\vspace{\medskipamount}
A~given spline curve may have various representations, which may differ
with the degree and with the knot sequence. The construction of a~representation
with additional knots is called \unl{knot insertion}. In particular,
the representation, whose all knots have the multiplicity (the number of
appearances) $n+1$ (i.e.\ there is $u_0=\cdots=u_n$, $u_{n+1}=\cdots=u_{2n+1}$,
 $u_{2n+2}=\cdots=u_{3n+2}$ etc.), is a~piecewise B\'{e}zier representation.

If the last (with the greatest index) left knot has the index
$k>n$, then the initial $k-n$ knots (starting from the left extremal)
and the initial $k-n$ control points are unnecessary in the curve
representation and they may (must in certain situations) be rejected.
Similarly, if the first (with the smallest index) right boundary knot
has the index $k<N-n$, then the last $N-n-k$ knots and control points
are unnecessary. The representations with unnecessary knots and control points
may be the effect of knot insertion (i.e.\ during the conversion to the piecewise
B\'{e}zier representation) or of constructing the B-spline representation
of the derivatives of a~B-spline curve.

A B-spline curve of degree~$n$, whose boundary knots have the multiplicity
$n$ or greater is called the \unl{curve with clamped ends}.
If the last (with the greatest index) left boundary knot has the index~$k$
(the knots are numbered from~$0$, therefore obviously $k\geq n$), then the
control point $\bm{d}_{k-n}$ is the curve point corresponding to
the parameter $u_k$, i.e.\ the left end of the domain. If $k=n$, then it is
the point $\bm{d}_0$; otherwise the points $\bm{d}_0,\ldots,\bm{d}_{k-n-1}$
have no influence on the shape of the curve (and they may be rejected
together with the knots $u_0,\ldots,u_{k-n-1}$).
A~similar rule concerns the right knot with the smallest index --- if it is
the knot $u_{N-n}$ of multiplicity~$n$ or $n+1$, then the control point
$\bm{d}_{N-n-1}$ is the end point of the curve (it corresponds to $t=u_{N-n}$).

A~curve, whose boundary knots have the multiplicity less than~$n$
is called a~\unl{free end curve}. Each end of the curve may be clamped
or free, independently on the other end.

\vspace{\medskipamount}
\unl{Closed B-spline curves} are represented in the same way as the other
B-spline curves. To be closed, a~B-spline curve of degree~$n$ must satisfy
the following conditions: the knot sequence $u_1,\ldots,u_{N-1}$ has to
consist of subsequent elements of an infinite sequence of numbers,
such that the sequence of differences is nonnegative and periodic, with the period
\begin{align*}
  K=N-2n,
\end{align*}
where $N>3n$. The knot sequence must be nondecreasing and there must be
a~positive number~$T$, such that
\begin{align*}
  u_{k+K}-u_k=T\qquad\mbox{for $k=1,\ldots,2n-1$.}
\end{align*}
The knots $u_0$ and $u_N$ have no influence on the shape of the curve,
but they must satisfy the conditions $u_0\leq u_1$ and $u_{N-1}\leq u_N$.

The sequence $\bm{d}_0,\ldots,\bm{d}_{N-n-1}$ has to consist of
consecutive elements of an infinite periodic sequence with the period
$N-2n$, i.e.\ there must be
\begin{align*}
  \bm{d}_{k+K} = \bm{d}_k\qquad\mbox{for $k=0,\ldots,n-1$.}
\end{align*}

\begin{sloppypar}
An application may use space saving representations
of closed B-spline curves, where the knots $u_{K+1},\ldots,u_{N-1}$ and
the control points $\bm{d}_{K},\ldots,\allowbreak\bm{d}_{N-n-1}$, possible to
reproduce based on the above conditions are absent.
To use the \texttt{libmultibs} procedures it is necessary to create
a~``working'' representation, with arrays containing all the knots and
control points.
\end{sloppypar}

Finding a~point and many other computations for closed B-spline curves
may be done with the procedures intended to use with ``ordinary'' B-spline
curves with free ends. The representation changes like knot insertion
or removal and degree elevation must be done with the procedures which
ensure that the new representation satisfies the conditions described above.
Such procedures have names with the word ``\texttt{Closed}'', and in most cases
they still have to be written.


\subsection{Tensor product B-spline patches}

A~\unl{B-spline patch} is defined with the formula
\begin{align}\label{eq:BSpatch:def}
  \bm{s}(u,v) =
  \sum_{i=0}^{N-n-1}\sum_{j=0}^{M-m-1}\bm{d}_{ij}N^n_i(u) N^m_j(v),
\end{align}
with two sets of B-spline functions of degrees $n$ and $m$ (different in
general), based on the knot sequences $u_0,\ldots,u_N$ and $v_0,\ldots,v_M$
respectively (also different in general, even if $n=m$). Both sequences must
be nondecreasing and long enough (there must be $N>2n$, $M>2m$, $u_n<u_{N-n}$
and $v_m<v_{M-m}$). The terminology and remarks from the previous section
apply to both these sequences.

\begin{sloppypar}
The array of control points $\bm{d}_{ij}$, which together with the knots
represent the patch, contains the coordinates of the points
$\bm{d}_{00},\bm{d}_{01},\ldots,\bm{d}_{0,N-n-1}$, then
$\bm{d}_{10},\bm{d}_{11},\ldots,\bm{d}_{1,N-n-1}$ etc., i.e.\
the consecutive columns of the control net of the patch.
\end{sloppypar}

Between the consecutive columns there may be unused spaces, which make it
possible to insert knots to the sequence ``$v$'' of the initial representation.
This is done as if the new knot was inserted to the representations of many
B-spline curves, whose control polygons are the columns of the patch control
net. After the knot insertion the length of the unused spaces is decreased by
the length of $d$ floating point numbers (where $d$ is the dimension of
the space, in which the patch is located). The pitch of the array in this case
is the distance of the beginnings of consecutive columns (measured in floating
point numbers).

The curves with clamped end and free end curves correspond to
the \unl{patches with} \unl{clamped boundary} and \unl{with free boundary}.
For example, if the ``$u$'' knot sequence satisfies the condition
$u_1=\cdots=u_n<u_{n+1}$, then the constant parameter curve for $u=u_n$ (one
of the four boundary curves of the patch) is a~B-spline curve of degree~$m$,
based on the knot sequence ``$v$'', whose control polygon is the first
column of the patch control net. Obviously, each of the four patch
boundary curves may be clamped or free, independently of the others.

The closed curves correspond to the closed patches, which may be tubes or tori.
One or both knot sequences, and the sequence of rows or columns of the control net
(interpreted as points) have to satisfy the conditions formulated for 
closed B-spline curves.


\subsection{NURBS curves and patches}

\unl{NURBS} (non-uniform rational B-spline) \unl{curves and patches}
are the curves and patches piecewise rational, whose relation with the
B-spline curves and patches is the same as the relation of the rational
B\'{e}zier curves and patches with the polynomial B\'{e}zier curves and
patches. One can choose one or two knot sequences and the control points
$\bm{d}_i$ or $\bm{d}_{ij}$ in the $d$-dimensional space and associate the
weight $w_i$ or $w_{ij}$ with each control point. Then the vectors in
the $d+1$-dimensional space
\begin{align}
  \bm{D}_i =
  \left[\begin{array}{c} w_i\bm{d}_i \\ w_i \end{array}\right]
  \qquad\mbox{or}\qquad
  \bm{D}_{ij} =
  \left[\begin{array}{c} w_{ij}\bm{d}_{ij} \\ w_{ij} \end{array}\right]
\end{align}
define a~curve or a~patch in this space; after dividing the first $d$
coordinates of a~point of the homogeneous curve or patch by the
$d+\mathord{\mbox{first}}$ (weight) coordinate one obtains the cartesian
coordinates of the point of the rational curve or patch.

The \texttt{libmultibs} procedures process such homogeneous representations
of the rational curves and surfaces.


\subsection{\label{ssect:Coons:patch:def}Coons patches}

Coons patches of class~$C^k$ are tensor product patches defined by
sufficiently smooth curves, which describe the boundary of the patch and
so called cross derivatives of order $1,\ldots,k$. For any~$k\in\N$ a~Coons
patch is defined by the formula
\begin{align}\label{eq:Coons:patch:def}
  \bm{p}(u,v) = \bm{p}_1(u,v)+\bm{p}_2(u,v)-\bm{p}_3(u,v),
\end{align}
where
\begin{align*}
  \bm{p}_1(u,v) &{}= \bm{C}(u)\hat{H}(v)^T, \quad
  \bm{p}_2(u,v) = \tilde{H}(u)\bm{D}(v)^T, \quad
  \bm{p}_3(u,v) = \tilde{H}(u)\bm{P}\hat{H}(v)^T,
\end{align*}
and
\begin{align*}
  \bm{C}(u) &{}= [\bm{c}_{00}(u),\bm{c}_{10}(u),\bm{c}_{01}(u),\bm{c}_{11}(u),
               \ldots,\bm{c}_{0k}(u),\bm{c}_{1k}(u)], \\
  \bm{D}(v) &{}= [\bm{d}_{00}(v),\bm{d}_{10}(v),\bm{d}_{01}(v),\bm{d}_{11}(v),
               \ldots,\bm{d}_{0k}(v),\bm{d}_{1k}(v)], \\
  \tilde{H}(u) &{}= [\tilde{H}_{00}(u),\tilde{H}_{10}(u),
                     \tilde{H}_{01}(u),\tilde{H}_{11}(u),\ldots,
                     \tilde{H}_{0k}(u),\tilde{H}_{1k}(u)], \\
  \hat{H}(v) &{}= [\hat{H}_{00}(v),\hat{H}_{10}(v),
                     \hat{H}_{01}(v),\hat{H}_{11}(v),\ldots,
                     \hat{H}_{0k}(v),\hat{H}_{1k}(v)].
\end{align*}
The curves $\bm{c}_{00},\ldots,\bm{c}_{1k}$ describe two opposite boundaries
of the patch and the cross derivatives at these boundaries. These curves
must have the same domain, denoted here by~$[a,b]$. Similarly, the curves
$\bm{d}_{00},\ldots,\bm{d}_{1k}$ describe the other pair of opposite
boundaries and cross derivatives and they also must have the same domain, say
$[c,d]$. The domain of the Coons patch is the rectangle $[a,b]\times[c,d]$.

The matrix~$\bm{P}$ of dimensions $(2k+2)\times(2k+2)$ consists of the points
of the given curves and the vectors of their derivatives of order~$1,\ldots,k$:
\begin{align}\label{eq:Coons:compat:cond}
  \bm{P} ={}& \left[\begin{array}{ccccc}
    \bm{c}_{00}(a) & \bm{c}_{10}(a) & \ldots & \bm{c}_{0k}(a) & \bm{c}_{1k}(a) \\
    \bm{c}_{00}(b) & \bm{c}_{10}(b) & \ldots & \bm{c}_{0k}(b) & \bm{c}_{1k}(b) \\
    \vdots & \vdots & & \vdots & \vdots \\
    \bm{c}^{(k)}_{00}(a) & \bm{c}^{(k)}_{10}(a) & \ldots &
    \bm{c}^{(k)}_{0k}(a) & \bm{c}^{(k)}_{1k}(a) \\
    \bm{c}^{(k)}_{00}(b) & \bm{c}^{(k)}_{10}(b) & \ldots &
    \bm{c}^{(k)}_{0k}(b) & \bm{c}^{(k)}_{1k}(b)
  \end{array}\right] = \nonumber \\
 &\left[\begin{array}{ccccc}
    \bm{d}_{00}(c) & \bm{d}_{00}(d) & \ldots & \bm{d}_{0k}(c) & \bm{d}_{0k}(d) \\
    \bm{d}_{10}(c) & \bm{d}_{10}(d) & \ldots & \bm{d}_{1k}(c) & \bm{d}_{1k}(d) \\
    \vdots & \vdots & & \vdots & \vdots \\
    \bm{d}^{(k)}_{00}(c) & \bm{d}^{(k)}_{00}(d) & \ldots &
    \bm{d}^{(k)}_{0k}(c) & \bm{d}^{(k)}_{0k}(d) \\
    \bm{d}^{(k)}_{10}(c) & \bm{d}^{(k)}_{10}(d) & \ldots &
    \bm{d}^{(k)}_{1k}(c) & \bm{d}^{(k)}_{1k}(d)
  \end{array}\right].
\end{align}
The curves, which define a~Coons patch, must satisfy the compatibility
conditions, expressed by the equality of the matrices above.

The functions $\tilde{H}_{mj}(u)$ and~$\hat{H}_{mj}(v)$ are elements of
so called local Hermite bases.
It is assumed that these functions are polynomials of degree $2k+1$, where
$k\in\{1,2\}$, though one might use other functions of class~$C^k$ instead,
e.g.\ spline functions of degree~$k+1$. These functions are given
by the formula
\begin{align*}
  \tilde{H}_{mj}(u) = (b-a)^jH_{mj}\Bigl(\frac{u-a}{b-a}\Bigr),\qquad
  \hat{H}_{mj}(v) = (d-c)^jH_{mj}\Bigl(\frac{v-c}{d-c}\Bigr).
\end{align*}

For $k=1$ there is
\begin{alignat*}{2}
  H_{00}(t) &{}= B^3_0(t)+B^3_1(t),\qquad &
  H_{10}(t) &{}= B^3_2(t)+B^3_3(t),\\
  H_{01}(t) &{}= \frac{1}{3}B^3_1(t),\qquad &
  H_{11}(t) &{}= -\frac{1}{3}B^3_2(t).
\end{alignat*}
As the polynomials used to define a~patch, i.e.\ in the interpolation
of the given curves for $k=1$ in both directions are cubic, the Coons
patches of class~$C^1$ are called \textbf{bicubic Coons patches},
though such a~patch may be defined with curves of any degree.

For $k=2$ the patch is defined with the polynomials of degree~$5$,
\begin{alignat*}{2}
  H_{00}(t) &{}= B^5_0(t)+B^5_1(t)+B^5_2(t),\qquad &
  H_{10}(t) &{}= B^5_3(t)+B^5_4(t)+B^5_5(t), \\
  H_{01}(t) &{}= \frac{1}{5}B^5_1(t)+\frac{2}{5}B^5_2(t), \qquad &
  H_{11}(t) &{}= -\frac{2}{5}B^5_3(t)-\frac{1}{5}B^5_4(t), \\
  H_{02}(t) &{}= \frac{1}{20}B^5_2(t), \qquad &
  H_{12}(t) &{}= \frac{1}{20}B^5_3(t),
\end{alignat*}
and therefore the Coons patches of class~$C^2$ are called \textbf{biquintic
Coons patches}.

Coons patches (bicubic and biquintic) may be defined with polynomial or
spline curves. In the former case the curves are B\'{e}zier curves and the
domain of the patch is the unit square $[0,1]^2$. The individual curves
defining a~patch do not have to have the same degree.

The domain of a~patch defined with spline curves (represented as B-spline
curves) may be dan arbitrary rectangle $[a,b]\times[c,d]$. The individual
curves do not need to have the same degree and they may be represented
with various knot sequences (but they must have the same domain,
determined by the boundary knots).

The library \texttt{libmultibs} contains procedures, which convert the
Coons representation of a~patch to the B\'{e}zier or B-spline form,
and fast procedures computing points and derivatives of Coons patches
at the points of a~regular net in the patch domain; these procedures
found their application in the constructions implemented in
the libraries \texttt{libg1hole} and \texttt{libg2hole}.



\subsection{Naming conventions}

The naming conventions are intended to simplify the package user (the
programmer) guessing the action done by a~procedure and guessing the use
of parameters (if two procedures have a~parameter with the same name,
its r\^{o}le is the same in both these procedures).

\begin{sloppypar}
Each procedure and a~macro intended to be called as a~procedure in the
\texttt{libmultibs} library has the name beginning with the prefix
\texttt{mbs\_}.
\end{sloppypar}

If after the prefix there is the word \texttt{multi}, the procedure
is intended to process a~number of curves simultaneously.
The number of curves is specified by the parameter named \texttt{ncurves}.

The suffix consists of two parts. The first part may be empty (if there is the
``\texttt{multi}'' after the prefix), or it indicates the kind of the
curve or patch processed by the procedure. The letter \texttt{C} denotes
a~curve, and \texttt{P} denotes a~patch. The digit denotes the dimension
of the space, in which the curve or the patch resides (e.g.\ \texttt{2}
denotes a~plane). The letter \texttt{R} after the digit denotes a~curve
or a~patch in the homogeneous representation. \textbf{Caution:}
the control points in this case have one coordinate more.
The second part of the suffix is the letter \texttt{f}, which denotes the single
precision (\texttt{float}) or \texttt{d}, which denotes the double precision%
\footnote{In the professional applications only double precision should be used,
unless even such a~precision is insufficient.}
of the floating point arithmetic used by the procedure to represent the data
and results and in the computations.

The main part of the name denotes the algorithm implemented by the
procedure. The macros and procedures with the same name differ with the
destination --- they are universal (if there is the \texttt{multi} part)
or specific for curves or patches in the space with the fixed dimension.
The most important main parts of the names are
\begin{mydescription}
  \item\texttt{deBoor} --- computing points of B-spline curves and patches
    with the de~Boor algorithm.
  \item\texttt{deBoorDer} --- computing points of B-spline curves and patches
    together with the first order derivatives, using the be~Boor algorithm.
  \item\texttt{BCHorner} --- computing points of B\'{e}zier curves and patches
    using the Horner scheme.
  \item\texttt{BCHornerDer} --- computing points of B\'{e}zier curves and patches
    together with the first order derivatives, using the Horner scheme.
  \item\texttt{BCFrenet} --- computing curvatures and Frenet frame vectors
    of B\'{e}zier curves.
  \item\texttt{BCHornerNv} --- computing the normal vector of a~B\'{e}zier patch.
  \item\texttt{KnotIns} --- insertion of a~single knot to B-spline
    curves using the Boehm algorithm.
  \item\texttt{KnotRemove} --- removing a~single knot.
  \item\texttt{*Oslo*} --- procedures related with inserting and removing
    a~number of knots simultaneously, using the Oslo algorithm.
  \item\texttt{MaxKnotIns} --- inserting knots so as to obtain a B-spline
    representation with all internal knots of multiplicity $n+1$ and
    the boundary knots of multiplicity $n$ or $n+1$, which is a~piecewise
    B\'{e}zier representation.
  \item\texttt{BisectB} --- division of B\'{e}zier curves into arcs, related with
    the division of the domain into two line segments of the same length,
    with the de~Casteljau algorithm.
  \item\texttt{DivideB} --- division of B\'{e}zier curves into arcs, related with
    the division of the domain into two line segments of arbitrary lengths,
    with the de~Casteljau algorithm.
  \item\texttt{BCDegElev} --- degree elevation of B\'{e}zier curves and patches.
  \item\texttt{BSDegElev} --- degree elevation of B-spline curves and patches.
  \item\texttt{MultBez} --- multiplication of polynomials and B\'{e}zier curves.
  \item\texttt{MultBS} --- multiplication of splines and B-spline curves.
  \item\texttt{BezNormal} --- computing normal vector B\'{e}zier patches.
  \item\texttt{BSCubicInterp} --- construction of cubic B-spline curves of
    interpolation.
  \item\texttt{ConstructApproxBS} --- construction of B-spline curves of
    approximation.
  \item\texttt{Closed} --- procedures, whose name contains this word, are
    intended to process closed B-spline curves.
\end{mydescription}

The \underline{formal parameters} of the procedures and macros may have
the following names:
\begin{mydescription}
  \item\texttt{spdimen} --- specifies the dimension $d$ of the space in which
    the curves reside, i.e.\ the number of coordinates of each point of this
    space.
    If the suffix of the name of a~procedure or a~macro contains the
    letter~\texttt{R}, which indicates a~rational object, then the control points
    have \texttt{spdimen}${}=d+1$ coordinates.
  \item\begin{sloppypar}\texttt{degree} --- specifies the degree of the curve representation.
    The degrees of a~patch with respect to its two parameters are specified by
    the parameters named \texttt{degreeu} and \texttt{degreev}.%
    \end{sloppypar}
  \item\texttt{lastknot} --- specifies the number~$N$, which is the index
    of the last knot in the knot sequence. The knot sequence consists of
    $N+1$ knots. For patches there are two parameters, \texttt{lastknotu} and
    \texttt{lastknotv}.
  \item\texttt{knots} --- array of floating point numbers, with the knots.
    Two arrays with two knot sequences being parts of a~B-spline patch
    representation are passed to the procedure as the parameters named
    \texttt{knotsu} and \texttt{knotsv}.
  \item\texttt{ctlpoints} --- array of control points. If the space
    dimension $d$ is $1$ (the procedure or macro processes scalar functions),
    the parameter pointing the appropriate array is called \texttt{coeff}.
  \item\texttt{pitch} --- pitch of the control point array, i.e.\ the
    difference between the indexes of the first coordinates of the first
    control points of two consecutive curves or columns of the control net
    in the \texttt{ctlpoints} array. Such arrays are always treated as
    arrays of \emph{floating point numbers}, therefore the pitch unit
    is always the length of one floating point number (even if the formal
    parameter type is e.g.\ \texttt{point3f*}).
\end{mydescription}
If the formal parameters are used to pass two representations,
e.g.\ the procedure constructs a~result representation based on the given one,
the parameter names are extended by \texttt{in} and \texttt{out}.
The given representations are described by the parameters, which appear
in the formal parameter list \emph{before} the parameters used to
describe the result.



