
%/* //////////////////////////////////////////////////// */
%/* This file is a part of the BSTools procedure package */
%/* written by Przemyslaw Kiciak.                        */
%/* //////////////////////////////////////////////////// */

\chapter{The \texttt{libcamera} library}

The \texttt{libcamera} library consists of procedures, which manage the
cameras, i.e.\ objects, which represent projections of the 3d space
onto a~plane, in order to make pictures. There are two kinds of projections:
perspective and parallel. The former are intended to make ``photographs'',
the latter are better to produce technical drawings.


\section{The camera}

\subsection{A description of the camera and the projection algorithm}

The data structure and the headers of procedures are giben in the header
files \texttt{cameraf.h} andf \texttt{camerad.h}.
Both files may be included via the file \texttt{camera.h}.

\vspace{\bigskipamount}
\cprog{%
typedef struct CameraRecf \{ \\
\ind{2}boolean parallel, upside, c\_fixed; \\
\ind{2}byte magnification; \\
\ind{2}short xmin, ymin, width, height; \\
\ind{2}float aspect; \\
\ind{2}point3f position; \\
\ind{2}float psi, theta, phi; \\
\ind{2}point3f g\_centre, c\_centre; \\
\ind{2}float xscale, yscale; \\
\ind{2}trans3f CTr, CTrInv; \\
\ind{2}vector4f cplane[6]; \\
\ind{2}union \{ \\
\ind{4}struct \{ \\
\ind{6}float f; \\
\ind{6}float xi0, eta0; \\
\ind{6}float dxi0, deta0; \\
\ind{4}\} persp; \\
\ind{4}struct \{ \\
\ind{6}float   wdt, hgh, diag; \\
\ind{6}boolean dim\_case; \\
\ind{4}\} para; \\
\ind{2}\} vd; \\
\} CameraRecf;}
The \texttt{CameraRecf} structure describes the camera, i.e.\ an object
representing a~perspective or a~parallel projection.%
\begin{figure}[ht]
  \centerline{\epsfig{file=camera.ps}}
  \caption{The camera system of coordinates and the frame for
      a~perspective projection}
\end{figure}

The structure fields contain the following information:
\begin{mydescription}
  \item[]\texttt{parallel} --- if its value is \texttt{false}
    (\texttt{0}), then the projection is perspective, else it is parallel.
  \item[]\texttt{upside} --- if its value is \texttt{false},
    then the $y$~axis of the image system of coordinates is oriented
    downward (like in XWindow system windows), else it is oriented
    up (like in the OpenGL library or in the default PostScript system
    of coordinates).
  \item[]\texttt{c\_fixed} --- this parameter specifies the changes of the
    camera rotations centre when the camera is moved --- if \texttt{false},
    then this point is fixed in the global system of coordinates, else
    if is fixed in the camera system.
  \item[]\texttt{magnification} --- by default this field has the value~$1$,
    which means that the axis unit in the image system of coordinate is the
    width or height of one pixel. Greater values select appropriately
    shorter units, which may be useful for supersampling.
  \item[]\texttt{xmin}, \texttt{ymin}, \texttt{width}, \texttt{height} ---
    coordinates of the upper left point of the frame and its dimensions in
    pixels.
  \item[]\texttt{aspect} --- the aspect factor, i.e.\ the ratio of the width
    and height of one pixel.
  \item[]\texttt{position} --- position of the view point (in global
    coordinates).
  \item[]\texttt{psi}, \texttt{theta},  \texttt{phi} --- Euler angles
    $\psi$, $\vartheta$, $\varphi$, which describe the direction of the
    camera.
  \item[]\texttt{g\_centre}, \texttt{c\_centre} --- coordinates of the
    centre of rotations of the camera, i.e.\ the point on the axes of
    rotations of the camera, in global and camera coordinates respectively.
  \item[]\texttt{xscale}, \texttt{yscale} --- factors of scaling of the axes
    $x$~and~$y$ of the camera system of coordinates.
  \item[]\texttt{Ctr}, \texttt{CTrInv} --- the transformation from the global
    to camera coordinates and its inverse.
  \item[]\texttt{cplane} --- representations of four halfspaces, ahose
    intersection is the visibility frustum. The halfspace
    $ax+by+cz+d>0$ is represented by the vector, whose coordinates are
    $a$, $b$, $c$, $d$.

    So far only four halfspaces are used, the other two (near and far) are
    to be done.
  \item[]\texttt{vd} --- a~union with data specific for the methods of
    projection. The structure \texttt{vd.persp} contains the data specific
    for perspective projections, while \texttt{vd.para} for the parallel
    ones.
  \item[]\texttt{vd.persp.f} --- focal length of the camera, in the units
    such that the diagonal of the frame has the length~$1$.
    Focal length~$1$ corresponds to a~standard photographic objective.
  \item[]\texttt{vd.persp.xi0}, \texttt{vd.persp.eta0} --- shift of pixels
   after the perspective projection.
  \item[]\texttt{vd.persp.dxi0}, \texttt{vd.persp.deta0} --- coordinates
    $x$, $y$ (in the camera system) of the frame centre. By default they are
    zero, and then the frame centre is located on the ``optical axis'' of
    the camera. Other values are necessary with cameras forming a~stereo
    pair.
  \item[]\texttt{vd.para.wdt}, \texttt{vd.para.hgh}, \texttt{vd.para.diag}
    --- dimensions (width, height, diagonal) of the frame, measured in the
    units of the global system of coordinates.
  \item[]\texttt{vd.para.dim\_case} --- the parameter, which specifies,
    which of the three above dimensions of the frame is specified by the
    user; $0$ --- diagonal, $1$ --- width, $2$ --- height.
    The other two dimensions will be computed by the library procedures.

    By default this parameter obtains the value~$0$, and the parameter
    \texttt{vd.para.diag} is set to~$1$.
\end{mydescription}

\subsubsection*{Projection algorithm:}

The image of the point $\bm{p}$, represented in the global coordinates, is
computed as follows:
\begin{enumerate}
  \item\begin{sloppypar}
    The point $\bm{p}$ is subject to the affine transformation represented
    by the \texttt{CTr} attribute of the camera. In this way the camera
    coordinates are obtained.%
    \end{sloppypar}
  \item For a~perspective projection the $x$ and~$y$ coordinates are divided
    by the~$z$ coordinate. Then the values of \texttt{xi0} and \texttt{eta0}
    are added to the quotients.

    For parallel projections this step is omitted.
  \item If the~$y$ axis is oriented upward (the \texttt{upside} attribute is
   nonzero) then the~$y$ coordinate is replaced by $2y_{\mathrm{min}}+h-y$,
   where $y_{\mathrm{min}}$ is the value of the attribute \texttt{ymin}, and~$h$
   is the value of the attribute \texttt{height}.
\end{enumerate}


\subsubsection*{Setting up the transformation to the camera coordinates:}

The transformation from the global to camera coordinates (done in the first
step of the algorithm describd above) is the composition of thee affine
transformations:
\begin{enumerate}
  \item Scaling of the axes $x$ and~$y$ by the factors, being values of the
    attributes \texttt{xscale} and~\texttt{yscale}.
  \item Rottaion described with use of the Euler angles, being values of the
    attributes \texttt{psi}, \texttt{theta}, \texttt{phi}.
  \item Translation, which sets the origin of the system at the point
    \texttt{position}.
\end{enumerate}
The values of the attributes, which specify the above transformations,
should be assigned by the procedures described later. These procedures
may compute the proper values representing a~composition of a~series of
camera movements from the default initial position.


\subsection{Camera procedures}

\cprog{%
void CameraInitFramef ( CameraRecf *CPos, \\
\ind{17}boolean parallel, boolean upside, \\
\ind{17}short width, short height, short xmin, short ymin, \\
\ind{17}float aspect );}
\hspace*{\parindent}The procedure \texttt{CameraInitFramef} sets initial
values of the \texttt{*CPos} attributes, which specify the kind of
projection and the size of the frame (in pixels), and the aspect factor
(ratio of the width and height of one pixel).

\begin{sloppypar}
The parameter \texttt{parallel} equal to \texttt{false} determines the
prespective projection, its value \texttt{true} results in a~paralell
projection.%
\end{sloppypar}

\begin{sloppypar}
The parameter \texttt{upside} equal to \texttt{false} causes assuming the
downward orientation of the $y$~axis of the image, its value
\texttt{true} --- upward.%
\end{sloppypar}

The parameters \texttt{width} and~\texttt{height} describe the width and
height of the frame (in pixels), and the parameters \texttt{xmin}
and~\texttt{ymin} the position of the upper left corner.

The call to this procedure should precede calling all other actions with
a~camera, but it is \emph{insufficient} to fully specify the projection.
This must be done by calling \texttt{CameraInitPosf} and perhaps a~number of
calls to the procedures changing the camera position. If the frame size is
to be changed without changing the current position (e.g.\ after changing
the size of a~program window), after calling \texttt{CameraInitFramef}
it is necessary to call the procedure \texttt{CameraSetMappingf}.

\vspace{\bigskipamount}
\cprog{%
void CameraSetMagf ( CameraRecf *CPos, byte mag );}
The procedure \texttt{CameraSetMagf} sets the magnification factor, e.g.\
for supersampling. The default units of the image axes are the width and
height of a~pixel. By calling this procedure with the parameter
\texttt{mag}${}=n$ (where $n$ is a~positive integer),
we decrease these units $n$~times, which may be useful during an image
synthesis with supersampling.

\vspace{\bigskipamount}
\cprog{%
void CameraSetMappingf ( CameraRecf *CPos );}
The procedure \texttt{CameraSetMappingf} computes the transformation
matrices between the global and camera coordinate systems. This procedure
is called by all procedures setting or changing the camera position,
and therefore calling it directly from applications is usually unnecessary.
One exception is after changing the size of the frame (with use of
\texttt{CameraInitFramef}) and the current camera position is to be left
unchanged.

\vspace{\bigskipamount}
\cprog{%
void CameraProjectPoint3f ( CameraRecf *CPos, const point3f *p, \\
\ind{28}point3f *q );}
The procedure \texttt{CameraProjectPoint3f} computes the image of a~point
\texttt{*p} in a~perspective or parallel projection. The coordinates of this
image are assigned to the attributes \texttt{x} and \texttt{y} of the
parameter~\texttt{*q}. Its \texttt{z} attribute is the depth of the point,
i.e.\ its signed distance from the plane contatining the camera position and
parallel to the projection plane. This may be needed by a~hidden line or
surface algorithm.

\vspace{\bigskipamount}
\cprog{%
void CameraUnProjectPoint3f ( CameraRecf *CPos, const point3f *p, \\
\ind{30}point3f *q );}
\begin{sloppypar}
The procedure \texttt{CameraUnProjectPoint3f} computes the counterimage
of a~point \texttt{p}. The coordinates $x$, $y$ of the point~\texttt{*q}
are specified in the image coordinates, the $z$~coordinate is the depth
(in the camera system). This is thus the inversion of the transformation
computed by the procedure \texttt{CameraProjectPoint3f}.%
\end{sloppypar}

The coordinates $x$, $y$, $z$ of the counterimage are assigned to the
appropriate attributes of the parameter~\texttt{*q}.

\vspace{\bigskipamount}
\cprog{%
void CameraProjectPoint2f ( CameraRecf *CPos, const point2f *p, \\
\ind{28}point2f *q );}
\begin{sloppypar}
The procedure \texttt{CameraProjectPoint2f} computes a~projection
of the point~$\bm{p}$, whose coordinates $x$, $y$ are these of the
parameter~\texttt{p}, and the coordinate~$z$ is~$0$.%
\end{sloppypar}

The coordinates $x$, $y$ of the image are assigned to the attributes of the
parameter~\texttt{*q}.

In principle using this procedure makes sense only with parallel
projections.

\vspace{\bigskipamount}
\cprog{%
void CameraUnProjectPoint2f ( CameraRecf *CPos, const point2f *p, \\
\ind{30}point2f *q );}
\begin{sloppypar}
The procedure \texttt{CameraUnProjectPoint2f} computes a~counterimage
of the point~$\bm{p}$, whose coordinates $x$, $y$ (in the image system) are
given by the parameter~\texttt{p}, and the $z$~coordinate (in the camera
system) is~$0$.%
\end{sloppypar}

The coordinates of the counterimage are assigned to the attributes of the
parameter~\texttt{*q}.

This procedure may be used only with parallel projections.

\vspace{\bigskipamount}
\cprog{%
void CameraRayOfPixelf ( CameraRecf *CPos, float xi, float eta, \\
\ind{25}ray3f *ray );}
\begin{sloppypar}
The procedure \texttt{CameraRayOfPixel} for a~point of the projection plane,
whose image coordinates are $x={}$\texttt{xi}, $y={}$\texttt{eta}, finds
the representation of a~ray, i.e.\ a~halfline, whose origin (for
a~perspective projection) is the viewer position, and which intersect the
projection plane at that point. For a~parallel projection the ray origin is
that point and its direction is the projection direction.%
\end{sloppypar}

\begin{sloppypar}
The ray origin is assigned to the~\texttt{p} attribute of the structure
\texttt{*ray} and the unit vector, representing the ray direction is
assigned to the attribute~\texttt{v}. The ray is specified in the global
system of coordinates. The main use of this procedure is ray tracing.%
\end{sloppypar}

\vspace{\bigskipamount}
\cprog{%
void CameraInitPosf ( CameraRecf *CPos );}
The procedure \texttt{CameraInitPosf} sets the camera to the default initial
position, in which the axes $x$, $y$ and~$z$ of the camera system of
coordinates coincide with the axes $x$, $y$, $z$ of the global system. The
focal length is set to~$1$. Before calling this procedure it is necessary to
specify the frame dimensions and the image aspect, by calling
\texttt{CameraInitFramef}.

After calling \texttt{CameraInitPosf} the camera is ready to projecting
points and to the manipulations with the position, direction and the focal
length.

\vspace{\bigskipamount}
\cprog{%
void CameraSetRotCentref ( CameraRecf *CPos, point3f *centre, \\
\ind{22}boolean global\_coord, boolean global\_fixed );}
The procedure \texttt{SetCameraRotCentref} may be used to specify
the point of axes of rotations of the camera. The parameter
\texttt{*centre} is this point, \texttt{global\_coord}
specifies, whether its coordinates are specified in the global system
(\texttt{true}) or in the camera system (\texttt{false}).
The parameter \texttt{global\_fixed} specifies, whether this point
is fixed in the global (\texttt{true}), or in the camera (\texttt{false})
system, when the camera is moved.

\vspace{\bigskipamount}
\cprog{%
void CameraMoveToGf ( CameraRecf *CPos, point3f *pos );}
The procedure \texttt{CameraMoveToGf} translates (without rotation) the
camera to the position \texttt{*pos}, specified in the global system.  

\vspace{\bigskipamount}
\cprog{%
void CameraTurnGf ( CameraRecf *CPos, \\
\ind{20}float psi, float theta, float phi );}
The procedure \texttt{CameraTurnGf} sets the camera orientation specified
by the Euler angles (precession, \texttt{psi}, nutation, \texttt{theta}, 
and revolution \texttt{phi}),
in the global coordinate system.

\vspace{\medskipamount}
\begin{sloppypar}\noindent
\textbf{Remark:} The way of representing the camera orientation will some
day
be changed, and using this procedure is therefore \emph{not recommended}.
\end{sloppypar}

\vspace{\bigskipamount}
\cprog{%
void CameraMoveGf ( CameraRecf *CPos, vector3f *v );}
The procedure \texttt{CameraMoveGf} translates the camera by the
vector~\texttt{v}, specified in the global system of coordinates.

\vspace{\bigskipamount}
\cprog{%
void CameraMoveCf ( CameraRecf *CPos, vector3f *v );}
The procedure \texttt{CameraMoveGf} translates the camera by the
vector~\texttt{v}, specified in the camera system of coordinates.

\vspace{\bigskipamount}
\cprog{%
void CameraRotGf ( CameraRecf *CPos, \\
\ind{19}float psi, float theta, float phi );}
The procedure \texttt{CameraRotGf} turns the camera. The rotation is
specified by the Euler angles in the global system of coordinates.  
The axis of the rotation passes through the point set with
the procedure~\texttt{SetCameraRotCentref} (default is the origin of
the global system of coordinates).

\vspace{\bigskipamount}
\cprog{%
\#define CameraRotXGf(Camera,angle) \bsl \\
\ind{2}CameraRotGf(Camera, 0.0, angle, 0.0) \\
\#define CameraRotYGf(Camera,angle) \bsl \\
\ind{2}CameraRotGf(Camera, 0.5 * PI, angle, -0.5 * PI) \\
\#define CameraRotZGf(Camera,angle) \bsl \\
\ind{2}CameraRotGf(Camera, angle, 0.0, 0.0)}
Three macrodefinitions, which turn the camera around the three axes of the
global system of coordinates.

\vspace{\bigskipamount}
\cprog{%
void CameraRotVGf ( CameraRecf *CPos, vector3f *v, float angle );}
The procedure \texttt{CameraRotVGf} turns the camera around the axis,
whose direction is that of the vector~\texttt{v}, by the
angle~\texttt{angle}.
The coordinates of the vector~\texttt{v} are specified in the global system   
of coordinates.

\vspace{\bigskipamount}
\cprog{%
void CameraRotCf ( CameraRecf *CPos, \\
\ind{19}float psi, float theta, float phi );}
The procedure \texttt{CameraRotCf} turns the camera. The rotation
is specified by the Euler angles in the camera system of coordinates.
The axis of the rotation passes through the point set with
the procedure~\texttt{SetCameraRotCentref} (default is the origin of
the global system of coordinates).

\vspace{\bigskipamount}
\cprog{%
\#define CameraRotXCf(Camera,angle) \bsl \\
\ind{2}CameraRotCf ( Camera, 0.0, angle, 0.0 ) \\
\#define CameraRotYCf(Camera,angle) \bsl \\
\ind{2}CameraRotCf ( Camera, 0.5 * PI, angle, -0.5 * PI ) \\
\#define CameraRotZCf(Camera,angle) \bsl \\
\ind{2}CameraRotCf ( Camera, angle, 0.0, 0.0 )}
Three macrodefinitions, which turn the camera around the three axes of the
camera system of coordinates.

\vspace{\bigskipamount}
\cprog{%
void CameraRotVCf ( CameraRecf *CPos, vector3f *v, float angle );}
The procedure \texttt{CameraRotVCf} turns the camera around the axis,
whose direction is that of the vector~\texttt{v}, by the
angle~\texttt{angle}.
The coordinates of the vector~\texttt{v} are specified in the camera system   
of coordinates.

\vspace{\bigskipamount}
\cprog{%
void CameraSetFf ( CameraRecf *CPos, float f );}
The procedure \texttt{CameraSetFf} sets the focal length of the camera.

\vspace{\bigskipamount}
\cprog{%
void CameraZoomf ( CameraRecf *CPos, float fchange );}
The procedure \texttt{CameraZoomf} changes the focal length of the camera
by the factor \texttt{fchange}, which must be positive.

\vspace{\bigskipamount}
\cprog{%
boolean CameraClipPoint3f ( CameraRecf *CPos, \\
\ind{28}point3f *p, point3f *q );}
The procedure \texttt{CameraClipPoint3f} checks, whether the image
of the point~\texttt{p} fits into the frame and if it does, then it
computes the image. Its coordinates are passed using the
parameter~\texttt{q}.
The value \texttt{true} indicates that the image has been computed,
\texttt{false} is returned for points outside the visibility pyramid.

\vspace{\bigskipamount}
\cprog{%
boolean CameraClipLine3f ( CameraRecf *CPos, \\
\ind{22}point3f *p0, float t0, point3f *p1, float t1, \\
\ind{22}point3f *q0, point3f *q1 );}
The procedure \texttt{CameraClipLine3f} clips the line segment
$\{\,(1-t)\bm{p}_0+t\bm{p}_1\colon t\in[t_0,t_1]\,\}$ to the visibility
pyramid.
If the intersection is nonempty, its end points are projected and returned
with use of the parameters
\texttt{q0} and~\texttt{q1}. The procedure value is then \texttt{true}.

The procedure is an implementation of the Liang-Barsky algorithm.

\vspace{\bigskipamount}
\cprog{%
boolean CameraClipPolygon3f ( CameraRecf *CPos, \\
\ind{30}int n, const point3f *p, \\
\ind{30}void (*output)(int n, point3f *p) );}
The procedure \texttt{CameraClipPolygon3f} finds the intersection of
a~polygon with the visibility pyramid, using the Sutherland-Hodgman 
algorithm.
The parameter~\texttt{n} specifies the number of vertices in the space,
whose coordinates are given in the array~\texttt{p}; the polygon boundary
is one closed polyline.

\begin{sloppypar}
The parameter \texttt{output} points to a~procedure, which will be called if
the intersection is nonempty. The parameter~\texttt{n} of this procedure
specifies the number of vertices of the intersection. Projections of those
vertices are given in the array~\texttt{p}.%
\end{sloppypar}



\newpage
\section{Stereo camera pair}

To make a~stereo pair of images it is necessary to place two cameras
in the space, and then to render the images using the cameras.
The procedures described in this section make it easier to manipulate with
such a~pair of cameras; each procedure corresponds to some procedure of   
manipulating with one camera and it should be used \emph{instead} of that 
procedure. To project points or to cast rays one should use the procedures
described in the previous section for each camera of the pair.
\begin{figure}[ht]
  \centerline{\epsfig{file=stereo.ps}}
  \caption{Stereo pair of cameras and their common frame}
\end{figure}%

The data structure and the headers of procedures related with the
stereo pair of cameras are described in the header file \texttt{stereo.h}.

\vspace{\bigskipamount}
\cprog{%
typedef struct StereoRecf \{ \\
\ind{2}point3f\ind{4}position; \\
\ind{2}float\ind{6}d; \\
\ind{2}float\ind{6}l; \\
\ind{2}CameraRecf left, right; \\
\ind{2}trans3f\ind{4}STr, STrInv; \\
\} StereoRecf;}
The structure \texttt{StereoRecf} contains two data structures, which
describe
the left and the right camera.

\vspace{\bigskipamount}
\cprog{%
void StereoInitFramef ( StereoRecf *Stereo, boolean upside, \\
\ind{17}short width, short height, short xmin, short ymin, \\
\ind{17}float aspect );}
\begin{sloppypar}
The procedure \texttt{StereoInitFramef} initializes the dimensions of the frame
of the cameras (in pixels) and the aspect factor. This is done by calling 
\texttt{CameraInitFramef} for each camera of the pair, with these
parameters.
This procedure should be called first in the sequence of procedure calls
of the camera initialization. After calling it the cameras are \emph{not}
ready to use.%
\end{sloppypar}

The parameter \texttt{upside} specifies the orientation of the~$y$ axis in
the image system of coordinates --- see the description of
the procedure \texttt{CameraInitFramef}.

\vspace{\bigskipamount}
\cprog{%
void StereoSetDimf ( StereoRecf *Stereo, \\
\ind{21}float f, float d, float l );}
The procedure \texttt{SetStereoDimf} initialize the dimensions of the
cameras,
using the units of length of the global system of coordinates (used to
represent objects to be drawn). The parameter~\texttt{f} specifies the
focal length, i.e.\ the ratio of the distance of the frame from the   
central points of projection and the diagonal of the frame.
The parameter~\texttt{d} specifies the distance between the pupils of the
eyes of the viewer (i.e.\ the distance between the centres of projections),
and the parameter~\texttt{l} specifies the distance of the viewer from the 
plane of the frame (i.e.\ from the monitor screen). The length of diagonal 
of the frame is therefore $l/f$ units of the global system. The same length
in inches depends on the monitor.

\vspace{\medskipamount}\noindent
\textbf{Remark.} The procedure \texttt{StereoInitPosf} initialises
the attributes $f=1$, $d=0$ and~$l=1$, and these values are not very
useful. A~more proper values must therefore be assigned, by calling 
\texttt{StereoSetDimf}.

\vspace{\bigskipamount}
\cprog{%
void StereoSetMagf ( StereoRecf *Stereo, char mag );}
The procedure \texttt{StereoSetMagf} sets the magnification factor
of the cameras (e.g.\ for antialiasing) by calling
\texttt{CameraSetMagf} for each camera. The default value of this
factor (after this procedure has not been called) is~$1$.

\vspace{\bigskipamount}
\cprog{%
void StereoSetMappingf ( StereoRecf *Stereo );}
The procedure \texttt{StereoSetMappingf} computes the positions of the centres
of projections of the cameras and it prepares the cameras for using
(e.g.\ projecting points), by calling \texttt{CameraSetMappingf}.
Before calling this procedure one should call
\texttt{StereoInitFramef} and \texttt{StereoInitPosf}.

The procedures described below, which manipulate with the cameras,
call this procedure, therefore typical applications need not call it
directly.

\vspace{\bigskipamount}
\cprog{%
void StereoInitPosf ( StereoRecf *Stereo );}
\begin{sloppypar}
The procedure \texttt{StereoInitPosf} moves both cameras to the default
position. Both cameras get the same position, assigned by
the procedure \texttt{CameraInitPosf}.%
\end{sloppypar}

\newpage
%\vspace{\bigskipamount}
\cprog{%
void StereoSetRotCentref ( StereoRecf *Stereo, \\
\ind{22}point3f *centre, \\
\ind{22}boolean global\_coord, boolean global\_fixed );}
The procedure sets the point, through which axes of rotations of the cameras
pass. The way of specifying it is the same as in case of a~single camera.   

\vspace{\bigskipamount}
\cprog{%
void StereoMoveGf ( StereoRecf *Stereo, vector3f *v );}
The procedure \texttt{StereoMoveGf} translates the pair of cameras
(without rotating) by the vector~$\bm{v}$, specified in the global
system of coordinates.

\vspace{\bigskipamount}
\cprog{%
void StereoMoveCf ( StereoRecf *Stereo, vector3f *v );}
The procedure \texttt{StereoMoveCf} translates the pair of cameras
(without rotating) by the vector~$\bm{v}$, specified in the
system of coordinates of the stereo pair.

\vspace{\bigskipamount}
\cprog{%
void StereoRotGf ( StereoRecf *Stereo, \\
\ind{19}float psi, float theta, float phi );}
The procedure \texttt{StereoRotGf} turns the pair of cameras. The rotation
is specified by the Euler angles $\psi$, $\vartheta$, $\varphi$ in the
global system of coordinates.

\vspace{\bigskipamount}
\cprog{%
\#define StereoRotXGf(Stereo,angle) \bsl \\                         
\ind{2}StereoRotGf ( Stereo, 0.0, angle, 0.0 ) \\
\#define StereoRotYGf(Stereo,angle) \bsl \\
\ind{2}StereoRotGf ( Stereo, 0.5*PI, angle, -0.5*PI ) \\
\#define StereoRotZGf(Stereo,angle) \bsl \\
\ind{2}StereoRotGf ( Stereo, angle, 0.0, 0.0 )}
The above macrodefinitions turn the stereo pair around axes parallel to the
axes $x$, $y$ and~$z$ of the global system of coordinates.

\vspace{\bigskipamount}
\cprog{%
void StereoRotVGf ( StereoRecf *Stereo, vector3f *v, float angle );}
The procedure \texttt{StereoRotVGf} turns the stereo pair around the axis,
whose direction is given by the vector~$\bm{v}$, specified in the global  
system of coordinates.

\vspace{\bigskipamount}
\cprog{%
void StereoRotCf ( StereoRecf *Stereo, \\
\ind{19}float psi, float theta, float phi );}
The procedure \texttt{StereoRotCf} turns the stereo pair of cameras.
The rotation is specified by the Euler angles $\psi$, $\vartheta$,  
$\varphi$ in the stereo pair system of coordinates.

\newpage
%\vspace{\bigskipamount}
\cprog{%
\#define StereoRotXCf(Stereo,angle) \bsl \\
\ind{2}StereoRotCf ( Stereo, 0.0, angle, 0.0 ) \\
\#define StereoRotYCf(Stereo,angle) \bsl \\
\ind{2}StereoRotCf ( Stereo, 0.5*PI, angle, -0.5*PI ) \\
\#define StereoRotZCf(Stereo,angle) \bsl \\
\ind{2}StereoRotCf ( Stereo, angle, 0.0, 0.0 )}
The above macrodefinitions turn the stereo pair around axes parallel to the
axes $x$, $y$ and~$z$ of the stereo pair system of coordinates.  

\vspace{\bigskipamount}
\cprog{%
void StereoRotVCf ( StereoRecf *Stereo, vector3f *v, float angle );}
The procedure \texttt{StereoRotVCf} turns the stereo pair of cameras
around the axis, whose direction is given by the vector~$\bm{v}$, specified
in the system of coordinates of the stereo pair.

\vspace{\bigskipamount}
\cprog{%
void StereoZoomf ( StereoRecf *Stereo, float fchange );}
The procedure \texttt{StereoZoomf} multiplies the focal length of the
cameras by the parameter \texttt{fchange}. It is better not to use it at all.


