
%/* //////////////////////////////////////////////////// */
%/* This file is a part of the BSTools procedure package */
%/* written by Przemyslaw Kiciak.                        */
%/* //////////////////////////////////////////////////// */

\thispagestyle{empty}
\vspace*{4.0cm}
\centerline{\Huge BSTools}
\vspace{0.75cm}
\centerline{\Huge procedure libraries}
\vspace{1.0cm}
\centerline{\Large Przemys{\l}aw Kiciak}
\vspace{1.0cm}
\centerline{\large Version 0.35, January, 26, 2015,}
\vspace{\smallskipamount}
\centerline{\large\TeX-processed \today.}

\vspace{1.0cm}
\centerline{Most of this file contains a~quick and dirty translation}
\centerline{of the Polish documentation,}
\centerline{which is also quick and dirty.}

\vspace{1.0cm}
\noindent
\textbf{Caution:} This documentation is incomplete
and it needs a thorough revision.

\newpage
\thispagestyle{empty}
\vspace*{\fill}
\noindent
The distribution of the software described in this document is subject
to the terms of the GNU licenses published by Free Software Foundation.
The procedures, whose sources are in the 
\texttt{../src} and \texttt{../include} subdirectories are distributed
on terms of the \textsl{GNU Lesser General Public License}, whose full text
is in the file \texttt{COPYING.LIB}. The demonstration programs, test programs
and the programs which generate the pictures for this documentation
(in the directories \texttt{../demo}, \texttt{../test} and \texttt{./pict})
are distributed on the terms of the \textsl{GNU General Public License},
given in the file \texttt{COPYING}.

\vspace{\bigskipamount}
\noindent
Copyright \copyright\ by Przemys{\l}aw Kiciak, 2005--2015.

\vspace*{\fill}

\tableofcontents

\chapter{Overview}

\section{Introduction}

I wrote the BSTools procedure package in order to make experiments
being part of my scientific work, and for pleasure. The main part of the package
consists of the procedures processing B\'{e}zier and B-spline curves and
surfaces, hence the name. The mathematical properties of the curves and surfaces
and the theoretical bases of the procedures processing these objects
are described in my book

\vspace{\medskipamount}
\centerline{\large\emph{Podstawy modelowania krzywych i~powierzchni}}
\vspace{\smallskipamount}
\centerline{\large\emph{zastosowania w~grafice komputerowej}}

\vspace{\medskipamount}
\noindent
published by Wydawnictwa Naukowo-Techniczne%
\footnote{%
Apart from my book there are also many other books, which describe the
algorithms implemented in the procedures described here; to my knowledge,
none of those books has been translated yet (before 2005) to Polish.}.
The BSTools package (version 0{.}12)
is an appendix to the second edition of this book (of the year 2005).
\emph{As opposed to} the book (whose copying, even fragments, \emph{must}
be preceded by getting a~permision of the publisher),
this package \emph{may be} freely copied and distributed, and
it may be moodified and used in any programs, on terms of the
FSF Lesser GNU Public License, to be read in the file
\texttt{COPYING.LIB}.

My second book,

\vspace{\medskipamount}
\centerline{\large\emph{Konstrukcje powierzchni g{\l}adko wype{\l}niaj{\k a}cych}}
\vspace{\smallskipamount}
\centerline{\large\emph{wielok{\k a}tne otwory}}

\vspace{\medskipamount}
\noindent
published by Oficyna Wydawnicza Politechniki Warszawskiej (prace naukowe,
Elektronika, z.~159, 2007) contains a~description of the constructions
of surfaces of class~$G^1$ and~$G^2$, implemented in the library
\texttt{libeghole} (the version 0.18 of the package, accompanying the book,
contains two libraries, \texttt{libg1hole} and~\texttt{libg2hole},
which have been merged and considerably extended).

\vspace{\medskipamount}
The procedures of this package may be used for any (hopefully decent)
purpose, for example to write a~modelling system or a~graphical application.
To do this, they must be made robust, i.e.\ it is necessary to implement
and test a~full system of error detection and signalling.
As it is known, the last person appropriate for making tests of any
procedure is its author (but there is no justification for him if
he does not do it). People interested in participation in this enterprise
and interested in using the package in applications are welcome.


\section{Short description of the libraries}

The BSTools package currently consists of the following libraries:
\begin{description}
\item[\texttt{libpkvaria}]--- varieties, like scratch memory management,
sorting etc.
\item[\texttt{libpknum}]--- numerical procedures used in various constructions
of B-spline curves, but suitable for general use.
\item[\texttt{libpkgeom}]--- geometric procedures.
\item[\texttt{libcamera}]--- perspective and parallel projections.
\item[\texttt{libpsout}]--- PostScript\raisebox{3pt}{\tiny(TM)} picture
generation.
\item[\texttt{libmultibs}]--- processing of B\'{e}zier and B-spline
curves and surfaces.
\item[\texttt{libraybez}]--- ray tracing (computing ray/patch intersections).
\item[\texttt{libeghole}]--- filling polygonal holes in piecewise bicubic
spline surfaces with $G^1$, $G^2$ and $G^1Q^2$ continuity.
\item[\texttt{libbsmesh}]--- procedures for processing meshes representing
surfaces.
\item[\texttt{libg1blending}]--- procedures of shape optimization of
B-spline patches of degree~$(2,2)$, of class~$G^1$.
\item[\texttt{libg2blending}]--- procedures of shape optimization of bicubic
B-spline patches and mesh surfaces of class~$G^2$.
\item[\texttt{libmengerc}]--- shape optimization by minimization of integral
Menger curvature of closed B-spline curves.
\item[\texttt{libbsfile}]--- Reading/writing files with data describing
curves and surfaces.
\item[\texttt{libxgedit}]--- support for interaction (using windows
and widgets) with an XWindow application (mainly for demonstration programs).
\end{description}

The procedures are written in~C, with no hardware or system dependencies,
with one exception: the sorting procedure in the \texttt{libpkvaria} library
assumes the little-endian byte ordering. A~migration to a~big-endian processor
(e.g.\ Motorola) requires the appropriate reimplementation of this procedure
(this has been done, but it has not been tested yet).


\section{Compilation}

The \texttt{Makefile}s are written for the Linux system. To compile the package,
the documentation and the demos, it is necessary to have
\begin{itemize}
  \item the GNU make program,
  \item the gcc compiler and the ar program,
  \item XWindow and OpenGL libraries (for demonstration programs),
  \item the \TeX\ system (for the documentation, which
    uses the \LaTeX$2_{\varepsilon}$ package and Concrete Roman and Euler
    font packages),
  \item plus Ghostscript and Ghostview programs for convenient browsing
    of the documentation and the pictures generated by test programs.
\end{itemize}
To compile the full package, run \texttt{make} from the main package directory.
This may be preceded by \texttt{make clean}, in order to force the compilation
of all sources.

The demonstration programs work in the XWindow system. There are no special
requirements (like Motif etc.). Some demonstration programs use OpenGL, and
the following libraries are needed: \texttt{libGL}, \texttt{libGLU} and
\texttt{libGLX}.


\section{Header files}

The header files are stored in the \texttt{../include} directory.
Each library may have more than one header file, to shorten the compilation
time for programs not using all procedures.
\begin{description}
\item[\texttt{libpkvaria}]--- the file \texttt{pkvaria.h}.
\item[\texttt{libpknum}]--- the files \texttt{pknumf.h} and \texttt{pknumd.h},
with procedure prototypes of IEEE-754 single precision (\texttt{float})
and double precision (\texttt{double}) floating-point arithmetic
versions respectively. Including the \texttt{pknum.h} file causes
including both above files, which helps to compile programs using procedures
of both precisions. 
\item[\texttt{libpkgeom}]--- the files \texttt{pkgeomf.h}, \texttt{pkgeomd.h} and
\texttt{pkgeom.h}, which make it possible to use the procedures of the
single, double and both precisions.

The convex hull procedures have a~separate file
\texttt{convh.h}, with the prototypes for both precisions.
\item[\texttt{libcamera}]--- the files \texttt{cameraf.h}, \texttt{camerad.h}
and \texttt{camera.h} with the descriptions of the cameras,
i.e.\ objects, which implement perspective and parallel projections,
in single, double, and both precisions.

The files \texttt{stereof.h}, \texttt{stereod.h} and \texttt{stereo.h}
describe pairs of such cameras, which may be used for the generation of
stereo pairs of pictures.

\item[\texttt{libpsout}]--- the file \texttt{psout.h} contains the prototypes
of all procedures in this library.
\item[\texttt{libmultibs}]--- the files \texttt{multibsf.h}, \texttt{multibsd.h}
and \texttt{multibs.h} describe procedures of single, double and both precisions.
\item[\texttt{libraybez}]--- the files \texttt{raybezf.h} (single precision),
\texttt{raybezd.h} (double precision) and \texttt{raybez.h} (both versions).
\item[\texttt{libeghole}]--- the files \texttt{eg1holef.h}, \texttt{eg2holef.h}
(single precision), \texttt{eg1holed.h} and \texttt{eg2holed.h}
(double precision). There are no header files for both precision versions together.
\item[\texttt{libbsmesh}]--- the file \texttt{bsmesh.h}
\item[\texttt{libg1blending}]--- the files \texttt{g1blengingf.h} and
\texttt{g1blendingd.h} (single and double precision respectively).
\item[\texttt{libg2blending}]--- the files \texttt{g2blendingf.h},
\texttt{g2blendingd.h} and \texttt{g2mblendingd.h}. Some procedures have the
double precision version only, because of the insufficient range of the
single precision numbers.
\item[\texttt{libbsfile}]--- the file \texttt{bsfile.h}, the input/output
procedures implemented at this point use double precision only.
\item[\texttt{libxgedit}]--- the files \texttt{xgedit.h} and
\texttt{xgledit.h}. Additional files, \texttt{xgergb.h} and
\texttt{xglergb.h} are not supposed to be included directly by applications
(they are included by \texttt{xgedit.h} and \texttt{xgledit.h}). These files
contain some colour definitions, with English colour names.
\end{description}

The procedures are compiled as C programs and the header files contain
the code, which causes the C++ compiler to see them in this way.
Therefore the C++ programs should be linked with these libraries without
problems.


\section{Linking order}

The procedures of some libraries refer to procedures of other libraries.
To link the program it is necessary to list the libraries in the proper 
order (otherwise the compiler may fail to resolve some references).
The proper order of the libraries is

\vspace{\medskipamount}
\centerline{\texttt{xgedit raybez bsfile g2blending g1blending bsmesh camera}}

\centerline{\texttt{eghole multibs psout pkgeom pknum pkvaria}}
\vspace{\medskipamount}

\noindent
and it should be preserved in the user-written \texttt{Makefile}s.
Unused libraries may of course be omitted.


\section{Principles of modification}

The GNU license does not limit the modifications, which one might want to do
(but the fact of modifying the software must be notified in the source code,
to make clear that it was not the original author, who made the mess).
Therefore any principles of making the modifications are only
the authors wishes.
\begin{enumerate}
\item Except of PostScript file generation and the \texttt{libxgedit}
  library all procedures are totally independent of any environment, in
  which they might work (and this should be preserved).
\item Making a~change in a~procedure of the single or double precision,
  should be accompnied by a~similar modification of the other version
  (if it exists; if not, it is desired to write that other version,
  using the same algorithm, though for some algorithms the single precision may
  be insufficient).
\item After each change the documentation should be updated, and a~test
  program should be written. I~would be grateful for notifying me
  about changes, so that I~can incorporate them into the future versions
  of the package.
\end{enumerate}

