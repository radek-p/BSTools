
%/* //////////////////////////////////////////////////// */
%/* This file is a part of the BSTools procedure package */
%/* written by Przemyslaw Kiciak.                        */
%/* //////////////////////////////////////////////////// */

\chapter{The \texttt{libpsout} library}

The \texttt{libpsout} library consists of procedures, which write
PostScript\raisebox{3pt}{\tiny(TM)} code to a~text file. The file
represents a~picture, defined by subsequent calls of the library
procedures, which produce commands of drawing lines etc.

The library procedures are basic, which actually write the PostScript
commands, and additional, which make it simpler to draw line
segments --- one can draw their parts of various widths, mark some
points, draw arrows etc.


\section{Basic procedures}

The phrase ,,the procedure draws a~line segment'' or whatever is to
be interpreted in such a~way that based on the parameter values the procedure
writes the text, whose processing by a~PostScript interpreter will cause
the appearance of the line segment on the picture.

\vspace{\bigskipamount}
\cprog{%
extern short ps\_dec\_digits;}
The variable \texttt{ps\_dec\_digits} determines the number of decimal digits
in the fractional parts of coordinates of points written to the PostScript
file. The default value is~$3$; if the resolution is set to $600$DPI and
there are no large scaling, this should be enough.

\vspace{\bigskipamount}
\cprog{%
void ps\_WriteBBox ( float x1, float y1, float x2, float y2 );}
The procedure \texttt{ps\_WriteBBox} called \emph{before} opening the
PostScript file sets up the dimensions of the bounding box, to be written
in the preamble. The picture should (but does not have to) fit in the
bounding box, which will be used by a~typesetting system (\TeX) to place
the picture on a~page.
The first two parameters are coordinates of the lower left corner, and
the last two parameters are coordinates of the upper right corner.
The parameters are specified in ``big points''
($1$ big point (\texttt{1bp} in~\TeX) is $1/72"$).

The appropriate numbers are easy to find using \texttt{GhostView}. The procedure
call with these numbers may be added to the program producing the picture,
which may then be recompiled and executed again.

\vspace{\bigskipamount}
\cprog{%
void ps\_OpenFile ( const char *filename, unsigned int dpi ); \\
void ps\_CloseFile ( void );}
The procedure \texttt{ps\_OpenFile} creates a~file, whose name is specified
by the parameter \texttt{filename} (if a~file of than name already exists,
then it is deleted) and writes a~PostScript header.
The header contains the coordinates of the bounding box
(if the procedure \texttt{ps\_WriteBBox} has been previously called)
and a~scaling command, which sets the initial unit length. This unit is determined
by the parameter \texttt{dpi}, e.g.\ if its value is $600$, then
the unit length is $1/600"$. Some procedures are written in such a~way that
symbols they draw (arrows etc.)\ look best with the unit that long.

The procedure \texttt{ps\_CloseFile} closes the PostScript file. It should
be called when the picture is finished.

\vspace{\bigskipamount}
\cprog{%
void ps\_Write\_Command ( char *command );}
The procedure \texttt{ps\_Write\_Command} writes an arbitrary text in the
PostScript file. Any PostScript command may be appended to the file,
even if there is no ``ready'' procedure in the library to write such a~command.
Therefore all possibilities of PostScript are available.

\vspace{\bigskipamount}
\cprog{%
void ps\_Set\_Gray ( float gray );}
The procedure \texttt{ps\_Set\_Gray} sets the specified gray level
int the current graphics state. The value of the parameter \texttt{gray}
should be in the interval $[0,1]$.

\vspace{\bigskipamount}
\cprog{%
void ps\_Set\_RGB ( float red, float green, float blue );}
The procedure \texttt{ps\_Set\_RGB} sets the colour with the specified
red, green and blue components in the current graphics state.
The values of the parameters \texttt{red}, \texttt{green} and~\texttt{blue}
should be in the interval $[0,1]$.

\vspace{\bigskipamount}
\cprog{%
void ps\_Set\_Line\_Width ( float w );}
The procedure \texttt{ps\_Set\_Line\_Width} sets the specified line width
in the current graphics state. The parameter~\texttt{w} should have
a~positive value.

\vspace{\bigskipamount}
\cprog{%
void ps\_Draw\_Line ( float x1, float y1, float x2, float y2 );}
The procedure \texttt{ps\_Draw\_Line} draws a~line segment, whose end points
have the coordinates \texttt{x1}, \texttt{y1} and \texttt{x2}, \texttt{y2}.
The line width, colour and other properties are determined by the current
settings in the graphics state.

\vspace{\bigskipamount}
\cprog{%
void ps\_Set\_Clip\_Rect ( float w, float h, float x, float y );}
The procedure \texttt{ps\_Set\_Clip\_Rect} sets the clipping rectangle
of dimensions \texttt{w} (width) and~\texttt{h} (height), whose lower
left vertex has the coordinates \texttt{x}, \texttt{y}.

The clipping is done in addition to all clipping paths specified before.
Cancellation of clipping may be done only in such a~way, that
we save the graphics state by calling \texttt{ps\_GSave~();},
then we set the clipping path and after drawing we restore the initial
graphics state by calling \texttt{ps\_GRestore~();}.

\vspace{\bigskipamount}
\cprog{%
void ps\_Draw\_Rect ( float w, float h, float x, float y );}
The procedure \texttt{ps\_Draw\_Rect} draws the edges of the~rectangle
of dimensions \texttt{w} (width) and~\texttt{h} (height), whose lower
left vertex has the coordinates \texttt{x}, \texttt{y}. The width and
colour of the lines drawn are determined by the current graphics state.

\vspace{\bigskipamount}
\cprog{%
void ps\_Fill\_Rect ( float w, float h, float x, float y );}
The procedure \texttt{ps\_Fill\_Rect} fills the rectangle of dimensions
\texttt{w} (width) and~\texttt{h} (height), whose lower left vertex
has the coordinates \texttt{x}, \texttt{y}. The colour of the
rectangle is determined by the current graphics state.

\vspace{\bigskipamount}
\cprog{%
void ps\_Hatch\_Rect ( float w, float h, float x, float y, \\
\ind{21}float ang, float d );}
The procedure \texttt{ps\_Hatch\_Rect} draws a~number of lines to hatch
the rectangle of dimensions \texttt{w} (width) and~\texttt{h} (height),
whose lower left vertex has the coordinates \texttt{x}, \texttt{y}.
The angle of inclination of the lines is specified by the parameter
\texttt{ang} (in radians), and their distance is the value of the parameter
\texttt{d}. The colour and width of the lines is determined by the
current graphics state.

\vspace{\bigskipamount}
\cprog{%
void ps\_Draw\_Polyline2f ( int n, const point2f *p ); \\
void ps\_Draw\_Polyline2d ( int n, const point2d *p );}
\begin{sloppypar}
The procedures \texttt{ps\_Draw\_Polyline2f} and~\texttt{ps\_Draw\_Polyline2d}
draw (open) polylines consisting of $n-1$~line segments, whose
vertices ($n$ points, i.e.\ $2n$ floating point numbers)
are given in the array~\texttt{p}. The colour and line width are
determined by the current graphics state.%
\end{sloppypar}

\vspace{\bigskipamount}
\cprog{%
void ps\_Draw\_Polyline2Rf ( int n, const point3f *p ); \\
void ps\_Draw\_Polyline2Rd ( int n, const point3d *p );}
\begin{sloppypar}
The procedures \texttt{ps\_Draw\_Polyline2Rf}
and~\texttt{ps\_Draw\_Polyline2Rd}
draw (open) polylines consisting of $n-1$~line segments, whose
vertices ($n$ points, i.e.\ $3n$ floating point numbers, the homogeneous
coordinates) are given in the array~\texttt{p}. The colour and line width
are determined by the current graphics state.%
\end{sloppypar}

\vspace{\bigskipamount}
\cprog{%
void ps\_Set\_Clip\_Polygon2f ( int n, const point2f *p ); \\
void ps\_Set\_Clip\_Polygon2d ( int n, const point2d *p );}
The procedures \texttt{ps\_Set\_Clip\_Polygon2f}
and~\texttt{ps\_Set\_Clip\_Polygon2d} set the clipping path, being a~closed
polyline with $n$ vertices given in the array~\texttt{p}.
The PostScript interpreter clips to all clipping paths set before
(except for the patchs set after saving the graphics state, which has then
been restored).

\vspace{\bigskipamount}
\cprog{%
void ps\_Set\_Clip\_Polygon2Rf ( int n, const point3f *p ); \\
void ps\_Set\_Clip\_Polygon2Rd ( int n, const point3d *p );}
The procedures \texttt{ps\_Set\_Clip\_Polygon2Rf}
and~\texttt{ps\_Set\_Clip\_Polygon2Rd} set the clipping path, being a~closed
polyline with $n$ vertices given in the array~\texttt{p}, which contains
their homogeneous coordinates.

\newpage
%\vspace{\bigskipamount}
\cprog{%
void ps\_Fill\_Polygon2f ( int n, const point2f *p ); \\
void ps\_Fill\_Polygon2d ( int n, const point2d *p );}
The procedures \texttt{ps\_Fill\_Polygon2f} and~\texttt{ps\_Fill\_Polygon2d}
fill a~polygon with $n$ vertices given in the array~\texttt{p}.

\vspace{\bigskipamount}
\cprog{%
void ps\_Fill\_Polygon2Rf ( int n, const point3f *p ); \\
void ps\_Fill\_Polygon2Rd ( int n, const point3d *p );}
The procedures \texttt{ps\_Fill\_Polygon2Rf} and~\texttt{ps\_Fill\_Polygon2Rd}
fill a~polygon with $n$ vertices (the homogeneous coordinates) given in the
array~\texttt{p}.

\vspace{\bigskipamount}
\cprog{%
void ps\_Draw\_BezierCf ( const point2f *p, int n ); \\
void ps\_Draw\_BezierCd ( const point2d *p, int n );}
The procedures \texttt{ps\_Draw\_BezierCf} and~\texttt{ps\_Draw\_BezierCd}
draw B\'{e}zier curves of degree~$n$,
whose $n+1$ control points are given in the array~\texttt{p}. For
$n>1$ a~polyline of~$50$ line segments is drawn.

The points of the curve are computed without using the \texttt{libmultibs}
library.

\vspace{\bigskipamount}
\cprog{%
void ps\_Draw\_Circle ( float x, float y, float r );}
The procedure \texttt{ps\_Draw\_Circle} draws the~circle with the radius~\texttt{r}
and the centre \texttt{(x,y)}.

\vspace{\bigskipamount}
\cprog{%
void ps\_Fill\_Circle ( float x, float y, float r );}
The procedure \texttt{ps\_Draw\_Circle} fills the circle with the radius~\texttt{r}
and the centre \texttt{(x,y)}.

\pagebreak[1]
\vspace{\bigskipamount}
\cprog{%
void ps\_Draw\_Arc ( float x, float y, float r, float a0, float a1 );}
The procedure \texttt{ps\_Draw\_Arc} draws an arc of a~circle with the
centre \texttt{(x,y)}, radius~\texttt{r} and the angles of beginning and
end point \texttt{a0} and~\texttt{a1}. The meaning of all parameters is just
like of the parameters of the PostScript operator \texttt{arc}, except that
the angles are specified in radians (not in degrees).

\vspace{\bigskipamount}
\cprog{%
void ps\_Mark\_Circle ( float x, float y );}
The procedure \texttt{ps\_Mark\_Circle} draws a~mark (small circle
with a~white dot) at \texttt{(x,y)}.

\vspace{\bigskipamount}
\cprog{%
void ps\_Init\_Bitmap ( int w, int h, int x, int y, byte b ); \\
void ps\_Out\_Line ( byte *data );}
The procedure \texttt{ps\_Init\_Bitmap} prepares outputting of a~monochrome
bitmap image (black--grey--white). The image is \texttt{w} pxels wide,
\texttt{h} pixels high (the pixel width and height are $1$ unit
of the current system of coordinates), and the lower left corner
is at~\texttt{(x,y)}. The parameter~\texttt{b} specifies the number of
bits per pixel, which has to be $1$, $2$, $4$ or~$8$.

After calling the procedure \texttt{ps\_Init\_Bitmap} it is necessary to call
\texttt{h}~times the procedure \texttt{ps\_Out\_Line}, whose parameter is
an array of $\lceil w/b\rceil$ bytes. Each byte describes $8/b$ packed pixels.
Each call of this procedure causes writing one row of pixels to the
PostScript file, from top of the image to the bottom.

The data are output in hexadecimal form, without any compression.
Therefore the PostScript file with such an image may be large.

\vspace{\bigskipamount}
\cprog{%
void ps\_Init\_BitmapP ( int w, int h, int x, int y ); \\
void ps\_Out\_LineP ( byte *data );}
The procedure \texttt{ps\_Init\_BitmapP} prepares outputting of a~monochrome
bitmap image (black--grey--white) in a~packed form. The image is
\texttt{w} pxels wide,
\texttt{h} pixels high (the pixel width and height are $1$ unit
of the current system of coordinates), and the lower left corner
is at~\texttt{(x,y)}. The colour of each pixel is specified by one byte.

After calling the procedure \texttt{ps\_Init\_BitmapP} it is necessary to call
\texttt{h}~times the procedure \texttt{ps\_Out\_LineP}, whose parameter is
an array of $w$ bytes.
Each call of this procedure causes writing one row of pixels to the
PostScript file, from top of the image to the bottom.

The data are output in hexadecimal form, with a~simple run-length encoding
compression. Therefore the PostScript file with such an image may smaller.

\vspace{\bigskipamount}
\cprog{%
void ps\_Init\_BitmapRGB ( int w, int h, int x, int y ); \\
void ps\_Out\_LineRGB ( byte *data );}
The procedure \texttt{ps\_Init\_BitmapRGB} prepares outputting of a~colour
bitmap image. The image is \texttt{w} pxels wide,
\texttt{h} pixels high (the pixel width and height are $1$ unit
of the current system of coordinates), and the lower left corner
is at~\texttt{(x,y)}. The colour of each pixel is specified by three bytes.

After calling the procedure \texttt{ps\_Init\_BitmapRGB} it is necessary to call
\texttt{h}~times the procedure \texttt{ps\_Out\_LineRGB}, whose parameter is
an array of $3w$ bytes.
Each call of this procedure causes writing one row of pixels to the
PostScript file, from top of the image to the bottom.

The data are output in hexadecimal form, without any compression.

\vspace{\bigskipamount}
\cprog{%
void ps\_Init\_BitmapRGBP ( int w, int h, int x, int y ); \\
void ps\_Out\_LineRGBP ( byte *data );}
The procedure \texttt{ps\_Init\_BitmapRGBP} prepares outputting of a~colour
bitmap image in a~packed form. The image is \texttt{w} pxels wide,
\texttt{h} pixels high (the pixel width and height are $1$ unit
of the current system of coordinates), and the lower left corner
is at~\texttt{(x,y)}. The colour of each pixel is specified by three bytes.

After calling the procedure \texttt{ps\_Init\_BitmapRGBP} it is necessary to call
\texttt{h}~times the procedure \texttt{ps\_Out\_LineRGBP}, whose parameter is
an array of $3w$ bytes.
Each call of this procedure causes writing one row of pixels to the
PostScript file, from top of the image to the bottom.

The data are output in hexadecimal form, with a~simple run-length encoding.
The decompression procedure is coded in PostScript. It is not extremely
fast nor effective, but often sufficient. Some day it might be
replaced by something better.

\vspace{\bigskipamount}
\cprog{%
void ps\_Newpath ( void );}
The procedure \texttt{ps\_Newpath} causes writing the command
\texttt{newpath}, which initializes a~path. This path may then be built
with the procedures \texttt{ps\_MoveTo} and \texttt{ps\_LineTo}, and then
it may be processed by the PostScript interpreter
in the way described by the procedure \texttt{ps\_Write\_Command}
(it may write \texttt{stroke} or anything else).

\vspace{\bigskipamount}
\cprog{%
void ps\_MoveTo ( float x, float y ); \\
void ps\_LineTo ( float x, float y );}
The procedures \texttt{ps\_MoveTo} and~\texttt{ps\_LineTo} output the PostScript
commands which build a~path: respectively \texttt{moveto} and~\texttt{lineto}
with appropriate parameters. The path constructed with these procedures may
be used in an arbitrary way.

\vspace{\bigskipamount}
\cprog{%
void ps\_ShCone ( float x, float y, float x1, float y1, \\
\ind{17}float x2, float y2 );}
The procedure \texttt{ps\_ShCone} draws a~shaded (grey) cone, i.e.\
triangle, whose vertices are \texttt{(x,y)}, \texttt{(x+x1,y+y1)}
and~\texttt{(x+x2,y+y2)}.

\vspace{\bigskipamount}
\cprog{%
void ps\_GSave ( void ); \\
void ps\_GRestore ( void );}
The procedure \texttt{ps\_GSave} writes the command \texttt{gsave}, which causes
saving the current graphics state (on the appropriate stack of the
PostScript interpreter).

The procedure \texttt{ps\_GRestore} writes the command \texttt{grestore}, which
restores the graphics state previously saved on the stack.

\vspace{\bigskipamount}
\cprog{%
void ps\_BeginDict ( int n ); \\
void ps\_EndDict ( void );}
The procedure \texttt{ps\_BeginDict} writes the command
\texttt{$n$ dict begin}, where $n$ is the number given as the parameter.
For the PostScript interpreter it is the order of creating a~new dictionary
with the capacity of~$n$ symbols, and opening it on top of the stack
of open dictionaries.

The procedure \texttt{ps\_EndDict} writes the command \texttt{end},
which causes removing the dictionary from the top of the dictionary stack.
Each call of \texttt{ps\_BeginDict} should be compelemnted by a~call
to~\texttt{ps\_EndDict}.

\vspace{\bigskipamount}
\cprog{%
void ps\_DenseScreen ( void );}
The procedure \texttt{ps\_DenseScreen} causes changing the current
rasterization pattern to the pattern of twice as large liniature.
Thin grey lines may look better printed with such a~pattern, though
the precision of reproducing grey levels is worse.

\vspace{\bigskipamount}
\cprog{%
void ps\_GetSize ( float *x1, float *y1, float *x2, float *y2 );}
The procedure \texttt{ps\_GetSize} may help to get the dimensions of the
rectangle bounding the picture (or rather its elements drawn before
calling this procedure). Howevet, this procedure does not take into
account any effects of the commands output by \texttt{ps\_Write\_Command}
(which may be \texttt{scale}, \texttt{translate} or drawing commands),
and it does not take into account clipping. Therefore it is
not a~very useful procedure (using GhostView is much better).

The parameters obtain the values of coordinates of the rectangle,
into which, as it seems to the library, the picture fits.
The units are determined by the resolution specified when the file
has been created.


\section{Additional procedures}

Additional procedures make it easier to draw line segments, whose parts
have various colours and widths. They may also have some parts marked
with symbols like dots, arrows, ticks etc.

\vspace{\bigskipamount}
\cprog{%
\#define tickl  10.0 \\
\#define tickw   2.0 \\
\#define tickd   6.0 \\
\#define dotr   12.0 \\
\#define arrowl 71.0 \\
\#define arroww 12.5}
\begin{sloppypar}
The abobe symbolic constants determine a~half of length (\texttt{tickl})
and width (\texttt{tickw}) of a~bar (tick) drawn across the current line,
dot radius and length and half of width of arrows.
\end{sloppypar}

These dimensions are chosen so that the symbols look good, if the unit length
(specified by the second parameter of \texttt{ps\_OpenFile}) is $1/600"$.

\vspace{\bigskipamount}
\cprog{%
void psl\_SetLine ( float x1, float y1, float x2, float y2, \\
\ind{19}float t1, float t2 );}
The procedure \texttt{psl\_SetLine} sets the line, whose segments and points
will be drawn and marked. The line passes through the points \texttt{(x1,y1)}
and~\texttt{(x2,y2)}, which must be different. To these two points
correspond the parameters \texttt{t1} and \texttt{t2}, which must be different.

Setting such a~line causes computing also its unit directional vector~$\bm{v}$,
which will be used by other procedures in various constructions.

\newpage
%\vspace{\bigskipamount}
\cprog{%
void psl\_GetPointf ( float t, float *x, float *y );}
The procedure \texttt{psl\_GetPointf} computes the point of the line
recently set by the procedure \texttt{psl\_SetLine}, corresponding to
the parameter~\texttt{t}. Its coordinates are assigned to the
parameters \texttt{*x} and \texttt{*y}.

\vspace{\bigskipamount}
\cprog{%
float psl\_GetDParam ( float dl );}
The procedure \texttt{psl\_GetDParam} computes the increment of the
line parameter, which corresponds to the translation by a~vector
of length \texttt{dl}.

\vspace{\bigskipamount}
\cprog{%
void psl\_GoAlong ( float s, float *x, float *y );}
The procedure \texttt{psl\_GoAlong} on entry gets the point
$\bm{p}=$\texttt{(*x,*y)}. On return the parameters \texttt{*x}
and~\texttt{*y} have values of coordinates of the image of~$\bm{p}$
in the translation along the current line by the distance~\texttt{s}.

\vspace{\bigskipamount}
\cprog{%
void psl\_GoPerp ( float s, float *x, float *y );}
The procedure \texttt{psl\_GoPerp} on entry gets the point
$\bm{p}=$\texttt{(*x,*y)}. On return the parameters \texttt{*x}
and~\texttt{*y} have values of coordinates of the image of~$\bm{p}$
in the translation prependicular to the current line by the
distance~\texttt{s}.

\vspace{\bigskipamount}
\cprog{%
void psl\_Tick ( float t );}
The procedure \texttt{psl\_Tick} draws a~tick on the current line, at the
point corresponding to the parameter~\texttt{t}, which is a~bar perpendicular
to the line.

\vspace{\bigskipamount}
\cprog{%
void psl\_BTick ( float t );}
The procedure \texttt{psl\_BTick} draws a~tick on the current line, at the
point corresponding to the parameter~\texttt{t}, which is a~bar perpendicular
to the line. This bar is thicker and longer than that drawn by
\texttt{psl\_Tick}. The idea is to draw it using the background colour
and then to draw the ordinary bar on that background.

\vspace{\bigskipamount}
\cprog{%
void psl\_HTick ( float t, boolean left );}
The procedure \texttt{psl\_HTick} draws a~tick on the current line, at the
point correspoding to the parameter~\texttt{t}, which is a~half of the bar
drawn by the procedure \texttt{psl\_Tick}. The parameter \texttt{left}
determines the side of the line to draw this half tick.

\vspace{\bigskipamount}
\cprog{%
void psl\_Dot ( float t );}
The procedure \texttt{psl\_Dot} marks the point of the current line corresponding
to the parameter~\texttt{t}. The mark is a~circle, whose radius is
\texttt{dotr}.

\vspace{\bigskipamount}
\cprog{%
void psl\_HDot ( float t );}
The procedure \texttt{psl\_HDot} marks a~point of the current line
corresponding to the parameter~\texttt{t}, by a~circle of slightly greater
radius. It is intended to draw the circle using the background colour
before drawing the proper circle using \texttt{psl\_Dot}.

\vspace{\bigskipamount}
\cprog{%
void psl\_TrMark ( float x, float y );}
The procedure \texttt{psl\_TrMark} marks the point \texttt{(x,y)} (not
related with the current line) by a~white isoscelses triangle with black edges.
One edge is horizontal and the top vertex is at the marked point.

\vspace{\bigskipamount}
\cprog{%
void psl\_BlackTrMark ( float x, float y );}
The procedure \texttt{psl\_BlackTrMark} marks the point \texttt{(x,y)}
(not related with the current line) by a~black isoscelses triangle.
One edge is horizontal and the top vertex is at the marked point.

\vspace{\bigskipamount}
\cprog{%
void psl\_HighTrMark ( float x, float y );}
The procedure a \texttt{psl\_HighTrMark} marks the point \texttt{(x,y)}
(not related with the current line) by a~white triangle with black edges.
One edge is horizontal and the top vertex is at the marked point.
The height of this triangle is greater than of that drawn by
\texttt{psl\_TrMark}.

\vspace{\bigskipamount}
\cprog{%
void psl\_BlackHighTrMark ( float x, float y );}
The procedure a \texttt{psl\_BlackHighTrMark} marks the point \texttt{(x,y)}
(not related with the current line) by a~black triangle.
One edge is horizontal and the top vertex is at the marked point.
The height of this triangle is greater than of that drawn by
\texttt{psl\_BlackTrMark}.

\vspace{\bigskipamount}
\cprog{%
void psl\_LTrMark ( float t ); \\
void psl\_BlackLTrMark ( float t ); \\
void psl\_HighLTrMark ( float t ); \\
void psl\_BlackHighLTrMark ( float t );}
The above procedures mark the point of the current line
corresponding to the parameter~\texttt{t}, using the
symbols drawn by \texttt{psl\_TrMark},
\texttt{psl\_BlackTrMark}, \texttt{psl\_HighTrMark}
and~\texttt{psl\_BlackHighTrMark} respectively. Essentially, they are
appropriate for horizontal lines.

\vspace{\bigskipamount}
\cprog{%
void psl\_Arrow ( float t, boolean sgn );}
The procedure \texttt{psl\_Arrow} marks the point of the current line
corresponding to the parameter~\texttt{t}, by an arrow (which is a~triangle)
having the direction of the line.
The parameter \texttt{sgn} selects the orientation of the arrow.

\vspace{\bigskipamount}
\cprog{%
void psl\_BkArrow ( float t, boolean sgn );}
The procedure \texttt{psl\_BkArrow} marks the point of the current line
corresponding to the parameter~\texttt{t}, drawing the area which is
a~background of the arrow to be drawn by\texttt{psl\_Arrow}.
The parameter \texttt{sgn} selects the orientation of the arrow.

\newpage
%\vspace{\bigskipamount}
\cprog{%
void psl\_Draw ( float ta, float tb, float w );}
The procedure \texttt{psl\_Draw} draws a~segment of the current line,
between the points corresponding to the parameters
\texttt{ta} and \texttt{tb}. The parameter~\texttt{w} specifies the
width of the line segment.

\vspace{\bigskipamount}
\cprog{%
void psl\_ADraw ( float ta, float tb, float ea, float eb, float w );}
The procedure \texttt{psl\_ADraw} draws a~segment of the current line,
whose end points are obtained as follows: first, the point
$\bm{p}_a$, which corresponds to \texttt{ta} is computed, and then
the unit vector having direction of the current line, multiplied by
the parameter~\texttt{ea} is added (see the description of the procedure
\texttt{psl\_SetLine}). The other end point of the segment is obtained in
a~similar way, using the parameters \texttt{tb} and \texttt{eb}.
The parameter ~\texttt{w} specifies the width of the line.

\vspace{\bigskipamount}
\cprog{%
void psl\_MapsTo ( float t );}
The procedure \texttt{psl\_MapsTo} marks the point of the current line
corresponding to the parameter~\texttt{t} with an arrow having a~shape
different than that of~\texttt{psl\_Arrow}, more appropriate
for commutative diagrams.

\vspace{\bigskipamount}
\cprog{%
void psl\_DrawEye ( float t, byte cc, float mag, float ang );}
The procedure \texttt{psl\_DrawEye} marks the point of the current line
corresponding to the parameter~\texttt{t} with an eye symbol.
It may be used to denote the viewer position on various schematic pictures.

The parameter \texttt{cc} should be $0$, $1$, $2$ or $3$, and it determines
the orientation of the symbol. The parameter \texttt{mag}
specifies the magnification, and the parameter \texttt{ang} specifies the
angle (in radians) of additional rotation of the image, which may be necessary
to obtain a~good looking effect.


\newpage
%\vspace{\bigskipamount}
\noindent
\textbf{Example:} The program below draws the picture shown in
Figure~\ref{fig:psout}.

\vspace{\medskipamount}
\noindent{\ttfamily
\#include <string.h> \\
\#include "psout.h" \\
int main ( void ) \\
\{ \\
\ind{2}ps\_WriteBBox ( 12, 13, 264, 80 ); \\
\ind{2}ps\_OpenFile ( "psout.ps", 600 ); \\
\ind{2}psl\_SetLine ( 200, 600, 2200, 100, 0.0, 4.0 ); \\
\ind{2}psl\_Draw ( 0.5, 1.5, 2.0 ); \\
\ind{2}psl\_Draw ( 2.0, 3.0, 6.0 ); \\
\ind{2}psl\_ADraw ( 3.5, 4.0, 0.0, -arrowl, 1.0 ); \\
\ind{2}psl\_DrawEye ( 0.0, 1, 1.2, 0.15 ); \\
\ind{2}psl\_HTick ( 1.0, false ); \\
\ind{2}psl\_HTick ( 1.5, true ); \\
\ind{2}psl\_Tick ( 2.0 ); \\
\ind{2}psl\_BlackHighLTrMark ( 2.5 ); \\
\ind{2}psl\_LTrMark ( 3.0 ); \\
\ind{2}psl\_Dot ( 3.5 ); \\
\ind{2}psl\_Arrow ( 4.0, true ); \\
\ind{2}ps\_CloseFile (); \\
\ind{2}exit ( 0 ); \\
\} /*main*/}
\begin{figure}[t]
  \centerline{\epsfig{file=psout.ps}}
  \caption{\label{fig:psout}A~line with marked points}
\end{figure}


