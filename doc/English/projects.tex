
%/* //////////////////////////////////////////////////// */
%/* This file is a part of the BSTools procedure package */
%/* written by Przemyslaw Kiciak.                        */
%/* //////////////////////////////////////////////////// */

\chapter{Obsolete projects}

Once upon a~time I~wrote a~very simple procedure of filling polygonal holes
in a~bicubic spline surface with biquintic patches so as to obtain the
tangent plane ($G^1$) continuity. This procedure is described in detail in
my book \emph{Podstawy mo\-de\-lo\-wa\-nia krzywych i~powierzchni}, and it may
have some educational value, which is why I~left it, even if the procedures
of filling polygonal holes in the \texttt{libeghole} library produce much
better results.

\section{\label{sect:obsolete:G1}Filling polygonal holes}

The demonstration program \texttt{polep} uses the procedure of filling
a~polygonal hole in a~generalized bicubic B-spline surface, with B\'{e}zier
patches of degree~$(5,5)$, joined with each other and with the patches
around the hole with tangent plane ($G^1$) continuity. A~detailed description
of this procedure and the underlying theory is in my book
\emph{Podstawy modelowania krzywych i~powierzchni} (in Polish).
The construction carried out by this procedure is much simpler and
less general than the constructions performed by the procedures
collected in the libraries \texttt{libg1hole} and \texttt{libg2hole};
it was developed much earlier and the experience gathered then helped
in developing the constructions implemented in these libraries.
Below is a~description of the representation of data for this procedure
and its parameters.

The single precision source code is in the file \texttt{g1holef.c},
and its header file is \texttt{g1holef.h}.
The corresponding double precision source files are \texttt{g1holed.c}
and \texttt{g1holed.h} respectively.

\vspace{\bigskipamount}
\cprog{%
boolean FillG1Holef ( int hole\_k, point3f*(*GetBezp)(int i, int j), \\
\ind{22}float beta1, float beta2, \\
\ind{22}point3f *hpcp );}
The procedure \texttt{FillG1Holef} constructs a~surface filling a~$k$-sided hole
in a~surface. The parameters are as follows:

The parameter \texttt{hole\_k} specifies the number $k$ of sides (and
corners) of the hole. Its value has to be~$3$, $5$, $6$, $7$ or~$8$.

The parameter \texttt{GetBezp} is a~pointer of a~procedure called by
\texttt{FillG1Holef} in order to get the data, i.e.\ the control points
of bicubic B\'{e}zier patches around the hole. The procedure has to
return the pointer to an array with the control points of the appropriate
patch.

The patches around the hole are numbered with pairs of numbers $(i,j)$,
as shown in Figure~\ref{fig:g1:patch:num:1}; the variable~$i$
(the parameter~\texttt{i}) has the value from $0$ to $k-1$, the variable~$j$
(the parameter~\texttt{j}) is either $1$ or~$2$.
The figure shows also the order in which the control points are to be
stored in the array. For each patch it is necessary to supply only $8$~points,
whose numbers are shown.
\begin{figure}[ht]
  \centerline{\epsfig{file=g1patches1.ps}}
  \caption{\label{fig:g1:patch:num:1}The numbering scheme of the
    patches around the hole}
  \centerline{and the order of their control points in the arrays}
\end{figure}

The patches surrounding the hole have to satisfy the compatibility
conditions given below, which concern their corners, partial
and mixed partial derivatives. The $m$-th control point of the
patch $(i,j)$ is denoted by $\bm{p}^{(i,j)}_m$. In addition,
$l=i+1\bmod k$.
\begin{itemize}
  \item Corner compatibility conditions:
    \begin{align*}
      &{}\bm{p}^{(i,1)}_0=\bm{p}^{(i,2)}_3 \quad\mbox{oraz}\quad
      \bm{p}^{(i,2)}_0=\bm{p}^{(l,1)}_3.
    \end{align*}
  \item Partial derivatives compatibility conditions:
    \begin{align*}
      &{}\bm{p}^{(i,1)}_0-\bm{p}^{(i,1)}_1=\bm{p}^{(i,2)}_2-\bm{p}^{(i,2)}_3, \\
      &{}\bm{p}^{(i,2)}_0-\bm{p}^{(i,2)}_1=\bm{p}^{(l,1)}_7-\bm{p}^{(l,1)}_3, \\
      &{}\bm{p}^{(i,2)}_0-\bm{p}^{(i,2)}_4=\bm{p}^{(l,1)}_2-\bm{p}^{(l,1)}_3.
    \end{align*}
  \item Mixed partial derivatives compatibility conditions:
    \begin{align*}
      &{}\bm{p}^{(i,1)}_4-\bm{p}^{(i,1)}_5=\bm{p}^{(i,2)}_6-\bm{p}^{(i,2)}_7, \\
      &{}\bm{p}^{(i,2)}_0-\bm{p}^{(i,2)}_1-\bm{p}^{(i,2)}_4+\bm{p}^{(i,2)}_5=
      \bm{p}^{(l,1)}_6-\bm{p}^{(l,1)}_7-\bm{p}^{(l,1)}_2+\bm{p}^{(l,1)}_3.
    \end{align*}
\end{itemize}

The parameters \texttt{beta1} and~\texttt{beta2} are factors, by which
some vectors constructed by the procedure \texttt{FillG1Holef} are multiplied.
By default, their value should be~$1$, but they may be modified to
improve the filleting surface shape, if necessary (i.e.\ if the effect
of using~$1$ is unsatisfactory).

The parameter \texttt{hpcp} points to the array, in which the procedure
is supposed to store the constructed control points of the B\'{e}zier
patches filling the hole.

The procedure returns \texttt{true} if the construction has been successful,
or \texttt{false} otherwise.

\vspace{\bigskipamount}
\cprog{%
extern void (*G1OutCentralPointf)( point3f *p ); \\
extern void (*G1OutAuxCurvesf)( int ncurves, int degree, \\
\ind{32}const point3f *accp, float t ); \\
extern void (*G1OutStarCurvesf)( int ncurves, int degree, \\
\ind{33}const point3f *sccp ); \\
extern void (*G1OutAuxPatchesf)( int npatches, int degu, int degv, \\
\ind{33}const point3f *apcp );}
The variables above are ``hooks'' for procedures, which may output the
partial results of the construction. Their initial value is \texttt{NULL}.
If the address of the appropriate procedure is assigned to any of those
variables before calling \texttt{FillG1Hole}, then this procedure will be
called and it may output the data given by the parameters or
draw the appropriate picture.

The procedure pointed by the variable \texttt{G1OutCentralPointf} obtains as
the parameter the pointer to the ``central'' point of the filleting surface
(i.e.\ the common point of the patches filling the hole). This procedure
is allowed to modify this point, i.e.\ assign it new coordinates, as such
an interference with the construction is possible and sometimes desirable.

The other procedures must not modify the data they get. The procedure
pointed by the variable \texttt{G1OutAuxCurvesf} is called with the
parameters, which describe the so called auxiliary curves --- B\'{e}zier
curves of degree \texttt{degree} (here it is always~$3$). Each curve is
given in a~separate call of this procedure.

The procedure pointed by the variable \texttt{G1OutStarCurvesf} is
called with the parameters, which describe the common boundary curves
of the patches filling the hole (the construction of those curves is
one of the first steps of the algorithm). The procedure parameters
describe the B\'{e}zier representation of degree~$3$ of the curves.
Each call is made to output one curve.%
\begin{figure}[ht]
  \centerline{\epsfig{file=g1patches2.ps}}
  \caption{\label{fig:g1:patch:num:2}A~surface with a~hole filled by the procedure
    \texttt{FillG1Holef}}
\end{figure}

The procedure pointed by the variable \texttt{G1OutAuxPatchesf} is
called with the parameters, which describe the so called auxiliary patches,
which determine the tangent planes at all points of the common curves
of the patches filling the hole. The auxiliary patches are of degree
$(3,1)$ (the parameters \texttt{ndegu} and \texttt{ndegv} are equal to
$3$ and~$1$ respectively). The parameters \texttt{apcp} is a~pointer of
an array with the control points of one auxiliary patch.

\vspace{\medskipamount}
\noindent\textbf{Caution:}
The current version of the procedure does not contain any code to handle
exceptional situations, which might cause the safe return in case of
failure. Such a~code has to be added and tested if the procedure is to be
built into a~``production'' software, to be used in industrial applications.
Moreover, most of the library procedures are not prepared to deal with
exceptional situations (\texttt{exit} is called in case of error),
and therefore the package may be used mainly for experiments
(and this is the cause, which made me write the package).
