
%/* //////////////////////////////////////////////////// */
%/* This file is a part of the BSTools procedure package */
%/* written by Przemyslaw Kiciak.                        */
%/* //////////////////////////////////////////////////// */

\chapter{The \texttt{libraybez} library}

The \texttt{libraybez} library consists of procedures, whose main
(but not only) purpose is supporting ray tracing, and more precisely
computing intersections of rays with B\'{e}zier patches. To do this,
the procedures build trees of recursive patch subdivision, to
accelerate solving the appropriate nonlinear equations, by eliminating
multiple computations of the same thing.

Possible extensions of this library should include constructing trees
for B-spline patches (also trimmed) and trees with additional attributes,
which would help in computing intersections of surfaces.

\section{Common definitions and procedures}

\cprog{%
typedef struct \{ \\
\ind{2}float xmin, xmax, ymin, ymax, zmin, zmax; \\
\} Box3f;}
\hspace*{\parindent} The structure \texttt{Box3f} represents a~rectangular
parallelpiped. In the representation of a~piece of a~patch it is used to
locate this piece in the space (the piece is in the associated
parallelpiped).

\vspace{\bigskipamount}
\cprog{%
typedef struct \{ \\
\ind{2}point3f\ind{2}p; \\
\ind{2}vector3f nv; \\
\ind{2}float\ind{4}u, v, t; \\
\} RayObjectIntersf, *RayObjectIntersfp;}
The structure \texttt{RayObjecIntersf} represents a~point of intersection of
a~ray with a~patch. It consists of the following fields:
\texttt{p} --- the point of intersection, \texttt{nv} --- normal vector of the
patch at this point, \texttt{u}, \texttt{v}, \texttt{t} ---
parameters pf the patch and the ray corresponding to the intersection point.


\section{Binary subdivision trees for polynomial patches}

\cprog{%
typedef struct \_BezPatchTreeVertexf \{ \\
\ind{2}struct \_BezPatchTreeVertexf \\
\ind{12}*left, *right, *up; \\
\ind{2}point3f\ind{3}*ctlpoints; \\
\ind{2}float\ind{5}u0, u1, v0, v1; \\
\ind{2}Box3f\ind{5}bbox; \\
\ind{2}point3f\ind{3}pcent; \\
\ind{2}float\ind{5}maxder; \\
\ind{2}short int level; \\
\ind{2}char\ind{6}divdir; \\
\ind{2}char\ind{6}pad; \\
\} BezPatchTreeVertexf, *BezPatchTreeVertexfp;}
\hspace*{\parindent}The structure \texttt{\_BezPatchTreeVertexf} represents
a~vertex of a~binary tree of recursive subdivision of a~polynomial
B\'{e}zier patch~$\bm{p}$.

The fields of this structure are used to store the following data:
\texttt{left}, \texttt{right}, \texttt{up} --- pointers to the vertices of
the left and right subtrees and to the parent vertex (the vertex,
whose the current vertex is the root of one of the subtrees) respectively,
\texttt{ctlpoints} --- pointer to the array with the control points
of the piece represented by this vertex,
\texttt{u0}, \texttt{u1}, \texttt{v0}, \texttt{v1}
--- numbers which describe the domain of the piece, $[u_0,u_1]\times[v_0,v_1]$,
\texttt{bbox} --- bounding box (rectangular parallelpiped) of the piece,
\texttt{pcent} --- the point
$\bm{p}((u_0\nolinebreak +\nolinebreak u_1)/2,
(v_0\nolinebreak +\nolinebreak v_1)/2)$, \texttt{maxder}
--- upper estimation of the length of the vectors of both partial derivatives
of this piece with respect to local parameters, \texttt{level} --- level of
the vertex in the tree, \texttt{divdir} --- indicator of the direction
of further division of the piece, \texttt{pad} --- unused (it aligns the
size of the structure to an even number of bytes).

\vspace{\bigskipamount}
\cprog{%
typedef struct \{ \\
\ind{2}unsigned char\ind{9}n, m; \\
\ind{2}unsigned int\ind{10}cpsize; \\
\ind{2}BezPatchTreeVertexfp root; \\
\} BezPatchTreef, *BezPatchTreefp;}
The structure \texttt{BezPatchTreef} represents a~tree of recursive binary
subdivision of a~polynomial B\'{e}zier patch. Its fields are the following:
\texttt{n}, \texttt{m} --- degree of the patch with respect to the
variables $u$~and~$v$, \texttt{cpsize} --- amount of memory needed to store
the control points, \texttt{root} --- pointer to the root of the tree.

\newpage
%\vspace{\bigskipamount}
\cprog{%
BezPatchTreefp \\
\ind{2}rbez\_NewBezPatchTreef ( unsigned char n, unsigned char m, \\
\ind{24}float u0, float u1, float v0, float v1, \\
\ind{24}point3f *ctlpoints );}
The procedure \texttt{rbez\_NewBezPatchTreef} creates a~tree of binary subdivision
of a~polynomial B\'{e}zier patch and it returns the pointer to the structure,
which represents this tree. Initially the tree consists only of the root,
which represents the entire patch.

The parameter \texttt{n}~and~\texttt{m} specify the degree of the patch
with respect to $u$~and~$v$ respectively. The parameters \texttt{u0},
\texttt{u1}, \texttt{v0} and~\texttt{v1} specify the domain of the patch,
i.e.\ the rectangle $[u_0,u_1]\times[v_0,v_1]$ (if the patch has been
obtained by dividing a~B-spline patch, then these numbers should be
the appropriate knots).

The parameter \texttt{ctlpoints} points to the array of control points
of the patch.

The value returned by the procedure is the pointer to the structure,
which describes the tree. The memory blocks for this structure and
for the structures representing the vertices are allocated with
\texttt{malloc}.

\vspace{\bigskipamount}
\cprog{%
void rbez\_DestroyBezPatchTreef ( BezPatchTreefp tree );}
The procedure \texttt{rbez\_DestroyBezPatchTreef} deallocates (by calling
\texttt{free}) the memory blocks used to represent a~tree of binary
patch division. The parameter \texttt{tree} is a~pointer to the structure
representing the tree.

\vspace{\bigskipamount}
\cprog{%
BezPatchTreeVertexp \\
\ind{2}rbez\_GetBezLeftVertexf ( BezPatchTreefp tree, \\
\ind{24}BezPatchTreeVertexfp vertex ); \\
BezPatchTreeVertexfp \\
\ind{2}rbez\_GetBezRightVertexf ( BezPatchTreefp tree, \\
\ind{25}BezPatchTreeVertexfp vertex );}
The procedures \texttt{rbez\_GetBezLeftVertexf}
and~\texttt{rbez\_GetBezRightVertexf}
return pointers to the vertex of the left and right subtree of a~vertex of a~tree
of patch subdivision respectively.

The parameters: \texttt{tree} --- pointer to the structure representing the
tree, \texttt{vertex} --- pointer to one of the vertices of this tree.

The procedures return the pointers to the appropriate (left or right) vertices.
If it does not exist, then the procedures divide the piece of the patch
represented by the vertex pointed by~\texttt{vertex}, they create
both root vertices of the subtrees and they return the pointer to one of
those new vertices. For each vertex, either both subtrees exist
or both are empty.

\newpage
%\vspace{\bigskipamount}
\cprog{%
int rbez\_FindRayBezPatchIntersf ( BezPatchTreef *tree, \\
\ind{26}ray3f *ray, \\
\ind{26}int maxlevel, int maxinters, \\
\ind{26}int *ninters, RayObjectIntersf *inters );}
The procedure \texttt{FindRayBezPatchIntersf} computes the common points
of a~ray (a~halfline) with a~polynomial B\'{e}zier patch in~$\R^3$.

The parameters: \texttt{tree} --- pointer to the tree of patch subdivision;
\texttt{ray} --- pointer to the ray (the structure \texttt{ray3f} is defined
in the header file \texttt{geomf.h}; \texttt{maxlevel} --- limit of the height
of the tree (the procedure will not require vertices beyond that level,
therefore they will not be created if the only reason to call the
procedures returning pointers to the vertices is the ray tracing);
\texttt{maxinters} --- capacity of the array \texttt{inters}, in which the
results are to be stored. The array must have at least that length,
and the procedure will terminate after computing at most that many
intersections. The number of intersection points found on return
is assigned to \texttt{*ninters}.

The number of intersection points is also the value of the procedure.


\section{Binary subdivision trees for rational B\'{e}zier patches}

Binary trees of recursive subdivision of rational B\'{e}zier patches
are constructed and processed in an almost identical way as
the trees for the polynomial patches. All structures and procedures
described in the previous section have their counterparts here.

\vspace{\bigskipamount}
\cprog{%
typedef struct \_RBezPatchTreeVertexf \{ \\
\ind{2}struct \_RBezPatchTreeVertexf \\
\ind{12}*left, *right, *up; \\
\ind{2}point4f\ind{3}*ctlpoints; \\
\ind{2}float\ind{5}u0, u1, v0, v1; \\
\ind{2}Box3f\ind{5}bbox; \\
\ind{2}point3f\ind{3}pcent; \\
\ind{2}float\ind{5}maxder; \\
\ind{2}short int level; \\
\ind{2}char\ind{6}divdir; \\
\ind{2}char\ind{6}pad; \\
\} RBezPatchTreeVertexf, *RBezPatchTreeVertexfp;}
\hspace*{\parindent}The structure \texttt{\_BezPatchTreeVertexf} represents
a~vertex of a~binary tree of recursive subdivision of a~rational
B\'{e}zier patch~$\bm{p}$.

The fields of this structure are used to store the following data:
\texttt{left}, \texttt{right}, \texttt{up} --- pointers to the vertices of
the left and right subtrees and to the parent vertex (the vertex,
whose the current vertex is the root of one of the subtrees) respectively,
\texttt{ctlpoints} --- pointer to the array with the control points
of the homogeneous patch representing the piece corresponding to this vertex,
\texttt{u0}, \texttt{u1}, \texttt{v0}, \texttt{v1}
--- numbers which describe the domain of the piece, $[u_0,u_1]\times[v_0,v_1]$,
\texttt{bbox} --- bounding box (rectangular parallelpiped) of the piece,
\texttt{pcent} --- the point
$\bm{p}((u_0\nolinebreak +\nolinebreak u_1)/2,
(v_0\nolinebreak +\nolinebreak v_1)/2)$, \texttt{maxder}
--- upper estimation of the length of the vectors of both partial derivatives
of this piece with respect to local parameters, \texttt{level} --- level of
the vertex in the tree, \texttt{divdir} --- indicator of the direction
of further division of the piece, \texttt{pad} --- unused (it aligns the
size of the structure to an even number of bytes).

\vspace{\bigskipamount}
\cprog{%
typedef struct \{ \\
\ind{2}unsigned char\ind{9}n, m; \\
\ind{2}unsigned int\ind{10}cpsize; \\
\ind{2}RBezPatchTreeVertexfp root; \\
\} RBezPatchTreef, *RBezPatchTreefp;}
The structure \texttt{RBezPatchTreef} representuje a~binary tree of recursive
subdivision of a~rational B\'{e}zier patch. Its fields are as follows:
\texttt{n}, \texttt{m} --- degrees of the patch with respect
to the cariables $u$~and~$v$,
\texttt{cpsize} --- amount of memory needed to store the control points,
\texttt{root} --- pointer to the root of the tree.

\vspace{\bigskipamount}
\cprog{%
RBezPatchTreefp \\
\ind{2}rbez\_NewRBezPatchTreef ( unsigned char n, unsigned char m, \\
\ind{24}float u0, float u1, float v0, float v1, \\
\ind{24}point4f *ctlpoints );}
The procedure \texttt{rbez\_NewRBezPatchTreef} creates a~tree of binary subdivision
of a~rational B\'{e}zier patch and it returns the pointer to the structure,
which represents this tree. Initially the tree consists only of the root,
which represents the entire patch.

The parameter \texttt{n}~and~\texttt{m} specify the degree of the patch
with respect to $u$~and~$v$ respectively. The parameters \texttt{u0},
\texttt{u1}, \texttt{v0} and~\texttt{v1} specify the domain of the patch,
i.e.\ the rectangle $[u_0,u_1]\times[v_0,v_1]$ (if the patch has been
obtained by dividing a~NURBS patch, then these numbers should be
the appropriate knots).

The parameter \texttt{ctlpoints} points to the array of control points
of the homogeneous patch in~$\R^4$.

The value returned by the procedure is the pointer to the structure,
which describes the tree. The memory blocks for this structure and
for the structures representing the vertices are allocated with
\texttt{malloc}.

\vspace{\bigskipamount}
\cprog{%
void rbez\_DestroyRBezPatchTreef ( RBezPatchTreefp tree );}
The procedure \texttt{rbez\_DestroyRBezPatchTreef} deallocates (by calling
\texttt{free}) the memory blocks used to represent a~tree of binary
patch division. The parameter \texttt{tree} is a~pointer to the structure
representing the tree.

\vspace{\bigskipamount}
\cprog{%
RBezPatchTreeVertexp \\
\ind{2}rbez\_GetRBezLeftVertexf ( RBezPatchTreefp tree, \\
\ind{24}RBezPatchTreeVertexfp vertex ); \\
RBezPatchTreeVertexfp \\
\ind{2}rbez\_GetRBezRightVertexf ( RBezPatchTreefp tree, \\
\ind{25}RBezPatchTreeVertexfp vertex );}
The procedures \texttt{rbez\_GetRBezLeftVertexf}
and~\texttt{rbez\_GetRBezRightVertexf}
return pointers to the vertex of the left and right subtree of a~vertex of a~tree
of patch subdivision respectively.

The parameters: \texttt{tree} --- pointer to the structure representing the
tree, \texttt{vertex} --- pointer to one of the vertices of this tree.

The procedures return the pointers to the appropriate (left or right) vertices.
If it does not exist, then the procedures divide the piece of the patch
represented by the vertex pointed by~\texttt{vertex}, they create
both root vertices of the subtrees and they return the pointer to one of
those new vertices. For each vertex, either both subtrees exist
or both are empty.

\vspace{\bigskipamount}
\cprog{%
int rbez\_FindRayRBezPatchIntersf ( RBezPatchTreef *tree, \\
\ind{26}ray3f *ray, \\
\ind{26}int maxlevel, int maxinters, \\
\ind{26}int *ninters, RayObjectIntersf *inters );}
The procedure \texttt{FindRayRBezPatchIntersf} computes the common points
of a~ray (a~halfline) with a~rational B\'{e}zier patch in~$\R^3$.

The parameters: \texttt{tree} --- pointer to the tree of patch subdivision;
\texttt{ray} --- pointer to the ray (the structure \texttt{ray3f} is defined
in the header file \texttt{geomf.h}; \texttt{maxlevel} --- limit of the height
of the tree (the procedure will not require vertices beyond that level,
therefore they will not be created if the only reason to call the
procedures returning pointers to the vertices is the ray tracing);
\texttt{maxinters} --- capacity of the array \texttt{inters}, in which the
results are to be stored. The array must have at least that length,
and the procedure will terminate after computing at most that many
intersections. The number of intersection points found on return
is assigned to \texttt{*ninters}.

The number of intersection points is also the value of the procedure.

\begin{figure}[ht]
  \centerline{\epsfig{file=raybezp.ps}}
  \caption{\label{fig:raybezp}Image of a~rational B\'{e}zier patch rendered by ray tracing}
  \centerline{with use of the procedure \texttt{rbez\_FindRayRBezPatchIntersf}. On the right}
  \centerline{side the pieces of the domain divided during the computation are shown}
\end{figure}
Figure~\ref{fig:raybezp} shows an image of a~rational B\'{e}zier patch
of degree~$(5,5)$ obtained with use of this procedure.
The full source code of the program which rendered this image is
given in the file \texttt{../cpict/raybez.c}.


