
%/* //////////////////////////////////////////////////// */
%/* This file is a part of the BSTools procedure package */
%/* written by Przemyslaw Kiciak.                        */
%/* //////////////////////////////////////////////////// */

\chapter{The \texttt{libpkgeom} library}

The \texttt{libpkgeom} library consists of procedures, which implement basic
operations on points and vectors in two-, three- and four-dimensional
spaces. The operations are: addition, subtraction, multiplication,
interpolation and affine transformations. In addition, there is a~procedure
of computing the convex hull of a~set of points in the plane. Other
procedures of computational geometry will also be placed in this library.

All names of data types and procedures end with the letter \texttt{f} or
\texttt{d}, which indicates the representation of coordinates --- of single
(\texttt{float}) or double (\texttt{double}) precision.

\section{Point and vector operations}

\cprog{%
typedef struct point2f \{ \\
\ind{2}float x, y; \\
\} point2f vector2f; \\
\mbox{} \\
typedef struct point3f \{ \\
\ind{2}float x, y, z; \\
\} point3f vector3f; \\
\mbox{} \\
typedef struct point4f \{ \\
\ind{2}float X, Y, Z, W; \\
\} point4f vector4f;}
\hspace*{\parindent}%
Points and vectors are represented with pairs, triples and quadruples
of numbers. An essential property of these representations is the absence
of any additional data. Due to that, for example an array of $n$~points
in a~plane may be passed to a~procedure, which processes an array
of $2n$ numbers. Therefore these structures should not be converted to
C++ classes, and in particular no classes with additional attributes
may be defined.

A~structure of type \texttt{point3f} may represent a~point in the $3$D space
or a~point of a~plane. In the latter case the fields
\texttt{x}, \texttt{y}, \texttt{z} describe homogeneous coordinates
of this point --- its cartesian coordinates are equal to \texttt{x/z}
and~\texttt{y/z}. Analoguously, a~structure of type \texttt{point4f}
consists of fields, whose values are homogeneous coordinates of a~point in the
three-dimensional space.

\newpage
%\vspace{\bigskipamount}
\cprog{%
typedef struct ray3f \{ \\
\ind{2}point3f p; \\
\ind{2}vector3f v; \\
\} ray3f;}
Structures of type \texttt{ray3f} represent rays, i.e.\ halflines in~$\R^3$,
with the origin at the point~\texttt{p} and with the direction described by the
vector~\texttt{v}.

\vspace{\bigskipamount}
\cprog{%
typedef union trans2f \{ \\
\ind{2}struct \{ \\
\ind{4}float a11, a12, a13; \\
\ind{4}float a21, a22, a23; \\
\ind{2}\} U0; \\
\ind{2}struct \{ \\
\ind{4}float a[2][3]; \\
\ind{4}short detsgn; \\
\ind{2}\} U1; \\
\} trans2f; \\
\mbox{} \\
typedef union trans3f \{ \\
\ind{2}struct \{ \\
\ind{4}float a11, a12, a13, a14; \\
\ind{4}float a21, a22, a23, a24; \\
\ind{4}float a31, a32, a33, a34; \\
\ind{2}\} U0; \\
\ind{2}struct \{ \\
\ind{4}float a[3][4]; \\
\ind{4}short detsgn; \\
\ind{2}\} U1; \\
\} trans3f;}
Structures of type \texttt{trans2f} and~\texttt{trans3f} represent affine
transformations of the two- and three-dimensional spaces. The representation
consists of a~matrix $3\times 3$ or $4\times 4$, whose last row is either
$[0,0,1]$ or $[0,0,0,1]$. Therefore this row is not stored. The field
\texttt{detsgn} describes the sign of the determinant of this matrix.

\vspace{\bigskipamount}
\cprog{%
void SetPoint2f ( point2f *p, float x, float y ); \\
\#define SetVector2f(v,x,y) SetPoint2f ( v, x, y ) \\
void SetPoint3f ( point3f *p, float x, float y, float z ); \\
\#define SetVector3f(v,x,y,z) SetPoint3f ( v, x, y, z ) \\
void SetPoint4f ( point4f *p, float X, float Y, float Z, float W ); \\
\#define SetVector4f(v,X,Y,Z,W) SetPoint4f ( v, X, Y, Z, W )}
The above procedures and macros initialize point and vector
representations.

\vspace{\bigskipamount}
\cprog{%
void TransPoint2f ( const trans2f *tr, const point2f *p, \\
\ind{20}point2f *q ); \\
void TransPoint3f ( const trans3f *tr, const point3f *p, \\
\ind{20}point3f *q );}
The above procedures compute the image $\bm{q}$ of a~point~$\bm{p}$
in an affine transformation of the two- or three-dimensional space.

\vspace{\bigskipamount}
\cprog{%
void TransVector2f ( const trans2f *tr, const vector2f *v, \\
\ind{21}vector2f *w ); \\
void TransVector3f ( const trans3f *tr, const vector3f *v, \\
\ind{21}vector3f *w );}
The above procedures compute the image~$\bm{w}$ of the vector~$\bm{v}$
in a~linear transformation, which is the linear part of the affine
transformation represented by the variable \texttt{*tr}.

\vspace{\bigskipamount}
\cprog{%
void TransContra3f ( const trans3f *tri, const vector3f *v, \\
\ind{21}vector3f *w );}
The procedure \texttt{TransContra3f} computes the image~$\bm{w}$ of the
vector~$\bm{v}$ in a~linear transformation, whose matrix is the transposition
of the matrix of the linear part of the affine transformation represented by
the parameter~\texttt{*tri}. If the vector~$\bm{v}$ is the normal vector
of some plane~$\pi$, and the transformation represented by~\texttt{*tri}
is the \emph{inverse} of some transformation~$A$, then the computed
vector~$\bm{w}$ is the normal vector of the plane~$A(\pi)$.

\vspace{\bigskipamount}
\cprog{%
void Trans3Point2f ( const trans3f *tr, const point2f *p, \\
\ind{21}point2f *q );}
The procedure \texttt{Trans3Point2f} applies the affine transformation
\texttt{*tr} to the point $\bm{p}\in\R^3$, whose first two coordinates are
the values of the fields \texttt{x}~and~\texttt{y} of the parameter~\texttt{*p},
and the third coordinate is~$0$.
The coordinates $x$~and~$y$ of the image are assigned to the appropriate fields
of the parameter~\texttt{*q}.

\vspace{\bigskipamount}
\cprog{%
void Trans2Point3f ( const trans2f *tr, const point3f *p, \\
\ind{21}point3f *q );}
The procedure \texttt{Trans2Point3f} computes the image of
a~point~$\bm{p}\in\R^2$, represented by homogeneous coordinates,
in an affine transformation.

\vspace{\bigskipamount}
\cprog{%
void Trans3Point4f ( const trans3f *tr, const point4f *p, \\
\ind{21}point4f *q );}
The procedure \texttt{Trans3Point4f} applies the affine transformation
\texttt{*tr} to the point~$\bm{p}\in\R^3$, whose four homogeneous coordinates
are the values of the fields of the parameter~\texttt{*p}.

The homogeneous coordinates of the image (such that the weight
coordinates of the point and its image are the same) are assigned to
the appropriate fields of the parameter~\texttt{*q}.

\vspace{\bigskipamount}
\cprog{%
void IdentTrans2f ( trans2f *tr ); \\
void IdentTrans3f ( trans3f *tr );}
The procedures \texttt{IdentTrans2f} and~\texttt{IdentTrans3f} initialize
the structures \texttt{*tr} to the values representing the identity mappings
of the two- and three-dimensional spaces respectively.

\vspace{\bigskipamount}
\cprog{%
void CompTrans2f ( trans2f *s, trans2f *t, trans2f *u ); \\
void CompTrans3f ( trans3f *s, trans3f *t, trans3f *u );}
The procedures \texttt{CompTrans2f} and~\texttt{CompTrans3f} compute
the composition of the affine transformations represented by the
parameters~\texttt{*t} and~\texttt{*u}, and assign it to the
parameter~\texttt{*s}. This composition is equivalent to the transformation
\texttt{*u} \emph{followed by}~\texttt{*t}.

\vspace{\bigskipamount}
\cprog{%
void GeneralAffineTrans3f ( trans3f *tr, \\
\ind{24}vector3f *v1, vector3f *v2, vector3f *v3 );}
The procedure \texttt{GeneralAffineTrans3f} computes the composition
of the transformation represented by the parameter~\texttt{*tr}
with the transformation, whose linear part is represented by the matrix
$[\bm{v}_1,\bm{v}_2,\bm{v}_3]$ (and the translation vector is~$\bm{0}$).
The composition is assigned to the parameter \texttt{*tr}.

\vspace{\bigskipamount}
\cprog{%
void ShiftTrans2f ( trans2f *tr, float tx, float ty ); \\
void ShiftTrans3f ( trans3f *tr, float tx, float ty, float tz );}
The procedures \texttt{ShiftTrans2f} and~\texttt{ShiftTrans3f} compute
the composition of the transformation represented by the
parameter~\texttt{*tr} and the translation by the vector
$[t_x,t_y]^T$ or $[t_x,t_y,t_z]^T$.
The composition is assigned to the parameter \texttt{*tr}.
 
\vspace{\bigskipamount}
\cprog{%
void RotTrans2f ( trans2f *tr, float angle );}
The procedure \texttt{RotTrans2f} computes the composition of the affine
transformation represented by~\texttt{*tr} with the rotation
around the point $[0,0]^T$ by the angle \texttt{angle}.
The composition is assigned to the parameter \texttt{*tr}.

\vspace{\bigskipamount}
\cprog{%
void Rot3f ( trans3f *tr, byte j, byte k, float angle );}
The procedure \texttt{Rot3f} computes the composition of the transformation
represented by the parameter \texttt{*tr} with the rotation around one of
the axes of the system of coordinates. The axis is specified by the parameters
\texttt{j}~and~\texttt{k}, which must be different numbers from the set
$\{1,2,3\}$. For example the rotation in the plane $xy$ (around the $z$ axis)
corresponds to \texttt{j}${}=1$, \texttt{k}${}=2$. The rotation angle is equal to
\texttt{angle}.
The composition is assigned to the parameter \texttt{*tr}.

\newpage
%\vspace{\bigskipamount}
\cprog{%
\#define RotXTrans3f(tr,angle) Rot3f ( tr, 2, 3, angle ) \\
\#define RotYTrans3f(tr,angle) Rot3f ( tr, 3, 1, angle ) \\
\#define RotZTrans3f(tr,angle) Rot3f ( tr, 1, 2, angle )}
The above macros call the procedure \texttt{Rot3f} in order to compute the
composition of the affine transformation represented by the parameter
\texttt{*tr} with a~rotation around the $x$, $y$, $z$ axes, i.e.\ in
the planes $yz$, $zx$ and~$xy$ respectively.

\vspace{\bigskipamount}
\cprog{%
void RotVTrans3f ( trans3f *tr, vector3f *v, float angle );}
\begin{sloppypar}
The procedure \texttt{RotVTrans3f} computes the composition of the
affine transformation represented by the parameter
\texttt{*tr} with the rotation around the line, which passes through
the point $[0,0,0]^T$ and has the direction of the \emph{unit}
vector~$\bm{v}$, by the angle \texttt{angle}.
The composition is assigned to the parameter \texttt{*tr}.
\end{sloppypar}

\vspace{\bigskipamount}
\cprog{%
void FindRotVEulerf ( const vector3f *v, float angle, \\
\ind{22}float *psi, float *theta, float *phi );}
The procedure \texttt{FindRotVEulerf} computes the Euler angles (precession
\texttt{*psi}, nutation \texttt{*theta} and revolution \texttt{*phi}),
representing the rotation around the line, whose direction is specified by
the \emph{unit} vector~$\bm{v}$ by the angle \texttt{angle}.

\vspace{\bigskipamount}
\cprog{%
float TrimAnglef ( float angle );}
\begin{sloppypar}
The procedure \texttt{TrimAnglef} returns the number~$\alpha$, which is an
element of the interval $[-\pi,\pi]$, and which differs from the parameter
\texttt{angle} by an integer multiplicity of~$2\pi$ plus the rounding error.
\end{sloppypar}

\vspace{\bigskipamount}
\cprog{%
void CompEulerRotf ( float psi1, float theta1, float phi1, \\
\ind{21}float psi2, float theta2, float phi2, \\
\ind{21}float *psi, float *theta, float *phi );}
The procedure \texttt{CompEulerRotf} computes the Euler angles $\psi$, $\theta$,
$\varphi$ of the rotation, which is the composition of two rotations
represented by the Euler angles $\psi_1$, $\theta_1$, $\varphi_1$ and
$\psi_2$, $\theta_2$, $\varphi_2$ respectively.

\vspace{\bigskipamount}
\cprog{%
void CompRotV3f ( const vector3f *v1, float a1, \\
\ind{18}const vector3f *v2, float a2, \\
\ind{18}vector3f *v, float *a );}
The procedure \texttt{CompRotV3f} computes the composition of two rotations
in~$\R^3$, given by unit vectors of their axes, $\bm{v}_1$, $\bm{v}_2$
and the angles $\alpha_1$, $\alpha_2$. The procedure computes the vector~$\bm{v}$
of the composition axis and the angle~$\alpha$.

\vspace{\bigskipamount}
\cprog{%
void EulerRotTrans3f ( trans3f *tr, \\
\ind{23}float psi, float theta, float phi );}
The procedure \texttt{EulerRotTrans3f} computes the composition of the
affine transformation represented by the initial value of the parameter
\texttt{*tr} with the rotation represented by the Euler angles
$\psi$, $\theta$, $\varphi$.

\vspace{\bigskipamount}
\cprog{%
void ScaleTrans2f ( trans2f *t, float sx, float sy ); \\
void ScaleTrans3f ( trans3f *tr, float sx, float sy, float sz );}
The procedures \texttt{ScaleTrans2f} and~\texttt{ScaleTrans3f} compute
the composition of the affine transformation represented by the initial
value of the parameter \texttt{*tr} with the scaling, whose coefficients are
$s_x$ and~$s_y$ or $s_x$, $s_y$ and~$s_z$.

\vspace{\bigskipamount}
\cprog{%
void MirrorTrans3f ( trans3f *tr, vector3f *n );}
The procedure \texttt{MirrorTrans3f} computes the composition of the affine
transformation represented by the initial value of the parameter
\texttt{*tr} with the symmetric reflection with respect to the plane, which
contains the origin of the coordinate system and whose normal vector
is~$\bm{n}$.

\vspace{\bigskipamount}
\cprog{%
boolean InvertTrans2f ( trans2f *tr ); \\
boolean InvertTrans3f ( trans3f *tr );}
The procedures \texttt{InvertTrans2f} and~\texttt{InvertTrans3f} compute
the inversion of the affine transformation represented by the initial value
of the parameter \texttt{*tr}, if it exists. In this case the procedure
returns \texttt{true}, otherwise it returns \texttt{false}.

\vspace{\bigskipamount}
\cprog{%
void MultVector2f ( double a, const vector2f *v, vector2f *w ); \\
void MultVector3f ( double a, const vector3f *v, vector3f *w ); \\
void MultVector4f ( double a, const vector4f *v, vector4f *w );}
The above procedures compute the vector $\bm{w}=a\bm{v}$.

\vspace{\bigskipamount}
\cprog{%
void AddVector2f ( const point2f *p, const vector2f *v, \\
\ind{19}point2f *q ); \\
void AddVector3f ( const point3f *p, const vector3f *v, \\
\ind{19}point3f *q );}
The above procedures compute the point $\bm{q}=\bm{p}+\bm{v}$.

\vspace{\bigskipamount}
\cprog{%
void AddVector2Mf ( const point2f *p, const vector2f *v, double t, \\
\ind{20}point2f *q ); \\
void AddVector3Mf ( const point3f *p, const vector3f *v, double t, \\
\ind{20}point3f *q );}
The above procedures compute the point $\bm{q}=\bm{p}+t\bm{v}$.

\vspace{\bigskipamount}
\cprog{%
void SubtractPoints2f ( const point2f *p1, const point2f *p2, \\
\ind{24}vector2f *v ); \\
void SubtractPoints3f ( const point3f *p1, const point3f *p2, \\
\ind{24}vector3f *v ); \\
void SubtractPoints4f ( const point4f *p1, const point4f *p2, \\
\ind{24}vector4f *v );}
The above procedures compute the vector $\bm{v}=\bm{p}_1-\bm{p}_2$.

\newpage
%\vspace{\bigskipamount}
\cprog{%
void InterPoint2f ( const point2f *p1, const point2f *p2, double t, \\
\ind{20}point2f *q ); \\
void InterPoint3f ( const point3f *p1, const point3f *p2, double t, \\
\ind{20}point3f *q ); \\
void InterPoint4f ( const point4f *p1, const point4f *p2, double t, \\
\ind{20}point4f *q );}
The above procedures compute the point $\bm{q}=\bm{p}_1+t(\bm{p}_2-\bm{p}_1)$.

\vspace{\bigskipamount}
\cprog{%
void MidPoint2f ( const point2f *p1, const point2f *p2, \\
\ind{18}point2f *q ); \\
void MidPoint3f ( const point3f *p1, const point3f *p2, \\
\ind{18}point3f *q ); \\
void MidPoint4f ( const point4f *p1, const point4f *p2, \\
\ind{18}point4f *q );}
The above procedures compute the point $\bm{q}=\frac{1}{2}(\bm{p}_1+\bm{p}_2)$.

\cprog{%
void Interp3Vectors2f ( const vector2f *p0, const vector2f *p1, \\
\ind{24}const vector2f *p2, \\
\ind{24}const float *coeff, vector2f *p ); \\
void Interp3Vectors3f ( const vector3f *p0, const vector3f *p1, \\
\ind{24}const vector3f *p2, \\
\ind{24}const float *coeff, vector3f *p ); \\
void Interp3Vectors4f ( const vector4f *p0, const vector4f *p1, \\
\ind{24}const vector4f *p2, \\
\ind{24}const float *coeff, vector4f *p );}
The above procedures compute the linear combination of three vectors given as
parameters; the coefficients of the combination are given in the array
\texttt{coeff}.

\vspace{\bigskipamount}
\cprog{%
void NormalizeVector2f ( vector2f *v ); \\
void NormalizeVector3f ( vector3f *v );}
The above procedures compute \texttt{*v\,:=\,}$\frac{1}{\|\bm{v}\|_2}\bm{v}$.

\vspace{\bigskipamount}
\cprog{%
double DotProduct2f ( const vector2f *v1, const vector2f *v2 ); \\
double DotProduct3f ( const vector3f *v1, const vector3f *v2 ); \\
double DotProduct4f ( const vector4f *v0, const vector4f *v1 );}
The above procedures compute the appropriate scalar products.

\newpage
%\vspace{\bigskipamount}
\cprog{%
double det2f ( const vector2f *v1, const vector2f *v2 ); \\
double det3f ( const vector3f *v1, const vector3f *v2, \\
\ind{15}const vector3f *v3 ); \\
double det4f ( const vector4f *v0, const vector4f *v1, \\
\ind{15}const vector4f *v2, const vector4f *v3 );}
The above procedures compute the determinants of the matrices
$2\times2$, $3\times3$ and~$4\times4$ respectively, whose columns are
given as the parameters.

\vspace{\bigskipamount}
\cprog{%
void Point3to2f ( const point3f *P, point2f *p ); \\
void Point4to3f ( const point4f *P, point3f *p );}
The above procedures compute the cartesian coordinates of a~point~$\bm{p}$
based on its homogeneous coordinates.

\vspace{\bigskipamount}
\cprog{%
void Point2to3f ( const point2f *p, float w, point3f *P ); \\
void Point3to4f ( const point3f *p, float w, point4f *P );}
The above procedures compute the homogeneous coordinates of a~point~$\bm{p}$
with the weight coordinate~$w$, based on the cartesian coordinates.

\vspace{\bigskipamount}
\cprog{%
void CrossProduct3f ( const vector3f *v1, const vector3f *v2, \\
\ind{22}vector3f *v );}
The procedure \texttt{CrossProduct3f} computes the vector product of the
vectors
$\bm{v}_1$ and~$\bm{v}_2$.

\vspace{\bigskipamount}
\cprog{%
void OrtVector2f ( const vector2f *v1, const vector2f *v2, \\
\ind{19}vector2f *v ); \\
void OrtVector3f ( const vector3f *v1, const vector3f *v2, \\
\ind{19}vector3f *v );}
The procedures \texttt{OrtVector2f} and~\texttt{OrtVector3f} compute the vector
$\bm{v}=\bm{v}_2-\frac{\scp{\bm{v}_1}{\bm{v}_2}}{\scp{\bm{v}_1}{\bm{v}_1}}\bm{v}_1$.

\vspace{\bigskipamount}
\cprog{%
void CrossProduct4P3f ( const vector4f *v0, const vector4f *v1, \\
\ind{24}const vector4f *v2, vector3f *v );}
The procedure \texttt{CrossProduct4P3f} computes the first three coordinates
of the vector in~$\R^4$, which is the vector product of three vectors
$\bm{v}_1$, $\bm{v}_2$ and~$\bm{v}_3$.

\vspace{\bigskipamount}
\cprog{%
void OutProduct4P3f ( const vector4f *v0, const vector4f *v1, \\
\ind{22}vector3f *v );}
The procedure \texttt{OutProduct4P3f} computes the vector
\begin{align*}
  \bm{v} = \left[\begin{array}{c}
     X_0W_1-W_0X_1 \\ Y_0W_1-W_0Y_1 \\ Z_0W_1-W_0Z_1
  \end{array}\right].
\end{align*}


\newpage
\section{Boxes}

Rectangles and rectangular parallelpipeds are useful in various applications,
especially for estimating the locations of more complicated geometrical
figures. The types defined below describe such boxes. In the future
basic procedures of processing such boxes will be developed as a~part
of the \texttt{libpkgeom} library.

\vspace{\bigskipamount}
\cprog{%
typedef struct Box2f \{ \\
\mbox{} \ \ \ float x0, x1, y0, y1; \\
\mbox{} \ \} Box2f; \\
\mbox{} \\
typedef struct Box3f \{ \\
\mbox{} \ \ \ float x0, x1, y0, y1, z0, z1; \\
\mbox{} \ \} Box3f;}


\section{Finding the convex hull}

The headers of the procedures of finding the convex hull
(for both versions: of the single and double precision)
are in the file \texttt{convh.h}.

\vspace{\bigskipamount}
\cprog{%
void FindConvexHull2f ( int *n, point2f *p );}
The procedure \texttt{FindConvexHull2f} finds the convex hull of a~set of
$n$ points of the plane.
These points are given in the array~\texttt{p}. The initial value of
the parameter \texttt{*n} is the number of the points.
The final contents of the array consists of some of the points,
namely the subsequent vertices of the polygon being the convex hull
of the points. The number of vertices of the hull is the final value
of the parameter~\texttt{*n}.


